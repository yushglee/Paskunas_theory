\documentclass[leqno]{amsart}
\usepackage{amssymb}
\usepackage{amsmath} 
\usepackage{enumitem}
\usepackage{hyperref}
\usepackage{mathrsfs}
\usepackage{color}
\usepackage{mathtools,caption,bbm,euscript}
\usepackage[table,dvipsnames]{xcolor}
\usepackage{tikz-cd}
\usepackage[utf8]{inputenc}
\usepackage[OT2,T1]{fontenc}
\hypersetup{
 colorlinks=true,
 linkcolor=DarkOrchid,
 filecolor=blue,
 citecolor=olive,
 urlcolor=orange,
 pdftitle={Paskunas' theory},
 %pdfpagemode=FullScreen,
 }
\usepackage{booktabs}

%[label=(\alph*)]
%[label=(\Alph*)]
%[label=(\roman*)]
%[label={(\bfseries R\arabic*)}]


\setlength{\textwidth}{\paperwidth}
\addtolength{\textwidth}{-2in}
\calclayout

\tikzset{
  symbol/.style={
    draw=none,
    every to/.append style={
      edge node={node [sloped, allow upside down, auto=false]{$#1$}}}
  }
}

\newcommand{\smat}[1]{\left( \begin{smallmatrix} #1 \end{smallmatrix} \right)}
\newcommand{\mat}[1]{\left( \begin{smallmatrix} #1 \end{smallmatrix} \right)}
\newcommand{\dBr}[1]{\llbracket{#1}\rrbracket}
\newcommand{\leg}[2]{\left(\frac{#1}{#2}\right)}

% double bracket
\makeatletter
\newsavebox{\@brx}
\newcommand{\llangle}[1][]{\savebox{\@brx}{\(\m@th{#1\langle}\)}%
  \mathopen{\copy\@brx\kern-0.5\wd\@brx\usebox{\@brx}}}
\newcommand{\rrangle}[1][]{\savebox{\@brx}{\(\m@th{#1\rangle}\)}%
  \mathclose{\copy\@brx\kern-0.5\wd\@brx\usebox{\@brx}}}
  \newcommand{\llbracket}[1][]{\savebox{\@brx}{\(\m@th{#1[}\)}%
  \mathopen{\copy\@brx\kern-0.5\wd\@brx\usebox{\@brx}}}
\newcommand{\rrbracket}[1][]{\savebox{\@brx}{\(\m@th{#1]}\)}%
  \mathclose{\copy\@brx\kern-0.5\wd\@brx\usebox{\@brx}}}
\makeatother


\newcommand{\Gp}{\mathcal{G}_{\Qp}} %Galois group over \Qp
\newcommand{\laMod}{\textnormal{Mod}^{\textnormal{ladm}}}
\newcommand{\aMod}{\textnormal{Mod}^{\textnormal{adm}}}
\newcommand{\lfMod}{\textnormal{Mod}^{\textnormal{lfin}}}
\newcommand{\Rep}{\textnormal{Rep}}
\newcommand{\Ord}{\textnormal{Ord}}
\newcommand{\V}{\check{\mathbf{V}}} %Colmez
\newcommand{\Fr}{\textnormal{Fr}}


\newcommand{\phk}{\mathcal{H}_{\wt{k}}} %Hecke algebra
\newcommand{\ordp}{e}


\newcommand{\vk}{v_{-\wt{k}}} % lowest weight vector
\newcommand{\lk}{\mathnormal{l}_{\wt{k}}} % projection to lowest weight
\newcommand{\blk}{\mathnormal{l}_{\bwt{k}}} % projection to lowest weight
\newcommand{\Lk}{L_{\wt{k}}} 
\newcommand{\pf}{\hat{f}} % algebraic modular form
\newcommand{\pfk}{\hat{f}_{\wt{k}}}
\newcommand{\euF}{\EuScript{F}} % Hida family


\newcommand{\ww}{\boldsymbol{\omega}}
\DeclareMathOperator{\an}{an}
\DeclareMathOperator{\ord}{ord}
\DeclareMathOperator{\cts}{cts}


%%% Linear algebraic groups
\DeclareMathOperator{\GL}{GL}
\DeclareMathOperator{\SL}{SL}
\DeclareMathOperator{\gl}{\mathfrak{gl}}
\DeclareMathOperator{\mtr}{tr}
\DeclareMathOperator{\diag}{diag}
\DeclareMathOperator{\Ad}{Ad}
\DeclareMathOperator{\vol}{vol}

\DeclareMathOperator{\val}{val}
\DeclareMathOperator{\Lie}{Lie}
\DeclareMathOperator{\Pol}{Pol_p}

%%% Adelic rings
\newcommand{\Q}{{\mathbf{Q}}}
\newcommand{\Z}{{\mathbf{Z}}}
\newcommand{\Qp}{\mathbf{Q}_p}
\newcommand{\Zp}{\mathbf{Z}_p}
\newcommand{\Ql}{\mathbf{Q}_\ell}
\newcommand{\Zl}{\mathbf{Z}_\ell}
\newcommand{\R}{\mathbf R}
\newcommand{\C}{\mathbf C}
\newcommand{\A}{\mathbf A}
\newcommand{\hZ}{{\hat{\mathbf{Z}}}}
\newcommand{\dd}{\mathfrak{d}} %different
\newcommand{\DD}{\mathcal{D}}  %discriminant

\newcommand{\arch}{\mathbf{a}}
\newcommand{\fin}{\mathbf{h}}

\newcommand{\F}{{\mathcal{F}}} %global 
\newcommand{\OF}{{\mathcal{O}_{\F}}}
\newcommand{\K}{{\mathcal{K}}} %global quadratic
\newcommand{\OK}{\mathcal{O}_{\K}}
\newcommand{\kk}{F} %local
\newcommand{\E}{E} %local quadratic


\DeclareMathOperator{\Sel}{Sel}
\DeclareMathOperator{\Gal}{Gal}
\DeclareMathOperator{\Nr}{\mathsf{N}}
\DeclareMathOperator{\Tr}{Tr}
\newcommand{\qch}{\epsilon} % quadratic character of K/F


%%% Fonts
\newcommand{\oeu}{\EuScript{O}}
\newcommand{\eeu}{\EuScript{E}}
\newcommand{\feu}{\EuScript{F}}
\newcommand{\geu}{\EuScript{G}}
\newcommand{\keu}{\EuScript{K}}

\newcommand{\oo}{\mathcal O}
\newcommand{\bs}{\mathcal S}
\newcommand{\id}{\mathbf{1}}

\newcommand{\1}{\mathbf{1}} 
\newcommand{\bfe}{\mathbf e}
\newcommand{\bff}{\mathbf f}

\newcommand{\bX}{\mathbb{X}}
\newcommand{\bY}{\mathbb{Y}}
\newcommand{\bV}{\mathbb{V}}
\newcommand{\bW}{\mathbb{W}}

\newcommand{\fa}{\mathfrak a}
\newcommand{\fg}{\mathfrak g}
\newcommand{\fc}{\mathfrak c}
\newcommand{\fC}{\mathfrak C}
\newcommand{\B}{\mathfrak B}
\newcommand{\fs}{\mathfrak s}
\newcommand{\fm}{\mathfrak m}
\newcommand{\fn}{\mathfrak n}
\newcommand{\fl}{\mathfrak l}
\newcommand{\fp}{\mathfrak p}
\newcommand{\bfp}{\overline{\mathfrak p}}
\newcommand{\fq}{\mathfrak q}
\newcommand{\bfq}{\overline{\mathfrak q}}

\newcommand{\btheta}{\boldsymbol{\theta}}
\newcommand{\bdelta}{\boldsymbol{\delta}}


\newcommand{\fG}{\mathfrak{G}}
\newcommand{\fX}{\mathfrak{X}}
\newcommand{\euW}{\EuScript{W}}


%%% Categorical
\DeclareMathOperator{\Ext}{Ext}
\DeclareMathOperator{\End}{End}
\DeclareMathOperator{\Hom}{Hom}
\DeclareMathOperator{\Inj}{Inj}
\DeclareMathOperator{\Isom}{Isom}
\DeclareMathOperator{\Aut}{Aut}
\DeclareMathOperator{\Ind}{Ind}
\DeclareMathOperator{\cInd}{c-Ind}
\DeclareMathOperator{\coker}{coker}
\DeclareMathOperator{\rank}{rank}
\DeclareMathOperator{\corank}{corank}


\DeclareMathOperator{\Res}{Res}
\DeclareMathOperator{\rec}{rec}



\newcommand{\ee}{\mathbf e}


\newtheorem*{theorem*}{Theorem}
\newtheorem{thm}{Theorem}[section]
\newtheorem{lem}[thm]{Lemma}
\newtheorem{prop}[thm]{Proposition}
\newtheorem{cor}[thm]{Corollary}


\theoremstyle{definition}
\newtheorem{definition}[thm]{Definition}
\newtheorem{defn}[thm]{Definition}
\theoremstyle{remark}
\newtheorem{rem}[thm]{Remark}
\newtheorem*{Remark*}{Remark}
\newtheorem{ack}{Acknowledgement}

\newcommand{\red}[1]{\textcolor{Red}{#1}}



\begin{document}
\title{Paskunas' theory}
\author[Y-S.~Lee]{Yu-Sheng Lee}
\address{Department of Mathematics, University  of Michigan, Ann Arbor, MI 48109, USA}
\email{yushglee@umich.edu}
\date{\today}

\maketitle
\setcounter{tocdepth}{1}
\tableofcontents





\section{Paskunas theory}

Let $L$ be a finite extension of  $\Qp$
with the ring of integers  $\oo$
and uniformizer  $\varpi$.
Let  $k$ be the residue field.

\subsection{generically reducible deformation}

Let $\chi_1,\chi_2\colon \Gp\to k^\times$ be continuous characters
such that $\chi_1\chi_2^{-1}\neq \id,\omega^{\pm1}$.
We recall 
the structure of the universal deformation ring $R$
of the $2$-dimensional pseudo-representation $\chi=\chi_1+\chi_2$ from \cite[\S B.1]{pask}.

The generic assumption that $\chi_1\chi_2^{-1}\neq \id,\omega^{\pm1}$
implies that there exists non-split extensions
\[
    0\to \chi_1\to \rho_{12}\to \chi_2\to 0\quad
    0\to \chi_2\to \rho_{21}\to \chi_1\to 0
\]
which are unique up to isomorphisms;
and that the universal deformation rings
$R_{ij}$ of the Galois representations $\rho_{ij}$
are formally smooth of relative dimension $5$ over $\oo$.

Denote by $\tilde{\rho_{ij}}$ the universal deformation,
one may choose bases and think of which as group homomorphisms
$\tilde{\rho_{ij}}\colon \Gp\to \GL_2(R_{ij})$
so that 
$\rho_{12}=\smat{\chi_1&*\\&\chi_2}$ and
$\rho_{21}=\smat{\chi_1&\\*&\chi_2}$.
Then trace induces $\theta\colon R\cong R_{ij}$ by \cite[Prop B.17]{pask}.
Since $R^{red}$ is formally smooth of relative dimension $4$ over  $\oo$,
the reducibility ideal  $\tau\subset R$ is a principal ideal generated by 
an element in $c\in\fm_R\setminus \fm_R^2$. 
Moreover, let $\tau_{ij}\subset R_{ij} $ be the ideal 
generated by the $(j,i)$-entry of  $ \tilde{\rho_{ij}}(g)$
for all $g\in \Gp$,
then  $\theta$ maps  $\tau$ to  $\tau_{ij}$ by \cite[Prop B.23]{pask}

Let $\tilde{\rho}_{12}^c\colon \Gp\to \GL_2(R_{12})$ be the representation defined by
\[
	\tilde{\rho}_{12}^c(g)\coloneqq 
	\smat{\theta(c)&\\&1}
	\tilde{\rho}_{12}(g)
	\smat{\theta(c)^{-1}&\\&1}
\]
which is a deformation of $\rho_{21}$ to $R_{12}$
and induces an isomorphism $\alpha\colon R_{21}\to R_{12}$,
for which the diagram
\[
	\begin{tikzcd}
		R_{21} \arrow[r,"\alpha"] &
		R_{12}\\
		R \arrow[u,"\theta"] \arrow[r,equal] &
		R \arrow[u,"\theta"]
	\end{tikzcd}
\]
commutes by \cite[Prop B.24]{pask}.

Identify $\tilde{\rho}_{21}$ and $\tilde{\rho}_{12}^c$,
it can be shown that 
$\Hom_{\Gp}(\tilde{\rho}_{12}, \tilde{\rho}_{21})$ and
$\Hom_{\Gp}(\tilde{\rho}_{21}, \tilde{\rho}_{12})$
are free modules over $R$ generated respectively by
\[
	\Phi_{12}=\smat{\theta(c)&\\&1} \text{ and }
	\Phi_{21}=\smat{1&\\&\theta(c)}.
\]
Consequently, the ring $\End_{\Gp}(\tilde{\rho}_{12}\oplus \tilde{\rho}_{21})$
is isomorphic to 
$\smat{R& R\Phi_{12}\\ R\Phi_{21}& R}$,
a free $R$-module of rank  $4$,
with the center isomorphic to  $R$
by \cite[Prop B.26]{pask}.

\subsection{Ordinary parts}


In \cite{eme},
Emerton has defined a functor of ordinary parts
$\Ord_P\colon \laMod_{G,\zeta}(A)\to \laMod_{T,\zeta}(A)$
where $A$ is local complete Noetherian  $\oo$-algebra
with a finite residue field,
which satisfies that
Would you please get in touch with them to talk about how the course is structured?

$\Ord_P(\Ind_{\bar{P}}^GU)\cong U$
and that passage to ordinary parts
induces an isomorphism
\begin{equation}\label{eq:adj}
	\Hom_{A[G]}(\Ind_{\bar{P}}^GU,V)\cong
	\Hom_{A[T]}(U,\Ord_PV)
\end{equation}
by \cite[Thm 4.4.6]{eme}.

By abuse of notation,
let $\chi_1,\chi_2\colon \Qp^\times\to k^\times$
be the smooth characters obtained through
the geometrically normalized reciprocity,
sending a geometric Frobenius $\Fr$ to $p\in \Qp^\times$.
The characters satisfy the generic assumption
$\chi_1\chi_2^{-1}\neq \id,\omega^{\pm1}$.
Let $\chi\colon T\to k^\times$
be the character  $\chi=\chi_1\otimes\chi_2\omega^{-1}$
and  $\chi^s\alpha=\chi_2\otimes \chi_1\omega^{-1}$, and
\[
\pi_1\coloneqq \Ind_{P}^G\chi\cong
\Ind_{P}^G\chi_1\otimes\chi_2\omega^{-1}\quad
\pi_2\coloneqq \Ind_{P}^G\chi^s\alpha\cong 
\Ind_{P}^G\chi_2\otimes\chi_1\omega^{-1} \in \laMod_{G,\zeta}(\oo)
\]
be the induced irreducible representations
with central characters $\zeta=\chi_1\chi_2\varepsilon^{-1}$.
These are irreducible by \cite[Thm 30]{barthel}.
And by \cite[Thm 33]{barthel}, there exists
a smooth irreducible $KZ$-representation  $\sigma$
and an exact sequence
\begin{equation}
	0\to \cInd_{KZ}^B\sigma\to
	\cInd_{KZ}^B\sigma\to \pi_i\to 0
\end{equation}
where the injectiveness follows from \cite[Thm 19]{barthel}.
\cite{barthel}

There exists the unique non-split extension
\[
	0\to \pi_1\to \kappa_{21}\to \pi_2\to 0
\]


By \cite[Prop 7.1]{pask},
let $\iota\colon \pi_1\hookrightarrow \tilde{J}_1$
be the injective envelope of $\pi_1$
in $\laMod_{G,\zeta}(\oo)$,
then $\Ord_P(\iota)\colon \Ord_P(\pi_1)\to \Ord_P(\tilde{J}_1)$
is isomorphic to the injective envelope $\tilde{J}_{\chi^s}$ 
of $\chi^s=\Ord_P(\pi_1)$
in $\laMod_{T,\zeta}(\oo)$.
Furthermore, 
fix an isomorphism $\tilde{J}_{\chi^s}\to \Ord_P(\tilde{J}_1)$,
the morphism in $\Hom_{G}(\Ind_{\bar{P}}^G(\tilde{J}_{\chi^s}), \tilde{P}_1)$
induced by the adjunction formula \eqref{eq:adj}
is injective.
A similar result also holds for
the injective envelope $\pi_2\to\tilde{J}_2$ of $\pi_2$
in $\laMod_{G,\zeta}(\oo)$
and the injective envelope $\tilde{J}_{(\chi^s\alpha)^s}$ of $(\chi^s\alpha)^s$
in $\laMod_{T,\zeta}(\oo)$.

Passing to the Pontryagin dual,
let $\tilde{P}_i=\tilde{J}_i^\vee$ denote
the projective envelopes of $\pi_i^\vee$
in  $\fC_{G,\zeta}(\oo)$;
and $\tilde{M}_1, \tilde{M}_2$ denote
the Pontryagin duals of 
$\Ind_P^G\tilde{J}_\chi\cong\Ind_{\bar{P}}^G \tilde{J}_{\chi^s}$ and 
$\Ind_P^G\tilde{J}_{\chi^s\alpha}\cong
\Ind_{\bar{}}^G\tilde{J}_{(\chi^s\alpha)^s}$,
where $ \tilde{J}_\chi$ and $\tilde{J}_{\chi^s\alpha}$
are respectively the injective envelopes
of $\chi$ and  $\chi^s\alpha$ in  $\lfMod_{T,\zeta}(\oo)$.
The surjections
$p_i\colon \tilde{P}_i\twoheadrightarrow \tilde{M}_i$
obtained by previous discussions
extend to the exact sequences
\begin{equation}\label{eq:exact}
	0\to \tilde{P}_{2}\xrightarrow{\Phi_{12}} 
	\tilde{P}_{1}\xrightarrow{p_1} \tilde{M}_1\to 0 \text{ and }
	0\to \tilde{P}_{1}\xrightarrow{\Phi_{21}} 
	\tilde{P}_{2}\xrightarrow{p_2} \tilde{M}_2\to 0
\end{equation}
by \cite[Cor 7.7]{pask}.

Observe that $\Ord_P(\tilde{M}_i^\vee)=\tilde{J}_{\chi^s}=\tilde{J}_{\chi_1}$
through identify $\laMod_{T,\zeta}(\oo)$
with $\laMod_{\Qp^\times}(\oo)$ through 
identifying $\Qp^\times\cong \{\smat{1&\\&*}\}\subset T$.


Moreover, it follows from \cite[Cor 7.2]{pask} that
compositions of the surjections $\tilde{P}_i\to \tilde{M}_i$
induce surjective ring homomorphisms
\[
	\End_{\fC_{G,\zeta}(\oo)}(\tilde{P}_i)\twoheadrightarrow
	\End_{\fC_{G,\zeta}(\oo)}(\tilde{M}_i)\cong
	\End_{\fC_{T,\zeta}(\oo)}(\Ord_P(\tilde{M}_i)^\vee)\cong
	\oo\llbracket x,y\rrbracket
\]


Let $\Rep_{\Gp}(\oo)$
be the category of continuous $\Gp$-representations
on compact $\oo$-modules, 
and $\V\colon \C_{G,\zeta}(\oo)\to \Rep_{\Gp}(\oo)$
be the exact covariant functor defined in
\cite[\S 5.7]{pask}.
Note that 
$\V(\pi_i^\vee)=\chi_i$
and $\V(\kappa^\vee)=\rho_{21}$ is a non-split exact sequence
 \[
	0\to\chi_2\to \rho_{21}\to \chi_1\to 0.
\]
And $\V(\tilde{P}_1)$
is the universal deformation of $\rho_{21}$ with determinant  $\zeta\varepsilon$
by \cite[Cor 8.7]{pask}.

\begin{lem}
	Let $\fa=\ker(\tilde{E}_1\to \tilde{E})$,
	then $\fa$ is isomorphic to the reducibility ideal  $\tau$
	under the isomorphism  $ \tilde{E}_1\cong R$.
	Moreover, the 
\[
	\begin{tikzcd}
		0 &
		\tilde{M}_2 \arrow[l] \arrow[d] &
		\tilde{P}_2 \arrow[l,"p_2"] \arrow[d,"\Phi_{12}"] &
		\tilde{P}_1 \arrow[l,"\Phi_{21}"] \arrow[d,equal] & 0 \arrow[l]\\
		0 &
		\tilde{P}_1/\tau\tilde{P}_1 \arrow[l] \arrow[d] &
		\tilde{P}_1 \arrow[l] \arrow[d,"p_1"] &
		\tilde{P}_1 \arrow[l] & 0 \arrow[l]\\
				      & \tilde{M}_1 
				      & \tilde{M}_1 \arrow[l,equal] & & 
	\end{tikzcd}
\]
\end{lem}
\begin{proof}
Since $\Hom_{\C_{G,\zeta}(\oo)}(\tilde{P}_2,\tilde{M}_1)\cong
\Hom_{\laMod_{T,\zeta}(\oo)}(\tilde{J}_{\chi^s},\tilde{J}_{(\chi^s\alpha)^s})=0$
by the adjunction formula \eqref{eq:adj},
(see the discussion in the begining of \cite[\S 7.2]{pask})
taking $\Hom(\tilde{P}_2,\cdot)$ to the first exact sequence in
\eqref{eq:exact} gives
$\Phi_{12}\colon \End_{\C_{G,\zeta}(\oo)}(\tilde{P}_2)\cong 
\Hom_{\C_{G,\zeta}(\oo)}(\tilde{P}_2,\tilde{P}_1)$.

But by \cite[Lem 8.10]{pask}
and the results on the structures
of deformation rings recalled in the previous subsection,
$\V$ induces an isomorphism
 \[
	\V(\Phi_{12})\colon 
	R\cong R_{12}\cong \End_{\Gp}(\tilde{\rho}_{12})\cong
	\Hom_{\Gp}(\tilde{\rho}_{12},\tilde{\rho}_{21})\cong R\Phi_{12}.
\]
We thus conclude that 
$\Phi_{12}\circ \Phi_{21}$
is induced by a generator of the 
reducibility ideal $\tau\subset R\cong \End_{\C_{G,\zeta}(\oo)}(\tilde{P}_1)$.


Since $\Ord_P(\Phi_{ij})\in \Hom_{\C_{T,\zeta}}()\cong \Hom(\tilde{J},\tilde{J})=0$,
we see $\tau\subset \ker(\tilde{E}\to \oo\llbracket x,y\rrbracket)$,
thus equality holds.
\end{proof}

\subsection{injective elements}

We consider in this subsection 
an object $N\subset \laMod_{G,\zeta}(\oo)$ 
satisfying the assumption
\begin{equation}\label{cond:adm_inj}\tag{\text{adm-inj}}
	N\in \aMod_{G,\zeta}(\oo) 
	\text{ and is injective as a $\GL_2(\Zp)$-representation}.
\end{equation}

\begin{lem}
    When $N$ is in the abstract setup,
    let $n_i=\dim_k\Ext^n(\pi_i,N)$
    and $I_n=\tilde{J}_1^{n_1}\oplus\tilde{J}_2^{n_2}$ for $n\geq 0$,
    we have an injective resolution
    \begin{equation*}
        0\to N\to I_0 \to\cdots \to I_n \to\cdots.
    \end{equation*}
\end{lem}
\begin{proof}
    Since $N$ is admissible, the socle $\textnormal{soc}(N)$ is a finite direct sum of $\pi_1$ and $\pi_2$.
    Precisely we should have
    $\textnormal{soc}(N)=\pi_1^{0_1}\oplus \pi_2^{0_2}$ and $I_0$
    is the injective envelope of which. 
    Then $\textnormal{soc}(N)\hookrightarrow I_0$ factors as
    \begin{equation*}
    \begin{tikzcd}
       & \oplus \tilde{J}_i \\
       \text{soc}(N) \arrow[r,hookrightarrow] \arrow[ur,hookrightarrow] & M\arrow[u,dashed,"\phi"].
    \end{tikzcd}
\end{equation*}
Since $\text{soc}(N)\hookrightarrow N$ is essential, the map $\phi$ is injective.
We wish to take 
$0\to N\to J_0\to J_0/N\to 0$
and apply the same construction to $J_0/N$.
But apply $\Hom(\pi_i,\cdot)$ to the exact sequence  and use $\Hom(\pi_i,\tilde{J}_i)\cong \End(\pi_i)$, we see that
\begin{equation*}
    \begin{tikzcd}[column sep=0.8em, row sep=2ex]
        &&
        \Hom(\pi_i,\text{soc}(N)) \arrow[d,symbol=\subset] \arrow[drr,"\sim"]
        &&&&&&&\\
        0\arrow[rr] && \Hom(\pi_i,N) \arrow[rr] 
        && \Hom(\pi_i,J_0) \arrow[rr]
        && \Hom(\pi_i,J_0/N) \arrow[rr] 
        &&  \Ext^1(\pi_i,N) \arrow[rr] && 0
    \end{tikzcd}
\end{equation*}
and $\Hom(\pi_i,J_0/N)\cong \Ext^1(\pi_i,N)$.
We thus obtain $J_0/N\hookrightarrow I_1$
and inductively the rest of the injective resolution.
\end{proof}


\begin{lem}
When $N$ is in the abstract setup and $M=N^\vee$,
there exists a projective resolution
\begin{equation*}
0\to \tilde{P}_\B^r\to \tilde{P}_\B^r\to M\to 0
\end{equation*}
\end{lem}
\begin{proof}
By the previous lemma it suffices to prove that 
$0_i=1_i$ and $n_i=0$ for $n\geq 2$
then take $r=\max\{0_1,0_2\}$.
Recall that an irreducible $\pi_i$ can be expressed by
Taking the Ext functor to the exact sequence gives
\begin{equation*}
    \begin{tikzcd}[row sep=2ex]
        \Ext^{i-1}_{G}(\pi, N)\arrow[r] &
        \Ext^{i}_{G}(\text{c-ind}_{KZ}^G\sigma, N)\arrow[r] \arrow[d,symbol={=}] &
        \Ext^{i}_{G}(\text{c-ind}_{KZ}^G\sigma, N)\arrow[r] \arrow[d,symbol={=}] &
        \Ext^{i}_{G}(\pi, N)\\ 
        & \Ext^i_K(\sigma ,N) &
         \Ext^i_K(\sigma ,N) &
    \end{tikzcd}
\end{equation*}
But since $N$ is injective as a $K=\GL_2(\Zp)$-representation, we have
\begin{equation*}
    0 \to \Hom_G(\pi_i,N)\to \Hom_K(\sigma_i,N)\to \Hom_K(\sigma_i,N)\to \Ext^1_G(\pi_i,N)\to 0
\end{equation*}
It follows that $\dim_k\Hom_G(\pi_i,N)=\dim_k \Ext^1_G(\pi_i,N)$ and we have the desired result.
\end{proof}

We can write
$A=\smat{A_{11} & A_{12}\Phi_{12}\\A_{21}\Phi_{21} & A_{22}}$ where $A_{ij}\in M_2(R)$.
To shorten the notation, we define the right exact ordinary functor on $\fC_T(\oo)$ by
$\Ord_PM=((\Ord_PM^\vee)^\vee)^s\in \fC_T(\oo)$, in particular $\Ord_P\tilde{P}_1\cong \tilde{P}_\chi$ and $\Ord_P\tilde{P}_2\cong \tilde{P}_{\chi^s\alpha}$. 
Since $\Ord_P(\Phi_{ij})=0$ and $\Ord_P(\tilde{E}_{21})\cong \tilde{E}_\chi\cong R^{red}$ by  and similarly for $\Ord_P(\tilde{E}_{12})$,
the image of the resolution under $\Ord_P$ is
\begin{equation*}
    (\tilde{P}_\chi\oplus\tilde{P}_{\chi^s\alpha})^{\oplus r} \xrightarrow{\overline{A}_{11}+\overline{A}_{22}}
    (\tilde{P}_\chi\oplus\tilde{P}_{\chi^s\alpha})^{\oplus r} \to \Ord_PM\to 0
\end{equation*}

\begin{prop}    
	The matrix $\overline{A}_{11}\in M_r(R^{red})$ is injective.
    In particular apply $\Hom(\tilde{P}_\chi,\cdot)$ to the above image of $\Ord_P$ gives an exact sequence
    \begin{equation}
    0\to \End(\tilde{P}_\chi)^{\oplus r} \xrightarrow{\overline{A}_{11}}
    \End(\tilde{P}_\chi)^{\oplus r} \to \Hom(\tilde{P}_\chi,\Ord_PM)\to 0.
    \end{equation}
\end{prop}
\begin{proof}
    Since $\Ord_P$ preserves admissibility, the augmented representation $\Ord_P M\in \fC_T(\oo)$
    is finitely generated over $\oo\llbracket (1+p\Z_p)\rrbracket$, which is a formal power series ring of one variable.
    Since both $\tilde{E}_\chi$ and $\tilde{E}_{\chi^s\alpha}$ are isomorphic to $R^{red}\cong \oo\llbracket X,Y\rrbracket$,
    $\Hom(\tilde{P}_\chi,\Ord_PM)$ must be an $R^{red}$-torsion module.
    If $\ker(\overline{A}_{11})\neq 0$, then $\Hom(\tilde{P}_\chi,\Ord_PM)\otimes(frac R^{red})a\neq 0$
    gives a contradiction.
\end{proof}

\begin{prop}
	When $N$ is in the abstract setup,
    the representation $M=N^\vee\in \fC_G(\oo)$ has no $x$-torsion, where $x$ is a generator of the reducibility ideal of $R$.
\end{prop}
\begin{proof}
    recall that by, $x=\varphi_{ij}\circ \varphi_{ji}$ injective on $\tilde{P}_i$. Apply the snake lemma to 
    \begin{equation*}
    \begin{tikzcd}
        0 \arrow[r] & \tilde{P}_\B^{\oplus r} \arrow[d,"x",hookrightarrow] \arrow[r,"A"] & \tilde{P}_\B^{\oplus r} \arrow[d,"x",hookrightarrow] \arrow[r] & M \arrow[d,"x"] \arrow[r] & 0 \\ 
        0 \arrow[r] & \tilde{P}_\B^{\oplus r} \arrow[r,"A"] & \tilde{P}_\B^{\oplus r}  \arrow[r] & M  \arrow[r] & 0 
    \end{tikzcd}
\end{equation*}
Using the same argument as the previous proposition,
$\overline{A}_{11}$ and $\overline{A}_{22}$ are injective and$\Phi_{ij}\circ \Phi_{ji}=x$, so $\overline{A}\in M_{2r}(R^{red})$ is also injective.
If follows that $\overline{A}\colon (\tilde{P}_\B/(x))^{\oplus r}\hookrightarrow (\tilde{P}_\B/(x))^{\oplus r}$ 
on the cokernel is an injection. Consequently we have $M[x]=0$.
\end{proof}

\begin{prop}
    When $N\in  \laMod_{G,\zeta}(\oo)^\B$ is in the abstract setup , we have an $R^{red}$-modules isomorphism
    \begin{equation*}
        \Hom(\tilde{P}_1,M^{red})=\Hom(\tilde{P}_{\chi^s\alpha}, \Ord_PM)\oplus \Hom(\tilde{P}_\chi,\Ord_PM).
    \end{equation*}
\end{prop}
\begin{proof}
Write $M^{red}=M/(x)$ and similarly for $\tilde{P}_\B^{red}, \tilde{P}_i^{red}$, and $\Phi_{ij}^{red}\colon \tilde{P}_i^{red}\to \tilde{P}_j^{red}$ (recall that $R$ is the center of the category).
We have $0\to (\tilde{P}_\B^{red})^{\oplus r}\xrightarrow{\overline{A}}(\tilde{P}_\B^{red})^{\oplus r}\to M^{red}\to 0$ from the proof above.
Since $\Hom(\tilde{P}_1,(\tilde{P}_2^{red})^{\oplus r})\cong (R^{red} \Phi_{12}^{red})^{\oplus r}$ and $\Phi_{21}^{red}\circ \Phi_{12}^{red}=0$,
$\overline{A}$ stabilizes the subspace $\Hom(\tilde{P}_1,(\tilde{P}_2^{red})^{\oplus r})\subset \Hom(\tilde{P}_1,(\tilde{P}_\B^{red})^{\oplus r})$.
Consider then the diagram
\begin{equation*}
    \begin{tikzcd}[column sep=small]
        0 \arrow[r] & (\Hom(\tilde{P}_1,\tilde{P}_2^{red})^{\oplus r} \arrow[d,"\overline{A}_{22}",hookrightarrow] \arrow[r] & (\Hom(\tilde{P}_1,\tilde{P}_\B^{red})^{\oplus r} \arrow[d,"\overline{A}",hookrightarrow] \arrow[r] & (\Hom(\tilde{P}_1,\tilde{P}_1^{red})^{\oplus r} \arrow[d,"\overline{A}_{11}",hookrightarrow] \arrow[r] & 0 \\ 
        0 \arrow[r] &(R^{red}\Phi_{12}^{red})^{\oplus r} \arrow[d] \arrow[r] & (R^{red}\Phi_{12}^{red}\oplus R^{red})^{\oplus r} \arrow[d] \arrow[r] & (R^{red})^{\oplus r}\arrow[d] \arrow[r] & 0 \\ 
        0 \arrow[r] & (R^{red} \Phi_{12})^{\oplus r}/\overline{A}_{22}  \arrow[r] & \Hom(\tilde{P}_1,M^{red}) \arrow[r] & (R^{red})^{\oplus r}/\overline{A}_{11}\arrow[r] & 0 
    \end{tikzcd}
\end{equation*}
Apply  to the two columns on the sides,  we have an exact sequence of $R^{red}$-modules
\begin{equation*}
    0\to \Hom(\tilde{P}_{\chi^s\alpha}, \Ord_PM)\to \Hom(\tilde{P}_1,M^{red})\to \Hom(\tilde{P}_\chi,\Ord_PM)\to 0
\end{equation*}
which is split since the $R^{red}$-actions via $\tilde{P}_\chi$ and $\tilde{P}_{\chi^s\alpha}$ are distinct.
\end{proof}

\bibliographystyle{amsalpha}
\bibliography{biblio}
\end{document}
