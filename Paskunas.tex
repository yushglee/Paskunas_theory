\documentclass[leqno]{amsart}
\usepackage{amssymb}
\usepackage{amsmath} 
\usepackage{enumitem}
\usepackage{hyperref}
\usepackage{mathrsfs}
\usepackage{color}
\usepackage{mathtools,caption,bbm,euscript}
\usepackage[table,dvipsnames]{xcolor}
\usepackage{tikz-cd}
\usepackage[utf8]{inputenc}
\usepackage[OT2,T1]{fontenc}
\hypersetup{
 colorlinks=true,
 linkcolor=DarkOrchid,
 filecolor=blue,
 citecolor=olive,
 urlcolor=orange,
 pdftitle={Pask\={u}nas' theory},
 %pdfpagemode=FullScreen,
 }
\usepackage{booktabs}
%[label=(\alph*)]
%[label=(\Alph*)]
%[label=(\roman*)]
%[label={(\bfseries R\arabic*)}]


\setlength{\textwidth}{\paperwidth}
\addtolength{\textwidth}{-2in}
\calclayout

\tikzset{
  symbol/.style={
    draw=none,
    every to/.append style={
      edge node={node [sloped, allow upside down, auto=false]{$#1$}}}
  }
}

\newcommand{\smat}[1]{\left( \begin{smallmatrix} #1 \end{smallmatrix} \right)}

% double bracket
\makeatletter
\newsavebox{\@brx}
\newcommand{\llangle}[1][]{\savebox{\@brx}{\(\m@th{#1\langle}\)}%
  \mathopen{\copy\@brx\kern-0.5\wd\@brx\usebox{\@brx}}}
\newcommand{\rrangle}[1][]{\savebox{\@brx}{\(\m@th{#1\rangle}\)}%
  \mathclose{\copy\@brx\kern-0.5\wd\@brx\usebox{\@brx}}}
  \newcommand{\llbracket}[1][]{\savebox{\@brx}{\(\m@th{#1[}\)}%
  \mathopen{\copy\@brx\kern-0.5\wd\@brx\usebox{\@brx}}}
\newcommand{\rrbracket}[1][]{\savebox{\@brx}{\(\m@th{#1]}\)}%
  \mathclose{\copy\@brx\kern-0.5\wd\@brx\usebox{\@brx}}}
\makeatother




\newcommand{\fF}{\mathbb{F}} % fintie field
\newcommand{\euF}{\EuScript{F}} %Hida family
\newcommand{\M}{\mathbf{M}} % modular form
\newcommand{\bnu}{\boldsymbol{\nu}}
\newcommand{\wt}[1]{\underline{ #1 }}
\newcommand{\bwt}[1]{\underline{\boldsymbol { #1 }}}
\newcommand{\TT}{\mathbb{T}} % Hecke 
\newcommand{\fG}{\mathfrak{G}}
\newcommand{\fX}{\mathfrak{X}}
\newcommand{\GG}{\mathbf G} 
\newcommand{\Iw}{\textnormal{Iw}} 

\DeclareMathOperator{\Spec}{Spec}

\newcommand{\bw}{\overline{w}}
%%% Block theory

\newcommand{\aMod}{\textnormal{Mod}^{\textnormal{adm}}}
\newcommand{\laMod}{\textnormal{Mod}^{\textnormal{l.adm}}}
\newcommand{\lfMod}{\textnormal{Mod}^{\textnormal{lfin}}}
\newcommand{\fgMod}{\textnormal{Mod}^{\textnormal{fg.aug}}}
\newcommand{\Ban}{\textnormal{Ban}^{\textnormal{adm}}}
\DeclareMathOperator{\Mod}{\textnormal{Mod}}
\DeclareMathOperator{\Rep}{Rep}
\newcommand{\B}{\mathfrak B} 
\newcommand{\fC}{\mathfrak C}
\DeclareMathOperator{\soc}{soc}
\DeclareMathOperator{\V}{\check{\mathbf{V}}} %Colmez


%%% p_adic Hodge

\newcommand{\Gp}{\mathcal{G}_{\Qp}} %Galois group over \Qp
\newcommand{\Fr}{\textnormal{Fr}} %geometric Frobenius
\newcommand{\frob}{\textnormal{frob}} %arithmetic Frobenius
\newcommand{\dR}{\textnormal{dR}}
\newcommand{\pst}{\textnormal{pst}}
\newcommand{\cris}{\textnormal{cris}}

\DeclareMathOperator{\Gal}{Gal}

\DeclareMathOperator{\Ord}{Ord}
\DeclareMathOperator{\Irr}{Irr}
\DeclareMathOperator{\WD}{WD}
\DeclareMathOperator{\rec}{rec}
\DeclareMathOperator{\Rec}{Rec}
\DeclareMathOperator{\Art}{Art}

\newcommand{\cont}{\textnormal{cont}}
\newcommand{\cts}{\textnormal{cts}}
\newcommand{\alg}{\textnormal{alg}}
\newcommand{\sm}{\textnormal{sm}}
\newcommand{\adm}{\textnormal{adm}}
\newcommand{\ps}{\textnormal{ps}}
\newcommand{\red}{\textnormal{red}}
\newcommand{\fin}{\textnormal{fin}}
\newcommand{\an}{\textnormal{an}}
\newcommand{\ord}{\textnormal{ord}}


%%% Linear algebraic groups
\DeclareMathOperator{\GL}{GL}
\DeclareMathOperator{\SL}{SL}
\DeclareMathOperator{\gl}{\mathfrak{gl}}
\DeclareMathOperator{\mtr}{tr}
\DeclareMathOperator{\diag}{diag}
\DeclareMathOperator{\Ad}{Ad}
\DeclareMathOperator{\vol}{vol}
\DeclareMathOperator{\Sym}{Sym}

\DeclareMathOperator{\Lie}{Lie}

\newcommand{\bs}{\mathcal{S}}
\newcommand{\id}{\mathbf{1}}

%%% Adelic rings
\newcommand{\Q}{{\mathbf{Q}}}
\newcommand{\Z}{{\mathbf{Z}}}
\newcommand{\Qp}{\mathbf{Q}_p}
\newcommand{\Zp}{\mathbf{Z}_p}
\newcommand{\Ql}{\mathbf{Q}_\ell}
\newcommand{\Zl}{\mathbf{Z}_\ell}
\newcommand{\R}{\mathbf R}
\newcommand{\C}{\mathbf C}
\newcommand{\A}{\mathbf A}
\newcommand{\dd}{\mathfrak{d}} %different
\newcommand{\DD}{\mathcal{D}}  %discriminant
\DeclareMathOperator{\Nr}{\mathsf{N}} %norm
\DeclareMathOperator{\Tr}{Tr} %trace

\newcommand{\arch}{\mathbf{a}}
\newcommand{\finite}{\mathbf{h}}

\newcommand{\F}{{\mathbf{F}}} %global field
\newcommand{\K}{{\mathbf{K}}} %global quadratic
\newcommand{\kk}{F} %local field
\newcommand{\E}{E} %local quadratic
\newcommand{\qch}{\epsilon} % quadratic character of K/F

\newcommand{\oo}{\mathcal{O}} %ring of integer
\DeclareMathOperator{\val}{val}


%%% Fonts

\newcommand{\oeu}{\EuScript{O}}
\newcommand{\eeu}{\EuScript{E}}
\newcommand{\feu}{\EuScript{F}}
\newcommand{\geu}{\EuScript{G}}
\newcommand{\keu}{\EuScript{K}}

\newcommand{\fa}{\mathfrak{a}}
\newcommand{\fc}{\mathfrak{c}}
\newcommand{\fg}{\mathfrak{g}}
\newcommand{\fk}{\mathfrak{k}}
\newcommand{\fs}{\mathfrak{s}}
\newcommand{\fm}{\mathfrak{m}}
\newcommand{\fn}{\mathfrak{n}}
\newcommand{\fl}{\mathfrak{l}}
\newcommand{\fp}{\mathfrak{p}}
\newcommand{\fq}{\mathfrak{q}}
\newcommand{\bfp}{\overlin{\mathfrak p}}
\newcommand{\bfq}{\overline{\mathfrak q}}

\newcommand{\btheta}{\boldsymbol{\theta}}
\newcommand{\bdelta}{\boldsymbol{\delta}}


%%% Categorical
\DeclareMathOperator{\End}{End}
\DeclareMathOperator{\Aut}{Aut}
\DeclareMathOperator{\Hom}{Hom}
\DeclareMathOperator{\Ext}{Ext}
\DeclareMathOperator{\Tor}{Tor}
\DeclareMathOperator{\Ind}{Ind}
\DeclareMathOperator{\cInd}{c-Ind}
\DeclareMathOperator{\nInd}{n-Ind}
\DeclareMathOperator{\coker}{coker}
\DeclareMathOperator{\Image}{Im}
\DeclareMathOperator{\rank}{rank}
\DeclareMathOperator{\corank}{corank}
\DeclareMathOperator{\Res}{Res}




\newtheorem{thm}{Theorem}[section]
\newtheorem{lem}[thm]{Lemma}
\newtheorem{prop}[thm]{Proposition}
\newtheorem{cor}[thm]{Corollary}


\theoremstyle{definition}
\newtheorem{defn}[thm]{Definition}


\theoremstyle{remark}
\newtheorem{rem}[thm]{Remark}
\newtheorem{ack}{Acknowledgement}




\begin{document}
\title{Pask\={u}nas' theory}
\author[Y-S.~Lee]{Yu-Sheng Lee}
\address{Department of Mathematics, University  of Michigan, Ann Arbor, MI 48109, USA}
\email{yushglee@umich.edu}
\date{\today}

\maketitle
\setcounter{tocdepth}{1}
\tableofcontents

\section{Notations}

Throughout the article, $\F$ is a totally real field
and $\K$ is a totally imaginary quadratic extension over $\F$.
Denote by $\arch=\Hom(\F, \C)$ 
the set of archimedean places of $\F$,
and by $\fin$ the set of finite places of $\F$.
Let $\dd_\K$ and $\dd_{\K/\F}$ denote respectively 
the absolute and the relative ideals of different in $\K$,
recall that $\dd_\K=\dd_{\K/\F}\dd_\F$,
where $\dd_\F$ denotes 
the absolute ideals of different in $\F$.

We fix an odd prime $p$ throughout the article
and assume that $p$ is prime to the class number $h_\K$,
the number of roots of unity in $\K$,
and satisfies the following ordinary condition.
\begin{equation}\label{cond:ord}\tag{ord}
\text{Every finite place of $\F$ above $p$ is split in $\K$}.
\end{equation}
We fix an embedding $\iota_\infty:\bar{\Q}\to \C$
and an isomorphism $\iota:\C\cong \C_p$,
and write $\iota_p=\iota\circ\iota_\infty:\bar{\Q}\to \C_p$.


Given a place $v$ of $\F$, archimedean or finite,
let $w\mid v$ denote a place $w$ of $\K$ above $v$.
Then $\K_w$ and $\F_v$ are respectively
the completions of the fields $\K$ and $\F$ at $w$ and $v$.
When $v\in \fin$ we denote by $\oo_w$ and $\oo_v$ 
the rings of integers of $\K_w$ and $\F_v$.
Let $|\cdot|_v$ be the norm on $\F_v$,
which is the usual absolute value when $v\in \arch$
and $q_v=|\varpi_v|_v^{-1}$,
for any choice of uniformizer $\varpi_v$ in $\oo_v$,
is the cardinality of the residue field $\oo_v/(\varpi_v)$
when $v\in \fin$.
For $w\mid v$, define $|a|_w=|\Nr_{\K_w/\F_v}(a)|_v$.


Denote by $\A=\A_{\F}$ the ring of adeles over $\F$,
by $\A_{\infty}$ and $\A_{f}$ respectively
the archimedean and the finite components of $\A$.
Let $\qch_{\K/\F}$ denote 
the quadratic character on $\A_\F^\times/\F^\times$
associated to $\K/\F$ by the global class field theory,
$\qch_v$ denote the component on $\F_v^\times$ 
when $v\in \fin$.
If $\eta$ is a character of $\A_\K^1/\K^1$, 
we denote
by $\tilde{\eta}(\alpha)\coloneqq \eta(\alpha/\alpha^c)$
the Hecke character which is the base change of $\eta$ 
to $\A_\K^\times/\K^\times$.

\subsection{CM types}

Denote respectively by $S_p$ and $S_p^\K$ the set of places above $p$
of $\F$ and $\K$.
Identify $I_\K=\Hom(\K,\bar{\Q})$ with
$\Hom(\K,\C)$ and $\Hom(\K,\C_p)$ by compositions with $\iota_\infty$ and $\iota_p$.
Given $\sigma\in I_\K$,
let $w_\sigma\in S_p^\K$ be the place induced by
$\sigma_p\coloneqq \iota_p\circ \sigma\in\Hom(\K,\C_p)$.
For $w\in S_p^\K$, define
\[
    I_w=\{\sigma\in I_\K\mid w=w_\sigma \}=\Hom(\K_w,\C_p)
\]
and decompose $I_\K=\sqcup_{w\mid p}I_w$.
For a subset $\Sigma\subset I_\K$
define $\Sigma_p=\{w_\sigma\mid \sigma\in \Sigma\}$.
We write
$\Sigma^c=\{\sigma c\mid \sigma\in \Sigma\}$ and 
$\Sigma_p^c=\{cw\mid w\in \Sigma_p\}$.
We fix throughout the article a $p$-ordinary CM type,
which is a subset $\Sigma\subset I_\K$ such that
\[
    \Sigma\sqcup \Sigma^c=I_\K,\quad
    \Sigma_p\sqcup \Sigma_p^c=S_p^\K.
\]
The $p$-ordinary CM type $\Sigma$
always exists by the assumption \eqref{cond:ord},
and is identified with $\arch=\Hom(\F,\C)$ by restrictions.
When $v\in S_p$ decomposes into $v=w\bw$,
we understand always that $w\in \Sigma_p$.

\subsection{Characters}

Given 
$\kappa=\sum_{\sigma\in \Sigma} a_\sigma\sigma+b_\sigma\sigma c\in \Z[I_\K]$,
an algebraic Hecke character 
$\chi\colon \A_\K^\times/\K^\times\to \C^\times$ 
has type $\kappa$ if
\[
    \chi_\infty(\alpha)=
    \iota_\infty \left(\prod_{\sigma\in \Sigma} 
    \sigma(\alpha)^{a_\sigma}\sigma(c \alpha)^{b_\sigma}\right),\quad
    \alpha\in \K^\times.
\]
For $\alpha_\infty=(\alpha_\sigma)\in \A_{\K,\infty}^\times$
and $\alpha_p=(\alpha_w,\alpha_{\bw})\in \prod_{v\in S_p}\K_v^\times$, 
define
\[
    \alpha_\infty^\kappa=
    \prod_{\sigma\in \Sigma} 
    (\alpha_\sigma)^{a_\sigma}(\bar{\alpha}_\sigma)^{b_\sigma}\in \C^\times,\quad
    \alpha_p^\kappa=
    \prod_{w\in \Sigma_p}
    \prod_{\sigma\in I_w}
    \sigma_p(\alpha_w)^{a_\sigma}\sigma_p(\alpha_{\bw})^{b_\sigma}\in \C_p^\times,
\]
for example,
$(2\pi)^\Sigma=(2\pi)^{[\F:\Q]}$
when $2\pi$ is embedded diagonally in $\A_{\K,\infty}^\times$.
Define the $p$-adic avatar of $\chi$ by
\[
    \hat{\chi}\colon \A_\K^\times\to \bar{\Z}_p^\times,\quad
    \hat{\chi}(\alpha)=\iota(\chi(\alpha)\alpha_\infty^{-\kappa})\alpha_p^{\kappa}
\]
where $\alpha_\infty$ and $\alpha_p$ are respectively 
the archimedean component and the components above $p$ of $\alpha\in \A_\K^\times$.


\subsection{Matrices}
When $R$ is an $\F$-algebra and 
$m=(m_{ij})\in \text{M}_{r,s}(\K\otimes_\F R)$,
we denote by 
$m^\intercal=(m_{ji}), 
m^c=(m^c_{ij})$, and
$m^*=(m^c_{ji})$
respectively the transpose, conjugate, and conjugate-transpose of $m$.

When $r=s$ and $g\in \GL_r(\K\otimes_\F R)$ is invertible, we write
$g^{-\intercal}=(g^{-1})^\intercal$ and $g^{-*}=(g^{-1})^*$.
We write $\mtr(m)$ for the trace of a square matrix $m$,
and reserve $\Tr$ for the traces between fields extensions.

When $v=w\bw$ is a place that is split in $\K$,
identify $\K_w=\F_v=\K_{\bw}$ and 
write $\K_v=\F_v^2$, 
where the first component corresponds to $\K_w$.
Then $m=(m_w,m_{\bw})\in M_n(\K\otimes_\F\F_v)=M_n(\F_v)\times M_n(\F_v)$ 
denotes an element in $m\in M_n(\K\otimes_\F\F_v)$ and its components.

\subsection{Representations of $p$-adic groups}

Let $\oo$ be the ring of integers of a finite extension $E$
over  $\Qp$.
When $G$ is a $p$-adic analytic group,
let $\Mod_G(\oo)$ be the category
of all $\oo[G]$-modules.
We refer the readers to \cite[\S 2]{emeI} and \cite[\S 2]{pask}
for the notations and definitions of the following subcategories.
\[
\begin{tikzcd}
	\fgMod_{G}(\oo) \arrow[r,leftrightarrow] &
	\laMod_{G}(\oo) \arrow[r,hookrightarrow] &
	\aMod_{G}(\oo) \arrow[r,hookrightarrow] &
	\Mod^{\sm}_{G}(\oo) \\
					       &&
	\lfMod_{G}(\oo) \arrow[ru,hookrightarrow] &
\end{tikzcd}
\]



\section{Modular forms on definite unitary groups}

Let $G$ be the definite unitary group over $\F$,
such that for any $\F$-algebra $R$
\[
    G(R)=\{g\in \GL_{n}(\K\otimes_\F R) \mid gg^*=\id_n\}.
\]
In this section we recall from \cite{ger}
the notion of algebraic modular forms on $G$
and results on the associated Galois representation.
Some results from \textit{loc.cit}
regarding Hida theory of ordinary forms
are generalized
to that of $P$-ordinary forms,
for a more general parabolic subgroup $P$,
after incorporating Emerton's functor in \cite{emeI}
We then introduce 
the big Hecke algebra acting on 
the completed cohomology of $P$-ordinary forms,
which admits a Galois pseudo-representation
of $\Gal_\K$.
Borrowing from the idea in \cite{pan},
we can deduce a density result
of crystalline points in the Hecke algebra,
which is crucial for checking 
the local-global compatibility in the next section.


\subsection{Algebraic modular forms}

When $v=w\bw$ is place of $F$ that is split in $\K$,
the group $G(\F_v)$ is a subgroup 
of  $\GL_n(\F_v\otimes_\F\K)\cong \GL_n(\K_w)\times\GL_n(\K_{\bw})$.
Write $g_v=(g_w,g_{\bw})\in G(\F_v)$,
the map $g_v\mapsto g_w$ then defines an isomorphism
$\iota_w\colon G(\F_v)\cong \GL_n(\F_v)$
such that 
$\iota_w(g_v)=\iota_{\bw}(g_v)^{-\intercal}$.
In particular,
for each $v\in S_p$
let  $w\mid v$ be such that  $w\in \Sigma_p$
and write $G_{w}\coloneqq\GL_n(\K_w)$,
we identify $G(\F_v)$ with $G_w$ 
via $\iota_w$ and define 
\[
	G_p\coloneqq\prod_{w\in \Sigma_p}G_w,\quad
	K_p\coloneqq\prod_{w\in \Sigma_p}K_w,\quad
	K_w\coloneqq\GL_n(\oo_w).
\]

Let $B_n\subset \GL_n$ be the subgroup of
upper trigiangular matrices
and $B_n=T_nN_n$ be the Levi decomposition,
where $T_n$ is the diagonal torus.
We identify the set of algebraic characters $X^*(T_n)$
with  $\Z^n$.
The Weyl group $W_n$ of $\GL_n$
then acts on  $\Z^n$ by
$(wk)(t)=k(w^{-1}tw)$.
Let $w_0\in W_n$ denote the longest element.

Following \cite[Def 2.3]{ger},
we say  $k=(k_1,\cdots,k_n)\in \Z^n$
is a dominant weight if $k_1\geq \cdots\geq k_n$.
In which case
$\xi_k\coloneqq \Ind_{B_n}^{\GL_n}(w_0k)$
is the algebraic representation 
of $\GL_n$ of highest weight $k$.
More generally,
we say $\wt{k}=(k_\sigma)\in (\Z^n)^{\Sigma}$
is a dominant 
if $k_\sigma=(k_{\sigma,1},\cdots,k_{\sigma,n})$
is a dominant weight for each $\sigma\in \Sigma$.
When this is the case,
and $\oo$ is the ring of integers  
of a finite extension $E$ over  $\Qp$
that contains $\iota_p(\sigma(\K))$
for all  $\sigma\in I_\K$,
we let $\xi_{\wt{k}}$ denote 
the algebraic $G_p$-representation over $\oo$ given by
\begin{equation}\label{def:algrep}
	\xi_{\wt{k}}=\bigotimes_{\sigma\in \Sigma}
	\Ind_{B_n}^{\GL_n}(w_0k_{\sigma}),\quad
	\xi_{\wt{k}}(g)=
	\otimes_{w\in \Sigma_p}
	\otimes_{\sigma\in I_w}\xi_{k_\sigma}(g_w)\,
	\text{ for } g=(g_w)\in G_p.
\end{equation}

\begin{rem}
	In previous work, 
	we have defined $\rho_k$ as the 
	algebraic representation of lowest weight  $-k$.
	Thus $\xi_k$ is isomorphic to the representation 
	$\rho^k(g)\coloneqq \rho_k(g^{-\intercal})$.
\end{rem}




\begin{defn}\label{def:algform}
Let $M_{\wt{k}}$ be the finite free $\oo$-module
on which $K_p$ acts by $\xi_{\wt{k}}$
if $\wt{k}\in (\Z^n)^{\Sigma}$ is dominant. 
For any $\oo$-module $M$ and any function
$f\colon G(\F)\backslash G(\A_f)\to M\otimes_{\oo}M_{\wt{k}}$,
let $k=(k^p,k_p)\in G(\A_f^p)\times K_p$ acts on $f$ by 
$(k\cdot f)(g)=\xi_{\wt{k}}(k_p)\cdot f(gk)$.
We say $f$ is an algebraic modular form of 
weight $\wt{k}$ and coefficients in $M$
if $f$ is invariant by some open compact subgroup
$U\subset G(\A_f^p)\times K_p$.
Let $S_{\wt{k}}(M)$
denote the space of such modular forms.
Then for $U$ as above we 
denote the subspace of forms
of level $U$ by
\begin{equation}
S_{\wt{k}}(U,M)=
S_{\wt{k}}(M)^U=
\left\{ f: G(\F)\backslash G(\A_f)/U^p 
\rightarrow M\otimes_{\oo}M_{\wt{k}}
\mid f(gu)=\xi_{\wt{k}}(u_p)^{-1}\cdot f(g), u\in U\right\} 
\end{equation}
Note that when $M$ is an  $E$-module,
the above action can be extended 
to  $G(\A_f)$
and the notation  $S_{\wt{k}}(U,M)$
makes sense for any open compact subgroup
$U\subset G(\A_f)$.

We will omit $\wt{k}$ 
when $\xi_{\wt{k}}$ is the trivial representation
and simply write
$S(M)$ and  $S(U,M)$.
\end{defn}


Since $\GG(\F)\backslash \GG(\A_f)/U$ is a finite set
for any open compact subgroup $U\subset G(\A_f)$,
any modular form $f\in S_{\wt{k}}(U,M)$ 
is determined by its values on a finite set of points.
Throughout the section,
we fix an open compact subgroup 
$U^p\subset G(\A_f^p)$ satisfying 
\begin{equation}\label{cond:small}\tag{$U^p$-\text{small}}
	\GG(\A_f)=\bigsqcup_{i\in I}
	\GG(\F)t_i U,\quad
	\GG(\F)\cap t_iUt_i^{-1}=\{1\} \text{ for all } i\in I
	\text{ for } U=U^pK_p
\end{equation}
Then the space $S_{\wt{k}}(U^pU_p,M)$ is
isomorphic to a finite direct sum of 
$M\otimes_{\oo}M_{\wt{k}}$
for any open compact subgroup $U^p\subset K_p$,
where the isomorphism
is given by 
evaluating at a set of representatives
for  $G(\F)\backslash G(\A_f)/U^pU_p$.

\subsection{Emerton's functor of ordinary parts}

Only in this subsection,
let $G$ denote a  $p$-adic reductive group,
$P$ be a parabolic subgroup,
with Levi decomposition $P=QU$.
We briefly recall the functor
of ordinary parts 
$\Ord_P\colon \Mod_G^{\sm}(\oo)\to \Mod_Q^{\sm}(\oo)$
defined in \cite{emeI}.
Fix an open compact subgroup $P_0\subset P$.
Let  $Q_0=P_0\cap Q$ and $U_0=P_0\cap U$
and define $Z_Q^+=Z_Q\cap Q^+$,
where  $Z_Q$ is the center of $Q$ and
\[
	Q^+=\{m\in Q\mid mU_0m^{-1}\subset U_0\}.
\]
If  $V$ is a $P$-representation of over $\oo$
and  $m\in Q^+$,
as in \cite[Def 3.1.3]{emeI} we define
\begin{equation}\label{def:hUm}
	 h_{U}(m)\colon V^{U_0}\to V^{U_0}\quad
	 h_{U}(m)(v)=\sum_{u\in N_0/m U_0 m^{-1}}um\cdot v
\end{equation}
We now recall the definition of the functor
and refer to \cite[Def 3.1.3]{emeI}
for the notations in below.
\begin{equation}\label{def:OrdP}
	\Ord_P\colon \Mod_G^{\sm}(\oo)\to \Mod_Q^{\sm}(\oo)
	\Ord_P(V)=\Hom_{\oo[Z_Q^+]}(\oo[Z_Q], V^{U_0})_{Z_Q-\fin}.
\end{equation}
Here $Z_Q^+$ acts by translation on the left; 
by $h_U$ on the right.
And the action of $Q=Z_Q\cdot Q^+$ is induced by 
havig $Z_Q$ act by translation on the left and 
$Q^+$ act by $h_U$ on the right.

\subsection{Hecke operators and $P$-ordinary forms}

We now return to the previous settings,
so $G$ is the definite unitary group over $\F$
and  $U^p$ satisfies \eqref{cond:small}.
Let $R=R(U^p)$ be a finite set of
finite places of $\F$ containing $S_p$ such that
$\iota_w^{-1}(\GL_n(\oo_w))\subset U^p$
when  $v\notin R$
and  $v=w\bw$ is split in  $\K$.

Recall that $B_n=T_nN_n$ is the Levi decomposition
for the group of upper triangular matrices in  $\GL_n$.
We put $B=\prod_{w\in \Sigma_p}B_w$
for $B_w=B_n(\K_w)$,
which admits decompositions 
$B=TN$ and  $B_w=T_wN_w$
where the subgroup  $T, N, T_w, N_w$ are defined similarly.
Given integers $c\geq b\geq 0$ with  $c>0$, 
we define 
the open compact subgroup 
$\Iw(p^{b,c})=\prod_{w\in \Sigma_p}\Iw(w^{b,c})$, where 
for each $w\in \Sigma_p$
\begin{equation}\label{def:Iwahori}
	\Iw(w^{b,c})=\{
	k\in K_w\mid 
	\iota_w(k) \text{ mod } \varpi_w^c \in B_n(\oo/\varpi_w^c)
	\text{ and }
	\iota_w(k) \text{ mod } \varpi_w^b \in N_n(\oo/\varpi_w^b)
	\}.
\end{equation}
In \cite{ger},
the Hecke operators on $S_{\wt{k}}(U^p\Iw(p^{b,c}),M)$
are defined as the following double-coset operators.

If $v=w\bw$ is split in  $\K$ and $v\neq R$,
for $1\leq j\leq n$ let 
\begin{equation}\label{def:hecke_away_p}
	T_w^{(j)}=
	\left[\iota_w^{-1}\left(
	\GL_n(\oo_v)
	\begin{pmatrix}
		\varpi_v\id_{j}&\\&\id_{n-j}
	\end{pmatrix}
	\GL_n(\oo_v)
	\right)\right],
	\text{ note that }
	T_{\bw}^{(j)}=(T_{w}^{{n}})^{-1}T_w^{(n-j)}.
\end{equation}

If $w\in \Sigma_p$, let  
$\alpha_w^{(j)}=\iota_w^{-1}
\left(\begin{smallmatrix}
\varpi_v\id_{j}&\\&\id_{n-j} 
\end{smallmatrix}\right)$ for $1\leq j\leq n$,
and $u\in \iota_w^{-1}(T_n(\oo_w))$, define
\begin{equation}\label{def:hecke_at_p}
	U_{\wt{k},w}^{(j)}=
	(w_0\wt{k})^{-1}(\alpha_{w}^{(j)})\cdot
	[\Iw(p^{b,c})\alpha_w^{(j)}\Iw(p^{b,c})]
	\text{ and }
	\langle u\rangle= (w_0\wt{k})^{-1}(u)\cdot 
	[\Iw(p^{b,c})u\Iw(p^{b,c})].
\end{equation}
Here $w_0\wt{k}$ is viewed as an algebraic character of $T$ 
using the same recipe as in \eqref{def:algrep}.
\begin{rem}
	Our definition of $\langle u\rangle$
	is different from that of \cite{ger}
	by the factor $(w_0\wt{k})^{-1}(u)$.
\end{rem}
Put $B_0=B\cap K_p$,
and $T^+, N_0$ be as in last subsection.
Then $N_0T^+$ is a monoid
and we can define the following action of which
on  $S_{\wt{k}}(U^p\Iw(p^{b,c}),M)$
extending the trivial action of $N_0$
\begin{equation}\label{def:T_act}
	(nt\cdot f)(g)=(w_0\wt{k})^{-1}(t)\xi_{\wt{k}}(nt)\cdot f(gnt).
\end{equation}
The same formula \eqref{def:hUm}
then defines an endomorphism $h_N(t)$
for $S_{\wt{k}}(U^p\Iw(p^{b,c}),M)$
for $t\in T^+$.
In particular we have
$U_{\wt{k},w}^{(j)}=h_N(\alpha_{w}^{(j)})$ and $\langle u\rangle= h_N(u)$
for $\alpha_w^{(j)}$ and $u$ as in \eqref{def:hecke_at_p}. 


More generally, 
we may pick a standard parabolic subgroup
$P_w\supset B_w$ for each  $w\in \Sigma_p$
and let  $P_w=Q_wU_w$ be the Levi decomposition.
We then put $P=\prod_{w\in \Sigma_p}P_w, P_0=P\cap K_p$,
and define the subgroups $Q,U, Q_0, U_0$ similarly.
For $w\in \Sigma_p$,
let  $\Iw^P(w^{b,c})$ 
be defined as in \eqref{def:Iwahori}
upon replacing $B_n$ and $N_n$
with $P_w$ and  $U_w$.

\begin{defn}\label{def:hecke}
Let $\Iw^P(p^{b,c})=\prod_{w\in \Sigma_p}\Iw^P(w^{b,c})$
for integers  $c\geq b\geq 0$ with $c>0$
and define the action of $U_0T^+$ 
on $S_{\wt{k}}(U^p\Iw^P(p^{b,c}),M)$
extending the trivial action of $U^0$ as in \eqref{def:T_act}.
We define the Hecke operators $T_w^{(j)}$
as in \eqref{def:hecke_away_p}; and 
$U_{\wt{k},w}^{(j)}=h_U(\alpha_w^{(j)}),\langle u\rangle=h_U(u)$,
for $\alpha_w^{(j)}$ and $u$ as in \eqref{def:hecke_at_p}.
We also define $U_P$
as the product of all  $U_{\wt{k},w}^{(j)}$
for which $\alpha_w^{(j)}\in Z_Q$.
\end{defn}


\begin{lem}
The Hecke operators defined above commutes with each other
and are equivariant with respect to the inclusions
$ S_{\wt{k}}(U^p\Iw^P(p^{b,c}),M)\hookrightarrow
S_{\wt{k}}(U^p\Iw^P(p^{b',c'}),M)$
if $b'\geq b$ and $c'\geq c$.
\end{lem}
\begin{proof}
That each $T_w^{(j)}$ commutes with other Hecke operators is classical,
and the equivariance is clear.
For the Hecke operators at  $w\in \Sigma_p$,
the commutivity follows from \cite[Lem 3.1.4]{emeI},
and the equivariance follows from 
that of the action \eqref{def:T_act}.
See also \cite[Lem 2.10]{ger} for the proof when $P=B$.
\end{proof}


When $M$ is either a finite  $\oo$-module
or the Pontryagin dual of which,
the operator $e_P\coloneqq\lim_{n\to \infty}(U_P)^{n!}$
converges to an idempotent 
on $S_{\wt{k}}(U^p\Iw^P(p^{b,c}),M)$.
We then define the space of $P$-ordinary forms by
\[
	S_{\wt{k}}^{P-\ord}(U^p\Iw^P(p^{b,c}),M)\coloneqq
	e_PS_{\wt{k}}(U^p\Iw^P(p^{b,c}),M)
\]
and $S_{\wt{k}}^{P-\ord}(U^p\Iw^P(p^{b,c}),E)\coloneqq 
S_{\wt{k}}^{P-\ord}(U^p\Iw^P(p^{b,c}),\oo)\otimes_{\oo}E$.
Alternatively,
$S_{\wt{k}}^{P-\ord}(U^p\Iw^P(p^{b,c}),M)$
can be defined as the subspace on which the action of
any  $U_{\wt{k},w}^{(j)}$ such that 
$\alpha_w^{(j)}\in Z_Q$ is invertible.
Note that when $P=B$ the definition coincides with 
that of ordinary forms in \cite[Def 2.13]{ger}.

\begin{defn}\label{def:ord_hecke}
	We let $\TT^P_{\wt{k}}(U^p\Iw^P(p^{b,c}),M)$
	be the $\oo$-subalgebra in 
	$\End_{\oo}S_{\wt{k}}(U^p\Iw^P(p^{b,c}),M)$
	generated by all
	$T_w^{(j)}$, for $1\leq j\leq n$,
	and $(T_w^{(n)})^{-1}$ at $w\mid v\notin R$;
	and all $U_{\wt{k},w}^{(j)}$ 
	for which $\alpha_w^{(j)}$  belongs to  $Z_Q$
	and all $\langle u\rangle$.
	We also define 
	$\TT^P_{\wt{k}}(U^p\Iw^P(p^{b,c}),E)\cong
	\TT^P_{\wt{k}}(U^p\Iw^P(p^{b,c}),\oo)\otimes_{\oo}E$,
	which acts faithfully on 
	$S_{\wt{k}}^{P-\ord}(U^p\Iw^P(p^{b,c}),E)$.
\end{defn}

\begin{lem}\label{lem:control}
	For $M$ as in the definition
	the inclusions below are isomorphisms.
	\begin{align*}
	&S_{\wt{k}}^{P-\ord}(U^p\Iw^P(p^{b,b}),M)\hookrightarrow	
	S_{\wt{k}}^{P-\ord}(U^p\Iw^P(p^{b,c}),M)\quad 
	\text{ for } c\geq b\geq 1\\
	&S_{\wt{k}}^{P-\ord}(U^p\Iw^P(p^{0,1}),M)\hookrightarrow	
	S_{\wt{k}}^{P-\ord}(U^p\Iw^P(p^{0,c}),M)\quad \text{ for } c\geq 1
	\end{align*}
\end{lem}
\begin{proof}
	It suffices to show that 
	$(U_P)^{n!}S_{\wt{k}}(U^p\Iw^P(p^{b,c}),M)
	\subset S_{\wt{k}}(U^p\Iw^P(p^{b,b}),M)$
	for $n$ sufficiently large. 
	Since $\Iw^P(p^{b,c})$ admits Iwahori decompositions,
	this follows from \cite[Lem 3.3.2]{emeI}.
	The same argument also applies to 
	$S_{\wt{k}}(U^p\Iw^P(p^{0,c}),M)$.
	See also \cite[Lem 2.19]{ger} for the proof when $P=B$.
\end{proof}

\begin{lem}\label{lem:PtoB}
	For any $b\geq 1$
	the inclusion 
	$S_{\wt{k}}^{B-\ord}(U^p\Iw(p^{b,b}),M)\subset
	S_{\wt{k}}^{P-\ord}(U^p\Iw^P(p^{b,b}),M)$
	is equivariant
	for Hecke operators in 
	Definition \ref{def:ord_hecke}.
	This induces a homomorphism of $\oo$-algebras
	\[
		\TT^P_{\wt{k}}(U^p\Iw^P(p^{b,b}),M)\to
		\TT^B_{\wt{k}}(U^p\Iw(p^{b,b}),M)
	\]
\end{lem}
\begin{proof}
	Since $\Iw^P(p^{b,b})\subset \Iw(p^{b,b})$,
	it suffices to show that the Hecke operators 
	are equivariant with respect to the natural inclusions
	$S_{\wt{k}}(U^p\Iw(p^{b,b}),M)\subset 
	S_{\wt{k}}(U^p\Iw^P(p^{b,b}),M)$.
	This is clear for $T_w^{(j)}$ and $\langle u\rangle$.
	And for  $U_{\wt{k},w}^{(j)}$ such that 
	$\alpha=\alpha_w^{(j)}\in Z_Q$, this follows from that 
	the following set of representatives for 
	$N_0/\alpha N_0\alpha^{-1}$
	\[
	\begin{pmatrix}
		\id_j&X\\&\id_{n-j}
	\end{pmatrix},\quad
	X \text{ runs through a set of representatives of }
	M_{j,n-j}(\oo_w/\varpi_w)
	\]
	as given in \cite[Lem 2.10]{ger}, is also
	a set of representatives for 
	$U_0/\alpha U_0\alpha^{-1}$.
\end{proof}

\subsection{Weights independence}

Let $P\subset G_p$ be a parabolic subgroup 
as in last subsection.
When $\wt{k}=(k_\sigma)\in (\Z^n)^{\Sigma}$ is dominant,
let $\pi_{k_{\sigma}}$
denote the algebraic $Q_w$-representation
$\Ind_{B_w\cap Q_w}^{Q_w}(\omega_0 k_\sigma)$
when $\sigma\in I_w$
and let  $\pi_{\wt{k}}$
be the $Q$-representation
as defined by \eqref{def:algrep},
which we extend to $P$ via the projection  $P=QU\to Q$.
The following proposition
is a generalization of \cite[Prop 2.22]{ger}
to $P$-ordinary forms.

\begin{lem}
	Let $\pi_{\wt{k}}^*$ be the contragredient
	representation and
	$\varpi$ be a uniformizer of $\oo$,
	For $b\geq 1$ let $A=\varpi^{-b}\oo/\oo$,
	then there exists an isomorphism
	\[
		\epsilon_{\wt{k}} \colon 
		S_{\wt{k}}^{P-\ord}(U^p\Iw^P(p^{b,b}),A)\cong 
		\Hom_{\oo}(\pi_{\wt{k}}^*(\oo),
		S^{P-\ord}(U^p\Iw^P(p^{b,b}),A)).
	\]
	Moreover, the isomorphism is equivariant 
	with respect to all the Hecke operators
	in Definition \ref{def:ord_hecke}
	and the following action of $u\in Q_0$
	\begin{align*}
	&u\cdot F(g)=\xi_{\wt{k}}(u)\cdot F(gu),\quad
	F(g)\in S_{\wt{k}}^{P-\ord}(U^p\Iw^P(p^{b,b}),A)\\
	&u\cdot \phi(v^*)(g)=
	\phi(\pi^*_{\wt{k}}(u^{-1})\cdot v^*)(gu),\quad
	\phi\in \Hom_{\oo}(\pi_{\wt{k}}^*(\oo),
	S^{P-\ord}(U^p\Iw^P(p^{b,b}),A))
	\end{align*}
\end{lem}

\begin{proof}
	By inductions in steps
	we can fix an isomorphism 
	$\xi_{\wt{k}}\cong \Ind_{P}^{G_p}\pi_{\wt{k}}$.
	Let $ev\colon \xi_{\wt{k}}\to \pi_{\wt{k}}$
	be the evaluation at the identity.
	For $F(g)\in S_{\wt{k}}(U^p\Iw^P(p^{b,b}),A)$,
	we define 
	$\epsilon_{\wt{k}}(F)$ as 
	\begin{equation}\label{eq:wt_indep}
	\epsilon_{\wt{k}}(F)\colon 
	\pi^*_{\wt{k}}(\oo)\rightarrow
	S(U^p\Iw^P(p^{b,b}),A)\quad
	v^*\mapsto [g\mapsto v^*(ev(F(g)))].
	\end{equation}
	By the assumption on $b$,
	the action of $\Iw^P(p^{b,b})$ on 
	$A\otimes_{\oo}\pi_{\wt{k}}(\oo)$
	is trivial.
	Thus the function defined above is indeed 
	a modular form
	of trivial weight.
	It is also straightforward to verify
	that the map is equivariant with respect
	to the Hecke operators and the $Q_0$-action.


	To construct the reversed map,
	note that if $\mu$ is a weight character of $T$ in  
	$\pi_{\wt{k}}$, then it is also a weight character 
	of $T$ in $\xi_{\wt{k}}$.
	We fix weight vectors $v_\mu\in \xi_{\wt{k}}$
	and $v^*_\mu\in \pi_{\wt{k}}^*$
	such that $v^*_{\mu}(ev(v_\mu))=1$.
	Now, let $\alpha_P\in Z_Q^+$ be the product
	of all $\alpha_w^{(j)}\in Z_Q$, $\alpha=\alpha_P^r$,
	and $\{x_i\}_{i\in I}$
	be a set of represntatives 
	for $U_0/\alpha U_0\alpha^{-1}$,
	we put 
	\begin{align*}
		\varphi\colon 
		\Hom_{\oo}(\pi_{\wt{k}}^*(\oo),&
		S(U^p\Iw^P(p^{b,b}),A))\longrightarrow
		S_{\wt{k}}(U^p\Iw^P(p^{b,b}),A)\\
		\phi&\mapsto 
		F_\phi(g)=\sum_{i\in I} \sum_{\mu}
		\xi_{\wt{k}}(x_i)\cdot 
		\phi(v^*_\mu)(gx_i\alpha)v_\mu
	\end{align*}
	where $\mu$ runs through the weight characters in 
	$\pi_{\wt{k}}$.
	To show that the resulting function 
	defines a modular form,
	let $u\in \Iw^P(p^{b,b})$, 
	then as explained in \cite[Prop 2.22]{ger}
	there exists a bijection $i\mapsto i'$ of $I$
	such that 
	 \[
		ux_i=x_{i'}v_i,\quad
		v_i\in\alpha\Iw^P(p^{b,b})\alpha^{-1} 
		\cap \Iw^P(p^{b,b})
	\]
	Since each $v_i$ is reduced to the identity matrix 
	modulo $\varpi^r$ and thus acts trivially on 
	$\xi_{\wt{k}}(A)$,
	\[
		\xi_{\wt{k}}(u)\cdot F_\phi(gu)=
		\sum_{i\in I}\sum_{\mu}
		\xi_{\wt{k}}(x_i'v_i)\cdot 
		\phi(v^*_\mu)(gx_i'v_i\alpha)v_\mu=
		\sum_{i\in I}\sum_{\mu}
		\xi_{\wt{k}}(x_i')\cdot 
		\phi(v^*_\mu)(gx_i'\alpha)v_\mu=F_\phi(g)
	\]
	and indeed $F_\varphi(g)\in 
	S_{\wt{k}}(U^p\Iw^P(p^{b,b}),A)$.

	At last, we observe that for each $\mu$ 
	the composition
	$\epsilon_{\wt{k}}(F_\phi)$ is the homomorphism
	\[
		v_\mu^*\mapsto \sum_{i\in I}\phi(v_\mu^*)
		(gx_i\alpha) =U_P^r\phi(v_\mu^*)(g)
	\]
	On the other hand 
	if we decompose $F$ with respect to a choice of 
	weight vectors
	$F(g)=\sum_\mu F_\mu(g)v_\mu+
	\sum_{\mu'}F_{\mu'}(g)v_{\mu'}$, 
	with $\mu$ goes through weight vectors 
	that also appears in $\pi_{\wt{k}}$
	and $\mu'$ goes through the complement,
	then we have
	$\mu(\alpha)=(w_0\wt{k})(\alpha)$ for all $\mu$
	and  $\varpi^r(w_0\wt{k})(\alpha)\mid \mu'(\alpha)$
	for all $\mu'$.
	Therefore
	\begin{multline*}
	U_P^rF(g)=
	\sum_{i\in I}
	\sum_\mu \xi_{\wt{k}}(x_i)\cdot F_\mu(gx_i\alpha)v_\mu+
	\sum_{i\in I}
	\sum_{\mu'}\frac{\mu'(\alpha)}{(w_0\wt{k})(\alpha)}
	\xi_{\wt{k}}(x_i)\cdot F_{\mu'}(gx_i\alpha)v_{\mu'}\\=
	\sum_{i\in I}
	\sum_\mu \xi_{\wt{k}}(x_i)\cdot F_\mu(gx_i\alpha)v_\mu=
	\sum_{i\in I}
	\sum_\mu \xi_{\wt{k}}(x_i)\cdot
	\epsilon_{\wt{k}}(F)(v^*_\mu)(gx_i\alpha)
	=F_{\epsilon_{\wt{k}}(F)}(g).
	\end{multline*}

	We thus have the following commutative diagram,
	from which the proposition follows.
	\[
	\begin{tikzcd}
		S_{\wt{k}}(U^p\Iw^P(p^{b,b}),A)
		\arrow[r,"\epsilon_{\wt{k}}"]
		\arrow[d,"U_P^r"]
		& \Hom_\oo(\pi^*_{\wt{k}}(\oo), S(U^p\Iw^P(p^{b,b}),A))
		\arrow[d,"U_P^r"]
		\arrow[dl,"\varphi"]\\
		S_{\wt{k}}(U^p\Iw^P(p^{b,b}),A)
		\arrow[r,"\epsilon_{\wt{k}}"]
		& \Hom_\oo(\pi^*_{\wt{k}}(\oo), S(U^p\Iw^P(p^{b,b}),A))
	\end{tikzcd}	
	\]
\end{proof}


Let $S_{\wt{k}}^{P-\ord}(U^p,E/\oo)=
\varinjlim_{b}
S_{\wt{k}}^{P-\ord}(U^p\Iw^P(p^{b,b}),E/\oo)$
be the injective limit under inclusions
and define 
\[
	\TT^P_{\wt{k}}(U^p,E/\oo)=
	\varprojlim_{b}
	\TT^P_{\wt{k}}(U^p\Iw^P(p^{b,b}),E/\oo)
\]
Since $S_{\wt{k}}^{P-\ord}(U^p,E/\oo)$
is also the injective limit of 
$S_{\wt{k}}^{P-\ord}(U^p\Iw^P(p^{b,b}),\varpi^{-b}\oo/\oo)$
and $\pi_{\wt{k}}^*(\oo)$ is finite over $\oo$,
the following proposition
follows immediately from the previous lemma.

\begin{prop}\label{prop:wt_indep}
	There exists the following isomorphism
	which is equivariant with respect to the 
	Hecke operators and the $Q_0$-action 
	defined in previous lemma.
	\[
		\epsilon_{\wt{k}} \colon 
		S_{\wt{k}}^{P-\ord}(U^p,E/\oo)\cong 
		\Hom_{\oo}(\pi_{\wt{k}}^*(\oo),
		S^{P-\ord}(U^p,E/\oo)).
	\]
	In particular, this isomorphism 
	induces the following surjective homomorphism
	between the Hecke algebras
	\[
		\varphi_{\wt{k}}\colon 
		\TT^P(U^p,E/\oo)\twoheadrightarrow
		\TT^P_{\wt{k}}(U^p,E/\oo).
	\]
\end{prop}


\subsection{Completed homology and cohomology}

Recall that when $\wt{k}$ is the trivial weight,
the action of $K_p$
on  $S(M)$ as in Definition \ref{def:algrep} 
for any  $\oo$-module $M$
is simply the right translation,
which extends to $G_p$
and coincides with that of $N_0T^+$ in \eqref{def:T_act}.
In particular,
when $U^p$ satisfies \eqref{cond:small}
and $\{U_p\}$ is the filtered system of 
all the compact open subgroups in $K_p$
and for $A=E/\oo$ or  $\oo/\varpi^{r}$, we have
\begin{equation}\label{eq:complete}
	S(U^p,A)\coloneqq
	\varinjlim_{U_p}S(U^pU_p,A)\in 
	\Mod^{\adm}_{G_p}(\oo).
\end{equation}
Moreover, Let $P=QU$ be a parabolic subgroup 
as in the previous subsection.
Since 
\[
	S(U^p,E/\oo)^{U_0}=
	\varinjlim_{b}
	S(U^p\Iw^P(p^{b,b}),\varpi^{-b}\oo/\oo)
\]
is the injective limits of finite $\oo$-modules,
It follows from \cite[Lem 3.1.5]{emeI} and \cite[Prop 3.2.4]{emeI}
that $S^{P-\ord}(U^p,E/\oo)\cong \Ord_P(S(U^p,E/\oo))$,
where the latter is an object in $\aMod_Q(\oo)$
by \cite[Thm 3.3.3]{emeI}.
We define the $P$-ordinary completed homology and cohomology by
\begin{align}
	M(U^p)&=
	\Ord_P(S(U^p,E/\oo))^\vee
	\coloneqq \Hom_\oo(\Ord_P(S(U^p,E/\oo)),E/\oo)\\
	S(U^p)&=\Hom_\oo(E/\oo, \Ord_P(S(U^p,E/\oo)))
	\cong \varprojlim_r \Ord_P(S(U^p,\oo/\varpi^{r}))
\end{align}
Let $\Hom_\oo^{\cts}(M(U^p),\oo)$
be the set of
$\Phi\in \Hom_\oo(M(U^p),\oo)$ 
such that for any positive integer $r$,
there exists $b$ sufficiently large so that 
the reduction of $\Phi$ modulo $\varpi^r$
factors through
the Pontryagin dual of 
$S^{P-\ord}(U^p\Iw^P(p^{b,b}),E/\oo)\subset \Ord_P(S(U^p,E/\oo)$. 
It can be verified that 
\[
	M(U^p)\cong \Hom_\oo(S(U^p),\oo),\qquad
	S(U^p)\cong \Hom_\oo^{\cts}(M(U^p),\oo).
\]
From the above isomorhisms we see that
$\TT^P(U^p,E/\oo)$ acts faithfully
on  $M(U^p)$, and  $S(U^p)$.
In fact, 
let $S_{\wt{k}}^{P-\ord}(U^p,\oo)
=\varinjlim_{b}S_{\wt{k}}^{P-\ord}(U^p\Iw^P(p^{b,b}),\oo)$
and 
$\TT^P_{\wt{k}}(U^p,\oo)=\varprojlim_bS_{\wt{k}}^{P-\ord}(U^p\Iw^P(p^{b,b}),\oo)$.
Then there exists an isomorphism of $\oo$-algebras 
$\TT^P(U^p,\oo)\cong \TT^P(U^p,E/\oo)$
as in \cite[Lem 2.17]{ger},
which also coincides with the fact that
$S^{P-\ord}(U^p,\oo)$ is dense in $S(U^p)$.

\begin{defn}\label{def:big_hecke}
	From now on, let $\TT^P(U^p,\oo)$
	be the big $P$-ordinary Hecke algebra 
	acting faithfully on each of 
	$S^{P-\ord}(U^p,E/\oo), \Ord_P(S(U^p,E/\oo)), M(U^p)$
	and $S(U^p)$. 
	We still denote 
	\[
		\varphi_{\wt{k}}\colon \TT^P(U^p,\oo)\twoheadrightarrow
		\TT^P_{\wt{k}}(U^p,\oo)
	\]
	for the surjective homomorphism of $\oo$-algebras
	induced by Proposition \ref{prop:wt_indep}.
	Moreover, 
	let $S(U^p)_E\coloneqq S(U^p)\otimes_{\oo}E$
	be the $E$-Banach space $G_p$-representation with 
	the unit ball $S(U^p)$. 
	We also define $\TT^P(U^p,E)=\TT^P(U^p,\oo)\otimes_{\oo}E$,
	which acts faithfully on $S(U^p)_E$.
\end{defn}


\begin{lem}\label{lem:inj}
	The restriction of
	$\Ord_P(S(U^p,E/\oo))$ to $Q_0$ 
	is an injective object
	in $\Mod^{\sm}_{Q_0}(\oo)$.
\end{lem}
\begin{proof}
	Following the strategy of the proof of 
	\cite[Prop 3.2.4]{pan}, 
	it suffices to show the surjectivity of
	\[
		\Hom_{\oo[Q_0]}(\pi,\Ord_P(S(U^p,E/\oo)))\to 
		\Hom_{\oo[Q_0]}(\pi_1,\Ord_P(S(U^p,E/\oo)))
	\]
	when $\pi_{1}\hookrightarrow \pi$ 
	is an injective morphism between admissible $Q_0$
	representations that are finite $\oo$-modules.
	We first note that since $\pi_1$ is admissible
	and $\oo$-finite,
	any homomorphism on the right factors 
	through 
	\begin{multline*}
		\Hom_{\oo[Q_0]}(\pi_1,
		\Hom_{\oo[Z_Q^+]}
		(\oo[Z_Q], S(U^p\Iw^P(p^{b,b}),
		\varpi^{-r}\oo/\oo)))\\=
		\Hom_{\oo[Z_Q^+]}(\oo[Z_Q],
		\Hom_{\oo[Q_0]}(\pi_1, 
		S(U^p\Iw^P(p^{b,b}),\varpi^{-r}\oo/\oo)))
	\end{multline*}
	for some $b$ and  $r$ sufficiently large,
	on which 
	the $Z_Q$-fintieness condition is automatic
	by \cite[Lem 3.1.5]{emeI}.

	Now, enlarging $b$ if necessary,
	we may assume that 
	the $Q_0$-actions on 
	$\pi$ and $\pi_1$ is trivial on
	$Q_0\cap \Iw(p^{b,b})$. 
	Let $U_0$ acts trivially,
	we may then extend $\pi$ and  $\pi_1$
	to of $\Iw(p^{0,b})$.
	Let $\pi^\vee$ be the Pontryagin dual.
	Define
	\[
		S_{\pi^\vee}(U^p\Iw(p^{0,b}))=
		\{
			F\colon G(\F)\backslash G(\A_f)\to 
			\pi^\vee\mid 
			F(gu)=\pi^\vee(u_p)\cdot F(g),\,
			u\in \Iw(p^{0,b})
		\}
	\]
	then there exists an isomorphism
	\[
		\epsilon\colon 
		S_{\pi^\vee}(U^p\Iw(p^{0,b}))\cong 
		\Hom_{\oo[Q_0]}(\pi,
		S(U^p\Iw^P(p^{b,b}),\varpi^{-r}\oo/\oo)))
		\quad \epsilon(F)\colon
		v\mapsto [g\mapsto v(F(g))]
	\]
	and similarly for $\pi_1$.
	But the smallness assumption implies
	that $S_{\pi^\vee}(U^p\Iw(p^{0,b}))$
	are $S_{\pi_1^\vee}(U^p\Iw(p^{0,b}))$
	are direct sums of 
	$\pi^\vee$ and  $\pi_1^\vee$ 
	on the same indexing set
	$G(\F)\backslash G(\A_f)/U^p\Iw(p^{0,b})$.
	Therefore 
	$S_{\pi^\vee}(U^p\Iw(p^{0,b}))\to 
	S_{\pi_1^\vee}(U^p\Iw(p^{0,b}))$ is 
	surjective
	as $\pi^\vee\to \pi_1^\vee$ is surjective.
	Apply localization to the $P$-ordinary parts
	and apply \cite[Lem 3.1.5]{emeI} again,
	we see that
	\[
		\Hom_{\oo[Z_Q^+]}(\oo[Z_Q],
		S_{\pi^\vee}(U^p\Iw(p^{0,b})))\to 
		\Hom_{\oo[Z_Q^+]}(\oo[Z_Q],
		S_{\pi^\vee_1}(U^p\Iw(p^{0,b})))
	\]
	is also surjective, from which 
	the proposition follows.
\end{proof}


\begin{prop}\label{prop:density}
The subspace $S^{\alg}(U^p)_E$ 
of $Q_0$-algebraic vectors, defined as
\[
\Image\left(\bigoplus_{\wt{k}}\Hom_{E[Q_0]}(\pi_{\wt{k}}^*(\oo), S(U^p)_E)
\otimes_E \pi_{\wt{k}}^*(E)\rightarrow S(U^p)_E\right)
\]
where $\wt{k}$ ranges through all dominant weights,
is dense in the $E$-Banach space $S(U^p)_E$.
\end{prop}
\begin{proof}
	Since $\Ord_P(S(U^p,E/\oo))$ is an injective object
	in $\Mod_{Q_0}^{\sm}(\oo)$
	by Lemma \ref{lem:inj},
	we may follow the strategy of 
	\cite[Prop 3.2.9]{pan}
	and use \cite[Cor 3.2.6]{pan}
	to reduce the statement to that of
	$\mathcal{C}(Q_0,E)$,
	the space of continuous  $E$-valued
	functions on $Q_0$.
	The density result then follows from
	\cite[Prop 6.A.17]{Pask14}.
\end{proof}


\begin{prop}\label{prop:wt_space}
	There exists a Hecke-equivariant isomorphism
	\[
	S_{\wt{k}}^{P-\ord}(U^p\Iw^P(p^{0,1}),E)\cong 
	\Hom_{\oo[Q_0]}(\pi_{\wt{k}}^*(\oo), S(U^p)_E)
	\]
\end{prop}
\begin{proof}
	By Proposition \ref{prop:wt_indep},
	there exists an isomorphism
	\[
		\Hom_\oo(E/\oo, S_{\wt{k}}^{P-\ord}(U^p,E/\oo))\cong 
		\Hom_\oo(E/\oo,
		\Hom_{\oo}(\pi_{\wt{k}}^*(\oo),
		S^{P-\ord}(U^p,E/\oo)))=
		\Hom_{\oo}(\pi_{\wt{k}}^*(\oo), S(U^p))
	\]
	that is equivariant with respect to 
	the Hecke operators and 
	the $Q_0$-actions defined.
	Taking the subspaces of $Q_0$-invariant subspaces.
	This gives
	$\Hom_{\oo[Q_0]}(\pi_{\wt{k}}^*(\oo), S(U^p))$
	on the right hand side.

	On the other hand, since
	$\Hom_\oo(E/\oo, S_{\wt{k}}^{P-\ord}(U^p,E/\oo))\cong
	\varprojlim_r S_{\wt{k}}^{P-\ord}(U^p,\oo/(\varpi^r))=
	S_{\wt{k}}^{P-\ord}(U^p,\oo)$,
	the $Q_0$-invariant subspace 
	on the left hand side
	is $S_{\wt{k}}^{P-\ord}(U^p\Iw^P(p^{0,1}),\oo)$
	by Lemma \ref{lem:control}.
	The claimed result now follows by
	tensoring both subspaces with $E$.
\end{proof}

\begin{rem}
	The above results generalize
	\cite[Prop 3.2.9]{pan} and 
	\cite[\S 3.2.10]{pan},
	which deals with the case 
	when $n=2$ and $P=G_p$,
	in which the $P$-ordinary condition is empty.
\end{rem}


\subsection{Hecke algebras and Galois representations}


Recall taht for a dominant $\wt{k}\in (\Z^n)^{\Sigma}$,
the space $S_{\wt{k}}(\bar{\Q}_p)$
admits the $G(\A_f)$-action defined in 
Definition \ref{def:algform}.
Let $\mathcal{A}$ be
the space of automorphic forms on $G(\A)$, 
and $\xi_{\wt{k}}^*(\C)$ be the
$G(\A_\infty)$-representation over $\C$
defined by the inclusions
$G(\F_\sigma)\subset \GL_n(\F_\sigma\otimes \K)=\GL_n(\C)$
for each $\sigma\in \Sigma$.
By \cite[Prop 3.3.2]{CHT},
there exists an $G(\A_f)$-equivariant isomorphism
\[
	\iota\colon S_{\wt{k}}(\bar{\Q}_p)\otimes_{\iota,\bar{Q}_p}\C
	\rightarrow \Hom_{G(\A_\infty)} (\xi_{\wt{k}}^*(\C), \mathcal{A})\quad
	\iota(F)\colon v^*\mapsto 
	[g\mapsto \xi_{\wt{k}}(g_\infty)\xi_{\wt{k}}(g_p)\cdot F(g_f)].
\]


\begin{prop}\cite[Prop.2.27]{ger}
	Let $\pi$ be an irreducible constituent of the
	$G(\A_f)$-representation $S_{\wt{k}}(\bar{\Q}_p)$,
	then there exist a unique 
	continuous semisimple representation
	\[
	r_\pi: \Gal_\K \rightarrow \GL_n(\bar{\Q}_p)\quad
	\text{ satisfying }
	r_\pi^c \cong r_\pi^{\vee} \epsilon^{1-n}
	\]
	where $\epsilon$ is the $p$-th cyclotomic character,
	with the following properties.
\begin{enumerate}[label=(\alph*)]
\item Let $v=w\bw$ be a prime-to-$p$ place that is split in $\K$
and $\pi_w$ be the $\GL_n(\K_w)$-representation
induced by $\iota_w\colon G(\F_v)\cong \GL_n(\K_w)$, then
\[
\WD\left(\left.r_\pi\right|_{D_w}\right)^{\mathrm{ss}} \cong
\Rec(\pi_w|\cdot|^{\frac{1-n}{2}})^{\mathrm{ss}}.
\]
Moreover, $r_\pi$ is unramified at $w$ if $\pi_v$ is unramified.
\item Let $v=w\bw$ with $v\in S_p$ and define $\pi_w$ as above.
The representation $r_\pi$ is potentially semistable at $w$ and  $\bw$.
Moreover $r_\pi$ is crystalline at $w$ 
if $\pi_v$ is unramified,
in which case 
the characteristic polynomial of the geometric Frobenius $\Fr_w$
on $\WD\left(D_{\mathrm{cris }}\left(\left.r_\pi\right|_{D_w}\right)\right)$
coincides with that of $\Rec(\pi_w|\cdot|^{\frac{1-n}{2}})^{\mathrm{ss}}$.
\item 
Let $k_{\sigma,j}=-k_{\sigma c, n-j+1}$
for $\sigma\notin \Sigma$.
If $w\mid p$ and  $\sigma\in I_w$, then 
$\dim_{\bar{\Q}_p}\operatorname{gr}^i
\left(r_\pi \otimes_{\sigma, \K_w} B_{\dR}\right)^{D_w}=1$
exactly when $i=k_{\sigma, j}+n-j$ 
for $j=1, \ldots, n$ and is equal to 0 otherwise.
\end{enumerate}
\end{prop}

From now on,
we restrict ourselves to the following situation
\begin{equation}\label{cond:parabolic}\tag{P}
	n=2,\, 
	P_w=G_w \text{ for a fixed }w\in \Sigma,\,
	P_{w'}=B_{w'} \text{ for } w'\neq w.
\end{equation}
By \cite[Lem 2.14]{ger}, 
the Hecke algebras
$\TT^P_{\wt{k}}(U^p\Iw^P(p^{b,b}),\oo)$
are finite flat reduced $\oo$-algebras.
\begin{defn}\label{def:rep_prime}
	Let $\fp\subset \TT^P_{\wt{k}}(U^p\Iw^P(p^{b,b}),\oo)$
	be a height-one prime ideal.
	By abuse of notation, 
	we also let $\fp$ denote the induced minimal prime
	in $\TT^P_{\wt{k}}(U^p\Iw^P(p^{b,b}),E)$.
	The quotient $\TT^P_{\wt{k}}(U^p\Iw^P(p^{b,b}),E)/\fp$
	is isomorphic to a finite extension $E_{\fp}$ of $E$.
	Let 
	$\lambda_\pi\colon \TT^P_{\wt{k}}(U^p\Iw^P(p^{b,b}),E)\to E_\pi$
	denote the associated homomorphism.

	Let $S_{\wt{k}}^{P-\ord}(U^p\Iw^P(p^{b,b}),E)_{\fp}$
	be the localization at $\fp$.
	We say an irreducible component $\pi$ as above
	belongs to  $\fp$ if 
	$\pi\cap S_{\wt{k}}^{P-\ord}(U^p\Iw^P(p^{b,b}),E)_{\fp}\neq 0$.
	The proposition then implies that
	$r_\pi$ satisfies
	\begin{equation}\label{eq:Gal_hecke_away_p}
		\mtr(r_\pi(\Fr_w))=\lambda_\fp(T_w^{(1)}),\quad
		\det(r_\pi(\Fr_w))=q_w\lambda_\fp(T_w^{(2)}),\,
	\end{equation}
	for $v=w\bw$ that is split in  $\K$ and $v\notin R$. 
	By Chebotarev's density theorem,
	this implies that $r_\pi$ is defined over  $E_{\fp}$
	and independent of the choice of $\pi$.
	Thus we also write $r_\fp=r_\pi$.
\end{defn}




Following \cite{ger},
we say a dominant weight $\wt{k}$ is sufficiently regular
(for $w'\neq w$)
if for each $w'\in \Sigma_p$  and $w'\neq w$,
there exists  $\sigma\in I_{w'}$
such that  $k_{\sigma,1}>k_{\sigma,2}$.

\begin{lem}\label{lem:galois_at_p}
	Let $r_{\fp}$ be the Galois representation
	associated to a minimal prime
	$\fp\subset \TT^P_{\wt{k}}(U^p\Iw^P(p^{0,1}),E)$.
	\begin{enumerate}[label=(\alph*)]
	\item The representation $r_\fp$ is crystalline at $w'\neq w$
	if $\wt{k}$ is sufficiently regular.
	Moreover, $r_\fp\vert_{D_{w'}}$ 
	fits into the exact sequence
	$0\to \psi_1\to r_{\fp}\vert_{D_{w'}} \to \epsilon^{-1}\psi_2\to 0$
	for characters $\psi_i\colon D_{w'}\to E_{\fp}^{\times}$
	\begin{equation}\label{eq:Gal_hecke_at_p}
	\begin{aligned}
		\psi_1\circ \Art_{w'}(\varpi_{w'})&=
		\lambda_{\fp}(U_{\wt{k},w'}^{(1)}) &
		\psi_1\circ \Art_{w'}(x)&=
		\lambda_{\fp}
		(\langle 
		\iota_{w'}^{-1}
		(\begin{smallmatrix}
			x&\\&1
		\end{smallmatrix})
		\rangle)\, \text{ for }x\in \oo_{w'}^{\times}\\
		\psi_2\circ \Art_{w'}(\varpi_{w'})&=
		\lambda_{\fp}(U_{\wt{k},w'}^{(2)})/
		\lambda_{\fp}(U_{\wt{k},w'}^{(1)}) &
		\psi_1\circ \Art_{w'}(x)&=
		\lambda_{\fp}
		(\langle 
		\iota_{w'}^{-1}
		(\begin{smallmatrix}
			1&\\&x
		\end{smallmatrix})
		\rangle)\, \text{ for }x\in \oo_{w'}^{\times}
	\end{aligned}
	\end{equation}
	\item The representation $r_\fp$ is 
	crystalline at $w$, with 
	\[
	\det r_\fp\circ \Art_w(\varpi_w)=
	\lambda_{\fp}(U_{\wt{k},w}^{(2)}),\quad
	\det r_\fp\circ \Art_w(x)=
	\lambda_{\fp}
	(\langle 
	\iota_{w}^{-1}
	(\begin{smallmatrix}
		x&\\&x
	\end{smallmatrix})
	\rangle)\, \text{ for }x\in \oo_{w}^{\times}
	\]
	If furthermore $\fp$ extends
	to a prime ideal in the $B$-ordinary Hecke algebra
	$\TT^P(U^p\Iw(p^{0,1}),E)$,
	then $r_\pi\vert_{D_w}$ 
	admits an exact sequence as above.
	\end{enumerate}
\end{lem}
\begin{proof}
This is a restatement of \cite[Cor 2.33]{ger}
for $w'\neq w$.
For the fixed place  $w\in \Sigma$,
note that $\pi_w$ is unramified 
by definition
for any $\pi$ belonging to $\fp$.
We then combine the corollary with \cite[Lem 2.31]{ger} to obtain 
the result.
\end{proof}


Since the Hecke algebra 
$\TT^P_{\wt{k}}(U^p\Iw^P(p^{b,b}),\oo)$
is reduced the homomorphisms
\[
	\prod_{\fp}\lambda_{\fp}\colon 
	\TT^P_{\wt{k}}(U^p\Iw^P(p^{b,b}),\oo)\to 
	\TT^P_{\wt{k}}(U^p\Iw^P(p^{b,b}),E)\to  
	\prod_{\fp}E_{\fp}
\]
where $\fp$ goes through all height-one primes,
is injective.
In particular $\TT^P_{\wt{k}}(U^p\Iw^P(p^{b,b}),\oo)$
can be identified with 
a closed subring of $\prod_{\fp}E_{\fp}$.
By the relation \eqref{eq:Gal_hecke_away_p}
and Chebotarev's density theorem,
the pseudo-representations  $r_{\fp}$
lift to a pseudo-representation
\begin{equation}\label{eq:pseudo_rep_finite}
	\Psi_{\wt{k}}\colon \Gal_{\K}
	\to \TT^P_{\wt{k}}(U^p\Iw^P(p^{b,b}),\oo)\quad
	\Psi_{\wt{k}}(\Fr_w)=T_w^{(1)}
\end{equation}
that are compatible among different levels of $\Iw^P(p^{b,b})$.
By abuse of notation,
we let 
$\Psi_{\wt{k}}\colon \Gal_{\K} \to \TT^P_{\wt{k}}(U^p,\oo)$
denote the big Galois pseudo-representation 
obtained through the inverse limit.

\begin{defn}\label{def:big_Gal}
Let $\Psi\colon \Gal_{\K}\to \TT^P(U^p,\oo)$	
denote the big Galois pseudo-representation
when $\wt{k}$ is the trivial weight.
Note that by the relation \eqref{eq:Gal_hecke_away_p}
and Chebotarev's density theorem again,
the composition of $\Psi$
with the surjection
$\varphi_{\wt{k}}\colon 
\TT^P(U^p,\oo)\to \TT^P_{\wt{k}}(U^p,\oo)$
is precisely  $\Psi_{\wt{k}}$.
\end{defn}


\section{Results on p-adic local Langlands}

We continue with the assumptions
in the previous section.
So $U^p\subset G(\A_f^p)$ is an
open compact subgroup satisfying \eqref{cond:small},
$P\subset G_p$ is a parabolic subgroup
as \eqref{cond:parabolic},
$\TT^P(U^p,\oo)$
is the big  $P$-ordinary Hecke algebra
acting on the completed homology 
and cohomology as defined in Definition \ref{def:big_hecke}.
We further assume that 
the fixed place $w\in \Sigma_p$
in \eqref{cond:parabolic}
is such that  $\K_w\cong \Qp$.
So that the decomposition group
$D_w$ is identified with
$\Gp$, the absolute Galois group of  $\Qp$.



Fix a maximal ideal $\fm\subset \TT^P(U^p,\oo)$
and let $\fF$ denote the residue field of the localization
$\TT^P(U^p,\oo)_{\fm}$.
We note that the same argument
as \cite[Prop 3.3.6]{pan} shows that
$\TT^P(U^p,\oo)\to \TT^P(U^p\Iw^P(p^{0,1}),\oo/\varpi)$ 
induces a bijection of maximal ideals
and thus $\fF$ is a finite field.
Enlarging the finite extension  $E$ over  $\Qp$
if necessary,
we may assume that  $\fF$ is also 
the residue field of the ring of integers  $\oo$ of  $E$.


Let  $\Psi_{\fm}$ denote 
the localization of the big Galois pseudo-representation.
We say $\Psi_{\fm}$ is 
residually reducible and locally generic at $w$ if
there exists characters
$\bar{\delta}_1, \bar{\delta}_2\colon \Gal_{\K}\to \fF$
such that 
\begin{equation}\tag{red.gen}\label{cond:red_gen}
	\Psi_\fm\equiv \bar{\delta}_1+\bar{\delta}_2
	\mod \fm,\quad
	\bar{\delta}_1\bar{\delta}_2^{-1} \vert_{D_w}
	\neq \id,\omega^{\pm}
\end{equation}

Let $P=QU$ be the Levi decomposition.
Our assumption on $w$ implies that
$Q$ is the product of  $\GL_2(\Qp)$
and a torus.
In this section,
we follow Urban's idea
and show that the localization
$M(U^p)_{\fm}$ belongs to a block
of $Q$-representations in the sense of  \cite{pask}
when  $\fm$ satisfies \eqref{cond:red_gen}.
As a consequence,
we show that $\TT^P(U^p,\oo)_{\fm}$
is Noetherian
and relate the ``reducible'' part of 
$M(U^p)_{\fm}$ to the space of 
$B$-ordinary modular forms.
We deduce from which
the fundamental exact sequence
that is critical for our
construction of the Euler systems
in next section.

\subsection{Generically reducible deformation}



We first recall from \cite[\S B.1]{pask}
the structure of the universal deformation ring 
$R$ of the $2$-dimensional pseudo-representation 
$\chi_1+\chi_2$,
where $\chi_1,\chi_2\colon \Gp\to \fF^\times$ 
are continuous characters satisfying the 
the following generic assumption
\begin{equation}\label{cond:generic}\tag{\text{gen}}
	\chi_1\chi_2^{-1}\neq \id,\omega^{\pm1}.
\end{equation}

By \textit{loc.cit}, the assumption
implies the existence of non-split extensions
\begin{equation*}
    0\to \chi_1\to \rho_{12}\to \chi_2\to 0\quad
    0\to \chi_2\to \rho_{21}\to \chi_1\to 0
\end{equation*}
which are unique up to isomorphisms;
and that the universal deformation rings
$R_{ij}$ of the Galois representations $\rho_{ij}$
are formally smooth of relative dimension $5$ over $\oo$.

Denote by $\tilde{\rho}_{ij}$ the universal deformation,
one may choose bases and think of which as group homomorphisms
$\tilde{\rho}_{ij}\colon \Gp\to \GL_2(R_{ij})$,
whose reduction modulo the maximal ideals are
$\rho_{12}=\smat{\chi_1&*\\&\chi_2}$ and
$\rho_{21}=\smat{\chi_1&\\ * &\chi_2}$.
The trace then induces 
$\theta\colon R\cong R_{ij}$
by \cite[Prop B.17]{pask}.
Since the reduced universal deformation ring
$R_{\red}$ is formally smooth of 
relative dimension $4$ over $\oo$,
the reducibility ideal  $\tau\subset R$ 
is a principal ideal generated by 
an element $x\in\fm_R\setminus \fm_R^2$,
where $\fm_R\subset R$ is the maximal ideal. 
Let $\tau_{ij}\subset R_{ij} $ be the ideal 
generated by the $(j,i)$-entry of  $ \tilde{\rho}_{ij}(g)$
for all $g\in \Gp$,
then  $\theta$ maps  
$\tau$ to  $\tau_{ij}$ by \cite[Prop B.23]{pask}

Let $\tilde{\rho}_{12}^x\colon \Gp\to \GL_2(R_{12})$ be the representation defined by
\begin{equation*}
	\tilde{\rho}_{12}^x(g)\coloneqq 
	\smat{\theta(x)&\\&1}
	\tilde{\rho}_{12}(g)
	\smat{\theta(x)&\\&1}^{-1}.
\end{equation*}
Then $\tilde{\rho}_{12}^x$
is a deformation of $\rho_{21}$ to $R_{12}$
and induces an isomorphism 
$\alpha\colon R_{21}\to R_{12}$,
for which the diagram
\begin{equation*}
	\begin{tikzcd}
		R_{21} \arrow[r,"\alpha"] &
		R_{12}\\
		R\arrow[u,"\theta"] \arrow[r,equal] &
		R\arrow[u,"\theta"]
	\end{tikzcd}
\end{equation*}
commutes by \cite[Prop B.24]{pask}.
We identify $\tilde{\rho}_{21}$ with 
$\tilde{\rho}_{12}^x$.

\begin{lem}\cite[Prop B.26]{pask}.
The modules
$\Hom_{\Gp}(\tilde{\rho}_{12}, \tilde{\rho}_{21})$ and
$\Hom_{\Gp}(\tilde{\rho}_{21}, \tilde{\rho}_{12})$
are free over $R\cong R_{12}$ and
generated respectively by
\begin{equation}\label{eq:Phi_ij}
	\Phi_{12}=\smat{\theta(x)&\\&1} \text{ and }
	\Phi_{21}=\smat{1&\\&\theta(x)}.
\end{equation}
And the algebra
$\End_{\Gp}(\tilde{\rho}_{12}\oplus \tilde{\rho}_{21})$
is isomorphic to the generalized matrix algebra
$\smat{R& R\Phi_{12}\\ R\Phi_{21}& R}$,
which is a free $R$-module of rank  $4$,
with the center isomorphic to  $R$.
\end{lem}

\subsection{Generically reducible block}

In the next two subsections,
we let $G$ and  $K$
denote  $\GL_2(\Qp)$ and $\GL_2(\Zp)$
respecitvely.
We also let $B=TN$
denote the subgroup of upper triangular matrices
and its Levi decomposition,
$\bar{B}$ be
the subgroup of lower triangular matrices,
and identify the center $Z\subset T$ with $\Qp^\times$.

We briefly replace the deformation rings
$R, R_{\red}, R_{12}$ and $R_{21}$
introduced in the previous subsection
by the variants with 
fixed determinant $\epsilon\zeta$,
where $\zeta\colon \Gp\to \oo^\times$
is a continuous character 
such that  $\epsilon\zeta\equiv \chi_1\chi_2$ 
modulo $\varpi$.
All the results there
still hold true, 
except that the relative dimensions
are decreased by  $2$.

Let $\Art\colon \Qp^\times\to \Gp^{\textnormal{ab}}$
denote the reciprocity map
so that $\Fr\coloneqq \Art(p)$
is a geometric Frobenius.
We identify characters
of $\Gp$ with that of $\Qp^\times$
through the composition with $\Art$.
For any character $\chi$ of 
of $T\cong \Qp^\times\times\Qp^\times$,
let $\chi^s$
denote the composition
with the involution
$(a,b)\mapsto (b,a)$.
Define the characters
$\chi=\chi_1\otimes\chi_2\omega^{-1}$ and
$\chi^s\alpha=\chi_2\otimes \chi_1\omega^{-1}$,
where  $\alpha=\omega\otimes\omega^{-1}$.
By \cite[Thm 30]{barthel},
\[
\pi_1\coloneqq \Ind_{B}^G\chi\cong
\Ind_{B}^G\chi_1\otimes\chi_2\omega^{-1}\quad
\pi_2\coloneqq \Ind_{B}^G\chi^s\alpha\cong 
\Ind_{B}^G\chi_2\otimes\chi_1\omega^{-1} 
\]
are irreducible representations
in $\Mod^{\sm}_{G,\zeta}(\oo)$.

\begin{defn}\label{def:block}
Let $\lfMod_{G,\zeta}(\oo), \lfMod_{T,\zeta}(\oo)$
be the subcategories
of representations that are 
locally of finite length 
and with central character $\zeta$;
and let $\fC_G(\oo), \fC_T(\oo)$
be the dual categories
of the Pontryagin duals.
For $\pi_1,\pi_2$ as before
the set $\B\coloneqq\{\pi_1,\pi_2\}$ 
defines a block 
in the sense of \cite[\S 5]{pask}.
Then $\lfMod_{G,\zeta}(\oo)^\B$
is the subcategory
of representations whose subquotients
all belong to $\B$;
and $\fC_G(\oo)^\B$
is the dual category.
\end{defn}

Let $\Ord\colon \Mod_{G,\zeta}^{\sm}(\oo)
\to \Mod_{T,\zeta}^{\sm}(\oo)$
denote the functor of $B$-ordinary
parts as recalled in \eqref{def:OrdP}.
The adjunction formula
in \cite[Thm 4.4.6]{emeI} states that
\begin{equation}\label{eq:adjunct}
	\Hom_{\oo[G]}(\Ind_{\bar{B}}^GU,V)
	\xrightarrow{\Ord}
	\Hom_{\oo[T]}(\Ord(\Ind_{\bar{B}}^GU),\Ord V)
	\cong
	\Hom_{\oo[T]}(U,\Ord V)
\end{equation}
is an isomorphism, where the last isomorphism
is induced by $\Ord(\Ind_{\bar{B}}^GU)\cong U$.


By \cite[Prop 7.1]{pask},
if $\iota\colon \pi_1\hookrightarrow \tilde{J}_1$
is the injective envelope of $\pi_1$
in $\lfMod_{G,\zeta}(\oo)$,
We have $\Ord\pi_1=\Ord(\Ind_B^G\chi)=\chi^s$
and $\Ord(\iota)\colon \chi^s \to \Ord(\tilde{J}_1)$
is isomorphic to an injective envelope
$\tilde{J}_{\chi^s}$ of $\chi^s$
in $\lfMod_{T,\zeta}(\oo)$.
Furthermore, the morphism 
$\iota_1\colon \Ind_{\bar{B}}^G(\tilde{J}_{\chi^s})\to
\tilde{J}_1$
induced through the adjunction formula \eqref{eq:adjunct}
by a fixed isomorphism
$\tilde{J}_{\chi^s}\to \Ord_B(\tilde{J}_1)$
is injective.
To simplify notations,
we identify $\lfMod_{T,\zeta}(\oo)$
with $\lfMod_{\Qp^\times}(\oo)$ through 
the map $\Qp^\times\cong \{\smat{1&\\&*}\}\subset T$
and write $\tilde{J}_{\chi_1}=\tilde{J}_{\chi^s}$.
Then for $ \tilde{J}_{\chi_2}$
and $ \tilde{J}_2$ defined as above
we also have the injective morphisms
$\iota_2\colon \Ind_{\bar{B}}^G(\tilde{J}_{\chi_2})\to
\tilde{J}_2$.


\begin{lem}\cite[Cor 7.7]{pask}
Let $\tilde{P}_{\chi_i^\vee}\coloneqq \tilde{J}_{\chi_i}^\vee
\in\fC_T(\oo)$ and
$\tilde{M}_i\coloneqq 
\Ind_{\bar{B}}^G(\tilde{J}_{\chi_i})^\vee,
\tilde{P}_i\coloneqq \tilde{J}_i^\vee\in\fC_G(\oo)$
for $i=1,2$.
The morphisms
$p_i\colon \tilde{P}_i\twoheadrightarrow \tilde{M}_i$
that are dual to
$\iota_i\colon 
\Ind_{\bar{B}}^G(\tilde{J}_{\chi_i})\hookrightarrow 
\tilde{J}_i$ 
extend to the exact sequences
\begin{equation}\label{eq:exact_PPM}
	0\to \tilde{P}_{2}\xrightarrow{\phi_{12}} 
	\tilde{P}_{1}\xrightarrow{p_1} \tilde{M}_1\to 0 
	\text{ and }
	0\to \tilde{P}_{1}\xrightarrow{\phi_{21}} 
	\tilde{P}_{2}\xrightarrow{p_2} \tilde{M}_2\to 0
\end{equation}
\end{lem}

Let $\Rep_{\Gp}(\oo)$
be the category of compact $\oo$-modules with
continuous actions of $\Gp$,
and $\V\colon \fC_G(\oo)\to \Rep_{\Gp}(\oo)$
be the Colmez functor introduced 
in \cite[\S 5.7]{pask},
which is exact and covariant.
By \cite[Cor 8.7]{pask},
for $(i,j)=(1,2)$ or  $(2,1)$,
there exists unique non-split extensions
in $\Mod^{\sm}_{G,\zeta}(\oo)$ 
\[
	0\to \pi_2\to \kappa_{12}\to \pi_1\to 0,\quad
	0\to \pi_1\to \kappa_{21}\to \pi_2\to 0
\]
such that
$\V(\pi_i^\vee)=\chi_i$, $\V(\kappa_{ij}^\vee)=\rho_{ij}$,
and $\V(\tilde{P}_j)=\tilde{\rho}_{ij}$
are the universal deformations
with determinant $\zeta\varepsilon$.

It then follows from \cite[Lem 8.10]{pask} that 
taking the Colmez functor 
$\V$ induces the isomorphisms below.
\begin{equation}\label{eq:end_deform}
\begin{split}
	\End_{\fC_{G}(\oo)}(\tilde{P_2})\cong R_{12}\cong R,\quad
	\Hom_{\fC_G(\oo)}(\tilde{P}_2, \tilde{P}_1)\cong R\Phi_{12}\\
	\Hom_{\fC_G(\oo)}(\tilde{P}_1, \tilde{P}_2)\cong R\Phi_{21},\quad
	\End_{\fC_{G}(\oo)}(\tilde{P_1})\cong R_{21}\cong R
\end{split}
\end{equation}
Write $ \tilde{P}_\B=\tilde{P}_1\oplus \tilde{P}_2$,
then $\tilde{E}_\B\coloneqq
\End_{\fC_G(\oo)}(\tilde{P}_\B)$
is isomorphic to 
$\End_{\Gp}(\tilde{\rho}_{12}\oplus \tilde{\rho}_{21})$.



By \cite[Prop 7.1]{pask}
any morphism 
$\End_{\fC_G(\oo)}(\tilde{P}_i, \tilde{M}_i)$
factors through
$\End_{\fC_G(\oo)}(\tilde{M}_i)\cong
\End_{\fC_T(\oo)}(\tilde{P}_{\chi_i^\vee})$,
which are isomorphic to
a formal power series ring
$ \oo\llbracket x,y\rrbracket$
of two variables
by \cite[Prop 3.34]{pask}.

\begin{lem}\label{lem:ker_red}
	The kernel of the following 
	surjective homomorphism 
	is the reducibility ideal $\tau\subset R$.
	\begin{equation}
	R\cong 
	\End_{\fC_G(\oo)}(\tilde{P}_i)
	\overset{p_i}{\twoheadrightarrow}
	\End_{\fC_G(\oo)}(\tilde{P}_i, \tilde{M}_i)=
	\oo\llbracket x,y\rrbracket
	\end{equation}
\end{lem}
\begin{proof}
It suffices to show that 
the image of $\tau$ consists of 
$\phi\in \End_{\fC_G(\oo)}(\tilde{P}_i)$
such that $p_i\circ \phi$ is trivial.
Let
\[
\begin{tikzcd}
	0 \arrow[r]&
	\tilde{P}_1  \arrow[r,"\phi_{21}"]  &
	\tilde{P}_2 \arrow[r,"p_2"] \arrow[dr,equal] &
	\tilde{M}_2  \arrow[r] & 0 \\
	0 & 
	\tilde{M}_1 \arrow[l]&
	\tilde{P}_1 \arrow[l,"p_1",swap]  &
	\tilde{P}_2  \arrow[l,"\phi_{12}",swap]  & 
	0  \arrow[l] 
\end{tikzcd}
\]
be the exact sequences in \eqref{eq:exact_PPM}.
Take $\Hom(\tilde{P}_j,\cdot)$.
The adjunction formula implies that
$\Hom_{\fC_G(\oo)}(\tilde{P}_j,\tilde{M}_i)\cong
\Hom_{\fC_T(\oo)}
(\tilde{P}_{\chi_j^\vee},\tilde{P}_{\chi_i^\vee})$,
which is trivial for $i\neq j$
since blocks in $\fC_T(\oo)$ are singletons
by the discussion in \cite[\S 7.2]{pask}.
Consequently there are isomorphisms
\[
	\phi_{12}\circ\colon
	\End_{\fC_G(\oo)}(\tilde{P}_2)\cong
	\Hom_{\fC_G(\oo)}(\tilde{P}_2, \tilde{P}_1)\quad
	\phi_{21}\circ\colon
	\End_{\fC_G(\oo)}(\tilde{P}_1)\cong
	\Hom_{\fC_G(\oo)}(\tilde{P}_1, \tilde{P}_2)
\]
Combined with the isomorphisms in \eqref{eq:end_deform},
we may assume that $\V(\phi_{ij})$ agree with 
$\Phi_{ij}$ in \eqref{eq:Phi_ij}.
Thus the image of 
$\phi_{ij}\circ\phi_{ji}\in End_{\fC_G(\oo)}(\tilde{P}_i)$,
which belongs to the kernel in question,
generates $\tau\subset R$,
and the lemma follows 
since  $R$ is formally smooth of relative dimension  $3$.
\end{proof}

\subsection{Univeral unitary completion}

Let $\Ban_{G,\zeta}(E)$
be the category of admissible 
$G$-representation with central character $\zeta$
on $E$-Banach spaces
as defined in \cite{pask}
and  $\B$ be the block defined earlier.
Identify the center of $\C_G(\oo)^\B$ with  $R$,
then  $R[\frac{1}{p}]$ acts on objects of $\Ban_{G,\zeta}(E)^{\B}$.
If $\fn\subset R[\frac{1}{p}]$ is a maximal ideal,
let $\Irr(\fn)$ be the set of
irreducible representation in  $\Ban_{G,\zeta}(E)^{\B}$
on which the action of $R[\frac{1}{p}]$ 
factors through $\fn$.

On the other hand,
let $T\colon \Gp\to R$ 
and $T_\fn\colon \Gp\to R[\frac{1}{p}]/\fn$ 
denote the universal deformation and 
the reduction.
Enlarge $E$ if necessary,
we assume $R[\frac{1}{p}]/\fn\cong E$.
Let $r_{\fn}\colon \Gp\to \GL_2(E)$
be the semisimple Galoi representation
such that $\mtr r_{\fn}=T_{\fn}$.
When $r_\fn$ is crystalline,
the associated Weil-Deligne representation,
defined by having $\Fr$ acts as $\varphi$
on  $D(r_\fn)\coloneqq (r_\fn\otimes_{\Qp}B_{\cris})^{\Gp}$,
is the sum of 
unramified characters $\chi_1$ and  $\chi_2$.
We then define the unramified
$\GL_2(\Qp)$ representation
$\pi_\fn$
as the un-normalized induction
$\Ind_B^G(\chi_1\otimes\chi_2|\cdot|^{-1})$
(the so-called Hecke correspondence in \cite{pan}).

\begin{rem}
	The representation 
	$\Ind_B^G(\chi_1\otimes\chi_2|\cdot|^{-1})$
	is irreducible under the assumption \eqref{cond:generic}.
	For otherwise
	$T_\fn$ would be of the form
	$\eta+\eta\epsilon$ for some character  $\eta$.
\end{rem}

\begin{lem}\label{lem:uni_completion}
	When $r_\fn$ is crystalline 
	and with 
	Hodge-Tate weights $\{-l,-l-k\}$,
	the universal unitary completion of 
	$\pi_\fn\otimes \pi_{-l,1-l-k}^*(E)$ belongs to 
	$\Irr(\fn)$.
\end{lem}
\begin{proof}
	When  $r_\fn$ is irreducible,
	this follows from \cite[Thm. 1.3]{CDP}
	since 
	$\pi_{-l,1-l-k}^*$,
	which is the algebraic representation of $\GL_2$
	of highest weight  $(l+k-1,l)$,
	is isomorphic to 
	$W_{l,k}\coloneqq \Sym^{k-1}\otimes\det^l$.
	In fact the universal unitary completion
	is the only elements in $\Irr(\fn)$
	in this case by \cite[Cor 8.14]{pask}.
	
	When  $r_\fn$ is reducible
	and $T_\fn=\psi_1+\psi_2$
	for crystalline characters
	$\psi_i\colon \Gp\to E^\times$,
	say of Hodge-Tate weights
	$1-l-k$ and  $-l$ respectively,
	then $\pi=\Ind_B^G(\chi_1\otimes\chi_2|\cdot|^{-1})$
	for the charaters
	\[
	\chi_1=\psi_1(\epsilon|\cdot|^{-1})^{-l-k},\quad
	\chi_2=\psi_2(\epsilon|\cdot|^{-1})^{-l}	
	\]
	By \cite[Thm 12.3]{pask} the universal unitary completion
	of $\pi\otimes W_{l,k}$
	is then $\Ind_B^G(\psi)_{cont}$ for
	\[
		\psi( (\begin{smallmatrix}
			a&b\\&d
		\end{smallmatrix}))
		=\chi_2(a)a^l\chi_1(d)|d|^{-1}d^{l+k-1}
		=\psi_2(a)\psi_1\epsilon^{-1}(d)
	\]
	since $\val_p(\chi_2(p))=-l$ and $\varepsilon(a)=a|a|$.
	And indeed by \cite[Cor 8.15]{pask} we have
	\[
	\Irr(\fn)=\{(\Ind_P^G\psi_1\otimes\psi_2\varepsilon^{-1})_{cont},
	(\Ind_P^G\psi_2\otimes\psi_1\varepsilon^{-1})_{cont}\}.
	\]
\end{proof}

\subsection{Local-global compatibility}\label{sub:uni_completion}

We now return to the setting in 
the beginning of the section,
in which we have assumed 
the existence of
a maximal ideal $\fm\subset \TT^P(U^p,\oo)$ 
and characters $\bar{\delta}_i$
such that \eqref{cond:red_gen} holds.
Thus the characters
$\chi_i\coloneqq\bar{\delta}_i\omega$
of $D_w\cong \Gp$
satisfies \eqref{cond:generic} and
\[
	\epsilon\Psi_\fm\vert_{D_w}\equiv
	\chi_1+\chi_2 \mod \fm.
\]
Fix a crystalline character 
$\zeta\colon \Gp\to \oo^\times$ 
such that $\epsilon\zeta\equiv \chi_1\chi_2$
as in the previous subsection.
We let $\B=\{\pi_1,\pi_2\}$
be the block of $\GL_2(\Qp)$
representations
in $\lfMod_{\GL_2(\Qp),\zeta}(\oo)$
as introduced in Definition \ref{def:block}.


To apply the results in the previous subsections,
we follow Urban and make the following twist.
Let $\Psi\colon \Gp\to R=R_{\fm}$ be 
the universal pseudo-deformation of $\chi_1+\chi_2$.
Since $\det\Psi\equiv \epsilon\zeta$ modulo the 
maximal ideal and the residue characteristic is odd,
the character
\[
	\xi=\epsilon\zeta(\det \Psi^u)^{-1}\colon
	\Gp\to 1+\fm_R
\]
admits a character $\xi^{1/2}$ which is the square root of $\xi$.
Then  $\Psi'\coloneqq \xi^{1/2}\Psi$ has determinant $\epsilon\zeta$
and induces $R_\fm^{\epsilon\zeta}\to R_\fm$,
where $R_\fm^{\epsilon\zeta}$ is the universal 
pseudo-deformation with fixed determinant $\epsilon\zeta$.

By the universal property,
the pseudo-representations
$\epsilon\Psi_{\fm}$
induces homomorphisms of $R_\fm$ 
to the Noetherian local $\oo$-algebras
$\TT^P(U^p,\Iw^P(p^{b,b}),\oo)_{\fm}$.
Let $R_{\fm}\to \TT(U^p,\oo)_{\fm}$
be the homomorphism
coming from the inverse limit.
Note that we cannot define the homomorphism
from the universal property
since we haven't shown that
$\TT(U^p,\oo)_{\fm}$ is Noetherian.
Let $\xi_\fm^{1/2}$ be the  composition 
of $\xi^{1/2}$ with the homomorphism,
so that 
\[
	\xi_\fm\coloneqq(\xi_\fm^{1/2})^2=
	\epsilon\zeta(\det \epsilon\Psi_\fm)^{-1}
	=\zeta(\epsilon\det \Psi_\fm)^{-1}\colon
	\Gp\to 1+\fm\subset \TT(U^p,\oo)_\fm.
\]
It follows from the density of crystalline points
that 
\[
	\epsilon\det\Psi_\fm\circ \Art\colon
	\Qp^\times\to \TT(U^p,\oo)_{\fm}\quad
	p\mapsto U_w^{(2)},\quad
	x\mapsto h_U(x)\,x\in\Zp^\times
\]
When $\Qp^\times$
is identified with the center of  $\GL_2(\Qp)$,
its action on $\Ord_P(S(U^p,E/\oo))_{\fm}$
through the above
is simply the usual right translation.
Let $(\Qp^\times)^\wedge$ be the pro-$p$ completion. 
We may thus assume that 
$\xi_{\fm}^{1/2}\circ \Art$
factors through $\llbracket (\Qp^\times)^\wedge\rrbracket$,
which acts on any smooth $\Qp^\times$-representation.


\begin{defn}\label{def:twist}
	For any smooth $\GL_2(\Qp)$-representation
	$V$,
	we let $V'=V(\xi_{\fm}^{1/2})$
	denote the twist 
	by $(\xi_{\fm}^{1/2}\circ\Art)\circ \det$.
	In particular, 
	$\Ord_P(S(U^p,E/\oo))_{\fm}'$
	has central character $\zeta$.

	Define $Q'=\GL_2(\Qp)\times Z_Q$
	and note that the natural 
	homomorphism  $Q'\to Q$
	is surjective with the kernel
	isomorphic to the center  $\Qp^\times$
	of  $\GL_2(\Qp)$.
	We also view
	$\Ord_P(S(U^p,E/\oo))_{\fm}'$
	as an admissible $Q'$-representation,
	with the usual $Z_Q$-action
	and twisted $\GL_2(\Qp)$-action.
	The restriction of which to $\GL_2(\Qp)$
	is locally admissible,
	and thus by \cite[Thm 2.3.8]{emeI}
	\[
		\Ord_P(S(U^p,E/\oo))_{\fm}'\in \lfMod_{\GL_2(\Qp),\zeta}(\oo).
	\]
\end{defn}


\begin{prop}\label{prop:compatibility}
	Let $\fC(\oo)$ be the dual category of 
	$\lfMod_{\GL_2(\Qp),\zeta}(\oo)$, then 
	$M(U^p)_{\fm}'\in \fC(\oo)^{\B}$
	and the $R_\fm^{\epsilon\zeta}$-action on which
	as the center of the category
	factors through $\TT(U^p,\oo)_\fm$.
\end{prop}

\begin{proof}
	For $\fp\subset \TT_{\wt{k}}^P(U^p\Iw^P(p^{0,1}),E)_{\fm}$,
	the associated representation $r_\pi$ defined 
	in Definition \ref{def:rep_prime}
	is crystalline at $D_w\cong \Gp$.
	Let $\pi_{\fp}$ denote the unramified
	$\GL_2(\Qp)$-representation 
	corresponding to the Weil-Deligne representation
	as in \eqref{eq:Gal_hecke_at_p}.
	By Proposition \ref{prop:wt_space} there exists
	a homomorphism between  $\GL_2(\Qp)$-representations
	\begin{equation}\label{eq:hom_wt}
		\pi_\fp\otimes \pi_{\wt{k}}^*(E)\to S(U^p)_\fm\otimes_{\oo}\E
	\end{equation}
	Use the same argument as \cite[Thm 3.5.5]{pan}
	and the density result in Proposition \ref{prop:density},
	it suffices to show that
	the universal unitary completion of
	the twisted $\GL_2(\Qp)$-representation
	of $\pi_{\fp}\otimes\pi_{\wt{k}}^*(E)$
	belongs to $\Irr(\fn)$.
	Here $\fn$ is the maximal ideal 
	of  $R_\fm^{\epsilon\zeta}[\frac{1}{p}]$ 
	which is the pull-back of $\fp$
	through the homomorphism induced by
	the pseudo-representation  $\xi_{\fm}^{1/2}\epsilon\Psi_\fm$.

	Recall that $r_\fp$
	has Hodge-Tate weights  $k_1+1,k_2$,
	where we write 
	$k_1=k_{\sigma,1}$ and $k_2=k_{\sigma,2}$
	for the unique $\sigma\in I_w$.
	Define $\xi_\fp^{1/2}=\lambda_\fp\circ \xi_\fm^{1/2}$
	and $\xi_\fp=(\xi_fp^{1/2})^2=\zeta(\epsilon\det r_\fp)^{-1}$.
	Suppose $\zeta$ has Hodge-Tate weight  $w_0$,
	then  $w_0-(k_1+k_2)$ is even as  $\xi_\fp\equiv \id$.
	Therefore  $\xi_{\fp}^{1/2}$
	is also crystalline,
	with Hodge-Tate weight $w_{\fp}\coloneqq \frac{1}{2}(w_0-k_1-k_2)$.
	Let $T\colon \Gp\to R_\fm^{\epsilon\zeta}$ 
	and $T_\fn\colon \Gp\to E$ be as last subsection.
	By definition  $T_\fn$ is the trace of the
	representation
	$\xi_{\fp}^{1/2}\epsilon r_{\fp}$,
	which has Hodge-Tate weights  $\{k_1',k_2'-1\}$ 
	for $k_1'=k_1+w_\fp, k_2'=k_2+w_\fp$.

	Since $\pi_\fp$ is also 
	the representation associated to  
	$D(\epsilon r_\fp)$ as in last subsection,
	the representation
	associated to  $D(\xi_\fp^{1/2}\epsilon r_\fp)$ is
	$\pi_\fp(\xi_\fp^{1/2}(\epsilon|\cdot|^{-1})^{w_\fp})$.
	Now, 
	twisting \eqref{eq:hom_wt} by $\xi_\fm^{1/2}$ gives
	\[
		(\pi_{\fp}\otimes \pi_{\wt{k}}^*(E))(\xi_\fp^{1/2})=
		\pi_\fp(\xi_\fp^{1/2}(\epsilon|\cdot|^{-1})^{w_\fp})
		\otimes (\pi_{\wt{k}}(\epsilon|\cdot|^{-1})^{w_{\fp}})^* \to 
		S(U^p)_\fm'\otimes_{\oo}E
	\]
	As $(\pi_{\wt{k}}(\epsilon|\cdot|^{-1})^{w_{\fp}})^*$
	is isomorphic to $\pi_{k_1,k_2}^*(E)$,
	the claim follows from Lemma \ref{lem:uni_completion}.
\end{proof}

On the other hand,
let $Z_Q\to \TT^p(U^p,\oo)_{\fm}^{\times}$
be the character
sending $x\in Z_Q^+$ to $h_U(x)$
and let 
$\chi\colon Z_Q\to \TT^P(U^p,\oo)/\fm=\fF^\times$
denote the residue character modulo $\fm$.
Let $Z_Q$ acts on $\pi_i\in \B$ by $\chi$.
Then the irreducible subquotients
of  $\Ord_P(S(U^p,E/\oo))_{\fm}'$
are all isomorphic to either $\pi_1$ or  $\pi_2$
by previus proposition.


\begin{defn}

Let $\lfMod_{Z_Q}(\oo)$
be the category of $\oo[Z_Q]$-modules
that are locally of finite length
and  $\fC_{Z_Q}(\oo)$ be the dual category.
We write 
$\tilde{J}_\chi\in \lfMod_{Z_Q}(\oo)$ and 
$\tilde{P}_\chi\in \fC_{Z_Q}(\oo)$
for the injective and projective envelopes
for $\chi$ and its Pontryagin dual.
The same argument in \cite[Prop 3.34]{pask}
shows that 
$\tilde{E}_\chi\coloneqq 
\End_{\fC_{Z_Q}(\oo)}(\tilde{P}_\chi)$
is the formal power series ring over $\oo$ 
of relative dimension $\dim\Hom^{\cont}(Z_Q,\fF)$
and $\tilde{P}_\chi$ is free of rank one over 
$\tilde{E}_\chi$.

\end{defn}


\begin{prop}\label{prop:envelope}
	Let $\tilde{P}_{i,\chi}=
	\tilde{P}_i\hat{\otimes}_{\oo}\tilde{P}_\chi$,
	and $R_{\fm, \chi}=R_\fm^{\epsilon\zeta}
	\hat{\otimes}_{\oo}\tilde{E}_\chi$.
	Then $\tilde{P}_{i,\chi}$
	is the projective envelope
	of $\pi_i$ and 
	$\End_{Q'}(R_{\fm, \chi}) \cong R_{\fm,\chi}$.
\end{prop}
\begin{proof}
	When $V\in \Mod_{Q'}(\oo)$
	is locally admissible and $v\in V$.
	Then  vector 
	$v$ is  $Z_Q$-finite by \cite[Lem 2.3.5]{emeI}
	and the $\oo[Q']-module$
	generated by $v$ is also 
	finitely-generated over  $\GL_2(\Qp)$
	and admissible.
	Thus it is also of finite length 
	by \cite[Thm 2.3.8]{emeI}.
	In other word,
	any locally admissible representation $V$
	is also locally of finite length.
	We may now apply \cite[Lem B.6]{GN}
	and use the same argument in \cite[Lem B.8]{GN}
	to prove the above lemma.
\end{proof}

\begin{cor}\label{cor:Hecke_Noetherian}
	Let $\tilde{P}_{\B,\chi}
	=\tilde{P}_\B\hat{\otimes}_{\oo}\tilde{P}_\chi$.
	Then $\mathbf{m}\coloneqq
	\Hom_{Q'}(\tilde{P}_{\B,\chi},M(U^p)_{\fm}')$
	is a finite faithful $\TT(U^p,\oo)_{\fm}$-module
	and the big Hecke algebra
	$\TT(U^p,\oo)$ is Noetherian.
\end{cor}
\begin{proof}
	Let $\tilde{E}_{\B,\chi}
	=\tilde{E}_\B\hat{\otimes}_{\oo}\tilde{E}_\chi$.
	Since $\Ord_P(S(U^p,E/\oo))_\fm'
	\in \fC_{Q'}(\oo)^{\B}$
	for $\B=\{\pi_1,\pi_2\}$,
	on which $M\to \Hom_{Q'}(\tilde{P}_{\B,\chi},M)$
	is an anti-equivalence,
	the action of $\TT(U^p,\oo)_\fm$
	on $\mathbf{m}$ is still faithful.

	Moreover, $\fm$ is finitely-generated
	over  $ \tilde{E}_{\B,\chi}$
	by \cite[Prop 4.17]{pask}.
	Since $ \tilde{E}_{\B,\chi}$
	is also finitely-generated
	over $R_{\fm,\chi}$
	and the action of which on $\mathbf{m}$
	factors through $\TT(U^p,\oo)_{\fm}$,
	we conclude that $\mathbf{m}$ 
	is fintie faithul over $\TT(U^p,\oo)_{\fm}$.
	As this induces an injective homomorphism
	of $R_{\fm,\chi}$-algebra
	$\TT(U^p,\oo)_{\fm}\to 
	\End_{R_{\fm,\chi}}(\mathbf{m})$,
	the Hecke algebra
	is also finite over $R_{\fm}$ and hence Noetherian.
\end{proof}


\subsection{Reducible part of completed homology}

Recall that the reducibility ideal 
$\tau\subset R^{\epsilon\zeta}=R^{\epsilon\zeta}_{\fm}$
is a principal ideal.
Let $R^{\epsilon\zeta,\red}$ be the quotient 
and denote $R_{\fm,\chi}^{\red}=R^{\epsilon\zeta,\red}
\hat{\otimes}_{\oo}\tilde{E}_\chi$.
If $M$ is a  $R_{\fm,\chi}$-module,
we define the reducible part $M^{\red}$ 
as the $R_{\fm,\chi}^{\red}$-module
$M/\tau M$.





\begin{lem}
	For some non-negative integer $r$
	there exists a projective resolution 
	\begin{equation}\label{eq:resolution}
	0\to \tilde{P}_{\B,\chi}^r\to 
	\tilde{P}_{\B,\chi}^r\to 
	M(U^p)_{\fm}'\to 0
	\end{equation}
\end{lem}
\begin{proof}
Identify $\Qp^\times$ with the center of $\GL_2(\Qp)$.
By \cite[Thm 33]{barthel} and \cite[Thm 19]{barthel}, 
there exists
smooth irreducible 
$\GL_2(\Zp)\Qp^\times$-representations $\sigma_i$
and exact sequences
\begin{equation}
	0\to 
	\cInd_{\GL_2(\Zp)\Qp^\times}^{\GL_2(\Qp)}\sigma_i\to
	\cInd_{\GL_2(\Zp)\Qp^\times}^{\GL_2(\Qp)}\sigma_i\to
	\pi_i\to 0
\end{equation}
Let $Z_Q$ acts on each 
of the above by $\chi$
and apply $\Ext^i_{Q'}(*,S')$
for $S'=\Ord_P(S(U^p,E/\oo))_{\fm}'$.
We obtain
\begin{equation*}
    \begin{tikzcd}[row sep=2ex]
	    \Ext^{i-1}_{Q'}
	    (\pi, S')\arrow[r] &
	    \Ext^{i}_{Q'}(\text{c-ind}_{KZ}^G\sigma, S')
	    \arrow[r] \arrow[d,symbol={=}] &
	    \Ext^{i}_{Q'}(\text{c-ind}_{KZ}^G\sigma, S')
	    \arrow[r] \arrow[d,symbol={=}] &
	    \Ext^{i}_{Q'}(\pi, S')\\ 
	 & \Ext^i_{\GL_2(\Zp)\times Z_Q}(\sigma ,S') &
	    \Ext^i_{\GL_2(\Zp)\times Z_Q}(\sigma ,S') &
    \end{tikzcd}
\end{equation*}
The same argument in Lemma \ref{lem:inj}
shows that $S'$ is still 
an injective $\GL_2(\Zp)$-representation
after the twist in Definition \ref{def:twist}.
Therefore the long exact sequence reduces to 
the following, in which 
all the terms 
are finite-dimensional over $\fF$
since $S'$ is $Q'$-admissible.
\begin{equation*}
	0 \to \Hom_{Q'}(\pi_i,S')\to 
	\Hom_{\GL_2(\Zp)\times Z_Q}(\sigma_i,S')\to 
	\Hom_{\GL_2(\Zp)\times Z_Q}(\sigma_i,S')\to 
	\Ext^1_{Q'}(\pi_i,S')\to 0
\end{equation*}



Write $a_i\coloneqq \dim_{\fF} \Hom(\pi_i,S')=
\dim_{\fF} \Ext^1(\pi_i,S')$
so that $\soc(S')=\pi_1^{a_1}\oplus \pi_2^{a_2}$.
Let $\tilde{J}_{i,\chi}$ denote 
the injective envelope of $\pi_i$.
Then the injective envelope 
$\soc(S')\hookrightarrow \tilde{J}=\tilde{J}_1^{'a_1}\oplus \tilde{J}_2^{'a_2}$
factors through an injective morphism 
$\phi_0\colon S'\to \tilde{J}$
since the inclusion $\soc(S')\hookrightarrow S'$
is essential.
Apply the same construction 
to $\soc(\coker(\phi_0))$
and use $\dim_{\fF}\Hom_{Q'}(\pi_i, \coker(\phi_0))=
\dim_{\fF}\Ext^1_{Q'}(\pi_i, S')=a_i$
gives another injective morphism
$\phi_1\colon \coker(\phi_0)\to \tilde{J}$,
which is also surjective as 
$\Hom(\pi_i,\coker(\phi_1))
\cong \Ext^1(\phi_i,\coker(\phi_0))
\cong \Ext^2(\pi, S')=0$.
Now the lemma follows by picking
$r=\max\{a_1,a_2\}$ and taking the Pontryagin dual.
\end{proof}

Let $A\colon \tilde{P}_\fm^r\to \tilde{P}_\fm^r$ 
denote the morphism
in the  resolution \eqref{eq:resolution}.
By the isomorphisms \eqref{eq:end_deform}
and Proposition \ref{prop:envelope},
the morphism 
can be represented by a matrix
$A=\smat{A_{11} & A_{12}\Phi_{12}\\A_{21}\Phi_{21} & A_{22}}$,
where $A_{ij}\in M_r(R_{\fm,\chi})$.
Let $\Ord$ be the functor of ordinary parts
on the upper triangular Borel subgroup of  $\GL_2(\Qp)$.
Then  
$\Ord(\tilde{P}_{i,\chi}^\vee)^\vee\cong 
\tilde{P}_{\chi_i^\vee,\chi}\coloneqq 
\tilde{P}_{\chi_i^\vee}\hat{\otimes}_{\oo}
\tilde{P}_\chi$.
And $\Ord(A)$
is represented by the matrix
$\smat{\tilde{A}_{11} & \\& \tilde{A}_{22}}$,
where $\bar{A}_{ij}\in M_r(R^{\red}_{\fm,\chi})$ denote 
the reduction of the matrices by Lemma \ref{lem:ker_red}.
We apply the right exact $\Ord$ to 
\eqref{eq:resolution} and obtain
\begin{equation}\label{eq:exact_ord}
	\tilde{P}_{\chi_1^\vee,\chi}^r\oplus 
	\tilde{P}_{\chi_2^\vee,\chi}^r
	\xrightarrow{\overline{A}_{11}\oplus\overline{A}_{22}}
	\tilde{P}_{\chi_1^\vee,\chi}^r\oplus 
	\tilde{P}_{\chi_2^\vee,\chi}^r
	\to (\Ord S')^\vee\to 0
\end{equation}


\begin{prop}    
	The matrices 
	$\bar{A}_{ii}\in M_r(R^{\red}_{\fm,\chi})$
	are injective and the sequence above 
	is also left exaxt.
\end{prop}
\begin{proof}
	We first note that 
	$\Ord(\Ord_P(S(U^p,E/\oo))=\Ord_B(S(U^p,E/\oo))$
	is an admissible $T$-representation
	for the diagonal torus $T\subset G_p$,
	thus so is $\Ord(S')$.
	But then the Pontryagin dual 
	$\Ord(S')^\vee$ is finitely-generated
	over  $\oo\llbracket T(\oo)^{\wedge}\rrbracket$
	by \cite[Lem 2.2.11]{emeI}.
	Apply $\Hom_{T}(\tilde{P}_{\chi_i^\vee,\chi},*)$
	to the sequence \eqref{eq:exact_ord} shows
\begin{equation*}
	\End_{T}(\tilde{P}_{\chi_i^\vee,\chi}^r)\to 
	\End_{T}(\tilde{P}_{\chi_i^\vee,\chi}^r)\to 
	\Hom_{T}(\tilde{P}_{\chi_i^\vee,\chi}, ,(\Ord S')^\vee)
	\to 0
\end{equation*}
	We note that the first two terms in which are 
	isomorphic to $\oo\llbracket x,y\rrbracket^{\oplus r}$;
	and the last term is torsion over $\oo\llbracket x,y\rrbracket$,
	since there is no injective ring homomorphism from
	$\oo\llbracket x,y\rrbracket$ to $\oo\llbracket X\rrbracket$.
	It is now clear that $\bar{A}_{ii}$
	must be injective,
	for otherwise a contradiction occurs after 
	tensor with the field of fraction of 
	$\oo\llbracket x,y\rrbracket$.
\end{proof}

\begin{cor}
	the $\End_{\fC_{G}(\oo)}(\tilde{P}_i)$-module
	$\Hom_{\fC_G(\oo)}(\tilde{P}_2, N^\vee)$
	has no $\tau$-torsion 
	under the isomorphism $\End_{\fC_{G}(\oo)}(\tilde{P}_2)\cong R$
	in \eqref{eq:end_deform}.
\end{cor}
\begin{proof}
	Identify the center of $\End_{\fC_{G}(\oo)}(\tilde{P}_\fm)$ 
	with $R$ by the isomorphisms
	in \eqref{eq:end_deform}
	and fix a generator $x$ of the reducibility ideal  $\tau\subset R$.
	The generator acts by, up to an invertible elements,
	$\phi_{ij}\circ\phi_{ji}$ on $\tilde{P}_i$.
	Thus $\tilde{P}_\fm[\tau]=0$,
	since by definition $\varphi_{ij}\circ\varphi_{ji}$ are injective.
	Then apply 
	$\Hom_{\fC_G(\oo)}(\tilde{P}_2,\cdot)$
	and the snake lemma to the diagram
    \begin{equation*}
    \begin{tikzcd}
        0 \arrow[r] & \tilde{P}_\fm^{\oplus r} 
	\arrow[d,"x",hookrightarrow] \arrow[r,"A"] & 
	\tilde{P}_\fm^{\oplus r} 
	\arrow[d,"x",hookrightarrow] \arrow[r] & 
	N^\vee \arrow[d,"x"] \arrow[r] & 0 \\ 
        0 \arrow[r] & \tilde{P}_\fm^{\oplus r}
	\arrow[r,"A"] & \tilde{P}_\fm^{\oplus r}
	\arrow[r] &N^\vee  \arrow[r] & 0 
    \end{tikzcd}
\end{equation*}
results in the following isomorphism,
whose right-hand-side
is trivial by the previous proposition.
\[
\Hom_{\fC_G(\oo)}(\tilde{P}_2, N^\vee)[\tau]\cong 
\ker\left(
	\Hom_{\fC_G(\oo)}(\tilde{P}_2, \tilde{P}_\fm^{\red})^{\oplus r}
\xrightarrow{\smat{\bar{A}_{11}& \bar{A}_{12}\Phi_{12}\\& \bar{A}_{22}}}
\Hom_{\fC_G(\oo)}(\tilde{P}_2, \tilde{P}_\fm^{\red})^{\oplus r} \right)
\]\qedhere
\end{proof}

\begin{cor}
	there exists an isomorphism between $R^{\red}$-modules
    \begin{align}
	    \Hom_{\fC_{G}(\oo)}(\tilde{P}_2,(N^\vee)^{\red})
	    &\cong
	    \Hom_{\fC_{T}(\oo)}(\tilde{P}_{\chi_1^\vee}, (\Ord_PN)^\vee)\oplus
	    \Hom_{\fC_{T}(\oo)}(\tilde{P}_{\chi_2^\vee},(\Ord_PN)^\vee)\\
	    &\cong
	    ((\Ord_PN)_{U_p\equiv \chi_2\omega^{-1}})^\vee\oplus
	    ((\Ord_PN)_{U_p\equiv \chi_1\omega^{-1}})^\vee
    \end{align}
\end{cor}
\begin{proof}
	Denote $\Hom_{\fC_{G}(\oo)}$ by $\Hom_G$ for short.
	We decompse the resulting sequence of cokernels
	from the proof of the previous corollary into the following diagram
\begin{equation*}
    \begin{tikzcd}
	    0 \arrow[r]& \Hom_{G}(\tilde{P}_2, \tilde{P}_2^{\red})^{\oplus r}
	    \arrow[r,"\bar{A}_{11}"] \arrow[d]&
	    \Hom_{G}(\tilde{P}_2, \tilde{P}_2^{\red})^{\oplus r}
	    \arrow[d] &&\\
	    0\arrow[r] & \Hom_{G}(\tilde{P}_2, \tilde{P}_\fm^{\red})^{\oplus r}
	    \arrow[r] 
	    \arrow[d] &
	    \Hom_{G}(\tilde{P}_2, \tilde{P}_\fm^{\red})^{\oplus r}
	    \arrow[d] \arrow[r]&
	    \Hom_{G}(\tilde{P}_2, (N^\vee)^{\red})\arrow[r]&0\\
	    0\arrow[r] & \Hom_{G}(\tilde{P}_2, \tilde{P}_1^{\red})^{\oplus r}
	    \arrow[r,"\bar{A}_{22}"] &
	    \Hom_{G}(\tilde{P}_2, \tilde{P}_1^{\red})^{\oplus r}&&
    \end{tikzcd}
\end{equation*}
But by the exactness of \eqref{eq:exact_ord}
and the proof of the proposition thereafter,
the cokernels on the first and last rows
are isomorphic, as  $R^{\red}$-modules,
to $\Hom_{\fC_T(\oo)}(\tilde{P}_{\chi_2^\vee}, (\Ord_PN)^\vee)$ and
$\Hom_{\fC_T(\oo)}(\tilde{P}_{\chi_1^\vee}, (\Ord_PN)^\vee)$ respectively.
\end{proof}  


\subsection{Fundamental exact sequence}

Let $r=\#\Sigma_p-1$ and  
$\Lambda=\oo\llbracket \GG^{2r+2} \rrbracket$.
$\Lambda'=\oo\llbracket \GG^{2r} \rrbracket$.
Let $R_{\fm}\times \Lambda'$
act on 
$M=\Hom(P\hat{\otimes}_{\oo}\Lambda', M(U^p)_\fm)$.
We are in the following set up:
\begin{align*}
	\Lambda\rightarrow R\times \Lambda' \to \TT
	&\text{ acts on } M\\
	\TT^{\red} &\text{ acts on }
	M^{\red}=M_1\oplus M_2
	\text{ which is finite(free) over }\Lambda
\end{align*}
And by previous results,
there exists $\lambda\colon \TT\to \Lambda$
and a nonzero $\Lambda$-algebra 
morphism $\Theta\colon M_2\to \Lambda$ 
such that 
\[
	\Theta(T\cdot m)=\lambda(T)\cdot \Theta(m).
\]
Let $\wp=\ker\lambda\subset \TT$,
and $\fq\subset \ker(\TT\to \End(M_1))$
is some ideal containing  $x=x_{\red}$.
Let $S_2=\ker(\Theta)\subset M_2$ 
and make the following assumptions.
\begin{enumerate}[label=(\alph*)]
	\item $\fq M_2/\fq S_2$ is $\Lambda$-torsion.
	\item $\TT$ is reduced
	\item $\TT$ acts faithfully on $M$.
\end{enumerate}

\begin{lem}
Let $S=\ker(M\to M^{\red}=M_1\oplus M_2\to M_2
\xrightarrow{\Theta}\Lambda)$, 
then there exists an exact sequence
\begin{equation}\label{eq:fund}
	M/S\to \fq M/\fq S\to \fq M_2/\fq S_2 \to 0
\end{equation}
\end{lem}
\begin{proof}
Since the image of the injection 
$\fq M^{\red}=\fq M/xM\hookrightarrow M^{\red}=M/xM$
is $\fq M_2$ and
\[
	\Image(\fq S^{\red}\to M^{\red})=\fq S+xM/xM=\fq S_2,
\]
we have $\fq M^{\red}/\fq S^{\red}=\fq M_2/\fq S_2$.
The exact sequence in question now follows from 
the isomorphism and applying the snake lemma to 
the commutative diagram
\[
\begin{tikzcd}
	S\arrow[r,"x"] \arrow[d]
	& \fq S \arrow[r] \arrow[d]
	& \fq S^{\red} \arrow[r] \arrow[d] & 0\\
	M\arrow[r,"x"]
	& \fq M \arrow[r]
	& \fq M^{\red} \arrow[r] & 0
\end{tikzcd}
\]
\end{proof}
Note that since 
$\Theta\colon M/S\cong M_2/S_2\hookrightarrow \Lambda$,
the $\TT$-actions on \eqref{eq:fund}
all factor through  $\lambda\colon \TT\to \Lambda$.
Let $K=K_\Lambda=\Lambda_\wp$
be the field of fraction of $\Lambda$.
As $\fq M_2/\fq S_2$ is a $\Lambda$-torsion module,
taking the tensor
$\otimes_{\TT}\TT_{\wp}=
\otimes_\Lambda(\Lambda\otimes_\TT\TT_\wp)=
\otimes_\Lambda K$ 
to the exact sequence then results in the surjection
\[
	K_\Lambda\twoheadrightarrow (\fq M/\fq S)_{\wp}
	= \fq M_\wp/\fq S_\wp
\]
From which 
we see that \eqref{eq:fund} is injective
if $\fq M_\wp/\fq S_\wp\neq 0$.


\begin{lem}
	The localization $\TT_\wp$
	is finite over $K\llbracket x\rrbracket$.
\end{lem}
\begin{proof}
	Let $R_\wp$ be the localization at 
	$\ker(R\times \Lambda'\to \TT
	\xrightarrow{\lambda}\Lambda)$,
	which contains $K\llbracket x\rrbracket$
	as a subring.
	Thus
	\[
	K\llbracket x\rrbracket\hookrightarrow
	R_\wp \longrightarrow T_\wp
	\text{ acts on } M_\wp
	\]
	In fact, the composition
	$K\llbracket x\rrbracket\to \TT_\wp$
	is injective 
	since  $x\neq 0$ acts nontrivially on $M_\wp$
	and $\TT_\wp$ is reduced and thus 
	without nilpotent elements.
	Now, 
	that  $M_\wp$ is finite over  $\TT_\wp$
	implies that  $M_\wp$ is a compact  $\TT_\wp$-module;
	and moreover 
	$\dim_K M_\wp/xM_\wp=(M_1\oplus M_2)_K<\infty$.
	Thus
	$M_\wp$ is finite over $K\llbracket x\rrbracket$
	by the topological Nakayama lemma,
	from which the lemma follows as
	$\TT_\wp\hookrightarrow 
	\End_{K\llbracket x\rrbracket}(M_\wp)$.
\end{proof}

\begin{prop}
	We have the fundamental exact sequence
	\begin{equation}
		0\to M/S\xrightarrow{x}
		\fq M/\fq S\to \fq M_2/\fq S_2 \to 0
	\end{equation}
\end{prop}
\begin{proof}
	Let $\TT, \mathbb{M}$ and  $\mathbb{S}$
	denote  $\TT_\wp, M_\wp$ and $S_\wp$.
	For each minimal prime $\fq_i\subset\TT$,
	let $\TT_i$ be the integral closure
	of  $K\llbracket x\rrbracket$
	in the finite extension
	$\textnormal{Frac}(\TT/\fq_i)/K((x))$.
	Then each $\TT_i$ is a discrete valuation ring and
\[
	\TT\hookrightarrow 
	\prod_{\fq_i:\text{ minimal}}\TT/\fq_i\hookrightarrow
	\tilde{\TT}\coloneqq\prod_i \TT_i
\]
As each $\fq\TT_i$ is a principal ideal,
we have $\fq\tilde{\TT}\cong \TT$ and 
\[
	(\fq \mathbb{M}/\fq \mathbb{S})\otimes_\TT
	\tilde{\TT}\cong
	\fq\mathbb{M}\otimes_\TT\tilde{\TT}/
	\fq\mathbb{S}\otimes_\TT\tilde{\TT}=
	\mathbb{M}\otimes_\TT\fq\tilde{\TT}/
	\mathbb{S}\otimes_\TT\fq\tilde{\TT} \cong 
	\mathbb{M}\otimes_\TT\tilde{\TT}/
	\mathbb{S}\otimes_\TT\tilde{\TT}\cong
	(\mathbb{M}/\mathbb{S})\otimes_\TT\tilde{\TT}=
	K\otimes_{\lambda,\TT}\tilde{\TT}.
\]
The last ring is nonzero since 
$\tilde{\TT}$ is integral over $\TT$
and the maximal ideal of  $ \tilde{\TT}$ 
restricts to that of $\TT$.
\end{proof}

\section{Euler systems}


\subsection{Deformation}

Ribet's lemma 
and cocycles from 
generically irreducible representations.

\subsection{construction}


\begin{equation*}
\begin{tikzcd}[column sep=tiny]
& M_{N\ell}/S_{N\ell}\arrow{dd} \arrow[rd,"\sim"]\arrow[rr]
&& \fq_{N\ell}\otimes M_{N\ell}/S_{N\ell}
	\arrow[rd,twoheadrightarrow]\arrow[rr]\arrow[dd]
&& \fq_{N\ell}^{\red}M_{N\ell}/S_{N\ell}\arrow[dd]\arrow[rd]\\
0 \arrow[crossing over]{rr} 
&& \Lambda_{N\ell}
	\arrow[crossing over]{dd} \arrow[crossing over]{rr} 
&& \fq_{N\ell}M_{N\ell}/\fq_{N\ell}S_{N\ell}
	\arrow[crossing over]{dd}\arrow[crossing over]{rr} 
&& \fq_{N\ell}M_{N\ell}^{\ord}/\fq_{N\ell}S^{\ord}_{N\ell}
	\arrow{dd} \arrow[rr] && 0\\
& M_{N}/S_{N}\arrow{rr}\arrow[rd,"\sim"]
&& \fq_{N}\otimes M_{N}/S_{N}
	\arrow{rr} \arrow[rd,twoheadrightarrow]
&& \fq_{N}^{\red}M_{N}/S_{N} \arrow[rd]& & & \\
0 \arrow[crossing over]{rr} 
&& \Lambda_{N} \arrow[crossing over]{rr} 
&& \fq_{N}M_{N}/\fq_{N}S_{N}\arrow[crossing over]{rr} 
&& \fq_{N}M_{N}^{\ord}/\fq_{N}S_{N}^{\ord} \arrow[rr] && 0
\arrow[from=2-3,to=4-3,crossing over]
\arrow[from=2-5,to=4-5,crossing over]
\end{tikzcd}
\end{equation*}

\[
\begin{tikzcd}
0&
H^1(K,\Lambda_N(\delta)) \arrow[r]\arrow[d,"res_p"]&
H^1(K,\fq_N\otimes \Lambda_N(\delta)) \arrow[r]&
H^1(K,\fq_N^{\red}\otimes \Lambda_N(\delta))\\
&H^1(K_w,\Lambda_N(\delta))
\end{tikzcd}
\]
Here $H^1(K_w,\Lambda_N(\delta))=H_{Iw}^1(\Qp,\oo())$
is free and crystalline?

Let $x=x_{\red}=x(\sigma_0,\tau_0)$
for $\sigma_0,\tau_0\in \Gp$
and  $y(\sigma)=\delta_2^{-1}(\sigma)x(\sigma,\tau_0)$,
then $res_p(y^{\red})=0$
and $res_\ell(y)=0$ if  $\ell\nmid Np$.

\section{Previous results}

Let $\chi$ be a character of $\A_{\K}^\times/\K^\times$
such that $\chi\mid_{\A_\F^\times}=\qch_{\K/\F}$
is the quadratic character associated to $\K/\F$.
Put $\chi_\circ=\chi|\cdot|_{\K}^{1/2}$ and $\chi^0=\bar{\chi}_\circ$.
We assumed the following condition.
\begin{equation}\label{cond:chi}\tag{$\chi$}
    \frac{L(\frac{1}{2},\chi)}{\Omega_\infty^{\Sigma}}\not\equiv 0\text{ mod } \fm
    \text{ and } \chi_\circ
    \text{ satisfies the conditions of \cite[Theorem A]{Hsieh12}}.
\end{equation}

Let $\fc$ be the conductor of $\chi$ and $\fs=\fs^c$ be a prime-to-$p\fc$
idea of $\K$ consists only of split primes.
Let $\K_{p^n\fs}$ be the ray class group of conductor $p^n\fs$ over $\K$,
$\K_{p^\infty\fs}=\bigcup_{n}\K_{p^n\fs}$,
and $\fG_{\fs}^-$ be the Galois group of the maximal pro-$p$ anticyclotomic
subextension in $\K_{p^\infty\fs}$.
We denote by $\fX_{\fs}^-$
the set of characters of $\fG_{\fs}^-$
which are $p$-adic avatars of 
algebraic Hecke characters $\eta$ such that 
\begin{enumerate}
    \item $\eta$ has infinity type $k\geq 0$.
    \item $\eta$ is ramified only at places dividing $p\fs$.
    \item the prime-to-$p$ part of the conductor of $\tilde{\eta}$ divides $\fs$.
\end{enumerate}
We can construct a Hida family $\euF^\circ_{\fs}\in \M(K(\fn),\Lambda_{\fs}^-)$
of level $\fn=\fc\fs$ over $\Lambda_{\fs}^-=R\llbracket \fG_{\fs}^-\rrbracket$ for 
a sufficiently large $p$-adic ring $R$,
which interpolates the $p$-adic modular form
associated to the pull-back theta lifts of $\eta$ when $\eta\in \fX_{\fs}^-$.

\begin{thm}\label{thm:intro1}
    Under the assumption \eqref{cond:chi}
    and suppose $p$ is ``large enough'' with respect to $\K$ and $\fc$,
    the Hida family $\euF^\circ_{\fs}$
    can be shown to be primitive inside $\M(K(\fn),\Lambda_{\fs}^-)$.
\end{thm}

Therefore
\begin{align*}
    U_w\theta^\square_{\Phi'}(\eta,\bnu)(g)&=
    (\chi^{-1}|\cdot|_\K^{-1/2})_w(\varpi_w)
    \theta^\square_{\Phi'}(\eta,\bnu)((\varpi_w,\varpi_w^{-1})g)
    =(\chi_\circ^{-1}\tilde{\eta})_w(\varpi_w)
    \theta^\square_{\Phi'}(\eta,\bnu)(g)\\
    T_w\theta^\square_{\Phi'}(\eta,\bnu)(g)&=
    (\chi_\circ^{-1}\tilde{\eta})_w(\varpi_w)
    \theta^\square_{\Phi'}(\eta,\bnu)(g)
\end{align*}
Since $\chi_\circ$ and $\eta$ has infinity type $\Sigma^c$ and $k$ respecitvely,
\begin{align}
    U_wf^\circ(\eta)&=
    \wt{k}(\varpi_w,1)
    (\chi_\circ^{-1}\tilde{\eta})_w(\varpi_w)f^\circ(\eta)=
    \hat{\chi}_\circ^{-1}(\varpi_w)\tilde{\hat{\eta}}(\varpi_w)f^\circ(\eta)\\
    T_wf^\circ(\eta)&=
    (q_w\chi_\circ(\varpi_w)
    +\chi_\circ^{-1}\tilde{\eta}(\varpi_w))
    f^\circ(\eta)
\end{align}
has a $p$-unit as an eigenvalue when $w\in \Sigma_p$.
Note that the formula also holds when $w\mid \fs$.


where $\rho_{\wt{k}}(g_\infty)=\rho^{\wt{k}}(g_\infty^{-\intercal})
\coloneqq \rho^{\bwt{k}}(g_\infty^{-\intercal},\id_2)$
for $\wt{k}\coloneqq (k,0)$



\begin{thm}\label{thm:intro2}
    Put $\mathcal{L}_\fn=\B_\fn(\euF^\circ_{\fs},U_\fs^{-1}\euF^\circ_{\fs})
    \in \Lambda_{\fs}^-$
    where $U_\fs$ is some Hecke action defined above $\fs$. Then
    \begin{equation*}
        \frac{1}{\Omega_p^{2k+4}}
        \int_{\fG_{\fn}}\hat{\eta}\,\mathcal{L}_\fn=
    C(\eta,\chi,\K)
    \frac{(2\pi i)^{k}\Gamma(k+2)L(1,\chi^{-2}\tilde{\eta})}{\Omega_\infty^{k+2\Sigma}}
    \prod_{w\mid p\fs}
    \varepsilon(1,(\chi^{2}\tilde{\eta}^{-1})_w,\psi_w)
    \frac{1-(\chi^{-2}\tilde{\eta})^{-1}_w(\varpi_v)}
    {1-(\chi^{-2}\tilde{\eta})_w(\varpi_v)q_v^{-1}}
    \end{equation*}
    up to powers of $2$
    when $\hat{\eta}\in \fX^+_{\fs}$.
    Here $C(\eta,\chi,\K)$ is a scalar depending on $\eta,\chi$ and $\K$
    which is a $p$-unit when \eqref{cond:chi} is satisfied 
    and $p$ is ``large enough''
    with respect to $\K$ and $\fc$. 
\end{thm}


\bibliographystyle{amsalpha}
\bibliography{biblio}
\end{document}

