\documentclass[leqno]{amsart}
\usepackage{amsmath} 
\usepackage{amssymb,mathtools,stmaryrd}
\usepackage{mathrsfs,euscript}
\usepackage[table,dvipsnames]{xcolor}
\usepackage{hyperref,tikz-cd,enumitem}
\usepackage[utf8]{inputenc}
\hypersetup{
 colorlinks=true,
 linkcolor=DarkOrchid,
 filecolor=blue,
 citecolor=olive,
 urlcolor=orange,
 pdftitle={Pask\={u}nas' theory},
}

\setlength{\textwidth}{\paperwidth}
\addtolength{\textwidth}{-2in}
\calclayout

\newtheorem{thm}{Theorem}[section]
\newtheorem{lem}[thm]{Lemma}
\newtheorem{prop}[thm]{Proposition}
\newtheorem{cor}[thm]{Corollary}

\theoremstyle{definition}
\newtheorem{defn}[thm]{Definition}

\theoremstyle{remark}
\newtheorem{rem}[thm]{Remark}
\newtheorem{ack}{Acknowledgement}

%%%%%%%%%% COMMONLY USED COMMAND %%%%%%%%%%%%%%%%

\newcommand{\smat}[1]{\left(\begin{smallmatrix} #1 \end{smallmatrix}\right)}
\newcommand{\id}{\mathbf{1}}
\newcommand{\oo}{\mathcal{O}} 
\newcommand{\eo}{\EuScript{O}}
\newcommand{\fF}{\mathbb{F}} % residue field

\newcommand{\Q}{{\mathbf{Q}}}
\newcommand{\Z}{{\mathbf{Z}}}
\newcommand{\Qp}{\mathbf{Q}_p}
\newcommand{\Zp}{\mathbf{Z}_p}
\newcommand{\Ql}{\mathbf{Q}_\ell}
\newcommand{\Zl}{\mathbf{Z}_\ell}
\newcommand{\R}{\mathbf R}
\newcommand{\C}{\mathbf C}
\newcommand{\A}{\mathbf A}
\newcommand{\dd}{\mathfrak{d}} %different
\newcommand{\DD}{\mathcal{D}}  %discriminant
\newcommand{\arch}{\mathbf{a}}
\newcommand{\finite}{\mathbf{h}}
\DeclareMathOperator{\Nr}{N}
\DeclareMathOperator{\Tr}{Tr}

\DeclareMathOperator{\End}{End}
\DeclareMathOperator{\Aut}{Aut}
\DeclareMathOperator{\Hom}{Hom}
\DeclareMathOperator{\Ext}{Ext}
\DeclareMathOperator{\Tor}{Tor}
\DeclareMathOperator{\Ind}{Ind}
\DeclareMathOperator{\cInd}{c-Ind}
\DeclareMathOperator{\nInd}{n-Ind}
\DeclareMathOperator{\Res}{Res}
\DeclareMathOperator{\Cor}{Cor}
\DeclareMathOperator{\Image}{Im}
\DeclareMathOperator{\coker}{coker}
\DeclareMathOperator{\rank}{rank}
\DeclareMathOperator{\corank}{corank}

\DeclareMathOperator{\Sym}{Sym}
\DeclareMathOperator{\Ad}{Ad}

\DeclareMathOperator{\Lie}{Lie}
\DeclareMathOperator{\GL}{GL}
\DeclareMathOperator{\SL}{SL}
\DeclareMathOperator{\UU}{U}
\DeclareMathOperator{\gl}{\mathfrak{gl}}
\DeclareMathOperator{\mtr}{tr}
\DeclareMathOperator{\diag}{diag}
\DeclareMathOperator{\chr}{char} 

\DeclareMathOperator{\vol}{vol}
\DeclareMathOperator{\val}{val}

\DeclareMathOperator{\Spec}{Spec}
\DeclareMathOperator{\Supp}{Supp}
\DeclareMathOperator{\Ass}{Ass}
\DeclareMathOperator{\Ann}{Ann}
\DeclareMathOperator{\Der}{Der}
\DeclareMathOperator{\Fitt}{Fitt}
\DeclareMathOperator{\depth}{depth}
\DeclareMathOperator{\length}{length}

\DeclareMathOperator{\Gal}{\mathcal{G}}
\DeclareMathOperator{\WD}{WD}
\DeclareMathOperator{\Rec}{Rec}
\DeclareMathOperator{\rec}{rec}
\DeclareMathOperator{\Art}{Art}
\newcommand{\Fr}{\textnormal{Fr}} %geometric Frobenius
\newcommand{\frob}{\textnormal{frob}} %arithmetic Frobenius
\newcommand{\dR}{\textnormal{dR}}
\newcommand{\pst}{\textnormal{pst}}
\newcommand{\st}{\textnormal{st}}
\newcommand{\cris}{\textnormal{cris}}

\newcommand{\cont}{\textnormal{cont}}
\newcommand{\cts}{\textnormal{cts}}

\newcommand{\fa}{\mathfrak{a}}
\newcommand{\fc}{\mathfrak{c}}
\newcommand{\ff}{\mathfrak{f}}
\newcommand{\fg}{\mathfrak{g}}
\newcommand{\fk}{\mathfrak{k}}
\newcommand{\fl}{\mathfrak{l}}
\newcommand{\fm}{\mathfrak{m}}
\newcommand{\fn}{\mathfrak{n}}
\newcommand{\fp}{\mathfrak{p}}
\newcommand{\fq}{\mathfrak{q}}
\newcommand{\fs}{\mathfrak{s}}
\newcommand{\ft}{\mathfrak{t}}

%%%%%%%%%% MORE SPECIFIC COMMAND %%%%%%%%%%%%%%%%

\newcommand{\bs}{\mathcal{S}} %Bruhat-Shwartz

%%% p-adic local Langlands

\DeclareMathOperator{\Mod}{\textnormal{Mod}}
\DeclareMathOperator{\laMod}{\textnormal{Mod}^{\textnormal{l.adm}}}
\DeclareMathOperator{\aMod}{\textnormal{Mod}^{\textnormal{adm}}}
\DeclareMathOperator{\fgMod}{\textnormal{Mod}^{\textnormal{fg.aug}}}
\DeclareMathOperator{\fC}{\mathfrak{C}} %dual category
\DeclareMathOperator{\Ban}{\textnormal{Ban}_{G,\zeta}^{\adm}}
\DeclareMathOperator{\Rep}{Rep}
\DeclareMathOperator{\V}{\check{\mathbf{V}}} %Colmez's functor
\DeclareMathOperator{\Ord}{Ord} %Emerton's functor
\DeclareMathOperator{\Irr}{Irr}
\DeclareMathOperator{\soc}{soc}

\newcommand{\Gp}{\mathcal{G}_{\Qp}} %Galois group over \Qp
\newcommand{\B}{\mathfrak B} %Paskunas' Block

\newcommand{\sm}{\textnormal{sm}}
\newcommand{\adm}{\textnormal{adm}}
\newcommand{\lfin}{\textnormal{lfin}}
\newcommand{\ps}{\textnormal{ps}}
\newcommand{\red}{\textnormal{red}}

\newcommand{\xx}{x_\textnormal{red}}

%%% Number Fields

\newcommand{\F}{{\mathcal{F}}} %global totally real
\newcommand{\K}{{\mathcal{K}}} %global CM
\newcommand{\qch}{\epsilon} % quadratic character of K/F
\newcommand{\bw}{\overline{w}}
\newcommand{\flw}{\bar{\fl}}

\newcommand{\cG}{\mathcal{G}}
\newcommand{\fG}{\mathfrak{G}}
\newcommand{\fX}{\mathfrak{X}}
\newcommand{\rg}[1]{\textnormal{Cl}_{#1}} %ray class gp
\newcommand{\rp}[1]{\mathfrak{C}_{#1}} %pro-p quot
\newcommand{\rs}[1]{H_{#1}} % sub of prop-p quot
\newcommand{\rk}[1]{\K({#1})} % ray class field

%%% Modular Forms

\newcommand{\wt}[1]{\underline{ #1 }}
\newcommand{\Iw}{\textnormal{Iw}} %Iwahori subgroup
\newcommand{\TT}{\mathbb{T}} % Hecke algebra
\newcommand{\euF}{\EuScript{F}} %Hida family
\newcommand{\ord}{\textnormal{ord}} %ordinary


\begin{document}
\title{Pask\={u}nas' theory}
\author[Y-S.~Lee]{Yu-Sheng Lee}
\address{Department of Mathematics, University  of Michigan, Ann Arbor, MI 48109, USA}
\email{yushglee@umich.edu}
\date{\today}

\maketitle
\setcounter{tocdepth}{1}
\tableofcontents

\section{Notations}

Throughout the article, $\F$ is a totally real field
and $\K$ is a totally imaginary quadratic extension over $\F$.
Denote by $\arch=\Hom(\F, \C)$ 
the set of archimedean places of $\F$,
and by $\finite$ the set of finite places of $\F$.

We fix an odd prime $p$ throughout the article
that satisfies the following ordinary condition.
\begin{equation}\label{cond:ord}\tag{ord}
\text{Every finite place of $\F$ above $p$ is split in $\K$}.
\end{equation}
We fix an embedding $\iota_\infty:\bar{\Q}\to \C$
and an isomorphism $\iota:\C\cong \C_p$,
and write $\iota_p=\iota\circ\iota_\infty:\bar{\Q}\to \C_p$.


Given a place $v$ of $\F$, archimedean or finite,
let $w\mid v$ denote a place $w$ of $\K$ above $v$.
Then $\K_w$ and $\F_v$ are respectively
the completions of the fields $\K$ and $\F$ at $w$ and $v$.
When $v\in \finite$ we denote by $\oo_w$ and $\oo_v$ 
the rings of integers of $\K_w$ and $\F_v$.
Let $|\cdot|_v$ be the norm on $\F_v$,
which is the usual absolute value when $v\in \arch$
and $q_v=|\varpi_v|_v^{-1}$,
for any choice of uniformizer $\varpi_v$ in $\oo_v$,
is the cardinality of the residue field $\oo_v/(\varpi_v)$
when $v\in \finite$.
For $w\mid v$, define $|a|_w=|\Nr_{\K_w/\F_v}(a)|_v$.


Denote by $\A=\A_{\F}$ the ring of adeles over $\F$,
by $\A_{\infty}$ and $\A_{f}$ respectively
the archimedean and the finite components of $\A$.
Let $\qch_{\K/\F}$ denote 
the quadratic character on $\A_\F^\times/\F^\times$
associated to $\K/\F$ by the global class field theory,
$\qch_v$ denote the component on $\F_v^\times$ 
when $v\in \finite$.
Let $c\in \Gal(\K/\F)$ be the unique complex conjugation.
When it should be clear from the context
we write $cz$, $z^c$ and $\bar{z}$ interchangeably 
for the action of $c$ on $z\in \K\otimes_\F R$, where $R$ is an $\F$-algebra,
induced by the action on the first factor.
We then define 
$\A_\K^1=\{z\in \A^\times_\K=\K\otimes_\F \A_\F \mid z\bar{z}=\Nr_{\K/\F}z=1\}$ and
$\K_v^1=\{z\in \K_v\coloneqq \K\otimes_\F\F_v\mid z\bar{z}=1\}$.
If $\eta$ is a character of $\A_\K^1/\K^1$, 
we denote
by $\tilde{\eta}(\alpha)\coloneqq \eta(\alpha/\alpha^c)$
the Hecke character which is the base change of $\eta$ 
to $\A_\K^\times/\K^\times$.

\subsection{CM types}

Denote respectively by $S_p$ and $S_p^\K$ the set of places above $p$
of $\F$ and $\K$.
Identify $I_\K=\Hom(\K,\bar{\Q})$ with
$\Hom(\K,\C)$ and $\Hom(\K,\C_p)$ by compositions with $\iota_\infty$ and $\iota_p$.
Given $\sigma\in I_\K$,
let $w_\sigma\in S_p^\K$ be the place induced by
$\sigma_p\coloneqq \iota_p\circ \sigma\in\Hom(\K,\C_p)$.
For $w\in S_p^\K$, define
\[
    I_w=\{\sigma\in I_\K\mid w=w_\sigma \}=\Hom(\K_w,\C_p)
\]
and decompose $I_\K=\sqcup_{w\mid p}I_w$.
For a subset $\Sigma\subset I_\K$
define $\Sigma_p=\{w_\sigma\mid \sigma\in \Sigma\}$.
We write
$\Sigma^c=\{\sigma c\mid \sigma\in \Sigma\}$ and 
$\Sigma_p^c=\{cw\mid w\in \Sigma_p\}$.
We fix throughout the article a $p$-ordinary CM type,
which is a subset $\Sigma\subset I_\K$ such that
\[
    \Sigma\sqcup \Sigma^c=I_\K,\quad
    \Sigma_p\sqcup \Sigma_p^c=S_p^\K.
\]
The $p$-ordinary CM type $\Sigma$
always exists by the assumption \eqref{cond:ord},
and is identified with $\arch=\Hom(\F,\C)$ by restrictions.
When $v\in S_p$ decomposes into $v=w\bw$,
we understand always that $w\in \Sigma_p$.

\subsection{Artin reciprocity}
We normalize the Artin reciprocity map
$\Art_w\colon \K_w^\times \to D_w^{\textnormal{ab}}$
so that $\Art_w(\varpi_w)=\Fr_w$
is the geometric Frobenius.

\subsection{Class groups}

When $\fs\subset\oo_\K$ is an ideal
we let $\cG_{\fs}$ denote the Galois group
of the maximal abelian extension over $\K$
that is unramified away primes dividing $p\fs$
and let $\fG_{\fs}$ be the maximal pro-$p$ quotient of which.
Moreover, when $\fs=\bar{\fs}$,
the complex conjugation $c\in \Gal(\K/\F)$
acts on $\cG_{\fs}$ and $\fG_{\fs}$ by conjugation.
Let $\cG_{\fs}^\pm=\{\gamma\in \cG_{\fs}\mid \gamma^c=\gamma^{\pm1}\}$.
We call $\cG_{\fs}^a\coloneqq \cG_{\fs}/\cG_{\fs}^+$
the anticyclotomic quotient of $\cG_{\fs}$, which can be 
identified with a subgroup of $\cG_{\fs}^-$ via the homomorphism
\[
    1-c\colon \cG_{\fs}\to \cG_{\fs}\quad \gamma\to \gamma/\gamma^c
\]
since $\cG_{\fs}^+=\ker(1-c)$ by definition.
We note that under the identification
$\cG_{\fs}^-/\cG_{\fs}^a$ is an abelian group of type $(2,\cdots,2)$.
In particular, 
if we define $\fG_{\fs}^{\pm}$ and $\fG_{\fs}^a$ similarly,
then $1-c\colon \fG_{\fs}^a\cong \fG_{\fs}^-$
when $p$ is odd.

Let $\rec\colon \K^\times\backslash\A_{\K,f}^\times \to \cG_{\fs}$
denote the reciprocity map,
which is normalized so that $\Fr_w\coloneqq \rec(\varpi_w)$
is the geometric Frobenius for each finite place $w$ of $\K$.
By abuse of notation, the homomorphism denoted by
\[
1-c\colon \K^\times\backslash\A_{\K,f}^\times\to
\K^\times\backslash\A_{\K,f}^\times\quad
z\mapsto z/\bar{z}
\]
which is compatible under the reciprocity map
with the homomorphism with the same name on the Galois side,
is an surjection on to the subgroup
$\K^1\backslash \A_{\K,f}^1$.
We then define the anticyclotimc
reciprocity map $\rec\colon \K^1\backslash\A_{\K,f}^1\to \cG_{\fs}^a$
by the commutative diagram
\begin{equation}\label{eq:anticyc_rec}
\begin{tikzcd}
    K^\times\backslash\A_{\K,f}^\times
    \arrow[r,twoheadrightarrow]\arrow[d,"\rec"]&
    K^1\backslash\A_{\K,f}^1 \arrow[r]\arrow[d,"\rec",dashed]&
    K^\times\backslash\A_{\K,f}^\times\arrow[d,"\rec"]\\
    \cG_{\fs} \arrow[r,twoheadrightarrow]&
    \cG_{\fs}^a\arrow[r]&
    \cG_{\fs}
\end{tikzcd}
\end{equation}


\subsection{Characters}
Given 
$\kappa=\sum_{\sigma\in \Sigma} a_\sigma\sigma+b_\sigma\sigma c\in \Z[I_\K]$,
an algebraic Hecke character 
$\chi\colon \A_\K^\times/\K^\times\to \C^\times$ 
has type $\kappa$ if
\[
    \chi_\infty(\alpha)=
    \iota_\infty \left(\prod_{\sigma\in \Sigma} 
    \sigma(\alpha)^{a_\sigma}\sigma(c \alpha)^{b_\sigma}\right),\quad
    \alpha\in \K^\times.
\]
For $\alpha_\infty=(\alpha_\sigma)\in \A_{\K,\infty}^\times$
and $\alpha_p=(\alpha_w,\alpha_{\bw})\in \prod_{v\in S_p}\K_v^\times$, 
define
\[
    \alpha_\infty^\kappa=
    \prod_{\sigma\in \Sigma} 
    (\alpha_\sigma)^{a_\sigma}(\bar{\alpha}_\sigma)^{b_\sigma}\in \C^\times,\quad
    \alpha_p^\kappa=
    \prod_{w\in \Sigma_p}
    \prod_{\sigma\in I_w}
    \sigma_p(\alpha_w)^{a_\sigma}\sigma_p(\alpha_{\bw})^{b_\sigma}\in \C_p^\times,
\]
for example,
$(2\pi)^\Sigma=(2\pi)^{[\F:\Q]}$
when $2\pi$ is embedded diagonally in $\A_{\K,\infty}^\times$.
Define the $p$-adic avatar of $\chi$ by
\[
    \hat{\chi}\colon \A_\K^\times\to \bar{\Z}_p^\times,\quad
    \hat{\chi}(\alpha)=\iota(\chi(\alpha)\alpha_\infty^{-\kappa})\alpha_p^{\kappa}
\]
where $\alpha_\infty$ and $\alpha_p$ are respectively 
the archimedean component and the components above $p$ of $\alpha\in \A_\K^\times$.


\subsection{Matrices}
When $R$ is an $\F$-algebra and 
$m=(m_{ij})\in \text{M}_{r,s}(\K\otimes_\F R)$,
we denote by 
$m^\intercal=(m_{ji}), 
m^c=(m^c_{ij})$, and
$m^*=(m^c_{ji})$
respectively the transpose, conjugate, and conjugate-transpose of $m$.

When $r=s$ and $g\in \GL_r(\K\otimes_\F R)$ is invertible, we write
$g^{-\intercal}=(g^{-1})^\intercal$ and $g^{-*}=(g^{-1})^*$.
We write $\mtr(m)$ for the trace of a square matrix $m$,
and reserve $\Tr$ for the traces between fields extensions.

When $v=w\bw$ is a place that is split in $\K$,
identify $\K_w=\F_v=\K_{\bw}$ and 
write $\K_v=\F_v^2$, 
where the first component corresponds to $\K_w$.
Then $m=(m_w,m_{\bw})\in M_n(\K\otimes_\F\F_v)=M_n(\F_v)\times M_n(\F_v)$ 
denotes an element in $m\in M_n(\K\otimes_\F\F_v)$ and its components.

\subsection{Representations of $p$-adic groups}

Let $\oo$ be the ring of integers of a finite extension $E$
over  $\Qp$.
When $G$ is a $p$-adic analytic group,
let $\Mod_G(\oo)$ be the category
of all $\oo[G]$-modules.
We refer the readers to \cite[\S 2]{emeI} and \cite[\S 2]{pask}
for the notations and definitions of the following subcategories.
\[
\begin{tikzcd}
	\fgMod_{G}(\oo) \arrow[r,leftrightarrow] &
	\laMod_{G}(\oo) \arrow[r,hookrightarrow] &
	\aMod_{G}(\oo) \arrow[r,hookrightarrow] &
	\Mod^{\sm}_{G}(\oo) \\
					       &&
	\Mod^{\lfin}_{G}(\oo) \arrow[ru,hookrightarrow] &
\end{tikzcd}
\]






\section{Modular forms on definite unitary groups}

Let $G$ be the definite unitary group over $\F$,
such that for any $\F$-algebra $R$
\begin{equation}\label{def:def_unitary}
    G(R)=\{g\in \GL_{n}(\K\otimes_\F R) \mid gg^*=\id_n\}.
\end{equation}
In this section we recall from \cite{ger}
the notion of algebraic modular forms on $G$
and results on the associated Galois representations.
Some results 
regarding Hida theory of ordinary forms
are partly generalized
to the case of $P$-lrdinary forms,
for a more general parabolic subgroup $P$,
after incorporating Emerton's functor in \cite{emeI}.
We then introduce 
the big Hecke algebra acting on 
the completed cohomology of $P$-ordinary forms,
which admits a Galois pseudo-representation
of $\Gal_\K$.
Using the techniques developed in \cite{pan},
we show the density
of crystalline points in the big Hecke algebra,
which is crucial for checking 
the local-global compatibility in the next section.


\subsection{Algebraic modular forms}

Let $B_n=T_nN_n\subset \GL_n$ be the subgroup of
upper triangular matrices and its Levi decomposition,
where $T_n$ is the diagonal torus.
We identify the set of algebraic characters $X^*(T_n)$
with  $\Z^n$.
Then the Weyl group $W_n$ of $\GL_n$
acts on $\Z^n$ by $(wk)(t)=k(w^{-1}tw)$.
We let $w_0\in W_n$ denote the longest element.
Following \cite[Def 2.3]{ger},
we say $k=(k_1,\cdots,k_n)\in \Z^n$
is dominant if $k_1\geq \cdots\geq k_n$.
When $k$ is dominant,
the algebraic representation 
$\xi_k\coloneqq \Ind_{B_n}^{\GL_n}(w_0k)$
is of highest weight $k$.



For each $v\in\finite$ that is split in $\K$
we fix a splitting $v=w\bw$
and write $g_v=(g_w,g_{\bw})$
for $g_v\in \GL_n(\F_v\otimes_\F\K)\cong \GL_n(\K_w)\times\GL_n(\K_{\bw})$.
When $g_v\in G(\F_v)$, the map
\begin{equation}
\iota_w\colon G(\F_v)\cong \GL_n(\K_w)\quad
\iota_w(g_v)=g_w
\end{equation}
defines an isomorphism satisfying
$\iota_w(g_v)=\iota_{\bw}(g_v)^{-\intercal}$.
We assume $w\in \Sigma_p$ when $v\in S_p$.
Write $G_{w}\coloneqq\GL_n(\K_w)$ and similarly 
for $B_w, T_w,$ and $N_w$. 
We will identify $G(\F_v)$ with $G_w$ 
via $\iota_w$ and define 
\[
	G_p\coloneqq\prod_{w\in \Sigma_p}G_w,\quad
	K_p\coloneqq\prod_{w\in \Sigma_p}K_w,\, \text{ where }
	K_w\coloneqq\GL_n(\oo_w).
\]

Let $\oo$ be the ring of integers 
of a finite extension $E$ over $\Qp$
that contains $\iota_p(\sigma(\K))$
for all $\sigma\in I_\K$
and let $\wt{k}=(k_\sigma)\in (\Z^n)^{\Sigma}$
be domiant in the sense that
$k_\sigma=(k_{\sigma,1},\cdots,k_{\sigma,n})$
is dominant for each $\sigma\in \Sigma$.
We let $\xi_{\wt{k}}$ denote 
the algebraic $K_p$-representation over $\oo$ given by
\begin{equation}\label{def:algrep}
	\xi_{\wt{k}}=\bigotimes_{\sigma\in \Sigma}
	\Ind_{B_n}^{\GL_n}(w_0k_{\sigma}),\quad
	\xi_{\wt{k}}(g)=
	\otimes_{w\in \Sigma_p}
	\otimes_{\sigma\in I_w}\xi_{k_\sigma}(g_w)\,
	\text{ for } g=(g_w)\in K_p.
\end{equation}
Note that over $E$, the representation $\xi_{\wt{k}}$ 
is an algebraic $G_p$-representation.

\begin{rem}
    In \cite{lee} we have used  $\rho_k$ to denote
	the algebraic representation of lowest weight  $-k$.
	Thus $\xi_k$ is isomorphic to the representation 
	$\rho^k(g)\coloneqq \rho_k(g^{-\intercal})$.
\end{rem}




\begin{defn}\label{def:algform}
When $A$ is an $\oo$-module and  
$\wt{k}\in (\Z^n)^{\Sigma}$ is dominant.
Let $g=(g^p,g_p)\in G(\A_f^p)\times K_p$ acts on a function
$f\colon G(\F)\backslash G(\A_f)\to A\otimes_{\oo}\xi_{\wt{k}}(\oo)$
by $(g\cdot f)(g_0)=\xi_{\wt{k}}(g_p)\cdot f(g_0g)$.
Note that the action can be extended to the whole $G(\A_f)$
when either $A$ is an $E$-module or $\xi_{\wt{k}}$
is the trivial representation.
We say $f$ is an algebraic modular form of
weight $\wt{k}$ and coefficients in $A$
if $f$ is invariant by some open compact subgroup
$U\subset G(\A_f^p)\times K_p$ by the above action
and let $S_{\wt{k}}(A)$
denote the space of all algebraic modular forms
of weight $\wt{k}$ and coefficients in $A$.
And for $U$ as above we write
\begin{equation}
S_{\wt{k}}(U,A)=
S_{\wt{k}}(A)^U=
\left\{ f: G(\F)\backslash G(\A_f)/U^p 
\rightarrow A\otimes_{\oo}\xi_{\wt{k}}(\oo)
\mid f(gu)=\xi_{\wt{k}}(u_p)^{-1}\cdot f(g), u\in U\right\} 
\end{equation}
We will write $S(A)=S_{\wt{k}}(A)$ and
$S(A,U)=S_{\wt{k}}(A,U)$
when $\xi_{\wt{k}}$ is the trivial representation.
\end{defn}


Since $G(\F)\backslash G(\A_f)/U$ is a finite set
for any open compact subgroup $U\subset G(\A_f)$,
any modular form $f\in S_{\wt{k}}(U,M)$ 
is determined by its values on a finite set of points.

Throughout the section,
we fix an ideal
$\fs=\bar{\fs}$ consisting only of 
split primes and  
a splitting $\fs=\ff\bar{\ff}$ such that $\oo_\K=\ff+\bar{\ff}$.
If $w\mid\ff$ is above $v\in\finite$, we let
$\Iw(\ff)_w\subset G(\F_v)$ be the open compact subgroup
whose image under $\iota_w$
is $\{k\in \GL_n(\oo_w)\mid k \bmod \ff \in B_n(\oo_w)\}$
and 
\begin{equation}\label{eq:Iwahori_s}
\Iw'(\ff)_w=\{k\in \Iw(w)\mid (\iota_w(k))_{11}\in \ft_w\}
\end{equation}
where $\ft_w\subset \oo_w^\times$ is the subgroup such 
that $\Delta_w\coloneqq \oo_w^\times/\ft_w$
is the maximal pro-$p$ quotient of  $\oo_w^\times$.
We fix an open compact subgroup 
$U^p\subset G(\A_f^p)$ 
and a decomposition 
$G(\A_f)=\bigsqcup_{i\in I} G(\F)t_i (U^pK_p)$
that satisfies
\begin{align}
    \label{cond:small}\tag{$U^p$-\text{small}}
	&G(\F)\cap t_i(U^pK_p)t_i^{-1}=\{1\} \text{ for all } i \\
    \label{cond:s-ram}\tag{$\fs$-\Iw}
    &\Iw'(\ff)_w\subset U^p \text{ for all } w\mid \ff
\end{align}
Consequenly,
for any open compact subgroup $U^p\subset K_p$
the space $S_{\wt{k}}(U^pU_p,A)$ is
isomorphic to a finite direct sum of 
$A\otimes_{\oo}\xi_{\wt{k}}(\oo)$
where the isomorphism is given by 
evaluating a modular form at a set of representatives
for $G(\F)\backslash G(\A_f)/U^pU_p$.

\subsection{Emerton's functor of ordinary parts}

Only in this subsection,
let $G$ denote a  $p$-adic reductive group,
$P=QU$ be a parabolic subgroup
and its Levi decomposition.
We briefly recall the functor
of ordinary parts 
$\Ord_P\colon \Mod_G^{\sm}(\oo)\to \Mod_Q^{\sm}(\oo)$
defined in \cite{emeI}.
Fix an open compact subgroup $P_0\subset P$.
Let  $Q_0=P_0\cap Q$ and $U_0=P_0\cap U$
and define $Z_Q^+=Z_Q\cap Q^+$,
where  $Z_Q$ is the center of $Q$ and
\[
	Q^+=\{m\in Q\mid mU_0m^{-1}\subset U_0\}.
\]
If  $V$ is a $P$-representation over $\oo$
and  $m\in Q^+$,
as in \cite[Def 3.1.3]{emeI} we define
\begin{equation}\label{def:hUm}
	 h_{U}(m)\colon V^{U_0}\to V^{U_0}\quad
	 h_{U}(m)(v)=\sum_{u\in N_0/m U_0 m^{-1}}um\cdot v
\end{equation}
We now recall the definition of the functor
and refer to \cite[Def 3.1.3]{emeI}
for the details of the notations.
\begin{equation}\label{def:OrdP}
	\Ord_P\colon \Mod_G^{\sm}(\oo)\to \Mod_Q^{\sm}(\oo)\quad
	\Ord_P(V)=\Hom_{\oo[Z_Q^+]}
    (\oo[Z_Q], V^{U_0})_{Z_Q-\textnormal{fin}}.
\end{equation}
Here $Z_Q^+$ acts by translation on the left; 
by $h_U$ on the right.
And the action of $Q=Z_Q\cdot Q^+$ is induced by 
having $Z_Q$ act by translation on the left and 
$Q^+$ act by $h_U$ on the right.

\subsection{Hecke operators and $P$-ordinary forms}

We now resume the previous settings,
in which $G$ is the definite unitary group
and $U^p$ satisfies \eqref{cond:small} and \eqref{cond:s-ram}
for a fixed ideal $\fs=\ff\bar{\ff}$ consisting only of split primes.
Given integers $c\geq b\geq 0$ with  $c>0$, 
we define 
the open compact subgroup 
$\Iw(p^{b,c})=\prod_{w\in \Sigma_p}\Iw(w^{b,c})$, where 
for each $w\in \Sigma_p$
\begin{equation}\label{def:Iwahori}
	\Iw(w^{b,c})=\{
	k\in K_w\mid 
    k \bmod \varpi_w^c \in B_n(\oo/\varpi_w^c)
	\text{ and }
	k \bmod \varpi_w^b \in N_n(\oo/\varpi_w^b)
	\}.
\end{equation}
Following \cite{ger}, we define 
the Hecke operators on $S_{\wt{k}}(U^p\Iw(p^{b,c}),A)$
as the following double-coset operators.

If $v=w\bw$ is split in  $\K$ and $\iota_w^{-1}(K_w)\subset U^p$, then
for $1\leq j\leq n$ let 
\begin{equation}\label{def:hecke_away_p}
	T_w^{(j)}=
	\left[\iota_w^{-1}\left(
	\GL_n(\oo_v)
	\begin{pmatrix}
		\varpi_v\id_{j}&\\&\id_{n-j}
	\end{pmatrix}
	\GL_n(\oo_v)
	\right)\right],
	\text{ note that }
	T_{\bw}^{(j)}=(T_{w}^{{n}})^{-1}T_w^{(n-j)}.
\end{equation}

If $w\in \Sigma_p$, let  
$\alpha_w^{(j)}=\iota_w^{-1}
\left(\begin{smallmatrix}
\varpi_v\id_{j}&\\&\id_{n-j} 
\end{smallmatrix}\right)$ for $1\leq j\leq n$,
and $u\in \iota_w^{-1}(T_n(\oo_w))$, define
\begin{equation}\label{def:hecke_at_p}
	U_{\wt{k},w}^{(j)}=
	(w_0\wt{k})^{-1}(\alpha_{w}^{(j)})\cdot
	[\Iw(p^{b,c})\alpha_w^{(j)}\Iw(p^{b,c})]
	\text{ and }
	\langle u\rangle_{\wt{k}}= (w_0\wt{k})^{-1}(u)\cdot 
	[\Iw(p^{b,c})u\Iw(p^{b,c})].
\end{equation}
Here $w_0\wt{k}$ is viewed as an algebraic character of $T$ 
using the same recipe as in \eqref{def:algrep}.
For $u$ as above,
we also let $\langle u\rangle=[\Iw(p^{b,c})u\Iw(p^{b,c})]$
denote the un-normalized version of the above operator.


If $w\mid \ff$,
for $\alpha_w^{(j)}$ defined as above
and $u\in \iota_w^{-1}(T_n(\oo_w))$, define 
\begin{equation}\label{def:hecke_at_s}
	U_{w}^{(j)}=
	[\Iw(\ff)_w\alpha_w^{(j)}\Iw(\ff)_w]
	\text{ and }
	\langle u\rangle= 
	[\Iw'(\ff)_wu\Iw'(\ff)_w]
\end{equation}
Note that by \eqref{cond:s-ram}
the action only depends on the first upper-left entry
and factors through $\Delta_w=\oo_w^\times/\ft_w$.
\begin{rem}
	Our definition of $\langle u\rangle$ at $w\in \Sigma_p$
	coincides with that of \cite{ger}.
    We choose to use $\langle u\rangle_{\wt{k}}$
    so that some of the Hecke-equivariance properties
    introduced below are easier to state.
    Also note that the operators $\langle u\rangle$,
    at either $w\in\Sigma_p$ or $w\in \ff$,
    are simply the usual action of $u$
    defined in Definition \ref{def:algform}.
\end{rem}

To relate the Hecke operators at $p$
with Emerton's functor.
Let $B\subset G_p$ be the Borel subgroup $\prod_{w\in \Sigma_p} B_w$
and define $B_0=B\cap K_p, T^+,$ and $N_0$ as in last subsection.
We define an action of the monoid $N_0T^+$ 
on $S_{\wt{k}}(U^p\Iw(p^{b,c}),A)$,
which extend the trivial action of $N_0$, by
\begin{equation}\label{def:T_act}
	(nt\cdot f)(g)=(w_0\wt{k})^{-1}(t)\xi_{\wt{k}}(nt)\cdot f(gnt)\quad
    n\in N_0, t\in T^+
\end{equation}
Thus we may define the operators $h_N(t)$
for $t\in T^+$
acting on  $S_{\wt{k}}(U^p\Iw(p^{b,c}),M)$
using the formula \eqref{def:hUm}.
In particular we can observe that
$U_{\wt{k},w}^{(j)}=h_N(\alpha_{w}^{(j)})$ and 
$\langle u\rangle_{\wt{k}}= h_N(u)$
for $\alpha_w^{(j)}$ and $u$ as in \eqref{def:hecke_at_p}. 

More generally, for each  $w\in \Sigma_p$
let $P_w=Q_wU_w\supset B_w$ 
be a standard parabolic subgroup
and its the Levi decomposition.
For integers  $c\geq b\geq 0$ with $c>0$
we define $\Iw^P(p^{b,c})=\prod_{w\in\Sigma_p}\Iw^P(w^{b,c})$ where
\begin{equation}\label{def:Iwahori_P}
	\Iw^P(w^{b,c})=\{
	k\in K_w\mid 
    k \bmod \varpi_w^c \in P_w(\oo/\varpi_w^c)
	\text{ and }
	k \bmod \varpi_w^b \in U_w(\oo/\varpi_w^b)
	\}.
\end{equation}
The Hecke operators $T_w^{(j)}$ and $\langle u\rangle$
as defined in \eqref{def:hecke_away_p} and 
\eqref{def:hecke_at_s} naturally 
acts on  $S_{\wt{k}}(U^p\Iw^P(p^{b,c}),A)$
since the definitions only depends on $U^p$.
To define the Hecke operators at $p$, put
$P=\prod_{w\in \Sigma_p}P_w$, and similarly for $Q$ and $U$.
And for $P_0=K_p\cap P$ define the subgroups $Q_0, U_0$ as before.

\begin{defn}\label{def:hecke}
Define the action of the monoid $U_0T^+$ 
on $S_{\wt{k}}(U^p\Iw^P(p^{b,c}),A)$ as in \eqref{def:T_act}.
We define the Hecke operators above $p$ acting on 
$S_{\wt{k}}(U^p\Iw^P(p^{b,c}),A)$ by
$U_{\wt{k},w}^{(j)}=h_U(\alpha_w^{(j)})$
and $\langle u\rangle_{\wt{k}}=h_U(u)$,
for $\alpha_w^{(j)}$ as in \eqref{def:hecke_at_p}
and $u\in \iota_w^{-1}(T_n(\oo_w))$.
We also define $U_P$
as the product of all $U_{\wt{k},w}^{(j)}$
for which $\alpha_w^{(j)}\in Z_Q$.
\end{defn}


\begin{lem}
The Hecke operators defined above commutes with each other
and are equivariant with respect to the inclusions
$ S_{\wt{k}}(U^p\Iw^P(p^{b,c}),A)\hookrightarrow
S_{\wt{k}}(U^p\Iw^P(p^{b',c'}),A)$
if $b'\geq b$ and $c'\geq c$.
\end{lem}
\begin{proof}
That each $T_w^{(j)}$ commutes with other Hecke operators is classical,
and the equivariance is clear.
For the Hecke operators at $p$,
the commutivity follows from \cite[Lem 3.1.4]{emeI},
and the equivariance follows from 
that of the action \eqref{def:T_act}.
See also \cite[Lem 2.10]{ger} for the proof when $P=B$.
\end{proof}

From this point on,
we assume that $A$ is either $E$, or a finite $\oo$-module
or the Pontryagin dual of a finite $\oo$-module.
In particular, let $\varpi\in \oo$ be a uniformizer, $A$ could be 
$\oo, \oo/\varpi^n\oo, \varpi^{-n}\oo/\oo,$ or $E/\oo$.
Except for the first case,
the operator $e_P\coloneqq\lim_{n\to \infty}(U_P)^{n!}$
converges to an idempotent.
We then define the space of $P$-ordinary forms 
with coefficient in $A$ by
\[
	S_{\wt{k}}^{P-\ord}(U^p\Iw^P(p^{b,c}),A)\coloneqq
	e_PS_{\wt{k}}(U^p\Iw^P(p^{b,c}),A)
\]
and $S_{\wt{k}}^{P-\ord}(U^p\Iw^P(p^{b,c}),E)\coloneqq 
S_{\wt{k}}^{P-\ord}(U^p\Iw^P(p^{b,c}),\oo)\otimes_{\oo}E$.
Alternatively,
$S_{\wt{k}}^{P-\ord}(U^p\Iw^P(p^{b,c}),A)$
is characterized as the subspace on which 
any  $U_{\wt{k},w}^{(j)}$ such that 
$\alpha_w^{(j)}\in Z_Q$ acts invertibly.
When $P=B$ we write 
$S_{\wt{k}}^{B-\ord}(U^p\Iw(p^{b,c}),A)=
S_{\wt{k}}^{\ord}(U^p\Iw(p^{b,c}),A)$,
which coincides with \cite[Def 2.13]{ger}.

\begin{defn}\label{def:ord_hecke}
	We let $\TT^P_{\wt{k}}(U^p\Iw^P(p^{b,c}),A)\subset 
    \End_{\oo}S_{\wt{k}}^{P-\ord}(U^p\Iw^P(p^{b,c}),A)$
	be the $\oo$-subalgebra
	generated by all
	$T_w^{(j)}$, for $1\leq j\leq n$,
	and $(T_w^{(n)})^{-1}$ as in \eqref{def:hecke_away_p},
	all $U_{\wt{k},w}^{(j)}$ and $\langle u\rangle_{\wt{k}}$,
	for which $\alpha_w^{(j)}$ belongs to $Z_Q$
    and $\iota_w(u)\in T_n(\oo_w)\cap Z_Q$,
    as in Definition \ref{def:hecke},
    and all $\langle u\rangle$,
    for $w\mid \ff$ and $u\in \iota_w^{-1}(T_n(\oo_w))$,
    as in \eqref{def:hecke_at_s}.
    When $P=B$
    we write $\TT^{\ord}_{\wt{k}}(U^p\Iw(p^{b,c}),A)=
    \TT^B_{\wt{k}}(U^p\Iw(p^{b,c}),A)$.
\end{defn}

\begin{lem}\label{lem:control}
	The following natural inclusions are also surjective.
    In particular we may restrict ourselves 
    to modular forms of levels $\Iw^P(p^{b,b})$ or $\Iw^P(p^{0,1})$
    at $p$ when considering $P$-ordinary forms.
	\begin{align*}
	&S_{\wt{k}}^{P-\ord}(U^p\Iw^P(p^{b,b}),A)\hookrightarrow	
	S_{\wt{k}}^{P-\ord}(U^p\Iw^P(p^{b,c}),A)\quad 
	\text{ for } c\geq b\geq 1\\
	&S_{\wt{k}}^{P-\ord}(U^p\Iw^P(p^{0,1}),A)\hookrightarrow	
	S_{\wt{k}}^{P-\ord}(U^p\Iw^P(p^{0,c}),A)\quad \text{ for } c\geq 1
	\end{align*}
\end{lem}
\begin{proof}
	It suffices to show that 
	$(U_P)^{n!}S_{\wt{k}}(U^p\Iw^P(p^{b,c}),A)
	\subset S_{\wt{k}}(U^p\Iw^P(p^{b,b}),A)$
	for $n$ sufficiently large. 
	Since $\Iw^P(p^{b,c})$ admits Iwahori decompositions,
	this follows from \cite[Lem 3.3.2]{emeI}.
	The same argument also applies to 
	$S_{\wt{k}}(U^p\Iw^P(p^{0,c}),A)$.
	See also \cite[Lem 2.19]{ger} for the proof when $P=B$.
\end{proof}

\begin{lem}\label{lem:PtoB}
	For any $b\geq 1$
	  we have inclusions
	$S_{\wt{k}}^{\ord}(U^p\Iw(p^{b,b}),A)\subset
	S_{\wt{k}}^{P-\ord}(U^p\Iw^P(p^{b,b}),A)$
    that are equivariant  with respect to the Hecke operators.
    In particular they induce homomorphisms of $\oo$-algebras
	\[
		\TT^P_{\wt{k}}(U^p\Iw^P(p^{b,b}),A)\to
		\TT^{\ord}_{\wt{k}}(U^p\Iw(p^{b,b}),A)
	\]
\end{lem}
\begin{proof}
	Since $U\subset N$ and consequently 
    $\Iw^P(p^{b,b})\subset \Iw(p^{b,b})$,
	it suffices to show that the natural inclusions
	$S_{\wt{k}}(U^p\Iw(p^{b,b}),A)\subset 
	S_{\wt{k}}(U^p\Iw^P(p^{b,b}),A)$
	are equivariant with respect to the Hecke operators.
	This is clear for $T_w^{(j)}$, $\langle u\rangle_{\wt{k}}$,
    and $\langle u\rangle$.
	For $U_{\wt{k},w}^{(j)}$ such that 
	$\alpha=\alpha_w^{(j)}\in Z_Q$, this follows from that
	the set of representatives 
	\[
	\begin{pmatrix}
		\id_j&X\\&\id_{n-j}
	\end{pmatrix},\quad
	X \text{ runs through a set of representatives of }
	M_{j,n-j}(\oo_w/\varpi_w)
	\]
    given in \cite[Lem 2.10]{ger} for $N_0/\alpha N_0\alpha^{-1}$
	is also a set of representatives for 
	$U_0/\alpha U_0\alpha^{-1}$.
\end{proof}

\subsection{Weights independence}

We keep the notations and assumptions in the 
previous subsection.
When $\wt{k}=(k_\sigma)\in (\Z^n)^{\Sigma}$ is dominant,
let $\pi_{k_{\sigma}}$ denote 
the algebraic $Q_w$-representation
$\Ind_{B_w\cap Q_w}^{Q_w}(\omega_0 k_\sigma)$
for each $\sigma\in I_w$ 
and let $\pi_{\wt{k}}$ be the $Q$-representation
as defined by \eqref{def:algrep}.
Let $P_0$ acts on $\pi_{\wt{k}}(\oo)$
via the projection $P=QU\to Q$.
The following proposition can be seen as
a generalization of \cite[Prop 2.22]{ger}.

\begin{lem}
	Let $\pi_{\wt{k}}^*$ denote the contragredient
	representation and $A=\varpi^{-r}\oo/\oo$.
	For each $b\geq r$ there exists 
    the following isomorphism 
    which is equivariant with respect to the Hecke operators
    in Definition \ref{def:ord_hecke}.
	\[
		\epsilon_{\wt{k}} \colon 
		S_{\wt{k}}^{P-\ord}(U^p\Iw^P(p^{b,b}),A)\cong 
		\Hom_{\oo}(\pi_{\wt{k}}^*(\oo),
		S^{P-\ord}(U^p\Iw^P(p^{b,b}),A)).
	\]
	Moreover, the isomorphism is $Q_0$-equivariant 
    with respect to
	the action of $u\in Q_0$ defined as follows.
	\begin{align*}
	&u\cdot F(g)=\xi_{\wt{k}}(u)\cdot F(gu),\quad
	F(g)\in S_{\wt{k}}^{P-\ord}(U^p\Iw^P(p^{b,b}),A)\\
	&u\cdot \phi(v^*)(g)=
	\phi(\pi^*_{\wt{k}}(u^{-1})\cdot v^*)(gu),\quad
	\phi\in \Hom_{\oo}(\pi_{\wt{k}}^*(\oo),
	S^{P-\ord}(U^p\Iw^P(p^{b,b}),A))
	\end{align*}
\end{lem}

\begin{proof}
	By inductions in steps
	we can fix an isomorphism 
	$\xi_{\wt{k}}\cong \Ind_{P}^{G_p}\pi_{\wt{k}}$.
	Let $ev\colon \xi_{\wt{k}}\to \pi_{\wt{k}}$
	be the evaluation at the identity.
	For $F(g)\in S_{\wt{k}}(U^p\Iw^P(p^{b,b}),A)$,
	we define 
	$\epsilon_{\wt{k}}(F)$ by
	\begin{equation}\label{eq:wt_indep}
	\epsilon_{\wt{k}}(F)\colon 
	\pi^*_{\wt{k}}(\oo)\rightarrow
	S(U^p\Iw^P(p^{b,b}),A)\quad
	v^*\mapsto [g\mapsto v^*(ev(F(g)))].
	\end{equation}
    As $b\geq r$,
	the action of $\Iw^P(p^{b,b})$ on 
	$A\otimes_{\oo}\pi_{\wt{k}}(\oo)$
	is trivial.
	Thus the function defined above is indeed 
	a modular form
	of trivial weight.
	It is also straightforward to verify
	that the map is equivariant with respect
    to the $Q_0$-action defined above
    and the Hecke operators away $p$.
	For the Hecke operators at $p$,
    the equivariance follows from that 
    $(w_0\wt{k})^{-1}(z)\cdot ev(\xi_{\wt{k}}(z)v)=
    (w_0\wt{k})^{-1}(z)\cdot \pi_{\wt{k}}(z)ev(v)=
    ev(v)$ when $z\in Z_Q$.


	To construct the reversed map,
	note that if $\mu$ is a weight character of $T$ in  
	$\pi_{\wt{k}}$, then it is also a weight character 
	of $T$ in $\xi_{\wt{k}}$.
	We fix weight vectors $v_\mu\in \xi_{\wt{k}}$
	and $v^*_\mu\in \pi_{\wt{k}}^*$
	such that $v^*_{\mu}(ev(v_\mu))=1$.
	Now, let $\alpha_P\in Z_Q^+$ be the product
	of all $\alpha_w^{(j)}\in Z_Q$ and $\alpha=\alpha_P^r$,
	Let $\{x_i\}_{i\in I}$
	be a set of represntatives 
	for $U_0/\alpha U_0\alpha^{-1}$,
	we put 
	\begin{align*}
		\varphi\colon 
		\Hom_{\oo}(\pi_{\wt{k}}^*(\oo),&
		S(U^p\Iw^P(p^{b,b}),A))\longrightarrow
		S_{\wt{k}}(U^p\Iw^P(p^{b,b}),A)\\
		\phi&\mapsto 
		F_\phi(g)=\sum_{i\in I} \sum_{\mu}
		\phi(v^*_\mu)(gx_i\alpha)\otimes
		\xi_{\wt{k}}(x_i) v_\mu
	\end{align*}
	where $\mu$ runs through the weight characters in 
	$\pi_{\wt{k}}$.
	To show that the resulting function 
	defines a modular form,
	let $u\in \Iw^P(p^{b,b})$, 
	then as explained in \cite[Prop 2.22]{ger}
	there exists a bijection $i\mapsto i'$ of $I$
	such that 
	 \[
		ux_i=x_{i'}v_i,\quad
		v_i\in\alpha\Iw^P(p^{b,b})\alpha^{-1} 
		\cap \Iw^P(p^{b,b})
	\]
	Since each $v_i$ is reduced to the identity matrix 
	modulo $\varpi^r$ and thus acts trivially on 
	$A\otimes_{\oo}\xi_{\wt{k}}(\oo)$,
	\[
		\xi_{\wt{k}}(u)\cdot F_\phi(gu)=
		\sum_{i\in I}\sum_{\mu}
		\phi(v^*_\mu)(gx_i'v_i\alpha)\otimes
		\xi_{\wt{k}}(x_i'v_i) v_\mu=
		\sum_{i\in I}\sum_{\mu}
		\phi(v^*_\mu)(gx_i'\alpha)\otimes
		\xi_{\wt{k}}(x_i')
        v_\mu=F_\phi(g).
	\]

	At last, we observe that for each $\mu$ 
	the composition
	$\epsilon_{\wt{k}}(F_\phi)$ is the homomorphism
	\[
		v_\mu^*\mapsto \sum_{i\in I}\phi(v_\mu^*)
		(gx_i\alpha) =U_P^r\phi(v_\mu^*)(g)
	\]
	On the other hand 
	if we decompose $F$ with respect to a choice of 
	weight vectors
	$F(g)=\sum_\mu F_\mu(g)v_\mu+
	\sum_{\mu'}F_{\mu'}(g)v_{\mu'}$, 
	with $\mu$ goes through weight vectors 
	that also appears in $\pi_{\wt{k}}$
	and $\mu'$ goes through the complement,
	then we have
	$\mu(\alpha)=(w_0\wt{k})(\alpha)$ for all $\mu$
	and  $\varpi^r(w_0\wt{k})(\alpha)\mid \mu'(\alpha)$
	for all $\mu'$.
	Therefore
	\begin{multline*}
	U_P^rF(g)=
	\sum_{i\in I}
	\sum_\mu \xi_{\wt{k}}(x_i)\cdot F_\mu(gx_i\alpha)v_\mu+
	\sum_{i\in I}
	\sum_{\mu'}\frac{\mu'(\alpha)}{(w_0\wt{k})(\alpha)}
	\xi_{\wt{k}}(x_i)\cdot F_{\mu'}(gx_i\alpha)v_{\mu'}\\=
	\sum_{i\in I}
	\sum_\mu \xi_{\wt{k}}(x_i)\cdot F_\mu(gx_i\alpha)v_\mu=
	\sum_{i\in I}
	\sum_\mu 
	\epsilon_{\wt{k}}(F)(v^*_\mu)(gx_i\alpha)\otimes
    \xi_{\wt{k}}(x_i)v_\mu
	=F_{\epsilon_{\wt{k}}(F)}(g).
	\end{multline*}

	We thus have the following commutative diagram,
	from which the proposition follows.
	\[
	\begin{tikzcd}
		S_{\wt{k}}(U^p\Iw^P(p^{b,b}),A)
		\arrow[r,"\epsilon_{\wt{k}}"]
		\arrow[d,"U_P^r"]
		& \Hom_\oo(\pi^*_{\wt{k}}(\oo), S(U^p\Iw^P(p^{b,b}),A))
		\arrow[d,"U_P^r"]
		\arrow[dl,"\varphi"]\\
		S_{\wt{k}}(U^p\Iw^P(p^{b,b}),A)
		\arrow[r,"\epsilon_{\wt{k}}"]
		& \Hom_\oo(\pi^*_{\wt{k}}(\oo), S(U^p\Iw^P(p^{b,b}),A))
	\end{tikzcd}	
	\]
\end{proof}


Let $S_{\wt{k}}^{P-\ord}(U^p,E/\oo)=
\varinjlim_{b}
S_{\wt{k}}^{P-\ord}(U^p\Iw^P(p^{b,b}),E/\oo)$
be the injective limit under inclusions
and define 
\[
	\TT^P_{\wt{k}}(U^p,E/\oo)=
	\varprojlim_{b}
	\TT^P_{\wt{k}}(U^p\Iw^P(p^{b,b}),E/\oo)
\]
Since $S_{\wt{k}}^{P-\ord}(U^p,E/\oo)$
is also the injective limit of 
$S_{\wt{k}}^{P-\ord}(U^p\Iw^P(p^{b,b}),\varpi^{-b}\oo/\oo)$
and $\pi_{\wt{k}}^*(\oo)$ is finite over $\oo$,
the following proposition
follows immediately from the previous lemma.

\begin{prop}\label{prop:wt_indep}
	We have the following isomorphism
	which is equivariant with respect to the 
	Hecke operators in Definition \ref{def:ord_hecke}
    and the $Q_0$-action  defined in previous lemma.
	\[
		\epsilon_{\wt{k}} \colon 
		S_{\wt{k}}^{P-\ord}(U^p,E/\oo)\cong 
		\Hom_{\oo}(\pi_{\wt{k}}^*(\oo),
		S^{P-\ord}(U^p,E/\oo)).
	\]
	In particular, this isomorphism 
	induces the following surjective homomorphism
	between the Hecke algebras
	\[
		\varphi_{\wt{k}}\colon 
		\TT^P(U^p,E/\oo)\twoheadrightarrow
		\TT^P_{\wt{k}}(U^p,E/\oo).
	\]
\end{prop}


\subsection{Completed homology and cohomology}

We continue with previous notations and assumptions.
Let $\{U_p\}$ goes through
all the compact open subgroups in $K_p$.
When $A=E/\oo, \oo/\varpi^{r}\oo$ or $\varpi^{-r}\oo/\oo$, 
\begin{equation}\label{eq:complete}
	S(U^p,A)\coloneqq
	\varinjlim_{U_p}S(U^pU_p,A)\in 
	\Mod^{\adm}_{G_p}(\oo)
\end{equation}
in which $G_p$ acts by the action 
defined in Definition \ref{def:algform},
which is non-other than the usual right translation
on modular forms of trivial weight.
Moreover, when $A=E/\oo$ we have
\[
	S(U^p,E/\oo)^{U_0}=
	\varinjlim_{b}
	S(U^p\Iw^P(p^{b,b}),\varpi^{-b}\oo/\oo)
\]
where each object on the right is 
a finite $\oo$-module.
It then follows from \cite[Lem 3.1.5]{emeI} and \cite[Prop 3.2.4]{emeI}
that $S^{P-\ord}(U^p,E/\oo)\cong \Ord_P(S(U^p,E/\oo))$,
and the object on the right belongs to $\Mod^{\adm}_Q(\oo)$
by \cite[Thm 3.3.3]{emeI}.
We define the $P$-ordinary completed homology and cohomology by
\begin{align}\label{eq:completed_coh}
	M_P(U^p)&=
	\Ord_P(S(U^p,E/\oo))^\vee
	\coloneqq \Hom_\oo(\Ord_P(S(U^p,E/\oo)),E/\oo)\\
	S_P(U^p)&=\Hom_\oo(E/\oo, \Ord_P(S(U^p,E/\oo)))
	\cong \varprojlim_r \Ord_P(S(U^p,\oo/\varpi^{r}))
\end{align}
Let $\Hom_\oo^{\cts}(M(U^p),\oo)$
be the set of
$\Phi\in \Hom_\oo(M(U^p),\oo)$ 
such that for any positive integer $r$,
there exists $b$ sufficiently large so that 
the reduction of $\Phi$ modulo $\varpi^r$
factors through
the Pontryagin dual of 
$S^{P-\ord}(U^p\Iw^P(p^{b,b}),E/\oo)\subset \Ord_P(S(U^p,E/\oo)$. 
It can be verified that 
\[
	M_P(U^p)\cong \Hom_\oo(S_P(U^p),\oo),\qquad
	S_P(U^p)\cong \Hom_\oo^{\cts}(M_P(U^p),\oo)
\]
from which we see that
$\TT^P(U^p,E/\oo)$ acts faithfully
on  $M_P(U^p)$, and  $S_P(U^p)$.
In fact, 
let $S_{\wt{k}}^{P-\ord}(U^p,\oo)
=\varinjlim_{b}S_{\wt{k}}^{P-\ord}(U^p\Iw^P(p^{b,b}),\oo)$
and 
$\TT^P_{\wt{k}}(U^p,\oo)=
\varprojlim_b\TT_{\wt{k}}^{P}(U^p\Iw^P(p^{b,b}),\oo)$
Then the same argument in \cite[Lem 2.17]{ger}
shows that  $\TT^P_{\wt{k}}(U^p,\oo)\cong \TT^P_{\wt{k}}(U^p,E/\oo)$
as $\oo$-algebras.
We remark that this is in line with the fact that 
$S^{P-\ord}(U^p,\oo)$ is dense in $S_P(U^p)$.

\begin{defn}\label{def:big_hecke}
    We call $\TT^P(U^p,\oo)\cong \TT^P(U^p,E/\oo)$
	the big $P$-ordinary Hecke algebra,
    which, through the above identifications,
    acts faithfully on the spaces
	$S^{P-\ord}(U^p,E/\oo)=\Ord_P(S(U^p,E/\oo)), M_P(U^p), S_P(U^p)$,
	and $S^{P-\ord}(U^p,\oo)$. 
	We also define $\TT^P(U^p,E)=\TT^P(U^p,\oo)\otimes_{\oo}E$,
	which acts faithfully
	on the $E$-Banach space $G_p$-representation
    $S_P(U^p)_E\coloneqq S_P(U^p)\otimes_{\oo}E$.
    When $P=B$ we write 
    $\TT^B(U^p,A)=\TT^{\ord}(U^p,A)$.
\end{defn}


\begin{lem}\label{lem:inj}
	The restriction of
	$\Ord_P(S(U^p,E/\oo))$ to $Q_0$ 
	is an injective object
	in $\Mod^{\sm}_{Q_0}(\oo)$.
\end{lem}
\begin{proof}
	Following the strategy of 
	\cite[Prop 3.2.4]{pan}, 
	it suffices to show the surjectivity of
	\[
		\Hom_{\oo[Q_0]}(\pi,\Ord_P(S(U^p,E/\oo)))\to 
		\Hom_{\oo[Q_0]}(\pi_1,\Ord_P(S(U^p,E/\oo)))
	\]
	when $\pi_{1}\hookrightarrow \pi$ 
	is an injective morphism between admissible $Q_0$
	representations that are finite $\oo$-modules.
    Recall the definition of $\Ord_P$ in \eqref{def:OrdP}
    and note that
	any homomorphism on the right factors through 
	\begin{multline*}
		\Hom_{\oo[Q_0]}(\pi_1,
		\Hom_{\oo[Z_Q^+]}
		(\oo[Z_Q], S(U^p\Iw^P(p^{b,b}),
		\varpi^{-r}\oo/\oo)))\\=
		\Hom_{\oo[Z_Q^+]}(\oo[Z_Q],
		\Hom_{\oo[Q_0]}(\pi_1, 
		S(U^p\Iw^P(p^{b,b}),\varpi^{-r}\oo/\oo)))
	\end{multline*}
	for some $b$ and $r$ sufficiently large
    since $\pi_1$ is admissible and $\oo$-finite,
	on which 
	the $Z_Q$-fintieness condition is automatic
	by \cite[Lem 3.1.5]{emeI}.

    Let $\Iw(p^{0,b})$ 
    acts on $\pi$ through $Q_0\cap \Iw(p^{0,b})$
    by the Iwahori decomposition
    and $\pi^\vee$ denote the Pontryagin dual of $\pi$.
	Define the space of modular forms with
    coefficients in $\pi^\vee$ by
	\[
		S_{\pi^\vee}(U^p\Iw(p^{0,b}))=
		\{
			F\colon G(\F)\backslash G(\A_f)\to 
			\pi^\vee\mid 
			F(gu)=\pi^\vee(u_p)\cdot F(g),\,
			u\in \Iw(p^{0,b})
		\}
	\]
	Enlarge $b$ if necessary, we may assume
	$Q_0\cap \Iw(p^{b,b})$ acts trivially on $\pi$.
	Then there exists an isomorphism
	\[
		\epsilon\colon 
		S_{\pi^\vee}(U^p\Iw(p^{0,b}))\cong 
		\Hom_{\oo[Q_0]}(\pi,
		S(U^p\Iw^P(p^{b,b}),\varpi^{-r}\oo/\oo)))
		\quad \epsilon(F)\colon
		v\mapsto [g\mapsto v(F(g))]
	\]
	and similarly for $\pi_1$.
    Since \ref{cond:small} implies that
	that $S_{\pi^\vee}(U^p\Iw(p^{0,b}))$
	are $S_{\pi_1^\vee}(U^p\Iw(p^{0,b}))$
	are direct sums of 
	$\pi^\vee$ and  $\pi_1^\vee$, 
    with the same indexing set
	$G(\F)\backslash G(\A_f)/U^p\Iw(p^{0,b})$,
    the natural map
	$S_{\pi^\vee}(U^p\Iw(p^{0,b}))\to 
	S_{\pi_1^\vee}(U^p\Iw(p^{0,b}))$ is then
	surjective
	as $\pi^\vee\to \pi_1^\vee$ is surjective.
	Take localization to the $P$-ordinary parts
	and apply \cite[Lem 3.1.5]{emeI} again,
	we see that
	\[
		\Hom_{\oo[Z_Q^+]}(\oo[Z_Q],
		S_{\pi^\vee}(U^p\Iw(p^{0,b})))\to 
		\Hom_{\oo[Z_Q^+]}(\oo[Z_Q],
		S_{\pi^\vee_1}(U^p\Iw(p^{0,b})))
	\]
	is also surjective, from which 
	the proposition follows.
\end{proof}


\begin{prop}\label{prop:density}
The subspace $S_P(U^p)_E^{\textnormal{alg}}$ 
of $Q_0$-algebraic vectors, defined as
\[
\Image\left(\bigoplus_{\wt{k}}\Hom_{E[Q_0]}(\pi_{\wt{k}}^*(\oo), S_P(U^p)_E)
\otimes_E \pi_{\wt{k}}^*(E)\rightarrow S_P(U^p)_E\right)
\]
where $\wt{k}$ ranges through all dominant weights,
is dense in the $E$-Banach space $S_P(U^p)_E$.
\end{prop}
\begin{proof}
	Since $\Ord_P(S(U^p,E/\oo))$ is an injective object
	in $\Mod_{Q_0}^{\sm}(\oo)$
	by Lemma \ref{lem:inj},
	we may follow the strategy of 
	\cite[Prop 3.2.9]{pan}
	and use \cite[Cor 3.2.6]{pan}
	to reduce the statement to that of
	$\mathcal{C}(Q_0,E)$,
	the space of continuous  $E$-valued
	functions on $Q_0$.
	The density result then follows from
	\cite[Prop 6.A.17]{Pask14}.
\end{proof}


\begin{prop}\label{prop:wt_space}
	There exists a Hecke-equivariant isomorphism
	\[
	S_{\wt{k}}^{P-\ord}(U^p\Iw^P(p^{0,1}),E)\cong 
	\Hom_{\oo[Q_0]}(\pi_{\wt{k}}^*(\oo), S_P(U^p)_E)
	\]
\end{prop}
\begin{proof}
    Apply $\Hom_\oo(E/\oo,*)$ to the  isomorphism 
    in Proposition \ref{prop:wt_indep},
    we obtain 
	\[
		\Hom_\oo(E/\oo, S_{\wt{k}}^{P-\ord}(U^p,E/\oo))\cong 
		\Hom_\oo(E/\oo,
		\Hom_{\oo}(\pi_{\wt{k}}^*(\oo),
		S^{P-\ord}(U^p,E/\oo)))=
		\Hom_{\oo}(\pi_{\wt{k}}^*(\oo), S_P(U^p))
	\]
	which is equivariant with respect to 
	the Hecke operators and the $Q_0$-actions defined there.
    Observe that 
	taking the $Q_0$-invariant subspaces gives
	$\Hom_{\oo[Q_0]}(\pi_{\wt{k}}^*(\oo), S(U^p))$
	on the right hand side.

	On the other hand, since
	$\Hom_\oo(E/\oo, S_{\wt{k}}^{P-\ord}(U^p,E/\oo))\cong
	\varprojlim_r S_{\wt{k}}^{P-\ord}(U^p,\oo/(\varpi^r))$,
	by Lemma \ref{lem:control}
	the $Q_0$-invariant subspace 
	on the left hand side
	is $S_{\wt{k}}^{P-\ord}(U^p\Iw^P(p^{0,1}),\oo)$.
	The claimed result now follows by
	tensoring both subspaces with $E$.
\end{proof}

\begin{rem}
	The above results generalize
	\cite[Prop 3.2.9]{pan} and 
	\cite[\S 3.2.10]{pan},
	which deals with the case 
	when $n=2$ and $P=G_p$,
	in which the $P$-ordinary condition is empty.
\end{rem}

\begin{cor}\label{cor:density}
    When $\wt{k}$ goes through all dominant weight, 
    there exists an injective homomorphism
    \[
        \TT^P(U^p,E)\to \prod_{\wt{k}}\TT^P(U^p\Iw^P(p^{0,1}),E)
    \]
\end{cor}
\begin{proof}
This is an immediate consequence of the previous two propositions.
\end{proof}

\begin{lem}\label{lem:coh_to_ord}
	There exists a Hecke-equivariant
	surjective homomorphism 
	$M_P(U^p)\to M^{\ord}(U^p)$. 
\end{lem}
\begin{proof}
By the definition above and Lemma \ref{lem:PtoB}
we have the following chain of 
inclusions
\[
	S_{\wt{k}}^{B-\ord}(U^p\Iw(p^{b,b}),E/\oo)\subset
	S_{\wt{k}}^{P-\ord}(U^p\Iw^P(p^{b,b}),E/\oo)
\]
which are equivariant with the Hecke operators
that are defined at both spaces.
Taking the injective limit and the Pontryagin dual 
gives the desired homomorphism.
\end{proof}
Combine the previous two lemmas,
we see there exists a surjection
$M_P(U^p)\to \Hom_{\Lambda_{\fs}}(S^{\ord}_\Lambda,\Lambda_{\fs})$.


\subsection{Hida family}

Recall that for each $w\mid \ff$
with $\fs=\ff\bar{\ff}$,
we let $\Delta_w$ denote the maximal pro-$p$ quotient
of $\oo_w^\times$.
And for each $w\in \Sigma_p$,
we write $T_n(w^b)=\{u\in T_n(\oo_w)\mid u\equiv \id \mod\varpi_w^b\}$.
Put $\Delta_{\fs}=\prod_{w\mid \ff}\Delta_w$
and $T_n(p^b)=\prod_{w\in\Sigma_p}T_n(w^b)$,
we define 
\begin{equation}\label{def:lambda_rings}
    \Lambda\coloneqq \oo\llbracket T_n(p^1)\rrbracket,\quad
    \Lambda_{\fs}\coloneqq\Lambda[\Delta_{\fs}].
\end{equation}
For $w\mid \ff$,
we identify $u\in \oo_w^\times$
with $\diag(u,1,\cdots,1)\in T_n(\oo_w)$.
Then $\Lambda_{\fs}$ acts 
on modular forms via the Hecke operators
$\langle u\rangle_{\wt{k}}$ and $\langle u \rangle$
defined in \eqref{def:hecke_at_p} and \eqref{def:hecke_at_s}.
In particular, there exists an $\oo$-algebra homomorphism
into the big $B$-ordinary Hecke algebra
defined in Definition \ref{def:ord_hecke}
\begin{equation}\label{eq:Lambda_hecke}
    \langle *\rangle\colon \Lambda_{\fs}\to 
    \TT^{\ord}(U^p, \oo).
\end{equation}


\begin{defn}\label{def:Hida_family}
Write $S_B(U^p)=S^{\ord}(U^p)$.
Suppose $U^p$ satisfies \eqref{cond:s-ram}
and $\fG$ is an abelian pro-$p$ group
that admits a homomorphism
$T_n(p^1)\times \Delta_\fs\to \fG$
with finite cokernel.
We define the space of of 
Hida families with tame level $U^p$ over $\fG$
written as $S^{\ord}_\fG(U^p)$,
consists of 
$S^{\ord}(U^p)$-valued measures $\euF$ on  $\fG$
such that 
\begin{equation}\label{eq:Hida_family}
	\langle u\rangle \cdot 
	\int_{\fG}\alpha\,d\euF=
	\alpha(u_1)\cdot \int_{\fG}\alpha\,d\euF\quad
    u\in \prod_{w\mid \ff}T_n(\oo_w)\times
    \prod_{w\in\Sigma_p}T_n(\oo_w)
\end{equation}
for continuous functions $\alpha$ on $\fG$.
Here we view each $T_n(w^1)$ 
as the maximal pro-$p$ quotient of $T_n(\oo_w)$.
\end{defn}

\begin{rem}
    Let $\{V_i\}$ be a system of open compact subgroups
    that defines the topology on $\fG$.
    For each $i$ let $b_i$ be such that
    $T_n(w^{b_i})$ is mapped into $V_i$.
    By definition a Hida family is a projective system of
    \[
        f_i\in S^{\ord}(U^p\Iw(p^{b,b}),\oo)\otimes_\oo
        \oo[\fG/V_i] \quad
        \text{ satisfying }\eqref{eq:Hida_family}.
    \]
    Since the Hecke operators commute with each other,
    we can let $\TT^{\ord}(U^p,\oo)$ acts on the sets
    of $f_i$ as above on
    $S^{\ord}(U^p\Iw(p^{b_i,b_i}),\oo)$.
    This induces an Hecke-action on 
    $S^{\ord}_{\fG}(U^p)$ for which taking integration
    is equivariant.
    
\end{rem}

\begin{lem}\label{prop:ord_to_dual}
    Write $M_B(U^p)=M^{\ord}(U^p)$, then there exists
    a surjective homomorphism
	\[
		\Phi\colon M^{\ord}(U^p)\otimes_{\Lambda_\fs}
        \oo\llbracket\fG\rrbracket \to 
		\Hom_{\oo\llbracket\fG\rrbracket}
		(S_\fG^{\ord}(U^p),\oo\llbracket\fG\rrbracket)
	\]
\end{lem}
\begin{proof}
Since $\Hom_\oo(S^{\ord}(U^p),\oo)\cong M^{\ord}(U^p)$,
given $m\in M^{\ord}(U^p)$ and
$\euF\in S_\fG^{\ord}(U^p)$
we define $\Phi(m)(\euF)\in\oo\llbracket\fG\rrbracket$ as the 
$\oo$-valued measure on $\fG$
by composing $m$ with $\euF$.
In other word, 
\begin{equation}
	\Phi(m)\colon \euF\mapsto
	[\alpha\mapsto m(f_\alpha)]
	\in \Hom_{\oo\llbracket\fG\rrbracket}
	(S_\fG^{\ord}(U^p),\oo\llbracket\fG\rrbracket)\quad
	m\in \Hom_\oo(S^{\ord}(U^p),\oo)\cong M^{\ord}(U^p)
\end{equation}
We then $\oo\llbracket\fG\rrbracket$-linearly extend $\Phi$ to
$M^{\ord}(U^p)\otimes_{\Lambda_\fs}
\oo\llbracket\fG\rrbracket$.

Let $I_\fs\subset\Lambda_{\fs}$
be the augmentation ideal,
then $M^{\ord}(U^p)/I_\fs M^{\ord}(U^p)$
is isomorphic to the Pontryagin dual of the subspace of
ordinary forms in $S^{\ord}(U^p,E/\oo)$ that are invariant by
$\prod_{w\mid \ff}T_n(\oo_w)\times \prod_{w\in\Sigma_p}T_n(\oo_w)$.
By Lemma \ref{lem:control}
this invariant subspace is 
$S^{\ord}(U^p\Iw(p^{0,1}),E/\oo)^{\Delta_{\fs}}$,
and the Pontryagin dual of which is isomorphic to
$\Hom_{\oo}(S^{\ord}(U^p\Iw(p^{0,1}),\oo)^{\Delta_{\fs}},\oo)$.
Let $H$ be the cokernel of 
$T_n(p^1)\times\Delta_\fs\to \fG$, 
then
\[
    M^{\ord}(U^p)\otimes_{\Lambda_{\fs}}
    \oo\llbracket \fG\rrbracket /I_\fs\cong 
    \Hom_{\oo}(S^{\ord}(U^p\Iw(p^{0,1}),\oo)^{\Delta_{\fs}},
    \oo[H])  
\]
On the other hand, \eqref{eq:Hida_family}
implies that
$S^{\ord}_\fG(\fn)/I_\fs S^{\ord}_\fG(\fn)$
consists of measures on the finite abelian group $H$
valued in 
$S^{\ord}(U^p\Iw(p^{0,1}),\oo)^{\Delta_{\fs}}$
that can be extended to a Hida family over $\fG$.
Therefore
\[
    \Hom_{\oo[H]}(S^{\ord}(U^p\Iw(p^{0,1}),\oo)^{\Delta_{\fs}}
    \otimes_{\oo}\oo[H],
    \oo[H])
    \twoheadrightarrow
    \Hom_{\oo\llbracket\fG\rrbracket}(S^{\ord}_\fG(U^p),
    \oo\llbracket\fG\rrbracket/I_\fs).
\]
Since 
$\Hom_{\oo}(S^{\ord}(U^p\Iw(p^{0,1}),\oo)^{\Delta_{\fs}}
,\oo[H])=
\Hom_{\oo[H]}(S^{\ord}(U^p\Iw(p^{0,1}),\oo)^{\Delta_{\fs}}
\otimes_{\oo}\oo[H], \oo[H])$, 
we obtain a surjective homomorphism
between $\Lambda_{\fs}$-modules
\[
    M^{\ord}(U^p)\otimes_{\Lambda_{\fs}}
    \oo\llbracket \fG\rrbracket /I_\fs
    \twoheadrightarrow 
    \Hom_{\oo\llbracket\fG\rrbracket}(S^{\ord}_\fG(U^p),
    \oo\llbracket\fG\rrbracket/I_\fs).
\]
Since the objects on both sides of the homomorphism $\Phi$
have compact topology, both objects 
are finitely-genreated over $\Lambda_{\fs}$
by the topologicla Nakayama lemma.
The surjective now follows from the above surjectivity
modulo $I_{\fs}$ and Nakayama lemma.
\end{proof}





\subsection{Hecke algebras and Galois representations}

When $\wt{k}\in (\Z^n)^{\Sigma}$ is dominant,
Let $G(\A_f)$ act on $S_{\wt{k}}(\bar{\Q}_p)$
as in Definition \ref{def:algform} and
let $G(\A)$ act on $\mathcal{A}$,
the space of automorphic forms on $G(\A)$,
by the usual right translations.
Let $\xi_{\wt{k}}^*(\C)$ be the
$G(\A_\infty)$-representation
defined by the inclusions
$G(\F_\sigma)\subset \GL_n(\F_\sigma\otimes \K)=\GL_n(\C)$
for each $\sigma\in \Sigma$.
By \cite[Prop 3.3.2]{CHT},
there exists an $G(\A_f)$-equivariant isomorphism
\begin{equation}\label{eq:p_to_infty}
	\iota\colon S_{\wt{k}}(\bar{\Q}_p)\otimes_{\iota,\bar{Q}_p}\C
	\rightarrow \Hom_{G(\A_\infty)} (\xi_{\wt{k}}^*(\C), \mathcal{A})\quad
	\iota(F)\colon v^*\mapsto 
	[g\mapsto \xi_{\wt{k}}(g_\infty)\cdot 
    \iota\left(\xi_{\wt{k}}(g_p)F(g_f)\right)].
\end{equation}


\begin{prop}\cite[Prop.2.27]{ger}
	Let $\pi$ be an irreducible constituent of the
	$G(\A_f)$-representation $S_{\wt{k}}(\bar{\Q}_p)$,
	then there exist a unique 
	continuous semi-simple Galois representation
	\[
	r_\pi: \Gal_\K \rightarrow \GL_n(\bar{\Q}_p)\quad
	\text{ satisfying }
	r_\pi^c \cong r_\pi^{\vee} \epsilon^{1-n}
	\]
	where $\epsilon$ is the $p$-th cyclotomic character,
	with the following properties.
\begin{enumerate}[label=(\alph*)]
\item Let $v=w\bw$ be a prime-to-$p$ place that is split in $\K$
and $\pi_w$ be the $\GL_n(\K_w)$-representation
induced by $\iota_w\colon G(\F_v)\cong \GL_n(\K_w)$, then
\[
\WD\left(\left.r_\pi\right|_{D_w}\right)^{\mathrm{ss}} \cong
\Rec(\pi_w|\cdot|^{\frac{1-n}{2}})^{\mathrm{ss}}.
\]
Moreover, $r_\pi$ is unramified at $w$ if $\pi_v$ is unramified.
\item Let $v=w\bw$ with $v\in S_p$ and define $\pi_w$ as above.
The representation $r_\pi$ is potentially semistable at $w$ and  $\bw$.
Moreover $r_\pi$ is crystalline at $w$ 
if $\pi_v$ is unramified,
in which case 
the characteristic polynomial of the geometric Frobenius $\Fr_w$
on $\WD\left(D_{\mathrm{cris }}\left(\left.r_\pi\right|_{D_w}\right)\right)$
coincides with that of $\Rec(\pi_w|\cdot|^{\frac{1-n}{2}})^{\mathrm{ss}}$.
\item 
Let $k_{\sigma,j}=-k_{\sigma c, n-j+1}$
for $\sigma\in \Sigma^c$.
If $w\mid p$ and  $\sigma\in I_w$, then 
$\dim_{\bar{\Q}_p}\operatorname{gr}^i
\left(r_\pi \otimes_{\sigma, \K_w} B_{\dR}\right)^{D_w}=1$
exactly when $i=k_{\sigma, j}+n-j$ 
for $j=1, \ldots, n$ and is equal to 0 otherwise.
\end{enumerate}
\end{prop}

From now on,
we restrict ourselves to the following situation
\begin{equation}\label{cond:parabolic}\tag{P}
	n=2,\, 
	P_{w_0}=G_{w_0} \text{ for a fixed }w_0\in \Sigma,\,
	P_{w'}=B_{w'} \text{ for } w'\neq w.
\end{equation}
By \cite[Lem 2.14]{ger}, 
the Hecke algebras
$\TT^P_{\wt{k}}(U^p\Iw^P(p^{b,b}),\oo)$
are then finite flat reduced $\oo$-algebras.
Consequently
$\TT^P_{\wt{k}}(U^p\Iw^P(p^{b,b}),E)$
is isomorphic to the product of $E_\fp$,
where $\fp$ goes through minimal prime ideals
and $E_{\fp}\coloneqq \TT^P_{\wt{k}}(U^p\Iw^P(p^{b,b}),E)/\fp$
are finite extensions over $E$.
By abuse of notation,
we also let $\fp$ denote 
the height one prime ideal $\fp\cap \TT^P_{\wt{k}}(U^p\Iw^P(p^{b,b}),\oo)$
in $\TT^P_{\wt{k}}(U^p\Iw^P(p^{b,b}),\oo)$.

\begin{defn}\label{def:rep_prime}
	Let 
	$\lambda_\fp\colon \TT^P_{\wt{k}}(U^p\Iw^P(p^{b,b}),E)\to E_\fp$
    be the projection associated 
    to a minimal prime $\fp\subset \TT^P_{\wt{k}}(U^p\Iw^P(p^{b,b}),E)$.
    Let $\pi$  be an irreducible constituent 
    of $S_{\wt{k}}(\bar{\Q}_p)$
    and $r_\pi$ be the corresponding Galois representation.
    We say $\pi$ belongs to $\fp$ if
	$\pi\cap S_{\wt{k}}^{P-\ord}(U^p\Iw^P(p^{b,b}),E)_{\fp}\neq 0$.
	This implies that
	\begin{equation}\label{eq:Gal_hecke_away_p}
		\mtr(r_\pi(\Fr_w))=\lambda_\fp(T_w^{(1)}),\quad
		\det(r_\pi(\Fr_w))=q_w\lambda_\fp(T_w^{(2)}),\,
	\end{equation}
	for primes $w$ as in \eqref{def:hecke_away_p}.
    Since the set of such primes has density one,
	by Chebotarev's density theorem
	the representation $r_\pi$ is defined over  $E_{\fp}$
	and independent of the choice of $\pi$.
	Thus we also write $r_\fp=r_\pi$.
\end{defn}



Following \cite{ger},
we say $\wt{k}$ is sufficiently regular
at $w\in \Sigma_p$
if there exists  $\sigma\in I_{w}$
such that  $k_{\sigma,1}>k_{\sigma,2}$.
\begin{lem}\label{lem:galois_at_p}
	Let $r_{\fp}$ be the Galois representation
	associated to a minimal prime
	$\fp\subset \TT^P_{\wt{k}}(U^p\Iw^P(p^{0,1}),E)$.
	\begin{enumerate}[label=(\alph*)]
	\item The representation $r_\fp$ is crystalline at $w'\neq w_0$
	if $\wt{k}$ is sufficiently regular at $w'$.
	Moreover, $r_\fp\vert_{D_{w'}}$ 
	fits into an exact sequence
	$0\to \psi_1\to r_{\fp}\vert_{D_{w'}} \to \epsilon^{-1}\psi_2\to 0$
	for characters $\psi_i\colon D_{w'}\to E_{\fp}^{\times}$ with
	\begin{equation}\label{eq:Gal_hecke_at_p'}
	\begin{aligned}
		\psi_1\circ \Art_{w'}(\varpi_{w'})&=
		\lambda_{\fp}(U_{\wt{k},w'}^{(1)}) &
		\psi_1\circ \Art_{w'}(x)&=
		\lambda_{\fp}
		(\langle 
		\iota_{w'}^{-1}
		(\begin{smallmatrix}
			x&\\&1
		\end{smallmatrix})
		\rangle)\, \text{ for }x\in \oo_{w'}^{\times}\\
		\psi_2\circ \Art_{w'}(\varpi_{w'})&=
		\lambda_{\fp}(U_{\wt{k},w'}^{(2)})/
		\lambda_{\fp}(U_{\wt{k},w'}^{(1)}) &
		\psi_1\circ \Art_{w'}(x)&=
		\lambda_{\fp}
		(\langle 
		\iota_{w'}^{-1}
		(\begin{smallmatrix}
			1&\\&x
		\end{smallmatrix})
		\rangle)\, \text{ for }x\in \oo_{w'}^{\times}
	\end{aligned}
	\end{equation}
	\item The representation $r_\fp$ is 
	crystalline at $w=w_0$, with 
	\begin{equation}\label{eq:Gal_hecke_at_p}
	\det r_\fp\circ \Art_w(\varpi_w)=
	\lambda_{\fp}(U_{\wt{k},w}^{(2)}),\quad
	\epsilon\det r_\fp\circ \Art_w(x)=
	\lambda_{\fp}
	(\langle 
	\iota_{w}^{-1}
	(\begin{smallmatrix}
		x&\\&x
	\end{smallmatrix})
	\rangle)\, \text{ for }x\in \oo_{w}^{\times}
	\end{equation}
	If furthermore $\fp$ is the restriction
	of a prime ideal in 
	$\TT^{\ord}(U^p\Iw(p^{0,1}),E)$,
	then $r_\pi\vert_{D_w}$ 
	admits an exact sequence as above.
	\end{enumerate}
\end{lem}
\begin{proof}
This is a restatement of \cite[Cor 2.33]{ger}
for $w'\neq w_0$.
Note that since $\fp\subset \TT_{\wt{k}}^P(U^p\Iw^P(p^{0,1}),E)$,
$\lambda_\fp(\langle u\rangle)=(w_0\wt{k})^{-1}(u)$ for 
$u\in T_n(\oo_{w'})$.
For $w=w_0$, the crystalline property
follows from the proposition above since $\pi_w$ is unramified 
by definition for any $\pi$ belonging to $\fp$.
The rest of the assertion then follows from 
\cite[Lem 2.31]{ger} and \cite[Cor 2.33]{ger}.
\end{proof}


Since the Hecke algebra 
$\TT^P_{\wt{k}}(U^p\Iw^P(p^{b,b}),\oo)$
is reduced the homomorphisms
\[
	\prod_{\fp}\lambda_{\fp}\colon 
	\TT^P_{\wt{k}}(U^p\Iw^P(p^{b,b}),\oo)\to 
	\TT^P_{\wt{k}}(U^p\Iw^P(p^{b,b}),E)\to  
	\prod_{\fp}E_{\fp}
\]
where $\fp$ goes through all height-one primes,
is injective and the image is
a closed subring of $\prod_{\fp}E_{\fp}$.
By the relation \eqref{eq:Gal_hecke_away_p}
and Chebotarev's density theorem,
the traces  $\mtr(r_{\fp})$
lift to pseudo-representations
\begin{equation}\label{eq:pseudo_rep_finite}
	T_{\wt{k}}(U^p\Iw^P(p^{b,b}))\colon \Gal_{\K}
	\to \TT^P_{\wt{k}}(U^p\Iw^P(p^{b,b}),\oo)\quad
	\Fr_w\mapsto T_w^{(1)}
\end{equation}
that are compatible among $\Iw^P(p^{b,b})$.
Taking the inverse limit,
we let 
$T_{\wt{k}}(U^p)\colon \Gal_{\K} \to \TT^P_{\wt{k}}(U^p,\oo)$
denote the big $P$-ordinary Galois pseudo-representation.
Similarly, we write 
$T_{\wt{k}}^{\ord}(U^p)\colon \Gal_{\K} \to \TT^{\ord}_{\wt{k}}(U^p,\oo)$
when we apply the same constructions to ordinary forms.

\begin{defn}\label{def:big_Gal}
Let $T(U^p)=T_{\wt{k}}(U^p)$ and 
$T^{\ord}(U^p)=T_{\wt{k}}^{\ord}(U^p)$
when $\wt{k}$ is the trivial weight.
Since all the pseudo-representations satisfy 
\eqref{eq:pseudo_rep_finite},
by Chebotarev's density theorem 
the pseudo-representations are comptatible among
the commutative diagram
\[
\begin{tikzcd}
    \TT(U^p,\oo) 
    \arrow[r] \arrow[d]&
    \TT_{\wt{k}}(U^p,\oo) 
    \arrow[d]\\
    \TT^{\ord}(U^p,\oo) 
    \arrow[r] &
    \TT^{\ord}_{\wt{k}}(U^p,\oo) 
\end{tikzcd}
\]
where the horizontal maps come from Proposition \ref{prop:wt_indep}
and the vertical maps come from Lemma \ref{lem:coh_to_ord}.
\end{defn}


\section{Results on p-adic local Langlands}

In this section $G$ still denote the definite unitary group
over $\F$ as in \eqref{def:def_unitary}, but with $n=2$.
And $\oo$ is the ring of integers of a finite extension
$E$ over $\Qp$ with residue field $\fF$.
We continue working with an open compact subgroup
$U^p\subset G(\A_f^p)$ satisfying 
\eqref{cond:small} and $\eqref{cond:s-ram}$,
for a fixed ideal $\fs=\bar{\fs}$ and a decomposition $\fs=\ff\bar{\ff}$.
Follow the notations from last section,
we let $P\subset G_p$ be a parabolic subgroup
as in \eqref{cond:parabolic}.
We further assume that the special place $w_0\in \Sigma_p$
in the condition is of degree one.
We thus identify $K_{w_0}=\Qp=K_{\bar{w}_0}$
and $\iota_{w_0}\colon G(\F_{v_0})\cong \GL_2(\Qp)=P_{w_0}$
on the automorphic side,
if $v_0\in S_p$ is the prime of $\F$ below $w_0$.
And we identify the decomposition group
$D_{w_0}$ with $\Gp$, the absolute Galois group of  $\Qp$,
on the Galois side.

Since we will only work with the above parabolic subgroup $P$,
to simplify notations we write
$M(U^p)=M_P(U^p)$ and 
$S(U^p)=S_P(U^p)$
for the objects \eqref{eq:completed_coh}
and $\TT(U^p,\oo)=\TT^P(U^p,\oo)$ for the big 
$P$-ordinary Hecke algebra in Definition \ref{def:big_hecke}.
Meanwhile, we still refer
$M^{\ord}(U^p), S^{\ord}(U^p)$, and $\TT^{\ord}(U^p,\oo)$
to the corresponding $B$-ordinary objects.
Let $T(U^p)\colon \Gal_\K\to \TT(U^p,\oo)$
be the big $P$-ordinary Galois pseudo-representation.


Fix a maximal ideal $\fm\subset \TT(U^p,\oo)$.
We note that the same argument
as in \cite[Prop 3.3.6]{pan} shows that
$\TT(U^p,\oo)\to \TT^P(U^p\Iw^P(p^{0,1}),\oo/\varpi)$ 
induces a bijection on the maximal ideals of the two rings.
Enlarging $E$ if necessary, we assume that
$\TT(U^p,\oo)/\fm$ coincides with the residue field $\fF$.
Let $\TT(U^p,\oo)_{\fm}$ denote the localization
of $T(U^p,\oo)$ and $T_{\fm}$ denote 
the localization of the pseudo-representation $T(U^p)$.
Throughout the section, we assume $T_{\fm}$ is 
residually reducible and locally generic at $w_0$,
in the sense that there exists characters
$\bar{\delta}_1, \bar{\delta}_2\colon \Gal_{\K}\to \fF^\times$
such that 
\begin{equation}\tag{red.gen}\label{cond:red_gen}
	T_\fm\equiv \bar{\delta}_1+\bar{\delta}_2
	\mod \fm,\quad
	\bar{\delta}_1\bar{\delta}_2^{-1} \vert_{D_{w_0}}
	\neq \id,\omega^{\pm}
\end{equation}
Here $\omega$ is the Teichmuler character of $\Qp^\times$,
identified with a character on $D_{w_0}\cong \Gp$
through the Artin reciprocity map 
$\Art\colon \Qp^\times\to \Gp^{\textnormal{ab}}$.

Let $Q$ be the Levi subgroup of the parabolic subgroup $P$,
then $Q$ is the product of  $\GL_2(\Qp)$ and a torus,
and the completed homology and cohomology
$M(U^p)$ and $S(U^p)$
are representations over $Q$.
In this section,
we follow \cite{urban}
and show that $M(U^p)_{\fm}$ belongs to a block
of $Q$-representations in the sense of \cite{pask}.
As a consequence,
we show that $\TT(U^p,\oo)_{\fm}$ is Noetherian
and relate the ``reducible'' part of 
$M(U^p)_{\fm}$ to $M^{\ord}(U^p)_{\fm}$,
here we view $M^{\ord}(U^p)$ as a $\TT(U^p,\oo)$ module
through the homomorphism
$\TT(U^p,\oo)\to \TT^{\ord}(U^p,\oo)$ 
induced by Lemma \ref{lem:coh_to_ord}.
At last,
we deduce a fundamental exact sequence that is critical
for our construction of the Euler systems in next section.

\subsection{Generically reducible deformation}

We first recall from \cite[\S B.1]{pask}
the structure of the universal deformation ring 
$R$ of a $2$-dimensional pseudo-representation 
$\chi_1+\chi_2$,
where $\chi_1,\chi_2\colon \Gp\to \fF^\times$ 
are continuous characters that satisfies the 
the following generic assumption
\begin{equation}\label{cond:generic}\tag{\text{gen}}
	\chi_1\chi_2^{-1}\neq \id,\omega^{\pm1}.
\end{equation}

By \textit{loc.cit}, the assumption
implies the existence of non-split extensions
of Galois representations
\begin{equation*}
    0\to \chi_1\to \rho_{12}\to \chi_2\to 0\quad
    0\to \chi_2\to \rho_{21}\to \chi_1\to 0
\end{equation*}
which are unique up to isomorphisms;
and that the universal deformation rings
$R_{ij}$ of the Galois representations $\rho_{ij}$
are formally smooth of relative dimension $5$ over $\oo$.


Let $\tilde{\rho}_{ij}\colon \Gp\to\GL_2(R_{ij})$
denote the universal deformations.
Under suitable choices of bases we may assume
their reductions modulo the maximal ideals give
$\rho_{12}=\smat{\chi_1&*\\&\chi_2}$ and
$\rho_{21}=\smat{\chi_1&\\*&\chi_2}$.
The trace then induces 
$\theta_{ij}\colon R\cong R_{ij}$
by \cite[Prop B.17]{pask}.
Let $R_{\red}$ be the quotient of $R$
that represents reducible deformations.
Then $R_{\red}$ is isomorphic
to the complete tensor of the
deformation rings for the characters $\chi_1$ and $\chi_2$.
Since the pro-$p$ completion of $\Qp^\times$
has rank 2 over $\Zp$,
we see that $R_{\red}$ is formally smooth of
relative dimension $4$ over $\oo$.
Consequently the reducibility ideal 
$\tau\coloneqq \ker(R\to R_{\red})$ 
is a principal ideal generated by 
an element $\xx\in\fm_R\setminus \fm_R^2$,
where $\fm_R\subset R$ is the maximal ideal.
By \cite[Prop B.23]{pask},
we have $\theta_{ij}(\tau)=\tau_{ij}$
for the ideal $\tau_{ij}\subset R_{ij}$
generated by the $(j,i)$-entry of $ \tilde{\rho}_{ij}(g)$
for all $g\in \Gp$.

Write $x=\xx$ and $\theta=\theta_{12}$.
By \cite[Prop B.24]{pask},
the representation 
$\tilde{\rho}_{12}^x\colon \Gp\to \GL_2(R_{12})$
defined by
\begin{equation*}
	\tilde{\rho}_{12}^x(g)\coloneqq 
	\smat{\theta(x)&\\&1}
	\tilde{\rho}_{12}(g)
	\smat{\theta(x)&\\&1}^{-1}
\end{equation*}
is a deformation of $\rho_{21}$ to $R_{12}$
and the induced map
$\alpha\colon R_{21}\to R_{12}$ is an isomorphism.
Moreover, the following diagram is commutative.
From now on we identify 
$R_{21}$ with $R_{12}$ and
$\tilde{\rho}_{21}$ with 
$\tilde{\rho}_{12}^x$.
\begin{equation*}
	\begin{tikzcd}
		R_{21} \arrow[r,"\alpha"] &
		R_{12}\\
		R\arrow[u,"\theta_{21}"] \arrow[r,equal] &
		R\arrow[u,"\theta_{12}",swap]
	\end{tikzcd}
\end{equation*}

\begin{lem}\cite[Prop B.26]{pask}
\label{lem:B26}
The modules
$\Hom_{\Gp}(\tilde{\rho}_{12}, \tilde{\rho}_{21})$ and
$\Hom_{\Gp}(\tilde{\rho}_{21}, \tilde{\rho}_{12})$
are free of rank one over $R\cong R_{12}$ and
generated respectively by
\begin{equation}\label{eq:Phi_ij}
	\Phi_{12}=\smat{\theta(x)&\\&1} \text{ and }
	\Phi_{21}=\smat{1&\\&\theta(x)}.
\end{equation}
Moreover, the algebra
$\End_{\Gp}(\tilde{\rho}_{12}\oplus \tilde{\rho}_{21})$
is isomorphic to the generalized matrix algebra
$\smat{R& R\Phi_{12}\\ R\Phi_{21}& R}$.
In particular it is free of rank $4$ over $R$ and
its center is isomorphic to $R$.
\end{lem}

\subsection{Generically reducible block}

In the next two subsections
we recall results regarding
$p$-adic local Langlands for $\GL_2(\Qp)$
from \cite{pask}.
To better align with the notation in the literature,
we temporarily let $G$, $K$ and $B=TN$ respectively
denote $\GL_2(\Qp)$, $\GL_2(\Zp)$
and the subgroup of upper triangular matrices in $\GL_2(\Qp)$
and its Levi decomposition.
We then write $\bar{B}$ for
the subgroup of lower triangular matrices in $\GL_2(\Qp)$
and identify the center $Z\subset T$ with $\Qp^\times$.

Let $\chi_1,\chi_2$ be characters satisfying
\eqref{cond:generic} and fix a continuous character 
$\zeta\colon \Gp\to \oo^\times$
such that $\epsilon\zeta\equiv \chi_1\chi_2$ 
modulo $\varpi$.
For the deformation rings
$R, R_{\red}, R_{12}$ and $R_{21}$
introduced earlier, we let 
$R^{\zeta\epsilon}, R^{\zeta\epsilon}_{\red}, 
R^{\zeta\epsilon}_{12}$ and $R^{\zeta\epsilon}_{21}$
denote the quotients that represent
deformations with the
fixed determinant $\epsilon\zeta$.
All the results in the last subsection
still hold true for these deformation rings,
except that the dimensions are decreased by $2$.

Recall that 
$\Art\colon \Qp^\times\to \Gp^{\textnormal{ab}}$
denotes the reciprocity map
so that $\Fr\coloneqq \Art(p)$
is a geometric Frobenius.
We identify characters of $\Gp$ 
with that of $\Qp^\times$ through $\Art$.
For any character $\chi$ of 
of $T\cong \Qp^\times\times\Qp^\times$,
let $\chi^s$
denote the character given by 
$\chi^s(a,b)=\chi(b,a)$.
Define 
$\chi=\chi_1\otimes\chi_2\omega^{-1}$ and
$\chi^s\alpha=\chi_2\otimes \chi_1\omega^{-1}$,
where  $\alpha=\omega\otimes\omega^{-1}$.
By \cite[Thm 30]{barthel},
the inductions
\[
\pi_1\coloneqq \Ind_{B}^G\chi\cong
\Ind_{B}^G\chi_1\otimes\chi_2\omega^{-1}\quad
\pi_2\coloneqq \Ind_{B}^G\chi^s\alpha\cong 
\Ind_{B}^G\chi_2\otimes\chi_1\omega^{-1} 
\]
are irreducible representations
in $\Mod^{\sm}_{G,\zeta}(\oo)$.

\begin{defn}\label{def:block}
Let $\Mod^{\lfin}_{G,\zeta}(\oo), \Mod^{\lfin}_{T,\zeta}(\oo)$
be the subcategories
of representations that are 
locally of finite length 
and with central character $\zeta$;
and let $\fC_G(\oo), \fC_T(\oo)$
be the dual categories
of the Pontryagin duals.
Then $\B\coloneqq\{\pi_1,\pi_2\}$,
for $\pi_1,\pi_2$ as above, is a block 
in the sense of \cite[\S 5]{pask}.
We then let $\Mod^{\lfin}_{G,\zeta}(\oo)^\B$
denote the subcategory
of representations whose subquotients
all belong to $\B$;
and $\fC_G(\oo)^\B$
be the dual category of Pontryagin duals.
\end{defn}

Let $\Ord\colon \Mod_{G,\zeta}^{\sm}(\oo)
\to \Mod_{T,\zeta}^{\sm}(\oo)$
denote Emerton's functor of $B$-ordinary parts.
When $V\in \Mod^{\sm}_G(\oo)$ and
$U\in \Mod^{\sm}_T(\oo)$,
the adjunction formula
from \cite[Thm 4.4.6]{emeI} states that
\begin{equation}\label{eq:adjunct}
	\Hom_{\oo[G]}(\Ind_{\bar{B}}^GU,V)
	\xrightarrow{\Ord}
	\Hom_{\oo[T]}(\Ord(\Ind_{\bar{B}}^GU),\Ord V)
	\cong
	\Hom_{\oo[T]}(U,\Ord V)
\end{equation}
is an isomorphism, where the last isomorphism
is induced by $\Ord(\Ind_{\bar{B}}^GU)\cong U$
from  \cite[Prop 4.3.4]{emeI}.

By \cite[Prop 7.1]{pask},
if $\iota\colon \pi_1\hookrightarrow \tilde{J}_1$
is the injective envelope of $\pi_1$
in $\Mod^{\lfin}_{G,\zeta}(\oo)$,
then we have $\Ord\pi_1=\Ord(\Ind_B^G\chi)=\chi^s$
and $\Ord(\iota)\colon \chi^s \to \Ord(\tilde{J}_1)$
is isomorphic to an injective envelope
$\tilde{J}_{\chi^s}$ of $\chi^s$
in $\Mod^{\lfin}_{T,\zeta}(\oo)$.
Furthermore, the morphism 
$\iota_1\colon \Ind_{\bar{B}}^G(\tilde{J}_{\chi^s})\to
\tilde{J}_1$
induced through the adjunction formula \eqref{eq:adjunct}
by a fixed isomorphism
$\tilde{J}_{\chi^s}\to \Ord_B(\tilde{J}_1)$
is injective.
To simplify notations,
we identify $\Mod^{\lfin}_{T,\zeta}(\oo)$
with $\Mod^{\lfin}_{\Qp^\times}(\oo)$ through 
the map $\Qp^\times\cong \{\smat{1&\\&*}\}\subset T$
and write $\tilde{J}_{\chi_1}=\tilde{J}_{\chi^s}$.
Then for $ \tilde{J}_{\chi_2}$
and $ \tilde{J}_2$ defined as above
we also have the injective morphisms
$\iota_2\colon \Ind_{\bar{B}}^G(\tilde{J}_{\chi_2})\to
\tilde{J}_2$.


\begin{lem}\cite[Cor 7.7]{pask}
\label{lem:proj_enve}
Let $\tilde{P}_{\chi_i^\vee}\coloneqq \tilde{J}_{\chi_i}^\vee
\in\fC_T(\oo)$ and
$\tilde{M}_i\coloneqq 
\Ind_{\bar{B}}^G(\tilde{J}_{\chi_i})^\vee,
\tilde{P}_i\coloneqq \tilde{J}_i^\vee\in\fC_G(\oo)$
for $i=1,2$.
The morphisms
$p_i\colon \tilde{P}_i\twoheadrightarrow \tilde{M}_i$
that are dual to
$\iota_i\colon 
\Ind_{\bar{B}}^G(\tilde{J}_{\chi_i})\hookrightarrow 
\tilde{J}_i$ 
extend to the exact sequences
\begin{equation}\label{eq:exact_PPM}
	0\to \tilde{P}_{2}\xrightarrow{\phi_{12}} 
	\tilde{P}_{1}\xrightarrow{p_1} \tilde{M}_1\to 0 
	\text{ and }
	0\to \tilde{P}_{1}\xrightarrow{\phi_{21}} 
	\tilde{P}_{2}\xrightarrow{p_2} \tilde{M}_2\to 0
\end{equation}
\end{lem}

Let $\Rep_{\Gp}(\oo)$
be the category of compact $\oo$-modules with
continuous actions of $\Gp$,
and $\V\colon \fC_G(\oo)\to \Rep_{\Gp}(\oo)$
be the Colmez functor introduced 
in \cite[\S 5.7]{pask},
which is exact and covariant.
By \cite[Cor 8.7]{pask},
for $(i,j)=(1,2)$ or  $(2,1)$,
there exists unique non-split extensions
in $\Mod^{\sm}_{G,\zeta}(\oo)$ 
\[
	0\to \pi_2\to \kappa_{12}\to \pi_1\to 0,\quad
	0\to \pi_1\to \kappa_{21}\to \pi_2\to 0
\]
such that
$\V(\pi_i^\vee)=\chi_i$, $\V(\kappa_{ij}^\vee)=\rho_{ij}$,
and $\V(\tilde{P}_j)=\tilde{\rho}^{\zeta\epsilon}_{ij}$
are the universal deformations
with determinant $\zeta\varepsilon$.

It then follows from \cite[Lem 8.10]{pask} that 
taking the Colmez functor 
$\V$ induces the isomorphisms below.
\begin{equation}\label{eq:end_deform}
\begin{split}
	\End_{\fC_{G}(\oo)}(\tilde{P_2})\cong 
    R^{\zeta\epsilon}_{12}\cong R^{\zeta\epsilon},\quad
	\Hom_{\fC_G(\oo)}(\tilde{P}_2, \tilde{P}_1)\cong
    R^{\zeta\epsilon}\Phi_{12}\\
	\Hom_{\fC_G(\oo)}(\tilde{P}_1, \tilde{P}_2)\cong
    R^{\zeta\epsilon}\Phi_{21},\quad
	\End_{\fC_{G}(\oo)}(\tilde{P_1})\cong 
    R^{\zeta\epsilon}_{21}\cong R^{\zeta\epsilon}
\end{split}
\end{equation}
Write $ \tilde{P}_\B=\tilde{P}_1\oplus \tilde{P}_2$,
then $\tilde{E}_\B\coloneqq
\End_{\fC_G(\oo)}(\tilde{P}_\B)$
is isomorphic to 
$\End_{\Gp}(\tilde{\rho}^{\zeta\epsilon}_{12}\oplus
\tilde{\rho}^{\zeta\epsilon}_{21})$.



By \cite[Prop 7.1]{pask}
any morphism 
$\End_{\fC_G(\oo)}(\tilde{P}_i, \tilde{M}_i)$
factors through
$\End_{\fC_G(\oo)}(\tilde{M}_i)$,
which is isomorphic to 
$ \End_{\fC_T(\oo)}(\tilde{P}_{\chi_i^\vee})$
by the adjunction formula \eqref{eq:adjunct}.
And by \cite[Prop 3.34]{pask}
the last endomorphism ring is isomorphic to 
a formal power series ring
$ \oo\llbracket x,y\rrbracket$
of two variables.

\begin{lem}\label{lem:ker_red}
	The kernel of the homomorphism
    $p_i\colon R^{\zeta\epsilon}\cong
    \End_{\fC_G(\oo)}(\tilde{P}_i)\twoheadrightarrow
	\End_{\fC_G(\oo)}(\tilde{P}_i, \tilde{M}_i)$
	is the reducibility ideal $\tau\subset R^{\zeta\epsilon}$.
    In other word the functor $\Ord$ induces
	\begin{equation}
	R^{\zeta\epsilon}_\red\cong 
	\Hom_{\fC_G(\oo)}(\tilde{P}_i, \tilde{M}_i)\cong
	\End_{\fC_G(\oo)}(\tilde{P}_{\chi_i^\vee})\cong
	\oo\llbracket x,y\rrbracket
	\end{equation}
\end{lem}
\begin{proof}
It suffices to show that 
the image of $\tau$ consists of 
$\phi\in \End_{\fC_G(\oo)}(\tilde{P}_i)$
such that $p_i\circ \phi$ is trivial.
Apply $\Hom(\tilde{P}_j,*)$
to the following exact sequences
from \eqref{eq:exact_PPM}.
\[
\begin{tikzcd}
	0 \arrow[r]&
	\tilde{P}_1  \arrow[r,"\phi_{21}"]  &
	\tilde{P}_2 \arrow[r,"p_2"] \arrow[dr,equal] &
	\tilde{M}_2  \arrow[r] & 0 \\
	0 & 
	\tilde{M}_1 \arrow[l]&
	\tilde{P}_1 \arrow[l,"p_1",swap]  &
	\tilde{P}_2  \arrow[l,"\phi_{12}",swap]  & 
	0  \arrow[l] 
\end{tikzcd}
\]
The adjunction formula implies that
$\Hom_{\fC_G(\oo)}(\tilde{P}_j,\tilde{M}_i)\cong
\Hom_{\fC_T(\oo)}
(\tilde{P}_{\chi_j^\vee},\tilde{P}_{\chi_i^\vee})$,
which are zeros for $i\neq j$
since blocks in $\fC_T(\oo)$ are singletons
by the discussion in \cite[\S 7.2]{pask}.
Consequently there are isomorphisms
\[
	\phi_{12}\circ\colon
	\End_{\fC_G(\oo)}(\tilde{P}_2)\cong
	\Hom_{\fC_G(\oo)}(\tilde{P}_2, \tilde{P}_1)\quad
	\phi_{21}\circ\colon
	\End_{\fC_G(\oo)}(\tilde{P}_1)\cong
	\Hom_{\fC_G(\oo)}(\tilde{P}_1, \tilde{P}_2)
\]
Combined with the isomorphisms in \eqref{eq:end_deform},
we may assume that $\V(\phi_{ij})$ agree with 
$\Phi_{ij}$ in \eqref{eq:Phi_ij}.
Thus the image of 
$\phi_{ij}\circ\phi_{ji}\in \End_{\fC_G(\oo)}(\tilde{P}_i)$,
which belongs to the kernel in question,
corresponds to the generator 
$\xx=\Phi_{ij}\circ\Phi_{ji}$ of 
$\tau\subset R^{\zeta\epsilon}$.
Now the lemma follows 
since $R^{\zeta\epsilon}$ is formally smooth of relative dimension  $3$.
\end{proof}

\subsection{Univeral unitary completion}

Let $\Ban(E)$
denote the category of admissible $E$-Banach space
representations of $G$ with central character $\zeta$
as defined in \cite{pask}.
Identify the center of $\fC_G(\oo)^\B$ with $R^{\zeta\epsilon}$,
then $R^{\zeta\epsilon}[\frac{1}{p}]$ acts on objects of $\Ban(E)^{\B}$.
If $\fn\subset R^{\zeta\epsilon}[\frac{1}{p}]$ 
is a maximal ideal,
let $\Irr(\fn)$ denote the set of
irreducible representations in  $\Ban(E)^{\B}$
on which the action of $R^{\zeta\epsilon}[\frac{1}{p}]$ 
factors through $\fn$.

On the other hand,
let $T\colon \Gp\to R^{\zeta\epsilon}$ 
and $T_\fn\colon \Gp\to R^{\zeta\epsilon}[\frac{1}{p}]/\fn$ 
denote the universal deformation and 
the reduction.
Enlarge $E$ if necessary,
we assume $R^{\zeta\epsilon}[\frac{1}{p}]/\fn\cong E$.
Let $r_{\fn}\colon \Gp\to \GL_2(E)$
be the semisimple Galoi representation
such that $\mtr r_{\fn}=T_{\fn}$.
When $r_\fn$ is potentially semi-stable,
we let $\Delta_{\fn}$ denote
the associated Weil-Deligne representation 
on $D_{\pst}(r_\fn)$
and $\pi_{\fn}$ denote the smooth irreducible
$\GL_2(\Qp)$-representation attached to $\Delta_{\fn}$
by local Langlands correspondence.
We normalize the correspondence so that 
when $r_{\fn}$ is crystalline, 
thus $\Delta_{\fn}$
is the sum of unramified characters $\mu_1$ and $\mu_2$,
the representation $\pi_\fn$ 
is the smooth un-normalized induction
$\Ind_B^G(\mu_1\otimes\mu_2|\cdot|^{-1})_{\sm}$.
(this is called the Hecke correspondence in \cite{pan}).

\begin{rem}
	The representation 
	$\Ind_B^G(\mu_1\otimes\mu_2|\cdot|^{-1})_{\sm}$
	is irreducible.
	For otherwise
	$T_\fn$ would be of the form
	$\eta+\eta\epsilon$ for some character  $\eta$,
    which contradicts 
    the assumption \eqref{cond:generic}.
\end{rem}

\begin{lem}\label{lem:uni_completion}
	When $r_\fn$ is crystalline of
	Hodge-Tate weights $\{-l,-l-k\}$
    after twisting by a finite character
    and $\pi_{-l,1-l-k}^*\cong 
    \Sym^{k-1}\otimes\det^l$ is the contragredient
    representation of the algebraic $\GL_2$-representation
    of highest weight $(-l,1-l-k)$,
	the universal unitary completion of 
	$\pi_\fn\otimes \pi_{-l,1-l-k}^*(E)$ belongs to 
	$\Irr(\fn)$.
\end{lem}
\begin{proof}
    Under our assumptions,
    the representation $r_{\fn}$
    is potentially semi-stable and
    the representation $\pi_{\fn}$
    is a smooth irreducible principal series.
    When $r_\fn$ is irreducible, $\Irr(\fn)$
    has only one object by \cite[Cor 8.14]{pask},
    which we denote by $\Pi_{\fn}$.
    On the other hand,
    the universal unitary completion $\Pi$ in question
    is a non-ordinary admissible absolutely 
    irreducible $E$-Banach space representation
    by \cite[Thm 12.3]{pask},
    and the $(\varphi,N)$-module
    associated to which
    coincides with $D_{\pst}(r_\fn)$
    by \cite[Thm. 1.3]{CDP}.
    Since the same is true for $\Pi_{\fn}$
    by definition,
    it follows from the same theorem
    that $\Pi=\Pi_{\fn}\in \Irr(\fn)$.

	When  $r_\fn$ is reducible
	and $T_\fn=\psi_1+\psi_2$
	for characters
	$\psi_i\colon \Gp\to E^\times$,
	say of Hodge-Tate weights
	$1-l-k$ and  $-l$ respectively,
	then $\pi=\Ind_B^G(\mu_1\otimes\mu_2|\cdot|^{-1})$
	for the charaters
	\[
	\mu_1=\psi_1(\epsilon|\cdot|^{-1})^{-l-k},\quad
	\mu_2=\psi_2(\epsilon|\cdot|^{-1})^{-l}	
	\]
	By \cite[Thm 12.3]{pask} the universal unitary completion
	of $\pi\otimes W_{l,k}$
	is then $\Ind_B^G(\psi)_{cont}$ for
	\[
		\psi( (\begin{smallmatrix}
			a&b\\&d
		\end{smallmatrix}))
		=\mu_2(a)a^l\mu_1(d)|d|^{-1}d^{l+k-1}
		=\psi_2(a)\psi_1\epsilon^{-1}(d)
	\]
	since $\val_p(\mu_2(p))=-l$ and $\varepsilon(a)=a|a|$.
	And indeed by \cite[Cor 8.15]{pask} we have
	\[
	\Irr(\fn)=\{(\Ind_B^G\psi_1\otimes\psi_2\varepsilon^{-1})_{cont},
	(\Ind_B^G\psi_2\otimes\psi_1\varepsilon^{-1})_{cont}\}.
	\]
\end{proof}

\subsection{Local-global compatibility}
\label{sub:compatible}

We now resume the notation and settings in
the beginning of the section.
Recall that
$M(U^p)=M_P(U^p)$ is the Pontryagin dual of
$\Ord_P(S(U^p,E/\oo))\in \Mod_Q^{\adm}(\oo)$,
where $Q$ is the product of $\GL_2(\Qp)$ and a split torus.
Let $\fm\subset \TT(U^p,\oo)$ and
$\bar{\delta}_1,\bar{\delta}_2\colon \Gal_\K\to \fF^\times$
be as in \eqref{cond:red_gen}.
The characters
$\chi_i\coloneqq\omega{\bar{\delta}_i}\vert_{D_{w_0}}$
then satisfies \eqref{cond:generic} and
\[
	\epsilon T_\fm\vert_{D_{w_0}}\equiv
	\chi_1+\chi_2 \mod \fm
\]
We fix a crystalline character 
$\zeta\colon \Gp\to \oo^\times$ 
such that $\epsilon\zeta\equiv \chi_1\chi_2$
as in the previous subsections
and define the block $\B=\{\pi_1,\pi_2\}$
of irreducible representations
in $\Mod^{\lfin}_{\GL_2(\Qp),\zeta}(\oo)$
as in Definition \ref{def:block}.

Let $T\colon \Gp\to R$ be 
the universal pseudo-deformation of $\chi_1+\chi_2$.
Since $\det T\equiv \zeta\epsilon$ modulo
the maximal ideal $\fm_R$ and
the residual characteristic $p$ is odd,
there exists a square-root character
\begin{equation}\label{eq:root_char}
	\xi^{1/2}
    \colon \Gp\to 1+
    \fm_{R},
    \, \text{ such that }
	\xi\coloneqq(\xi^{1/2})^2=\epsilon\zeta(\det T)^{-1}
    \colon \Gp\to 1+
    \fm_{R}.
\end{equation}
Note that $T'\coloneqq \xi^{1/2}T$ then has determinant 
$\zeta\epsilon$
and induces a homomorphism $R^{\epsilon\zeta}\to R$.



Recall that the homomorphisms
$\TT(U^p,\oo)\to \TT^P(U^p\Iw^P(p^{b,b}),\oo)$
induce bijections on maximal ideals
for each $b\geq 1$ by \cite[Prop 3.3.6]{pan}.
By abuse of notation
let $T(U^p\Iw^P(p^{b,b}))_\fm$
denote the localizations of the pseudo-representations
\eqref{eq:pseudo_rep_finite} at 
the maximal ideals corresponding to $\fm$.
By the universal property,
the pseudo-representations
$\epsilon\cdot T(U^p\Iw^P(p^{b,b}))_\fm$
induces homomorphisms of $R$ 
to the Noetherian local $\oo$-algebras
$\TT^P(U^p,\Iw^P(p^{b,b}),\oo)_{\fm}$,
which are compatible for $b\geq 1$.
The homomorphism 
induced from the inverse limit
\[
    R\to \TT(U^p,\oo)_{\fm}=
    \varprojlim_b\TT^P(U^p,\Iw^P(p^{b,b}),\oo)_{\fm}
\]
then sends the pseudo-representation
$T$ to $\epsilon T_\fm\vert_{D_{w_0}}$.
Note that the homomorphism is defined this way
since we do not know a priori whether
$\TT(U^p,\oo)_{\fm}$ is Noetherian.

Let $\xi_\fm^{1/2}$ denote the composition 
of $\xi^{1/2}$ with the homomorphism above, then
\begin{equation}\label{eq:root_charm}
	(\xi_\fm^{1/2})^2=
	\zeta\epsilon(\det \epsilon T_\fm\vert_{D_{w_0}})^{-1}
	=\zeta(\epsilon\det T_\fm\vert_{D_{w_0}})^{-1}\colon
	\Gp=D_{w_0}\to 1+\fm\TT(U^p,\oo)_\fm\subset \TT(U^p,\oo)_\fm
\end{equation}
\begin{lem}\label{lem:twist}
Identify 
$\K_{w_0}$ with $\Qp$,
$D_{w_0}$ with $\Gp$,
$\zeta$ with the character $\zeta\circ \Art$ on $\Qp^\times$,
and choose $\varpi_{w_0}=p$.
Then the action of the central torus
$\Qp^\times \subset \GL_2(\Qp)\subset Q$
on $\Ord_P(S(U^p,E/\oo))_{\fm}$
factors through $\TT(U^p,\oo)_{\fm}$ 
and is congruent to $\zeta$ modulo the maximal ideal
$\fm\TT(U^p,\oo)_{\fm}$.
\end{lem}
\begin{proof}

Since the central torus $\Qp^\times$
is a subset of $Z_Q^+$,
the action of which on
$\Ord_P(S(U^p,E/\oo))$ is given by $h_U$
and factors through $\TT(U^p,\oo)$ by Definition
\ref{def:ord_hecke}.
On the other hand,
Combine Lemma \ref{lem:galois_at_p}
with Corollary \ref{cor:density}, we see that 
the character 
$\epsilon\det T(U^p)\vert_{D_{w_0}}\circ \Art\colon
	\Qp^\times\to \TT(U^p,\oo)$ is given by
\[
	p\mapsto U_{w_0}^{(2)}=
    h_U\left(\smat{p&\\&p}\right),\quad
	x\mapsto \left\langle \smat{x&\\&x}\right\rangle=
    h_U\left(\smat{x&\\&x}\right),\,
	\text{ for }x\in\Zp^\times.
\]
After localizing everyting at $\fm$,
the assertion follows from 
the relation \eqref{eq:root_char}.
\end{proof}


To apply the results in the previous subsections,
we follow \cite{urban} and make the following twist.
\begin{defn}\label{def:twist}
View $\Ord_P(S(U^p,E/\oo))_{\fm}$
as a representation of
$Q'\coloneq \GL_2(\Qp)\times Z_Q$
through the natural surjective
homomorphism  $Q'\to Q$
whose kernel is isomorphic to the central torus
$\Qp^\times\subset \GL_2(\Qp)$.
We let $\Ord_P(S(U^p,E/\oo))_{\fm}'$
denote the twisted representation,
with the $\GL_2(\Qp)$-action is
twisted by the character
$(\xi_{\fm}^{1/2}\circ\Art)$.
from \eqref{eq:root_charm}.
The twisted $\GL_2(\Qp)$ has central character $\zeta$
by the lemma above.
\end{defn}
Since $\Ord_P(S(U^p,E/\oo))_{\fm}$
is an admissible $Q$-representation
and the character $\xi_{\fm}^{1/2}\circ \Art$
is continuous,
the twisted representation
$\Ord_P(S(U^p,E/\oo))_{\fm}'$ is also admissible
as $Q'$-representation.
In particular,
the restriction to $\GL_2(\Qp)$
is locally admissible, and thus by \cite[Thm 2.3.8]{emeI}
\[
    \Ord_P(S(U^p,E/\oo))_{\fm}'\in \Mod^{\lfin}_{\GL_2(\Qp),\zeta}(\oo).
\]



\begin{prop}\label{prop:compatibility}
    Let  $M(U^p)_{\fm}'$
    be the Pontryagin dual of $\Ord_P(S(U^p,E/\oo))_\fm'$
    and $\fC(\oo)$ be the dual category of
	$\Mod^{\lfin}_{\GL_2(\Qp),\zeta}(\oo)$
    as in Definition \ref{def:block}.
    Then $M(U^p)_{\fm}'\in \fC(\oo)^{\B}$
	and the $R^{\epsilon\zeta}$-action on which,
	as the center of the category,
	factors through $\TT(U^p,\oo)_\fm$.
\end{prop}

\begin{proof}

Recall that associated to each minimal prime
$\fp\subset \TT_{\wt{k}}^P(U^p\Iw^P(p^{0,1}),E)_{\fm}$
we have the Galois representation $r_\fp$
in Lemma \ref{lem:galois_at_p}
which is crystalline at $D_{w_0}\cong\Gp$.
Then there exists a unramified $\GL_2(\Qp)$-representation
$\pi_{\fp}$ that corresponds to $r_\fp\vert_{\Gp}$
under the Hecke correspondence introduced earlier.
And since $\pi_\fp\subset \varinjlim_b 
S^{P-\ord}_{\wt{k}}(U^p\Iw^P(p^{b,b}),E)$,
a slight modification of Proposition \ref{prop:wt_space}
gives homomorphisms
\begin{equation}\label{eq:hom_wt}
    \pi_\fp\otimes \pi_{\wt{k}}^*(E)\to S(U^p)_\fm\otimes_{\oo}E
\end{equation}
between $\GL_2(\Qp)$-representations.
At last we let $\fn\subset R^{\zeta\epsilon}[\frac{1}{p}]$
be the maximal ideal given by the pullback of $\fm\TT(U^p,\oo)_{\fm}$
under the homomorphism
$R^{\zeta\epsilon}\to R\to \TT(U^p,\oo)_{\fm}$,
where the first map is induced by $T'=\xi^{1/2}T$
and the second the the one introduced in the begining of
this subsection.

Combine the strategy of the proof of \cite[Thm 3.5.5]{pan}
with the density result in Proposition \ref{prop:density},
it suffices to show that
the universal unitary completions of
the twists of the $\GL_2(\Qp)$-representations
$\pi_{\fp}\otimes\pi_{\wt{k}}^*(E)$
belong to the set $\Irr(\fn)$
of irreducible $E$-Banach $\GL_2(\Qp)$-representations
associated to $\fn$.

Since $w_0\in \Sigma_p$ is of degree one
there is a unique $\sigma\in I_{w_0}$.
Write $k_1=k_{\sigma,1}$ and $k_2=k_{\sigma,2}$,
then $r_\fp\vert_{\Gp}$ is crystalline of Hodge-Tate
of Hodge-Tate weights  $k_1+1,k_2$
and $\epsilon\det r_{\fp}$
is of Hodge-Tate weight $k_1+k_2$.
Define $\xi_\fp^{1/2}=\lambda_\fp\circ \xi_\fm^{1/2}$
and $\xi_\fp=(\xi_\fp^{1/2})^2=\zeta(\epsilon\det r_\fp)^{-1}$.
Suppose $\zeta$ has Hodge-Tate weight $w_0$,
then the Hodge-Tate weight $w_0-(k_1+k_2)$ of
$\xi_\fp$ is even as $\xi_\fp$
is congruence to the trivial character.
Therefore the character $\xi_{\fp}^{1/2}$
is crystalline of Hodge-Tate weight 
$w_{\fp}\coloneqq \frac{1}{2}(w_0-k_1-k_2)$ 
up to a quadratic twist.

Therefore the representation
$r_{\fp}(\xi_{\fp}^{1/2}\epsilon)$
is crystalline up to a quadratic twist
and has Hodge-Tate weights  $\{k_1',k_2'-1\}$ 
for $k_1'=k_1+w_\fp, k_2'=k_2+w_\fp$.
And $\pi_\fp(\xi_\fp^{1/2}(\epsilon|\cdot|^{-1})^{w_\fp})$
is the smooth irreducible
representation associated to 
$D_{\pst}(r_{\fp}(\xi_\fp^{1/2}\epsilon))$
since $\pi_\fp$ is associated to 
$D_{\pst}(r_{\fp}(\epsilon))$.
Now twist \eqref{eq:hom_wt} by $\xi_\fm^{1/2}$ gives
\[
    (\pi_{\fp}\otimes \pi_{\wt{k}}^*(E))(\xi_\fp^{1/2})=
    \pi_\fp(\xi_\fp^{1/2}(\epsilon|\cdot|^{-1})^{w_\fp})
    \otimes \pi_{\wt{k}}^*(E)(\epsilon|\cdot|^{-1})^{-w_{\fp}} \to 
    S(U^p)_\fm'\otimes_{\oo}E
\]
Since  $T_\fn=\mtr(r_{\fn}(\xi_\fm^{1/2}\epsilon))$
and $\pi_{\wt{k}}^*(E)(\epsilon|\cdot|^{-1})^{-w_{\fp}}\cong 
\pi_{k_1',k_2'}^*(E)$,
the claim follows from Lemma \ref{lem:uni_completion}.
\end{proof}

By the proposition,
the twisted $\GL_2(\Qp)$-action 
on $\Ord_P(S(U^p,E/\oo))_\fm'$
has all the irreducible subquotients
isomorphic to either $\pi_1$ or $\pi_2$.
On the other hand,
the non-twisted $Z_Q$-action on which
is also locally admissible and
factors through the character
$Z_Q\to \TT(U^p,\oo)_{\fm}^{\times}$
that extends $h_U(x)$ for $x\in Z_Q^+$.
Let $\upsilon\colon Z_Q\to \TT^P(U^p,\oo)/\fm=\fF^\times$
denote the residual character.
Then the $Z_Q$-action 
on $\Ord_P(S(U^p,E/\oo))_{\fm}'$
has all the irreducible subquotients
isomorphic to $\upsilon$.





By abuse of notation
we let $\pi_1,\pi_2$ denote
the $Q'=\GL_2(\Qp)\times Z_Q$-representation
on which $Z_Q$ acts by $\upsilon$.
Combine the above discussion with the 
proposition,
we see that the irreducible subquotients
of the $Q'$-representation
$\Ord_P(S(U^p,E/\oo))_{\fm}'$
are all isomorphic to $\pi_1$ or $\pi_2$.


\begin{defn}

Let $T'=(\Qp^\times)^2\times Z_Q\subset Q'=\GL_2(\Qp)\times Z_Q$
and 
$\Mod^{\lfin}_{Q',\zeta}(\oo),\Mod^{\lfin}_{Q',\zeta}(\oo)$
be the cateogries of smooth representations of 
$Q'$ and $T'$ that are locally of finite type
and with central character $\zeta$
on $\GL_2(\Qp)$ or on $(\Qp^\times)^2$;
and let $\fC_{Q'}(\oo), \fC_{T'}(\oo)$
be the dual category of Pontryagin duals.
We write $\Hom_{Q'}$ for the space of morphisms
in either of the categories 
$\Mod^{\lfin}_{Q',\zeta}(\oo)$ or $\fC_{Q'}(\oo)$
when the context is clear.
We will similarly use $\Hom_{T'}$
when the context is clear.
\end{defn}

\begin{lem}\label{lem:end_comp}
Let  $\tilde{P}=\tilde{P}_{\upsilon^\vee}$
be the projective envelope of $\upsilon^\vee$
in $\fC_{Z_Q}(\oo)$,
which is the dual of the category 
of smooth $Z_Q$-representations
that are locally of finite length, and let
$\tilde{E}\coloneqq 
\End_{\fC_{Z_Q}(\oo)}(\tilde{P})$.
Let $Z_Q^\wedge$ denote the pro-$p$ completion
of the group $Z_Q$,
then we can identify
\[
    \tilde{E}=\tilde{P}
    =\oo\llbracket Z_Q^\wedge\rrbracket
\]
where the $Z_Q$-action on $\tilde{P}\in \fC_{Z_Q}(\oo)$
is the tautological one up to
twisting by a lift of $\upsilon$.
\end{lem}
\begin{proof}
This same argument in \cite[Prop. 3.34]{pask} 
shows that $\tilde{P}$
is free of rank one over $\tilde{E}$
and we may assume $\upsilon$ is the trivial character.
The description of the $Z_Q$-action
then follows from 
\cite[Cor. 3.27]{pask} and the proof of
\cite[Lem. 3.32]{pask}.
\end{proof}


\begin{prop}\label{prop:envelope}
    Write $\tilde{P}_{*,\fm}=
    \tilde{P}_*\hat{\otimes}_{\oo}\tilde{P}$
    for $\tilde{P}_*\in\{\tilde{P}_\B, \tilde{P}_1, \tilde{P}_2,
    \tilde{P}_{\chi_1^\vee}, \tilde{P}_{\chi_2^\vee}\}$
    introduced in Lemma \ref{lem:proj_enve}
	and write 
    $R_{\fm}=R^{\epsilon\zeta}
	\hat{\otimes}_{\oo}\tilde{E},
    R_{\fm}^\red=R^{\epsilon\zeta}_\red
	\hat{\otimes}_{\oo}\tilde{E},$.
	Then $\tilde{P}_{i,\fm}$
	is the projective envelope
	of the $Q'$-representation $\pi_i\boxtimes\upsilon$;
	$\tilde{P}_{\chi_i^\vee,\fm}$
	is the projective envelope
	of the $T'$-representation $\chi_i^\vee\boxtimes\upsilon$,
    and
    \begin{align}\label{eq:proj_1}
    &\Hom_{Q'}(\tilde{P}_{i,\fm}, \tilde{P}_{j,\fm}) \cong 
    \Hom_{\GL_2(\Qp)}(P_{i}, \tilde{P}_{j})
    \hat{\otimes}_\oo \End_{\fC_{\Z_Q}(\oo)}(\tilde{P})\cong 
    R^{\zeta\epsilon} \hat{\otimes}_\oo \tilde{E}
    = R_\fm\\\label{eq:proj_2}
    &\End_{T'}(\tilde{P}_{\chi_i^\vee,\fm}) \cong 
    \End_{(\Qp^\times)^2}(\tilde{P}_{\chi_i^\vee})
    \hat{\otimes}_\oo \End_{\fC_{\Z_Q}(\oo)}(\tilde{P})\cong 
    R^{\zeta\epsilon}_\red \hat{\otimes}_\oo \tilde{E}
    =R_\fm^{\red}
    \end{align}
\end{prop}
\begin{proof}
    Let $v$ be an admissible vector
    in a $Q'$-representation.
	Then $v$ is $Z_Q$-finite by \cite[Lem 2.3.5]{emeI}
	and the $\oo[Q']-module$
	generated by $v$ is admissible and
	finitely-generated over $\GL_2(\Qp)$,
	hence of finite length by \cite[Thm 2.3.8]{emeI}.
	In other word,
	any locally admissible representation $V$
	is also locally of finite length.
    That $\tilde{P}_{*,\fm}$ are
    the corresponding projective envelops now follows from 
    \cite[Lem B.6]{GN}
	and the argument in \cite[Lem B.8]{GN}.
    And the isomorphisms 
    \eqref{eq:proj_1} and
    \eqref{eq:proj_2} follows respectively from  
    \eqref{eq:end_deform} and 
    Lemma \eqref{lem:ker_red}.
\end{proof}


be the subcategory of $\Mod^{\lfin}_{Q',\zeta}(\oo)$
consisting of $Q'$-representations
with all irreducible subquotients
belonging to
$\B=\{\pi_1\boxtimes\upsilon, \pi_2\boxtimes\upsilon\}$
Let $\Mod^{\lfin}_{Q',\zeta}(\oo)^\B$
and $\fC_{\Q'}^\B$ be the dual category.
Similar to \cite[Prop.5.45]{pask},
general formalism on abelian categories 
asserts that 
\begin{equation}\label{eq:anti_equiv}
        \fC_{Q'}(\oo)^{\B}\ni M  \mapsto 
        \mathbf{m}(M)\coloneqq 
        \Hom_{Q'}(\tilde{P}_{\B,\fm}, M)
\end{equation}
is an anti-equivalence to the category 
of compact $\tilde{E}_{\B,\fm}$-modules for
$\tilde{E}_{\B,\fm}
=\tilde{E}_\B\hat{\otimes}_{\oo}\tilde{E}\cong \End_{Q'}(\tilde{P}_{\fm,\B})$.

\begin{cor}\label{cor:Hecke_Noetherian}
    Let $M(U^p)_\fm'=(\Ord_P(S(U^p,E/\oo))_\fm')^\vee$.
	The big Hecke algebra
	$\TT(U^p,\oo)_{\fm}$ is Noetherian and
	$\mathbf{m}\coloneqq
	\Hom_{Q'}(\tilde{P}_{\fm},M(U^p)_{\fm}')$
	is a finite faithful $\TT(U^p,\oo)_{\fm}$-module
\end{cor}
\begin{proof}
    It follows from the discussion following 
    Proposition \ref{prop:compatibility}
    that $M(U^p)_\fm'\in \fC_{Q'}(\oo)^\B$.
    Therefore the Hecke algebra
    $\TT(U^p,\oo)_\fm$ also acts faithfully
    on the image $\mathbf{m}$
    of the anti-equivalence \eqref{eq:anti_equiv}.
	Moreover,  by \cite[Prop 4.17]{pask},
    $\mathbf{m}$ is finitely-generated
	over $\tilde{E}_{\B,\fm}$ since 
    $\Ord_P(S(U^p,E/\oo))'_{\fm}$ is $Q'$-admissible.
    
	Since $\tilde{E}_{\fm}$ is finite over
    its center $R_{\fm}$ by Lemma \ref{lem:B26}
    and the action of 
    $R_{\fm}\coloneqq R^{\zeta\epsilon}\hat{\otimes}_\oo
    \tilde{E}$ factors through 
    $\TT(U^p,\oo)_\fm$ by Proposition
    \ref{prop:compatibility} and the discussion that 
    follows,
	we conclude that $\mathbf{m}$ 
	is a finite faithful $\TT(U^p,\oo)_{\fm}$-module.
    Therefore the induced $R_\fm$-algebra  homomorphism 
	$\TT(U^p,\oo)_{\fm}\to 
	\End_{R_{\fm}}(\mathbf{m})$ is injective.
    As $\mathbf{m}$ is finite over $R_\fm$,
    so is $\End_{R_\fm}(\mathbf{m})$,
    and hence $\TT(U^p,\oo)_\fm$
	is finite over $R_{\fm}$.
    From this we conclude that 
    $\TT(U^p,\oo)_\fm$ is Noetherian.
\end{proof}


\subsection{Reducible part of completed homology}

We continue working with the notations introduced
in last subsection.
In particular, we identify the generator $\xx$ of
$\tau=\ker(R^{\epsilon\zeta}\to R^{\epsilon\zeta}_{\red})$
with a generator of 
$\ker(R_\fm\to R_\fm^{\red})$.
For any $R_{\fm}$-module $M$
we call $M^{\red}\coloneqq M/\xx M$
the reducible part of $M$.
Let
\[
    \Ord\colon 
    \Mod^{\sm}_{Q'}(\oo)\to \Mod^{\sm}_{T'}(\oo)
\]
denote Emerton's functor 
with respect to the upper triangular Borel subgroup 
in $\GL_2(\Qp)\subset Q'$.
Then by definition
$\Ord(\Ord_P(S(U^p,E/\oo))_\fm')=S^{\ord}(U^p,E/\oo)_\fm'$.
Our goal in this subsection is to 
relate $(M(U^p)_\fm')^{\red}$
with $S^{\ord}(U^p,E/\oo)_\fm'$
following \cite{urban}.

\begin{lem}
	For some non-negative integer $r$
	there exists a projective resolution 
	\begin{equation}\label{eq:resolution}
	0\to \tilde{P}_{\B,\fm}^{\oplus r}\to 
	\tilde{P}_{\B,\fm}^{\oplus r}\to 
	M(U^p)_{\fm}'\to 0
	\end{equation}
\end{lem}
\begin{proof}
Identify $\Qp^\times$ with the center of $\GL_2(\Qp)$.
By \cite[Thm 33]{barthel} and \cite[Thm 19]{barthel}, 
there exists
smooth irreducible 
$\GL_2(\Zp)\Qp^\times$-representations $\sigma_i$
and exact sequences
\begin{equation}
	0\to 
	\cInd_{\GL_2(\Zp)\Qp^\times}^{\GL_2(\Qp)}\sigma_i\to
	\cInd_{\GL_2(\Zp)\Qp^\times}^{\GL_2(\Qp)}\sigma_i\to
	\pi_i\to 0
\end{equation}
Apply $\boxtimes\upsilon$ to the sequence above
then apply $\Ext^i_{Q'}(*,S')$
for $S'=\Ord_P(S(U^p,E/\oo))_{\fm}'$.
We obtain
\begin{equation*}
    \begin{tikzcd}[row sep=2ex]
	    \Ext^{i-1}_{Q'}
	    (\pi, S')\arrow[r] &
	    \Ext^{i}_{Q'}
        (\cInd_{KZ\times Z_Q}^{Q'}\sigma, S')
	    \arrow[r] \arrow[d,equal] &
	    \Ext^{i}_{Q'}
        (\cInd_{KZ\times Z_Q}^{Q'}\sigma, S')
	    \arrow[r] \arrow[d,equal] &
	    \Ext^{i}_{Q'}(\pi, S')\\ 
	 & \Ext^i_{\GL_2(\Zp)\times Z_Q}(\sigma ,S') &
	    \Ext^i_{\GL_2(\Zp)\times Z_Q}(\sigma ,S') &
    \end{tikzcd}
\end{equation*}
for $K=\GL_2(\Zp)$ and $Z=\Qp^\times$.
By Lemma \ref{lem:inj}
the restriction of $\Ord_P(S(U^p,E/\oo))_\fm$
is an injective object 
in the category of smooth $\GL_2(\Zp)$-representations.
And by \eqref{eq:root_charm} and Lemma \ref{lem:twist},
the action of $\Zp^\times$ on 
$\Ord_P(S(U^p,E/\oo))_\fm$ via 
$(\xi_{\fm}^{1/2})^2\circ\Art$
coincide with that of 
\[
    \xi'\colon 
    \Zp^\times\to 1+p\Zp\xrightarrow{\zeta\langle*\rangle^{-1}}
    \oo\llbracket 1+p\Zp\rrbracket
\]
where $\langle*\rangle$ is the tautological character
and $\oo\llbracket 1+p\Zp\rrbracket$ 
has the natural action 
induced from the structure 
of smooth $\Qp^\times$-representation.
Thus we could extend the definition of the twist 
to all smooth $\GL_2(\Zp)$-representations
using the square-root character $(\xi')^{1/2}$.
It follows that 
the twist $S'=\Ord_P(S(U^p,E/\oo))_\fm'$
is still injective
as a smooth $\GL_2(\Zp)$-representation.
Therefore the long exact sequence reduces to 
the following sequence, where
all the terms are finite-dimensional over $\fF$
since $S'$ is $Q'$-admissible.
\begin{equation*}
	0 \to \Hom_{Q'}(\pi_i,S')\to 
	\Hom_{\GL_2(\Zp)\times Z_Q}(\sigma_i,S')\to 
	\Hom_{\GL_2(\Zp)\times Z_Q}(\sigma_i,S')\to 
	\Ext^1_{Q'}(\pi_i,S')\to 0
\end{equation*}



Write $a_i\coloneqq \dim_{\fF} \Hom_{Q'}(\pi_i,S')=
\dim_{\fF} \Ext^1_{Q'}(\pi_i,S')$
so that $\soc(S')=\pi_1^{a_1}\oplus \pi_2^{a_2}$.
Let $\tilde{J}_{i,\fm}$ denote 
the injective envelope of $\pi_i$.
Then the injective envelope 
$\soc(S')\hookrightarrow \tilde{J}=
\tilde{J}_{1,\fm}^{'a_1}\oplus \tilde{J}_{2,\fm}^{'a_2}$
factors through an injective morphism 
$\phi_0\colon S'\to \tilde{J}$
since the inclusion $\soc(S')\hookrightarrow S'$
is essential.
Apply the same construction 
to $\soc(\coker(\phi_0))$
and use $\dim_{\fF}\Hom_{Q'}(\pi_i, \coker(\phi_0))=
\dim_{\fF}\Ext^1_{Q'}(\pi_i, S')=a_i$
gives another injective morphism
$\phi_1\colon \coker(\phi_0)\to \tilde{J}$,
which is also surjective as 
$\Hom(\pi_i,\coker(\phi_1))
\cong \Ext^1(\phi_i,\coker(\phi_0))
\cong \Ext^2(\pi, S')=0$.
Now the lemma follows by picking
$r=\max\{a_1,a_2\}$ and taking the Pontryagin dual.
\end{proof}

Let 
$A\colon \tilde{P}_{\B,\fm}^{\oplus r}\to\tilde{P}_{\B,\fm}^{\oplus r}$ 
denote the morphism in \eqref{eq:resolution} and
identify $\Phi_{ij}=\Phi_{ij}\otimes \id_{\tilde{P}}\in
\Hom_{Q'}(\tilde{P}_{j,\fm},\tilde{P}_{i,\fm})$.
By \eqref{eq:proj_1}
the morphism $A$
can be represented by a matrix
$A=\smat{A_{11} & A_{12}\Phi_{12}\\A_{21}\Phi_{21} & A_{22}}$,
where $A_{ij}\in M_r(R_{\fm})$.
By Lemma \ref{lem:proj_enve}
and Lemma \ref{lem:ker_red}
$\Ord(\tilde{P}_{i,\fm}^\vee)^\vee\cong 
\tilde{P}_{\chi_i^\vee,\fm}$
and by \eqref{eq:proj_2} $\Ord(A)$
is represented by the matrix
$\smat{\tilde{A}_{11} & \\& \tilde{A}_{22}}$,
where $\bar{A}_{ij}\in M_r(R^{\red}_{\fm})$ are 
the reductions of the matrix.
Apply the right exact functor
$\Ord$ to 
\eqref{eq:resolution} then gives
\begin{equation}\label{eq:exact_ord}
	\tilde{P}_{\chi_1^\vee,\fm}^{\oplus r}\oplus 
	\tilde{P}_{\chi_2^\vee,\fm}^{\oplus r}
	\xrightarrow{\overline{A}_{11}\oplus\overline{A}_{22}}
	\tilde{P}_{\chi_1^\vee,\fm}^{\oplus r}\oplus 
	\tilde{P}_{\chi_2^\vee,\fm}^{\oplus r}
	\to (S^{\ord}(U^p,E/\oo)_\fm')^\vee\to 0
\end{equation}


\begin{prop}    
The sequence \eqref{eq:exact_ord}
is also left exact.
\end{prop}
\begin{proof}
	Let $\Lambda$ be the complete group algebra
    as in \eqref{def:lambda_rings}.
    Since $S^{\ord}(U^p,E/\oo)_\fm'$
    is admissible as a representation of 
    the diagonal torus $T\subset G_p$,
    the Pontryagin dual
    $M^{\ord}(U^p)_\fm'\coloneqq (S^{\ord}(U^p,E/\oo)_\fm')^\vee$
    is finitely-generated over $\Lambda$
	by \cite[Lem 2.2.11]{emeI}.
	Apply $\Hom_{T}(\tilde{P}_{\chi_i^\vee,\chi},*)$
	to \eqref{eq:exact_ord} gives
\begin{equation*}
	\End_{T'}(\tilde{P}_{\chi_i^\vee,\fm})^r\xrightarrow{\bar{A}_{ii}}
	\End_{T'}(\tilde{P}_{\chi_i^\vee,\fm})^r\to 
	\Hom_{T'}(\tilde{P}_{\chi_i^\vee,\fm}, M^{\ord}(U^p)_\fm')
	\to 0
\end{equation*}
	The two terms on the left 
	are finite free over $R_{\fm}^{\red}$
	while the last term is torsion over $R_{\fm}^\red$
    since the relative dimension of $\Lambda$ is
	strictly smaller than that of $R_{\fm}^{\red}$.
	Passing to the field of fraction of $R_{\fm}^{\red}$,
	we see that the matrix $\bar{A}_{ii}$ has to be invertible
	and the result follows.
\end{proof}

By \eqref{eq:exact_ord},
the action of 
$\Qp^\times=\{\smat{1&\\&*}\}\subset \GL_2(\Qp)$
on $S^{\ord}(U^p,E/\oo)_{\fm}'$
has all irreducible subquotients 
belonging to $\{\chi_1,\chi_2\}$.
Since the blocks of $\Qp^\times$-representaions
are singletons by \cite[Lem 3.34]{pask},
standard results as in \cite[Prop. 5.34]{pask}
shows that 
\begin{equation}\label{eq:sub12}
 S^{\ord}(U^p,E/\oo)_{\fm}'=
 S^{\ord}(U^p,E/\oo)_{\fm,1}'\oplus S^{\ord}(U^p,E/\oo)_{\fm,2}'
\end{equation}
where $S^{\ord}(U^p,E/\oo)_{\fm,i}'$ is
the part in which all irreducible subquotients
of the $\Qp^\times$-action are isomorphic to $\chi_i$.
Let $M^{\ord}(U^p)_{\fm,i}'=S^{\ord}(U^p,E/\oo)_{\fm,1}^\vee$.
The same argument in Lemma \ref{lem:end_comp}
shows that $\tilde{P}_{\chi_i^\vee,\fm}$
are free of rank one over 
over $\End_T(\tilde{P}_{\chi_i^\vee,\fm})\cong R_\fm^{\red}$.
Therefore \eqref{eq:exact_ord} implies that
\begin{equation}\label{eq:separate}
    \coker\bar{A}_{ii}\cong 
    \Hom_{T'}(\tilde{P}_{\chi_i^\vee,\fm}, M^{\ord}(U^p)_\fm')\cong
    \Hom_{T'}(\tilde{P}_{\chi_i^\vee,\fm}, M^{\ord}(U^p)_\fm')
    \hat{\otimes}_{R^{\red}_\fm}\tilde{P}_{\chi_i^\vee,\fm}\cong
    M^{\ord}(U^p)_{\fm, i}'
\end{equation}
where the last isomorphism is the valuation map
as in \cite[Lem. 3.24]{pask}.


\begin{rem}
Since $U_{w_0}^{(1)}=h_N(\smat{p&\\&1})$
acts as $\smat{p&\\&1}$ on 
$S^{\ord}(U^p,E/\oo)_{\fm}$,
the discussion before Lemma \ref{lem:proj_enve}
implies that 
$U_{w_0}^{(1)}$ acts residually 
by $\chi_2\omega^{-1}(p)=\chi_2(p)$ on 
$S^{\ord}(U^p,E/\oo)_{\fm,1}'$ and 
$M^{\ord}(U^p)_{\fm, 1}'$;
by  $\chi_1\omega^{-1}(p)=\chi_1(p)$ on 
$S^{\ord}(U^p,E/\oo)_{\fm,2}'$ and 
$M^{\ord}(U^p)_{\fm, 2}'$.
\end{rem}

\begin{cor}\label{cor:no_torsion}
    The $R_\fm$-module $M(U^p)_{\fm}'$
    has no $\xx$-torsion.
\end{cor}
\begin{proof}
    By the anti-equivalence \eqref{eq:anti_equiv},
	it suffices to show that
	$\Hom_{Q'}(\tilde{P}_{j,\fm}, M(U^p)_{\fm}'[\xx])=0$
	for $j=1,2$.
    Since $R_\fm$ acts faithfully on $\tilde{P}_{i,\fm}$
    by \eqref{eq:proj_1}, 
	we can apply the snake lemma to the commutative diagram
    below then apply the exact functor
    $\Hom_{Q'}(\tilde{P}_{j,\fm}, *)$
    to the resulting sequence.
    \begin{equation*}
    \begin{tikzcd}
    0 \arrow[r] & 
    \tilde{P}_{\B,\fm}^{\oplus r} 
	\arrow[d,"\xx",hookrightarrow] \arrow[r,"A"] & 
	\tilde{P}_{\B,\fm}^{\oplus r} 
	\arrow[d,"\xx",hookrightarrow] \arrow[r] & 
	M(U^p)_{\fm}'
    \arrow[d,"\xx"]  \arrow[r] & 0 \\ 
    0 \arrow[r] & 
    \tilde{P}_{\B,\fm}^{\oplus r}
	\arrow[r,"A"] & 
    \tilde{P}_{\B,\fm}^{\oplus r}
	\arrow[r] &
    M(U^p)_{\fm}'  
    \arrow[r] & 0 
    \end{tikzcd}
\end{equation*}
From which we can observe that
$\Hom_{Q'}(\tilde{P}_{j,\fm}, M(U^p)_{\fm}'[x])$
is isomorphic to the kernel of the homomorphism below,
which is injective by the previous proposition.
\begin{equation}\label{eq:fil_ij}
	\smat{\bar{A}_{ii}& \bar{A}_{ij}\Phi_{ij}\\& \bar{A}_{jj}}\colon 
	\Hom_{Q'}(\tilde{P}_{j,\fm},\tilde{P}_{\B,\fm}^{\oplus r})^{\red}\to
	\Hom_{Q'}(\tilde{P}_{j,\fm},\tilde{P}_{\B,\fm}^{\oplus r})^{\red}
\end{equation}
\end{proof}


\begin{cor}\label{cor:fil_by_ord}
	We have the following exact sequence of
	$\TT(U^p,\oo)_\fm\times R_{\fm}^{\red}$-modules
\begin{equation}
	0\to M^{\ord}(U^p)_{\fm,1}'\to
	\Hom_{Q'}(\tilde{P}_{2,\fm},M(U^p)_{\fm}')^{\red}
    \xrightarrow{(*)}
	M^{\ord}(U^p)_{\fm,2}'\to0.
\end{equation}
A similarly exact sequence also exsits
for $\Hom_{Q'}(\tilde{P}_{1,\fm},M(U^p)_{\fm}')^{\red}$.
\end{cor}
\begin{proof}
    Apply $\Ord$ to the exact sequence
    obtained from the proof of the previous corollary
    with $j=2$.
    Use the exactness of \eqref{eq:exact_ord}
    and the isomorphism \eqref{eq:separate}
	gives the following commutative diagram
\begin{equation*}
    \begin{tikzcd}
	    0 \arrow[r]& 
	    \Hom_{Q'}(\tilde{P}_{2,\fm},\tilde{P}_{1,\fm}^{\oplus r})^{\red}
	    \arrow[r,"\bar{A}_{11}"] \arrow[d]&
	    \Hom_{Q'}(\tilde{P}_{2,\fm}, \tilde{P}_{1,\fm}^{\oplus r})^{\red}
	    \arrow[d] &&\\
	    0\arrow[r] & 
	    \Hom_{Q'}(\tilde{P}_{2,\fm},\tilde{P}_{\B,\fm}^{\oplus r})^{\red}
	    \arrow[r,"\eqref{eq:fil_ij}"] \arrow[d,"\Ord"] &
	    \Hom_{Q'}(\tilde{P}_{2,\fm},\tilde{P}_{\B,\fm}^{\oplus r})^{\red}
	    \arrow[d,"\Ord"] \arrow[r]&
	    \Hom_{Q'}(\tilde{P}_{2,\fm}, M(U^p)_{\fm}')^{\red}\arrow[r]
        \arrow[d,"\Ord", "(*)"'] &0\\
	    0\arrow[r] & 
	    \End_{T'}(\tilde{P}_{\chi_2^\vee,\fm})^{r}
	    \arrow[r,"\bar{A}_{22}"] &
	    \End_{T'}(\tilde{P}_{\chi_2^\vee,\fm})^{r}
        \arrow[r]&
        \Hom_{T'}(\tilde{P}_{\chi_2^\vee,\fm}, M^{\ord}(U^p)_\fm')
        \arrow[r]&0
    \end{tikzcd}
\end{equation*}
where the exactness of
the two vertical columns on the left follows from
$\Hom_{Q'}(\tilde{P}_{2,\fm}, \tilde{P}_{2,\fm})^{\red}
\cong \End_{T'}(\tilde{P}_{\chi_2^\vee,\fm})$
by Lemma \ref{lem:ker_red}
and the morphism $(*)$ is equivariant
with respect to $\TT(U^p,\oo)_\fm$.
Therefore the kernel of $(*)$
is isomorphic to the cokernel 
of the first row, 
which is the image under
composition with 
$\Phi_{12}\in \Hom_{Q'}(\tilde{P}_{2,\fm},\tilde{P}_{1,\fm})$
of the following exact sequence.
\[
0\to
\Hom_{Q'}(\tilde{P}_{1,\fm},\tilde{P}_{1,\fm}^{\oplus r})^{\red}
\xrightarrow{\bar{A}_{11}}
\Hom_{Q'}(\tilde{P}_{1,\fm},\tilde{P}_{1,\fm}^{\oplus r})^{\red}
\to
\Hom_{Q'}(\tilde{P}_{1,\fm}, M(U^p)_{\fm}')^{\red}
\to 0
\]
The same argument shows that 
the cokernel is Hecke-equivariantly
isomorphic under $\Ord$
to $M^{\ord}(U^p)_{\fm,1}'$.
\end{proof}  

Let $T_n(p^1)$ be the compact open subgroup of $T_n\subset G_p$
introduced in \eqref{def:lambda_rings}
and $\Lambda=\oo\llbracket T_n(p^1)\rrbracket$.
We can decompose $T_n(p^1)$
using $Z_Q$ and the subtorus 
$\Qp^\times=\{\smat{1&\\&*}\}$ in $\GL_2(\Qp)$ as follows
\[
    T_n(p^1)=T_n(p^1)\cap \{\smat{1&\\&*}\}\times 
    T_n(p^1)\cap Z_Q\subset T'=(\Qp^\times)^2\times Z_Q
    \subset Q'=\GL_2(\Qp)\times Z_Q
\]
which induces $\Lambda$-actions on smooth representations
of $Q'$ and $T'$.
Note that since $T_n(p^1)$ stabilizes 
$\{\smat{1&\Zp\\&1}\}$,
by the definition in $\eqref{def:hUm}$
the actions are compatible 
when we pass to the ordinary parts.

\begin{cor}\label{cor:Hecke_finite}
The module
$\mathbf{m}=\Hom_{Q'}(\tilde{P}_{\B,\fm},
M(U^p)_{\fm}')$
is finite free over 
$\Lambda\hat{\otimes}_{\oo}\oo\llbracket \xx\rrbracket 
=\Lambda\llbracket \xx\rrbracket$.
\end{cor}
\begin{proof}
    Given the restriction
    on the central character, we may identify 
    $\Mod_{T',\zeta}^{\lfin}(\oo)$
    with $\Mod_{\Qp^\times\times \Z_Q}^{\lfin}(\oo)$.
    Therefore $S^{\ord}(U^p,E/\oo)_\fm'$
    is $\Qp^\times\times Z_Q$-admissible
    and $M^{\ord}(U^p)_\fm'$ is finite over $\Lambda$.
    In fact, the assumption \eqref{cond:small}
    and \cite[Prop 2.20]{pask} 
    implies that $M^{\ord}(U^p)$
    is finite free over $\Lambda$.
    Since $\TT(U^p,\oo)$ is semi-local,
    each localization $M^{\ord}(U^p)_\fm'$
    is a direct summand of $M^{\ord}(U^p)$
    and hence finite free over $\Lambda$.
    Similarly  $M^{\ord}(U^p)_{\fm,i}'$   
    are finite free over $\Lambda$ for $i=1,2$.
    
    Therefore $\mathbf{m}^{\red}$ 
	is also finite free over $\Lambda$
    by the corollary above.
    Since the finite $R_\fm$-module
    $\mathbf{m}$ has a natural compact topology
    on which the $\Lambda$-action is continuous,
	by the topological Nakayama lemma
	$\mathbf{m}$ is finite over $\Lambda\llbracket x\rrbracket$.
    Lift a set of generators for $\mathbf{m}^{\red}$
    over $\Lambda$,
    which induces a surjective homomorphism
    $\Lambda\llbracket \xx\rrbracket^{\oplus r}\to 
    \mathbf{m}^{\red}$
    for some integer $r$.
    It then follows from Corollary 
    \ref{cor:no_torsion}
    that the induced homomorphism is also injective
    and therefore an isomorphism.
\end{proof}
\begin{rem}
Note that we actually proved that
$\Hom_{Q'}(\tilde{P}_{j,\fm},M(U^p)_\fm')$
has no $\xx$-torsion for $j=1,2$
in Corollayr \ref{cor:no_torsion},
therefore the same argument also shows that
$\Hom_{Q'}(\tilde{P}_{j,\fm},M(U^p)_\fm')$
are finite free over 
$\oo\llbracket\xx\rrbracket$ for $j=1,2$.
\end{rem}


\begin{defn}

Since $M(U^p)_{\fm}'$ is a faithful $\TT(U^p,\oo)_{\fm}$-module,
its image 
$\mathbf{m}$ of the anti-equivalence \eqref{eq:anti_equiv}
is also a faithful $\TT(U^p,\oo)_{\fm}$-module.
We let  $\tilde{\TT}(U^p,\oo)_\fm$
denote the $\Lambda\llbracket\xx\rrbracket$-subalgebra
of $\End_{\Lambda\llbracket\xx\rrbracket}(\mathbf{m})$
generated by the image of the injection.
\[
\TT(U^p,\oo)_\fm\hookrightarrow 
\End_{\Lambda\llbracket\xx\rrbracket}(\mathbf{m}).
\]
Then $\tilde{\TT}(U^p,\oo)_\fm$
is finite over $\Lambda\llbracket\xx\rrbracket$
since $\mathbf{m}$
is finite over $\Lambda\llbracket\xx\rrbracket$
by the corollary above.
In fact, 
since the image of $\xx$ is nonzero by
Corollary \ref{cor:no_torsion},
it is neither a nilpotent
as $\tilde{\TT}(U^p,\oo)_\fm$ is reduced by \cite[Lem 2.14]{ger}.
Therefore the natural map 
$\Lambda\llbracket\xx\rrbracket\to\tilde{\TT}(U^p,\oo)_\fm$
is an injection.
\end{defn}

\subsection{Fundamental exact sequence}
\label{sub:fund_exact_sequence}

Suppose $\fG=W\times \Delta$ is an abelian pro-$p$ group,
where $W$ is fintie free over $\Zp$
and $\Delta$ is a finite abelian group,
with a homomorphism
$T_n(p^1)\times\Delta_\fs\to \fG$ 
of finite cokernel.
Let  $\lambda^{\ord}\colon 
\TT^{\ord}(U^p,\oo)\to \oo\llbracket\fG\rrbracket$
be a homomorphism 
between $\Lambda_{\fs}$-algebras,
where the $\Lambda_{\fs}$-structure
of $\TT^{\ord}(U^p,\oo)$
is induced by \eqref{eq:Lambda_hecke}
and that of $\oo\llbracket\fG\rrbracket$
is induced by the group homomorphism
up to twisting by a character on which.
In this subsection, we write
\begin{equation}
    \lambda\colon 
    \TT(U^p,\oo)\to
    \TT^{\ord}(U^p,\oo)\xrightarrow{\lambda^{\ord}}
    \oo\llbracket\fG\rrbracket
\end{equation}
where $\TT(U^p,\oo)=\TT^P(U^p,\oo)\to \TT^{\ord}(U^p,\oo)$ is
induced by Lemma \ref{lem:coh_to_ord},
and the maximal ideal $\fm\subset \TT(U^p,\oo)$
satisfying \eqref{cond:red_gen}
is the pre-image by $\lambda$
of the maximal ideal of the local ring
$\oo\llbracket\fG\rrbracket$.
Let $\fm^\circ\subset \TT^{\ord}(U^p,\oo)$
be the maximal ideal obtained in the same way by $\lambda^{\ord}$.

Recall that the submodules
$S^{\ord}(U^p,E/\oo)_{\fm,i}'$
in \eqref{eq:sub12}
are determined by the action
of $\Qp^\times=\{\smat{1&\\&*}\}$.
Since this action
factor through $\TT^{\ord}(U^p,\oo)$,
up to twisting by the character
$\xi_\fm^{1/2}$ from \eqref{eq:root_charm}
which is congruent to the trivial character,
one of $\chi_i$, say $i=2$, coincide with 
the restriction of $\lambda^{\ord}\bmod\fm^\circ$
to $\Qp^\times$. Therefore we may assume
$M^{\ord}(U^p)_{\fm,2}'\cong M^{\ord}(U^p)'_{\fm^\circ}$.
Combine Corollary \ref{cor:fil_by_ord}
and Lemma \ref{prop:ord_to_dual}
we have the surjective homorphism between 
$\TT(U^p,\oo)_\fm$-modules
\begin{equation}\label{eq:chain_coh}
    \Hom_{Q'}(\tilde{P}_{2,\fm}, M(U^p)_\fm')
    \otimes_{\Lambda_{\fs}}\oo\llbracket\fG\rrbracket
    \twoheadrightarrow
    M^{\ord}(U^p)_{\fm^\circ}'
    \otimes_{\Lambda_{\fs}}\oo\llbracket\fG\rrbracket
    \twoheadrightarrow
    \Hom_{\oo\llbracket\fG\rrbracket}
    (S^{\ord}_\fG(U^p)_{\fm^\circ},
    \oo\llbracket\fG\rrbracket).
\end{equation}
Our goal in this subsection
is to formulate a fundamental exact sequence 
generalizing \cite[Prop 6.3.5]{urban}
under the following assumptions.
\begin{enumerate}[label=(C\arabic*)]
\item There exists a Hecke-equivariant homomorphism
$\Theta\colon
\Hom_{\oo\llbracket\fG\rrbracket}
(S^{\ord}_\fG(U^p)_{\fm^\circ},
\oo\llbracket\fG\rrbracket)\to 
\oo\llbracket\fG\rrbracket$,
where $\TT(U^p,\oo)_{\fm}'$
acts on 
$\oo\llbracket\fG\rrbracket$ through $\lambda$.
\label{cond:C1}
\item There exists an ideal 
$\fq\subset \TT(U^p,\oo)_{\fm}$
containing the image of $\xx$ under 
$R^{\zeta\epsilon}\to \TT(U^p,\oo)$
and some $F\in \Hom_{\oo\llbracket\fG\rrbracket}
(S^{\ord}_\fG(U^p)_{\fm^\circ},
\oo\llbracket\fG\rrbracket)$
with $\Theta(F)\neq 0$
such that
\[
    \fq M^{\ord}(U^p)_{\fm,1}'=0\quad
    \text{ and }\quad
    \fq F=0 \text{ in }
    \Hom_{\oo\llbracket\fG\rrbracket}
    (S^{\ord}_\fG(U^p)_{\fm^\circ},
    \oo\llbracket\fG\rrbracket).
\]
\label{cond:C2}
\item For $\Theta(F)$ as above 
and any character $\alpha$ of $\Delta$,
the image of $\Theta(F)$ under
$\oo\llbracket\fG\rrbracket\xrightarrow{\alpha}
\oo\llbracket W\rrbracket$
is nonzero.
\label{cond:C3}
\end{enumerate}

To simplify notations, 
we write
$\tilde{T}=\TT(U^p,\oo)_{\fm}$,
$M=\Hom_{Q'}(\tilde{P}_{2,\fm},M(U^p)_{\fm}')
\otimes_{\Lambda_{\fs}}\oo\llbracket\fG\rrbracket$, and
\[
M_1=M^{\ord}(U^p)_{\fm,1}'
\otimes_{\Lambda_{\fs}}\oo\llbracket\fG\rrbracket
\quad
M_2=M^{\ord}(U^p)_{\fm,2}'=M^{\ord}(U^p)_{\fm^\circ}
\otimes_{\Lambda_{\fs}}\oo\llbracket\fG\rrbracket.
\]
Composing the homomorphisms in \eqref{eq:chain_coh}
with $\Theta$
give maps of $M$ and $M_2$ to 
$\oo\llbracket\fG\rrbracket$.
We define $S_2\subset M_2$ and
$S\subset M$ as the kernels of the maps.
Note that $\xx M\subset S$ by definition.


\begin{lem}\label{lem:right_exact}
There exists an exact sequence
of $\oo\llbracket\fG\rrbracket\cong
\oo\llbracket W\rrbracket[\Delta]$-modules
\begin{equation}\label{eq:fund}
	M/S\xrightarrow{\xx} \fq M/\fq S\to 
	\fq M_2/\fq S_2 \to 0
\end{equation}
\end{lem}

\begin{proof}
Since $\xx\in \fq$ by \ref{cond:C2}
we have the commutative diagram
\[
\begin{tikzcd}
	S\arrow[r,"\xx"] \arrow[d]
	& \fq S \arrow[r] \arrow[d]
	& \fq S^{\red} \arrow[r] \arrow[d] & 0\\
	M\arrow[r,"\xx"]
	& \fq M \arrow[r]
	& \fq M^{\red} \arrow[r] & 0
\end{tikzcd}
\]
Applying the snake lemma
reduces the lemma to showing that
$\fq M^{\red}/\fq S^{\red}=\fq M_2/\fq S_2$.
Observe that $\fq M^{\red}=\fq M/\xx M$ 
is mapped injectively into $M^{\red}=M/\xx M$.
and by Corollary \ref{cor:fil_by_ord} we have
\begin{align*}
    M_1\to M/\xx M\to M_2\to 0
    &\Longrightarrow
	\Image(\fq M^{\red}\to M^{\red})=
    \fq M/\xx M\cong\fq M_2\\
    M_1\to S/\xx M\to S_2\to 0
    &\Longrightarrow
	\Image(\fq S^{\red}\to M^{\red})=
    \fq S+\xx M/\xx M\cong\fq S_2.
\end{align*}
where the last isomorphisms 
follows from \ref{cond:C2}
and from which we have the desired isomorphism.
\end{proof}


Let $\hat{\Delta}$ be the set 
of characters on the finite abelian group $\Delta$. Then
\[
    \oo\llbracket\fG\rrbracket=
    \oo\llbracket W\rrbracket[\Delta]\hookrightarrow
    \prod_{\alpha\in\hat{\Delta}}
    \oo\llbracket W\rrbracket_\alpha,\quad
    \oo\llbracket W\rrbracket_\alpha=
    \oo\llbracket W\rrbracket
    \text{ on which $\Delta$ acts by $\alpha$}
\]
We write 
$\oo\llbracket\fG\rrbracket=\Lambda_\fG$, 
$\oo\llbracket W\rrbracket=\Lambda_W$,
$K_W=\textnormal{Frac}\Lambda_W$.
And we write $\otimes_{\Lambda_\fG,\alpha}\Lambda_W$
for the tensor $\otimes_{\Lambda_\fG}\oo\llbracket W\rrbracket_\alpha$
and similarly for $\otimes_{\Lambda_\fG,\alpha}K_W$.

\begin{lem}\label{lem:inj_crit}
The sequence \eqref{eq:fund}
is left exact if 
$(\fq M/\fq S)\otimes_{\Lambda_\fG,\alpha}K_W\neq 0$
for all $\alpha\in\hat{\Delta}$.
\end{lem}
\begin{proof}
Since $\Theta$ induces $M/S\hookrightarrow
\oo\llbracket\fG\rrbracket$,
the module $M/S$ has no $p$-torsion
and we have
\[
\begin{tikzcd}
M/S
\arrow[r,"(a)"]\arrow[d,hookrightarrow] &
\fq M/\fq S
\arrow[r]\arrow[d] &
\fq M_2/\fq S_2
\arrow[r]\arrow[d] & 0\\
(M/S)\otimes_{\Lambda_\fG}\Lambda_\fG[1/p]
\arrow[r,"(b)"] &
(\fq M/\fq S)\otimes_{\Lambda_\fG}\Lambda_\fG[1/p]
\arrow[r] &
(\fq M_2/\fq S_2)\otimes_{\Lambda_\fG}\Lambda_\fG[1/p]
\arrow[r] & 0
\end{tikzcd}
\]
Therefore the map $(a)$ is injective if
$(b)$ is injective.
Moreover, observe that 
$\Lambda_\fG[1/p]=\prod_\alpha \Lambda_W[1/p]_\alpha$.
So $(b)$ is injective if and only if the map $(b)_\alpha$ below
is injective for all $\alpha\in\hat{\Delta}$.
\[
(M/S)\otimes_{\Lambda_\fG,\alpha}\Lambda_W[1/p]
\xrightarrow{(b)_\alpha}
(\fq M/\fq S)\otimes_{\Lambda_\fG,\alpha}\Lambda_W[1/p]
\to
(\fq M_2/\fq S_2)\otimes_{\Lambda_\fG,\alpha}\Lambda_W[1/p]\to 0
\]

Now, note that since $M/S$ and $\Lambda_\fG$ have no $p$-torsion,
$\Theta$ also induces
$(M/S)\otimes_{\Lambda_\fG}\Lambda_\fG[1/p]
\hookrightarrow\Lambda_\fG[1/p]$.
The localization at the primes corresponding to 
$\alpha\in\hat{\Delta}$ then also induces
$(M/S)\otimes_{\Lambda_\fG,\alpha}\Lambda_W[1/p]
\hookrightarrow\Lambda_W[1/p]$.
In particular we see that
$(M/S)\otimes_{\Lambda_\fG,\alpha}\Lambda_W[1/p]$
is torsion free over $\Lambda_W$, therefore
\[
\begin{tikzcd}
(M/S)\otimes_{\Lambda_\fG,\alpha}\Lambda_W[1/p]
\arrow[r,"(b)_\alpha"]\arrow[d,hookrightarrow] &
(\fq M/\fq S)\otimes_{\Lambda_\fG,\alpha}\Lambda_W[1/p]
\arrow[r]\arrow[d] &
(\fq M_2/\fq S_2)\otimes_{\Lambda_\fG,\alpha}\Lambda_W[1/p]
\arrow[r]\arrow[d] & 0\\
(M/S)\otimes_{\Lambda_\fG,\alpha}K_W
\arrow[r,"(b)_\alpha'"] &
(\fq M/\fq S)\otimes_{\Lambda_\fG,\alpha}K_W
\arrow[r] &
(\fq M_2/\fq S_2)\otimes_{\Lambda_\fG,\alpha}K_W
\arrow[r] & 0
\end{tikzcd}
\]
and the map $(b)_\alpha$ is injective if
$(b)_\alpha'$ is injective.
However, from
$(M/S)\otimes_{\Lambda_\fG}\Lambda_W[1/p]_\alpha
\hookrightarrow\Lambda_W[1/p]_\alpha$
we have
$(M/S)\otimes_{\Lambda_\fG,\alpha}K_W
\hookrightarrow K_W$.
There we can identify
$(M/S)\otimes_{\Lambda_\fG,\alpha}K_W=K_W$
since \ref{cond:C3}
implies that the image is nonzero.
On the other hand,
let $E_2\subset M_2$
be the preimage of
$\oo\llbracket\fG\rrbracket F$
under the last map in \eqref{eq:chain_coh}.
Then \ref{cond:C3} implies, by a similar arguent as above, that
$(M_2/(S_2+E_2))\otimes_{\Lambda_\fG,\alpha}K_W=0$.
And since $\fq(S_2+E_2)\subset \fq S_2$
as $\fq F=0$, we also have
\[
    0=\fq\otimes_{\TT}\left(M_2/(S_2+E_2)\right)
    \otimes_{\Lambda_\fG,\alpha}K_W
    =\fq\otimes_{\TT}(M_2/S_2)
    \otimes_{\Lambda_\fG,\alpha}K_W
    \twoheadrightarrow
    \fq M_2/\fq S_2\otimes_{\Lambda_\fG,\alpha}K_W
\]
Therefore $(b)_\alpha'$ is a surjective map
on a one-dimensional vector space over $K_W$,
which is injective is the target
$(\fq M/\fq S)\otimes_{\Lambda_\fG,\alpha}K_W$
is nonzero.
This conclude the proof of the lemma.
\end{proof}

Let $W'$
be the image of the 
homomorphism
$T_n(p^1)\to \fG\to W$
introduced in the beginning of the subsection,
note that by assumption
$W'$ is a finite free $\Zp$-submodule
of $W$ of finite index.
For each $\alpha\in \hat{\Delta}$,
let $\lambda_\alpha\colon \TT\to \Lambda_\fG\to \Lambda_W$
be the composition of $\lambda$ with
the map  $\Lambda_\fG\to \Lambda_W$
induced by $\alpha$.
Then the restriction of which to 
$\Lambda$ factors through $\Lambda_{W'}$.
We let $\wp\subset \TT$
and $\fp\subset \Lambda$
denote the kernels of the homomorphisms,
which are prime ideals
since $\Lambda_W$ and $\Lambda_{W'}$
are integral domains.

Define the ideals
$\fp_0=\fp\Lambda\llbracket \xx\rrbracket$
and $\fp_1=\fp_0+(\xx)$ in 
$\Lambda\llbracket \xx\rrbracket$.
Note that 
$\fp_1$ is the kernel of the restriction 
of $\lambda_\alpha$ to 
$\Lambda\llbracket \xx\rrbracket$
since $\lambda$ factors through $\TT^{\ord}(U^p,\oo)$
and therefore $\lambda(\xx)=0$.
In other word we have
$\wp\cap \Lambda\llbracket \xx\rrbracket=\fp_1$.

\begin{lem}
There exists a prime ideal 
$\wp_0\subset \wp_1\coloneqq\wp$
of $\TT$ such that 
$\wp_0\cap \Lambda\llbracket \xx\rrbracket=\fp_0$.
\end{lem}
\begin{proof}
Let $\Lambda\llbracket\xx\rrbracket_{\fp_1}$
be the localization,
by definition
\[
\Lambda\llbracket\xx\rrbracket_{\fp_1}/
\fp_0\Lambda\llbracket\xx\rrbracket_{\fp_1}\cong
\left(
\Lambda\llbracket\xx\rrbracket/\fp_0
\right)_{\fp_1}\cong
(\Lambda_{W'}\llbracket\xx\rrbracket)_{\fp_1}\cong 
K_{W'}\llbracket\xx\rrbracket
\]
here $K_{W'}$ denote the field of fraction
of $\Lambda_{W'}$.

Taking localization 
as an $\Lambda\llbracket\xx\rrbracket$-algebra,
the quotient
$\TT_{\fp_1}/\fp_0\TT_{\fp_1}$
is finite over
the complete discrete valuation ring
$K_{W'}\llbracket\xx\rrbracket$
since $\TT$ is finite
over $\Lambda\llbracket\xx\rrbracket$
by Corollary \ref{cor:Hecke_finite}.
Therefore
$\TT_{\fp_1}/\fp_0\TT_{\fp_1}$
is the direct product
of its localizations at the maximal ideals,
or equivalently 
at the primes ideal of $\TT$
that restricts to $\fp_1$.
In particular 
$\TT_{\wp}/\fp_0\TT_{\wp}=\TT_{\wp_1}/\fp_0\TT_{\wp_1}$
is finite over 
over $\Lambda\llbracket\xx\rrbracket$.
Now the lemma will follow
if we can prove that
$\TT_{\wp}/\fp_0\TT_{\wp}=\TT_{\wp_1}/\fp_0\TT_{\wp_1}$
is not Artinian.
For then any minimal ideal of which
restricts to the zero ideal in
$\Lambda\llbracket\xx\rrbracket$
and the pull-back of which to $\TT$
gives the desired prime ideal $\wp_0$.

To show that
$\TT_{\wp}/\fp_0\TT_{\wp}=\TT_{\wp_1}/\fp_0\TT_{\wp_1}$
is not Artinian,
note that 
$\mathbf{m}_\wp\neq 0$
for $\mathbf{m}=\Hom_{Q'}(\tilde{P}_{2,\fm},M(U^p)_\fm')$.
This follows from 
that $M=\mathbf{m}\otimes_{\Lambda_\fs}\Lambda_\fG$
and $\Theta$ and $\alpha$ induces a nonzero map
$M\to \Lambda_W$
by \ref{cond:C3}, and
therefore we have a surjetive map
\[
\mathbf{m}_\wp\otimes_{\Lambda_\fs}\Lambda_\fG=
M_\wp\to \Lambda_K\otimes_{\TT,\lambda_\alpha}\TT_\wp=K_W.
\]

Now, recall that
$\mathbf{m}$ is finite free over
$\Lambda\llbracket\xx\rrbracket$
by the remark below Corollary \ref{cor:Hecke_finite}.
This implies that the
$\TT_{\fp_1}/\fp_0\TT_{\fp_1}$-module
$(\mathbf{m}/\fp_0\mathbf{m})_{\fp_1}$
is also finite free over 
$(\Lambda\llbracket\xx\rrbracket/\fp_0)_{\fp_1}
\cong K_{W'}\llbracket \xx\rrbracket$.
Moreover, the module is again 
the direct sum of its localizations
at the maximal primes of
$\TT_{\fp_1}/\fp_0\TT_{\fp_1}$,
or equivalently the primes ideals of $\TT$
that restricts to $\fp_1$.
In particular the
$\TT_{\wp}/\fp_0\TT_{\wp}$-module
$\mathbf{m}_{\wp}/\fp_0\mathbf{m}_{\wp}$
is finite free over the complete discrete valuation ring
$K_{W'}\llbracket \xx\rrbracket$,
and of rank $r\geq 1$
by Nakayama's lemma because $\mathbf{m}_\wp\neq 0$.
As a consequence, we have the following commutative diagram
which implies that 
$\TT_{\wp}/\fp_0\TT_{\wp}$ is not Artinian.
\[
\begin{tikzcd}
\TT_{\wp}/\fp_0\TT_{\wp}\arrow[r]&
\End_{K_{W'}\llbracket\xx\rrbracket}
(\mathbf{m}_\wp/\fp_0\mathbf{m}_\wp)\\
K_{W'}\llbracket\xx\rrbracket
\arrow[u]\arrow[ur,hookrightarrow]
\end{tikzcd}
\]
\end{proof}

Let $\wp_0\subset \TT$ be a prime as above.
Then $\TT_{\wp}/\wp_0\TT_{\wp}$
is an integral domain and a finite extension
over $(\Lambda\llbracket\xx\rrbracket/\fp_0)_{\fp_1}\cong 
K_{W'}\llbracket\xx\rrbracket$.
Let $\tilde{\TT}$ be the localization
at any maximal ideal 
of the integral closure of
$K_{W'}\llbracket\xx\rrbracket$
in $\textnormal{Frac}(\TT_{\fp_1}/\wp_0\TT_{\fp_1})$.
Thus $\tilde{\TT}$ is also 
a discrete valuation ring
and $\fq\tilde{T}\cong \tilde{T}$,
from which we see 
\begin{multline*}
((\fq M/\fq S)\otimes_{\Lambda_\fG,\alpha} K_W)
\otimes_{\TT_\wp}\tilde{\TT}\cong 
(\fq M\otimes_{\Lambda_\fG,\alpha} K_W)
\otimes_{\TT_\wp}\tilde{\TT}/
(\fq S\otimes_{\Lambda_\fG,\alpha} K_W)
\otimes_{\TT_\wp}\tilde{\TT}\\\cong
(M\otimes_{\Lambda_\fG,\alpha} K_W)
\otimes_{\TT_\wp}\tilde{\TT}/
(S\otimes_{\Lambda_\fG,\alpha} K_W)
\otimes_{\TT_\wp}\tilde{\TT}\cong
((M/S)\otimes_{\Lambda_\fG,\alpha} K_W)
\otimes_{\TT_\wp}\tilde{\TT}=
K_W \otimes_{\TT_\wp}\tilde{\TT}
\end{multline*}
Since the last ring is nonzero
as the maximal ideal 
restricts to that of 
$\TT_{\wp}/\wp_0\TT_{\wp}$,
which is the kernel of the map
$\TT_\wp\to K_W$,
the module 
$(\fq M/\fq S)\otimes_{\Lambda_\fG,\alpha} K_W$
is nonzero as well.
From Lemma \eqref{lem:inj_crit}
we conclude that 
the sequence \eqref{eq:fund} is left exact as well.
We organize this result into the following corollary.


\begin{cor}\label{cor:fund}
	Under the setting of this subsection
    and the assumptions
    \ref{cond:C1},
    \ref{cond:C1}, and
    \ref{cond:C1},
    we have the following commutative diagram
    of exact sequences of Hecke algebras
    \begin{equation}
    \begin{tikzcd}
    & (M/S)
    \arrow[r,"\xx"]\arrow[d,equal] &
    (M/S)\otimes_{\TT}\fq
    \arrow[r]\arrow[d] &
    (M/S)\otimes_{\TT}\fq^\red
    \arrow[r]\arrow[d] &0\\
    0\arrow[r] &
    (M/S)
    \arrow[r,"\xx"] &
    \fq M/\fq S
    \arrow[r] &
    \fq M_2/\fq S_2
    \arrow[r] &0
    \end{tikzcd}
    \end{equation}
\end{cor}
\begin{proof}
The vertical maps are the natural ones
and the commutative properties 
follows from the isomorphism
$\fq M^\red/\fq S^\red=\fq M_2/\fq S_2$
shown in the proof of Lemma \ref{lem:right_exact}.
The exactness of the first row
is straightforward and
the exactness of the second row
follows from the discussions above.
\end{proof}

\begin{rem}
Aside from some slight modifications,
the formulation and the proof of the corollary
are entirely from \cite[Prop. 6.3.5]{urban},
in which the corollary is proven 
for the case of Eisenstein ideals on modular curves.
We remark that in both cases,
the proof use
the critical idea that 
given a ``Hida family'' with Hecke eigensystem
$\lambda^{\ord}$ whose pull-back to $\TT(U^p,\oo)$
satisfies \eqref{cond:red_gen},
there exists another family ``parametrized by $\xx$''
in the completed cohomology that
is perpendicular to the Hida family.
In the case of Eisenstein ideals
this family can be the generically cuspidal family
of overconvergent modular forms of a positive slope
that passes through a critical Eisenstein series.
We refer the reader to \cite[Thm 3.4.1]{urban}
for another flavor of the above corollary 
whose proof relies on the existence of this family.
\end{rem}


\section{Euler systems}

In this section,
we use the fundamental exact sequence
\eqref{eq:fund}, which is left exact
by Proposition \ref{cor:fund},
and the Hida family of theta lifts in \cite{lee}
to construct an anticyclotomic Euler system

We assume $\K/\F$
has the following properties.
\begin{enumerate}[label=($\K$\arabic*)]
\item $p$ is unramified in $\F$ and \eqref{cond:ord}
\label{cond:K1}
\item $\K/\F$ is generic, so $\oo_\K^\times\subset \F$
and $Cl_\F\to Cl_\K$ is injective
\label{cond:K2}
\item every prime above $2$ in $\F$ is split in $\K$
\label{cond:K3}
\item $-1$ is a square in the residue of every
ramified place. In other word $\qch(-1)=1$.
\label{cond:K4}
\end{enumerate}


$\K'/\K$ finite abelian extension
$\mu$ conductor prime to $p$
consist only of split primes
and $\psi^{-1}=\mu/\mu^c$.
Let $\psi^*=\epsilon\psi^{-1}$
and $\Psi^*=\psi^*\langle\cdot \rangle$
with values in $\oo\llbracket \fG_\fs^a\rrbracket$.

\subsection{Hida family of theta lifts}

Let $G=\UU(2)$ and $\UU(1)$
be the algebraic groups over $\F$
defined by \eqref{def:def_unitary}
for $n=2$ and $n=1$ respectively.
We can identify $\UU(1)(\A_\F)$
with the subgroup $\A_\K^1$ of norm-one
elements in $\A_\K^\times$.
Let $(\cdot,\cdot)$
denote the canonical Hermitian pairing on $\K^{\oplus 2}$
that defines the unitary group $G$.
We fix a nonzero $\delta=-\bar{\delta}\in \K$
and identify $G$ with the unitary group
associated to the skew-Hermitian pairing $\delta(\cdot,\cdot)$.
Then $(G, \UU(1))$ forms a dual reductive pair.

To define the Weil representation
on the dual reductive pair,
we fix a unitary Hecke character $\chi$ of 
$\A_{\K}^\times/\K^\times$
that restricts to $\qch_{\K/\F}$ on $\A_\F^\times$.
We further require that
$\chi_\sigma(z)=\frac{\bar{z}}{|z|}$
for all $\sigma\in \Sigma$.
In other word $\chi_\circ\coloneqq \chi|\cdot|^{1/2}$
is an algebraic Hecke character
of infinity type $\Sigma^c$.
We let $\fc$ denote the conductor of $\chi$.

In \cite{lee},
analogous to the method in
\cite{Hsieh2014} and \cite{wan},
we show that theta lifts
from $\UU(1)$ to $G$
can be obtained through a pull-back 
from the quasi-split unitary group
of signature $(2,2)$.
Assuming that $p$ is unramified in $\F$,
we obtain the integrality
and the interpolation of the theta lifts
using the theory of integral models
of PEL-type Shimura varieties
associated to quasi-split unitary groups
developed in \cite{Hida04}.

To give the precise statement,
let $K(\fc)\subset G(\A_f)$
be the open compact subgroup 
introduced in the beginning of \cite[\S 6]{lee},
which essentially depends only on the conductor 
$\fc$ of the auxiliary character $\chi$.

\begin{defn}\label{def:admchar}
We say an algebraic Hecke character $\eta$ 
of $\A_\K^\times/\K^\times$ is admissible
if the followings hold.
\begin{enumerate}
\item 
There exists an ideal $\fs=\fs^c$
consisting only of split primes such that 
$\eta$ is trivial on
$\A_{\K,f}^1\cap (1+\fs\widehat{\oo}_K)^\times$.
\item 
Identify $\A_{\K,\infty}^\times$ with $(\C^\times)^\Sigma$
via the fixed CM type $\Sigma$, then $\eta$ has the infinity type
\begin{equation}
\sum_{\sigma\in\Sigma}a_\sigma\sigma+b_\sigma\bar{\sigma},\quad
\text{ such that }
k_\sigma\coloneqq a_\sigma-b_\sigma\geq0\,
\text{ for all }\sigma\in\Sigma.
\end{equation}
\end{enumerate}
\end{defn}

When $\eta$ is an admissible Hecke character as above.
We write $k=\sum_{\sigma}k_\sigma\sigma$ and
$\wt{k}=(0,-k)$.
We also fix a splitting $\fs=\ff\bar{\ff}$
such that $\oo_\K=\ff+\bar{\ff}$ as in \S 2.
Then we define $K(\fs)$ as the open compact subgroup
obtained by replacing the component of $K(\fc)$ above 
each $w\mid \ff$ with 
$\{k\in \Iw(\ff)_w\mid (\iota_w(k))_{11}\in 1+\ff\oo_w\}$.
Let $K(\fs)^p$ denote the prime-to-$p$ part of $K(\fs)$.
The following proposition is a restatement 
of \cite[Prop. 7.5]{lee}.

\begin{prop}
When $\eta$ is admissible,
for $\wt{k}$ and $K(\fs)$ as above
there exists 
$f(\eta)\in S_{\wt{k}}^{\ord}(K(\fs)^p\Iw(p^{n,n}),\eo)$
such that $\iota(f(\eta))(v_{-\wt{k}})$
is the theta lift of $\eta$
(denoted by $\theta^\square_{\Phi'}(\eta,\nu)$
in \textit{loc.cit})
up to canonical CM periods
and explicit factors.
Here $\eo$
is the ring of integers
of some finite extension 
over the completion
of the maximal unramified extension
over $\Qp$,
$\iota$ is the map defined by \eqref{eq:p_to_infty},
and $v_{-\wt{k}}$ is a vector of lowest weight
in $\xi_{\wt{k}}^*(\C)$.
\end{prop}

\begin{rem}
In \cite{lee}
the proposition was stated 
for characters $\eta$
with conductors $\fs$ as above
that are coprime to $\fc$
and with infinity type
$\sum_{\sigma\in \Sigma}k_\sigma\sigma$
such that $k_\sigma\geq 0$ for all $\sigma\in \Sigma$.
But the choice of the Schwartz functions
at non-split places only depends
on the restriction of $\eta$ to $\A_\K^1$.
In particular, after replacing 
$\phi_{1,w}=\chi_w^{-1}\id_{\oo_v^\times}$
in $(\phi 2)$ \cite[p.14]{lee}
by $\phi_{1,w}=(\tilde{\eta}\chi_w^{-1})\id_{\oo_v^\times}$,
the formulae in
\cite[Prop 4.3]{lee} and \cite[Prop 4.5]{lee},
and thus \cite[Prop 7.5]{lee}, 
still hold true for any $\eta$ that is 
admissible as in Definition \ref{def:admchar}.
\end{rem}

Fix a prime-to-$p$ ideal $\fn=\fn^c$
consisting only of split primes
and an algebraic Hecke character $\eta$
such that $\eta$ is trivial on
$\A_{\K,f}^\times\cap (1+\fn\widehat{\oo}_\K)^\times$.
Then for a prime-to-$p\fn$ ideal $\fs=\fs^c$
consisting only of split primes,
let $\fX_\fs(\eta)$ be the collection of all
algebraic Hecke characters $\alpha$ such that 
$\eta\alpha$ is admissible and
the $p$-adic avatar
$\hat{\alpha}$ induces a character
on $\fG_\fs$.
Recall that 
$\fG_\fs$ denotes the Galois group of the maximal pro-$p$
abelian extenion over $\K$ of tame conductor $\fs$
and $\fG_\fs^a$ denotes the anticyclotomic quotient.
Fix a splitting of $\fn\fs$
and define $K(\fn\fs)$ as above.
We then have the following restatement of
\cite[Thm. 7.6]{lee}.
\begin{prop}\label{prop:family}
    There exists an
    $S^{\ord}(K(\fn\fs)^p)$-valued measure $\euF$
    on $\fG_\fs^a$ such that 
    \[
    \int_{\fG_\fs^a} \tilde{\alpha}^\wedge\,d\euF
    =\beta_{\wt{k}'}(f(\eta'))\quad\text{ for }
    \eta'=\eta\alpha\text{ and }
    \alpha\in \fX_\fs(\eta).
    \]
    Here $\wt{k}'$ is the weight associated
    to $\eta'$,
    $\tilde{\eta}'^\wedge$
    is the $p$-adic avatar
    of $\tilde{\eta}'=\eta/\eta^c$,
    and $\beta_{\wt{k}'}\colon 
    S^{\ord}_{\wt{k}'}(U^p)\to S^{\ord}(U^p)$
    is the injection which is essentially 
    induced by Proposition \ref{prop:wt_indep},
    since the representations $\pi_{\wt{k}}^*$ 
    are one-dimensional in the $B$-ordinary case.
\end{prop}
Note that $K(\fn\fs)^p$ satisfies \eqref{cond:s-ram}
and the reciprocity map
\eqref{eq:anticyc_rec}
defines a homomorphism
$T_1(p)\times \Delta_\fs\to \fG_\fs^a$.
Thus the measure above 
defines a Hida family over $\fG_\fs^a$
in the sense of Definition \ref{def:Hida_family}
since for each $\eta'=\eta\alpha$ with $\alpha\in \fX_\fs(\eta)$
the central character of $f(\eta')$ 
is the restriction of $\hat{\eta}'$ to $\A_{\K,f}^1$.
Furthermore, it is shown in \textit{loc.cit} that $f(\eta')$ 
is an eigenform with respect to 
the Hecke operators in \eqref{def:hecke_away_p},
\eqref{def:hecke_at_p},
and \eqref{def:hecke_at_s}, with eigenvalues
\begin{align}
T_w^{(1)}f(\eta')&=
(q_w\chi_\circ(\varpi_w)+\chi_\circ^{-1}\tilde{\eta}'(\varpi_w))
f(\eta'),\\
U_{\wt{k}',w}^{(1)}f(\eta')&=
\wt{k}'(\varpi_w,1)(\chi_\circ^{-1}\tilde{\eta}'(\varpi_w))
f(\eta'),\\
U_{w}^{(1)}f(\eta')&=
(\chi_\circ^{-1}\tilde{\eta}'(\varpi_w))
f(\eta').
\end{align}

Moreover, 
$B_\fs\colon 
S_{\fG_\fs^a}^{\ord}(K(\fn\fs)^p)\times 
S_{\fG_\fs^a}^{\ord}(K(\fn\fs)^p)
\to \eo\llbracket\fG_\fs^a\rrbracket$ 
be the pairing on Hida families
defined in \cite[\S 6.4.1]{lee}
and consider
$\mathcal{L}=
B_\fs(\euF,U_\fs^{-1}\euF)$,
where $U_{\fs}$ be the product of all $U_w^{(1)}$
(which depends on a fixed splitting of $\fs$).
The following restatement from 
\cite[Thm 7.7]{lee}
shows that 
$\mathcal{L}_\fs$
is a $p$-adic $L$-function.

\begin{prop}\label{prop:function}
For 
$\mathcal{L}\in \eo\llbracket\fG_\fs^a\rrbracket$
as above and $\eta'=\eta\alpha$ with $\alpha\in \fX_\fs(\eta)$,
up to powers of $2$ we have
\begin{multline*}
	\frac{1}{\Omega_p^{2k'+4}}
	\int_{\fG_{\fs}^a}\tilde{\alpha}^\wedge\,\mathcal{L}=
	c(\eta')C(\chi,\K)
	\frac{(2\pi i)^{k'}\Gamma(k'+2)
	}{\Omega_\infty^{k'+2\Sigma}}
    L(1,\chi^{-2}\tilde{\eta}')\\\cdot
	\prod_{w\mid \Sigma_p\mathfrak{C}}
	\varepsilon(1,(\chi^{2}\tilde{\eta}'^{-1})_w,\psi_w)
	(1-(\chi^{-2}\tilde{\eta}')(\varpi_{\bw})
	(1-(\chi^{-2}\tilde{\eta}')(\varpi_{\bw})q_{\bw}^{-1})
\end{multline*}
where $\fn\fs=\mathfrak{F}\mathfrak{F}^c$
is a fixed splitting for $\fn\fs$.
\end{prop}
We refer to the theorem for the precise 
meaning of the notations above.
Here we only note that
$\Omega_p$ and $\Omega_\infty$
are the canonical CM periods associated to $\K$,
$C(\chi,\K)$ and $c(\eta)$ are explict constants in $\eo$,
with $c(\eta)$ being an $p$-unit,
and $C(\chi,\K)$ being
essentially the square of the algebraic part
of the central values of $\chi$.
The following lemma shows that
our assumption on $\K$
guarantees a good choice of the auxiliary character $\chi$.

\begin{lem}
Assuming \ref{cond:K2}-\ref{cond:K4},
there exists a unitary Hecke character $\chi$
with the following properties.
\begin{enumerate}
\item The restriction of $\chi$ to $\A_F^\times$ is $\qch_{\K/\F}$.
\item The character $\chi_0=\chi|\cdot|^{1/2}_\K$
has the infinity type $\Sigma^c$.
\item There exists a split prime $\fl_\circ\neq \fl_\circ^s$
such that $\chi$ is trivial on 
$\A_{\K,f}^1\cap (1+\fl_\circ\fl_\circ^s\widehat{\oo}_\K)^\times$.
\item The central value $L(1/2,\chi)$ is nonzero.
\end{enumerate}
\end{lem}
\begin{proof}

Let $\chi_{can}$
be the canonical Hecke character 
with the infinity type $-\Phi=-\Sigma^c$
defined in \cite{Rohrlich} 
under the condition \ref{cond:K2}.
Then $\chi_{can}$ 
is ramified precisely at all $w\mid \mathfrak{d}_{\K/\F}$.
And the restriction of $\chi_{can}$ to $\oo_w^\times$
is the unique character 
of conductor $(\varpi_w)$ 
that extends $\qch_{\K_w/\F_v}$ for $w\mid v$.
In particular,
$\chi_{can}$ is trivial 
on $\A_{\K,f}^1\cap \widehat{\oo}_\K^\times$
by \ref{cond:K4}
and it follows from \cite[\S 8]{Rohrlich}
(see also \cite[Lem 2.1]{Rod})
that the root number $W(\chi_{can})=1$
under the condition \ref{cond:K3}.

Therefore the conditions in
\cite[Thm A]{Hsieh2012}
are satisfied for the self-dual character
$\chi_{can}^{-1}$.
Consequently for any split prime $\fl_\circ$
that is prime to $p$, there exists 
a twist of which by a finite-order 
anticyclotomic character
ramified at $\fl_\circ\fl^s_\circ$
for which the central $L$-value is nonzero
(in fact, the algebraic part of the 
the central $L$-value is nonzero modulo $p$).
We take $\chi$
for which $\chi_0=\chi|\cdot|^{1/2}_\K$
is such a twist.

Now the last property above follows by definition.
The third follows from the above discussion.
The second follows from that 
$\chi_0$ is a finite-order twist of $\chi_{can}^{-1}$,
which has the infinity type $\Sigma^c$.
And the first follows from that 
$\chi_{can}$ restricts to $\qch_{\K/\F}|\cdot|^{-1}_\F$.
\end{proof}

\begin{defn}

After fixing a Hecke character $\chi$
as in the lemma above,
for the finite order character 
$\psi^{-1}=\tilde{\mu}$
as in the beginning of the section
we let $\eta=\chi_0\mu$ and 
\[
    \euF_{\fs}\in S^{\ord}_{\fG_\fs^a}(U^p_{\fs}),\quad
    \mathcal{L}_\fs=B_\fs(\euF_{\fs}, U_\fs^{-1}\euF_{\fs})
    \in \eo\llbracket\fG_{\fs}^a\rrbracket
\]
be defined as in 
Proposition \ref{prop:family} and
Proposition \ref{prop:function}.
Here we let $\fn=\fn^s$ be the largest ideal
consisting only of split primes such that
$\eta$ is trivial on 
$\A_{\K,f}^1\cap (1+\fn\widehat{\oo}_\K)^\times$.
And for $\fs=\fs^c$ prime to $p\fn$
we let $U^p_\fs$ denote $K(\fn\fs)^p$.
We assume $U^p_{\fs}$
satisfies \eqref{cond:small} and \eqref{cond:s-ram}
for any $\fs$,
potentially after shrinking $K(\fn)$.
And we write $\euF_\fs=\euF$
and $\mathcal{L}_\fs=\mathcal{L}$
when $\fs=\oo_\K$.

Note that since $\eta=\chi_\circ\mu$
has the infinity type $\Sigma^c$,
the set $\fX_\fs(\eta)$ is equal to the set $\fX_\fs^+$ which
consists of all algebraic Hecke characters $\alpha$
such that $\tilde{\alpha}$
has the infinity type 
$\sum_{\sigma\in\Sigma}k_\sigma(\sigma-\sigma^c)$
with $k_\sigma>0$ for all $\sigma\in\Sigma$;
and that the $p$-adic avatar $\hat{\alpha}$
induces a character on $\fG_{\fs}$.

\end{defn}

\begin{prop}
The Hida familiy $\euF_\fs$
induces $\lambda_{\fs}\colon \TT^{\ord}(U^p,\eo)
\to \eo\llbracket\fG_\fs^a\rrbracket$ with
\begin{align*}
T_w^{(1)}&\mapsto\epsilon\hat{\chi}_0(\Fr_w)\cdot(\Psi^*(\Fr_w)+1)\\
U_w^{(1)}&\mapsto\epsilon\hat{\chi}_0(\Fr_w)\cdot(\Psi^*(\Fr_w))\\
\langle u\rangle&\mapsto \epsilon^{-1}\hat{\chi}^2_0\Psi^*(\rec u)
\end{align*}
Note that for $\Fr_w$ depends on the 
choice of the uniformizer  $\varpi_w$ when  $w\in \Sigma_p$.
\end{prop}

To simplify the notations,
for $\alpha\in \fX_\fs^+$ and $\eta'=\eta\alpha$
we write $\lambda=
\epsilon \chi^{-2}\tilde{\eta}'=\epsilon\psi^{-1}\tilde{\alpha}$,
$C(\lambda)=c(\chi_0\mu\alpha)C(\chi,\K)$,
and $W(\lambda)=\prod_{w\mid \Sigma_p\mathfrak{F}}
\varepsilon(1,(\psi\tilde{\alpha}^{-1})_w,\psi_w)$,
where $\mathfrak{F}$
is a fixed splitting for $\fn\fs$,
potentially switching $\fl_\circ$ and $\fl_\circ^c$,
we assume $\fl_\circ\mid \mathfrak{F}$.
Note that if $k'$ is the weight associated to $\eta'$,
then $\lambda$ has the infinity type
$2\Sigma+\sum_{\sigma\in \Sigma}k_\sigma(\sigma-\sigma^c)$.
The interpolation formula for $\mathcal{L}_\fs$
can then be stated as follows.
\begin{prop}
For $\alpha\in \fX_\fs^+$ and $\lambda=\epsilon\psi^{-1}\tilde{\alpha}$
as above we write
$\lambda^*=(\lambda)^{-c}\epsilon=\psi^{-1}\tilde{\alpha}$, then
\begin{equation*}
	\frac{1}{\Omega_p^{2k'+4}}
	\int_{\fG_{\fs}^a}\tilde{\alpha}^\wedge\,\mathcal{L}_\fs=
	C(\lambda)W(\lambda)\cdot 
	\frac{(2\pi i)^{k'}\Gamma(k'+2)
	}{\Omega_\infty^{k'+2\Sigma}}
    L(0,\lambda)\cdot
	\prod_{w\mid \Sigma_p\fs\fl_\circ}
	(1-\lambda(\varpi_{\bw}))
	(1-\lambda^*(\varpi_{\bw}))
\end{equation*}
\end{prop}

\begin{rem}
We may pick $\fl_\circ$
to be coprime to the conductor of $\psi$ and
such that $\psi(\fl_\circ)-1$
and $\omega\psi(\fl)-1$ are $p$-units.
Then the part with $w=\fl_\circ$
is a unit.

We should view this 
as $L(\Psi^D,\Sigma)$
\end{rem}





\subsection{Cocyles from reducible representation}

Let $\mathcal{G}$ be a group and 
$\Psi\colon \mathcal{G}\to A$
be a two-dimensional pseudo-representation
to a Henselian local ring $A$
of odd residual characteristic.
Recall that 
$\Psi$ is residually reducible
when there exists characters
$ \bar{\delta}_i$ of $\mathcal{G}$
to the residue field of  $A$
such that  
$\Psi\equiv \bar{\delta}_1+\bar{\delta}_2\mod \fm_A$.
We further assume that $\bar{\delta}_1\neq \bar{\delta}_2$,
so there exists $z\in \mathcal{G}$
with  $\bar{\delta}_1(z)\neq \bar{\delta}_2(z)$.
Then
\begin{equation}
    P_\Psi(z,X)=
    X^2-\Psi(z)X+\det(\Psi)(z) \equiv 
    (X-\bar{\delta}_1(z))(X-\bar{\delta}_2(z))
    \mod \fm_A
\end{equation}
The distinct roots $\bar{\delta}_i(z)$
lifts to two roots $\alpha,\beta$ of  $P_\Psi(z,X)$
by the Henselian property,
with $\alpha-\beta\in A^\times$.
We then define the functions
\begin{equation}
   a(\sigma)=
   \frac{\Psi(\sigma z)-\beta\Psi(\sigma)}{\alpha-\beta}\quad
   d(\sigma)=
   \frac{\Psi(\sigma z)-\alpha\Psi(\sigma)}{\beta-\alpha}\quad
   x(\sigma,\tau)=a(\sigma\tau)-a(\sigma)a(\tau).
\end{equation}
Then 
$a(\sigma)\equiv \bar{\delta}_1(\sigma)$,
$d(\sigma)\equiv \bar{\delta}_2(\sigma)$,
and $x(\sigma,\tau)$
generate the reducibility ideal
$I_\Psi$ of  $\Psi$.
In particular,
for any ideal $I$ containing $I_\Psi$,
the functions $a(\sigma),d(\sigma)$
defines characters to  $A/I$
lifting the characters
$\bar{\delta}_1$ and $\bar{\delta}_2$.
Now, if we let
$\bar{\delta}=\bar{\delta}_1\bar{\delta}_2^{-1}$, then
\begin{equation*}
    c(\sigma,\tau)\coloneqq \bar{\delta}_1^{-1}(\tau)
    \bar{\delta}_1^{-1}(\sigma)x(\sigma, \tau)\mod I_\Psi^2
\end{equation*}
defines a cocycle in 
$Z^1(\mathcal{G}\times \mathcal{G}, 
(I_\Psi/I_\Psi^2)(\bar{\delta}\boxtimes \bar{\delta}^{-1}))$.


\subsection{Main construction}
We now have all the ingredient needed 
for the construction of our Euler systems.
When $U^p=K(\fn)^p$,
let $M(U^p)$ be defined as in \eqref{eq:completed_coh}
and let $\TT(\fn,\oo)\coloneqq \TT(U^p,\oo)$
be the big Hecke algebra acting on which.
By Lemma \ref{lem:coh_to_ord}
and Proposition \ref{prop:ord_to_dual}
there exists a surjective homomorphism
\[
	M(U^p)\to M^{\ord}(\fn)\to 
	\Hom_{\Lambda_{fs}}(S_\Lambda^{\ord}(\fn),\Lambda_\fs)
\]
Let $\TT(\fn,\oo)\to \TT^{\ord}(\fn,\oo)$
be the induced homomorphism.
By abuse of notation, we write 
$\lambda_{\fn}\colon \TT(\fn,\oo)\to \TT^{\ord}(\fn,\oo)\to
\Lambda_{\fs}$ be the composition.
Let $\Psi\colon \Gal_\K\to \TT(\fn,\oo)$ be the 
big Galois pseudo-representation, then
\[
	\lambda_{\fn}\circ \Psi=
	\epsilon^{-1}\hat{\chi}_\circ+
	\hat{\chi}_\circ^{-1}\langle\cdot\rangle
\]
To apply the results from $p$-adic local Langlands
from \S\ref{sub:compatible},
we make the generic assumption
\begin{equation}\label{cond:chi_gen}\tag{$\chi$-gen}
\epsilon^{-1}\hat{\chi}_\circ^{2}\vert_{D_w}\neq
\id, \omega^{\pm}.
\end{equation}
In particular we let 
$\chi_1$ and  $\chi_2$ be the 
characters obtained from 
modulo 
$\hat{\chi}_\circ$ and $\epsilon\hat{\chi}_\circ$ 
respectively with the maximal ideal.
Note that $\lambda_{\fn}(U_{w}^{(1)})\equiv \chi_2(p)$
since $\epsilon(p)=1$.


\begin{defn}
	Using the notations from 
	\S\ref{sub:compatible}.
	Let $\TT_\fn=\TT(\fn,\oo)_\fm$,
	$M_\fn=\Hom_{Q'}(\tilde{P}_{2,\chi},M(U^p)_{\fm}')$,
	$M_i=M^{\ord}(U^p)_\fm^{U_{w}^{(1)}\equiv\chi_i(p)}$,
	and $M_2'=\Hom_{\Lambda_{\fs}}
	(S^{\ord}_{\Lambda}(\fn),\Lambda_{\fs})_{\fm_2}$,
	here $\fm_2$ is the maximal ideal
	of $\TT^{\ord}(\fn,\oo)$ obtained through 
	$\lambda_\fn$.
	We define 
	$\Theta_\fn\colon M_2'\to \Lambda_{\fs}$
	by $B_\fn(*,U_\fs^{-1}\euF^\circ_\fn)$.
\end{defn}


Let $\Psi_{\fn}\colon \Gal_K\to \TT_\fn$ be the localization
of the pseudo-representation
and let $R_{\fm}$ be the universal 
pseudo-deformation of $\chi_1+\chi_2$.
Note that  $R_\fm$ only depends on  $\chi_\circ$
and does not depends on the level  $\fn$.
In particular the pseudo-representations induces
the commutative diagram
\[
\begin{tikzcd}
	& \TT_{\fn\ell}\arrow[d,"\phi_{\fn\ell}^\ell"]\\
	R_{\fm}\arrow[r]\arrow[ur]
	& \TT_{\fn}
\end{tikzcd}
\]
Since the residue characters $\chi_i$ are distinct,
we may follow the previous subsection,
pick  $z\in \Gp\cong D_W$ 
and lift roots $\alpha,\beta\in R_\fm$
to form functions  $a(\sigma), d(\sigma), x(\sigma,\tau)$
in  $R_\fm$.

\begin{defn}
With the above choice of $z$
we also form the $\TT_\fn$-valued
function  $x_\fn(\sigma,\tau)$ 
from the pseudo-representation $\Psi_\fn$,
using the image of the roots  $\alpha,\beta$ in  $\TT_\fn$.
Thus the functions $x(\sigma,\tau)$
and  $x_\fn(\sigma,\tau)$ are compatible 
among the above diagram.
Moreover,
let $\sigma_0,\tau_0\in D_w$
be such that $ x=x(\sigma_0,\tau_0)$
is a generator of the reducibility ideal.
By abuse of notation we let
$x=x_\fn(\sigma_0,\tau_0)$
denote the image of which in $\TT_\fn$.
\end{defn}

\begin{prop}

The ideal 
$q_{\fn}\coloneqq
\{x_\fn(\sigma,\tau)\mid\sigma\in\Gal_K, \tau\in D_w\}
\subset \TT_\fn$
satisfies \ref{cond:C2}
in \S\ref{sub:fund_exact_sequence}.

\end{prop}

\begin{proof}

It is obvious by definition
that $x=x_\fn(\sigma_0,\tau_0)\in \fq_\fn$.
We also note that 
$\fq_\fn\subset\ker(\lambda_\fn)$
since  $\lambda_\fn\circ\Psi_\fn$ is reducible.
We still have to show that
$\fq_\fn$ acts trivially on  $M_1$
and that 
$\fq_\fn M_{\fn,2}/\fq_\fn S_{\fn,2}$
is torsion over $\Lambda_{\fs}$.

For the first statement,
let $\fp\subset \TT_\fn$
be a prime ideal that 
factors through  $M_1$. 
Enlarge $E$ if necessary,
we may assume  $\textnormal{Frac}(\TT_\fn/\fp)=E$,
then  $\Psi_\fn\bmod \fp$
is the trace of a Galois representation  $r_{\fp}$
as in \eqref{eq:Gal_hecke_at_p}
that is ordinary at $w$ as well and such that
\[
	r_\fp\vert_{D_w}\sim
	\smat{\psi_1&*\\&\psi_2}
\]
where $\epsilon\psi_i\equiv \chi_i$
and $x(\sigma,\tau)=b(\sigma)c(\tau)$.
In particular $q_\fn\subset \fp$.
Since such primes are dense in the image
of $\TT_{\fn}\to \End(M_1)$,
we conclude that $\fq_{\fn}$ acts trivially on $M_1$.

On the other hand,
the bilinear pairing $B_n$ induces
a homomorphism 
$S^{\ord}_\Lambda(\fn)\to\Hom_{\Lambda_{\fs}}
(S^{\ord}_\Lambda(\fn),\Lambda_{\fs})$
between $\Lambda_{\fs}$-modules.
Let $F_\fn\in M_2'$ be the image of which
and $E_{2,\fn}=\Lambda_{\fs}\cdot F_\fn$.
Since  $\Phi_\fn(F_{\fn})=\mathcal{L}_n$ is nonzero 
by definition, we have
\[
	M_{2,\fn}'/S_{2,\fn}'+E_{2,\fn}'\hookrightarrow
	\Lambda_{\fs}/\mathcal{L}_{\fn}
\]
is $\Lambda_{\fs}$-torsion.
Let $E_{\fn}\subset M_{\fn}$ be the 
preimage of $E_{\fn,2}'$,
then we have $\fq(S_\fn+E_\fn)=\fq S_\fn$, 
therefore 
$\fq_{\fn}M_{\fn}\fq_{\fn}S_{\fn}$ is a quotient of
the following, which is $\Lambda_{\fs}$-torsion.
\[
	\fq_\fn\otimes 
	M_\fn/(S_\fn+E_\fn)\cong 
	\fq_{\fn}\otimes 
	M'_\fn/(S'_\fn+E'_\fn)
\]
\end{proof}



\subsection{Compatibility between tame levels}



\begin{equation*}
	 \euF^\circ_{\fn\ell} \mod I_{\ell}=
	(1-\hat{\chi}_\circ(\Fr_\fl)\cdot 
	\langle \Fr_\fl\rangle^{-1} V_\fl)
	\euF^\circ_{\fn}.
\end{equation*}



\begin{align*}
	K_0(\fn)&=
	\{
	k\in K(\fc)\mid
	\iota_{w}(k_v)\equiv
	(\begin{smallmatrix}
		*&*\\&*
	\end{smallmatrix})\mod \varpi_\fl
	\text{ for } \fl\mid \fs
	\}\\
	K_0^n(\fn)&=
	\{
	k\in K_0(\fn)\mid
	\iota_{w}(k_v)\equiv
	(\begin{smallmatrix}
		*&*\\&*
	\end{smallmatrix})\mod \varpi_w^n
	\text{ for } w\in \Sigma_p
	\}
\end{align*}


Let $B_\fn\coloneqq S_\Lambda^{\ord}(\fn)\times
S_\Lambda^{\ord}(\fn)\to \Lambda_{\fs}^a$ 
by the $\Lambda_{\fs}$-bilinear pairing
on Hida families defined in \cite[\S 6.4.1]{lee}.
To compare the pairing between different
auxiliary levels $\fn=\fc\fs$,
define $I_{\ell}=\ker(\Lambda_{\fs\ell}\to \Lambda_{\fs})$
when $\ell\nmid \fs$.
The reduction 
$B_{\fn\ell}^{\ell}\coloneqq B_{\fn\ell}\bmod I_\ell$
is a $\Lambda_{\fs}$-valued pairing on 
$S_\Lambda^{\ord}(\fn\ell)/I_\fs S_\Lambda^{\ord}(\fn\ell)$.
This last space is by definition
the space of Hida families that are valued 
in $\varprojlim_r\varprojlim_n 
S^{\ord}((K^n_1(\fn))^{\fs}K_0(\fs),\oo/\varpi^r)$,
where the componets above $\fs$ in the subgroup $K^n_1(\fn)$
are replaced by that of in $K_0(\fn p)$.
The inclusion map 
$\id\colon S^{\ord}(K^n_1(\fn),M)\subset
S^{\ord}((K^n_1(\fn))^{\fs}K_0(\fs),M)$
and the map
$V_{\fl}\colon S^{\ord}(K^n_1(\fn),M)\to
S^{\ord}((K^n_1(\fn))^{\fs}K_0(\fs),M)$
given by $V_\fl\cdot f(g)=
f(g\iota_\fl^{-1}(\smat{1&\\&\varpi_\fl})$ induces
\[
\id, V_\fl\colon S^{\ord}_{\Lambda}(\fn)\to
S_\Lambda^{\ord}(\fn\ell)/I_\fs S_\Lambda^{\ord}(\fn\ell)
\]

\begin{prop}\label{prop:pair_at_deff_level}
	With notations above we have the following
	relation between $B_{\fn}$
	and $B_{\fn\ell}^\ell$.
	\begin{align*}
	%&B_{\fn}(T_{w}^{(1)}\cdot\euF_1,\euF_2)=
	%B_{\fn}(\euF_1,T_{w}^{(1)}\cdot\euF_2)\\
	&B_{\fn\ell}^{\ell}(\euF_1,\euF_2)=
	B_{\fn}(\euF_1,T_{\fl}^{(1)}\cdot \euF_2)\\
	&B_{\fn\ell}^{\ell}(\euF_1,V_{\fl}\cdot\euF_2)=
	(q_\ell+1) B_{\fn}(\euF_1,
	T_{\fl}^{(2)}\cdot\euF_2)\\
	&B_{\fn\ell}^\ell
	(V_{\fl}\cdot \euF_1,V_{\fl}\cdot\euF_2)=
	B_{\fn} (T_{\fl}^{(1)}\cdot\euF_1,
	T_{\fl}^{(2)}\cdot \euF_2)
	\end{align*}
\end{prop}
\begin{proof}
	By definition,
	the pairing $B_{\fn\ell}^\ell$
	and $B_{\fn}$ are defined by 
	interpolating the pairings 
	on modular forms
	$f_1,f_2\in S^{\ord}(K_0(\fn p^n),\oo)$
	given by 
	\[
	\sum_{G(\F)\backslash G(\A_f)/K_0(\fn\ell p^n)}
	f_1(g)f_2(\bar{g}\tau_{\fs\ell}^n)\text{ and }\quad
	\sum_{G(\F)\backslash G(\A_f)/K_0(\fn p^n)}
	f_1(g)f_2(\bar{g}\tau_{\fs}^n)
	\]
	Here $\tau_{\fs}$ is the product 
	$\tau_\fl\coloneqq
	\iota_{\fl}^{-1}(\smat{\varpi_\fl&\\&1})$
	for $\fl\mid \fs$
	and  $\tau_{\fs}^n$ is the product
	of $\tau_{\fs}$ and 
	$\iota_{w}^{-1}(\smat{\varpi_w^n&\\&1})$
	for all $w\in \Sigma_p$.
	Identify $G(\F_\ell)$ with  $\GL_2(\K_\fl)$
	via  $\iota_\fl$, and 
	write $K_\ell=\GL_2(\oo_\fl)$, we see that 
	\begin{multline*}
	\sum_{G(\F)\backslash G(\A_f)/K_0(\fn\ell p^n)}
	f_1(g)f_2(\bar{g}\tau_{\fs\ell}^n)=
	\sum_{G(\F)\backslash G(\A_f)/K_0(\fn p^n)}
	\sum_{k\in K_\ell/K_0(\ell)}
	f_1(gk)f_2(\bar{g}\bar{k}\tau_{\fs\ell}^n)\\=
	\sum_{G(\F)\backslash G(\A_f)/K_0(\fn p^n)} f_1(g)
	\sum_{k\in K_\ell/K_0(\ell)}
	f_2(\bar{g}\bar{k}\smat{1&\\&\varpi_\fl}
	\tau_{\fs}^n)=
	\sum_{G(\F)\backslash G(\A_f)/K_0(\fn p^n)} f_1(g)
	(T_\fl^{(1)}f_2)(\bar{g}\tau_{\fs}^n).
	\end{multline*}
	%That the pairing is symmetric 
	%follows from that 
	%$g\mapsto \bar{g}\smat{\varpi_\fl&\\&1}$
	%is an involution on 
	%$G(\F)\backslash G(\A_f)/K^n_0(\fn)$.
	%Now, we identify
	%$G(\F_\ell)$ and  $\GL_2(\oo_\fl)$
	%via  $\iota_{\fl}$, then
	%\begin{multline*}
	%B_{\fn}(T_{\fl}^{(1)}f_1, f_2)=
	%\sum_{G(\F)\backslash G(\A_f)/K^n_0(\fn)}
	%\left[\sum_b f_1(g\smat{\varpi_\fl&b\\&1})
	%+f_1(g\smat{1&\\&\varpi_\fl}) \right]
	%f_2(\bar{g}\tau_{\fs}(n))\\=
	%\sum_{G(\F)\backslash G(\A_f)/K^n_0(\fn)K^0(\ell)}
	%f_1(g\smat{\varpi_\fl&\\&1})
	%f_2(\bar{g}\tau_{\fs}(n))=
	%\sum_{G(\F)\backslash G(\A_f)/K^n_0(\fn)K^0(\ell)}
	%f_1(\bar{g}\tau_{\fs}(n))
	%f_2(g\smat{\varpi_\fl&\\&1})\\=
	%B_{\fn}(T_\fl^{(1)}\cdot f_2, f_1)=
	%B_{\fn}(f_1, T_\fl^{(1)}\cdot f_2)
	%\end{multline*}
	%The third equality follow from 
	%that $g\mapsto 
	%\bar{g}\smat{\varpi_\fl^{-1}&\\&1}\tau_\fs(n)$
	%is an involution on the underlying set.
	And the first equality follows.
	For the second equality, we have
	\begin{multline*}
	\sum_{G(\F)\backslash G(\A_f)/K_0(\fn\ell p^n)}
	f_1(g)(V_\fl f_2)(\bar{g}\tau_{\ell\fs}^n)=
	\sum_{G(\F)\backslash G(\A_f)/K_0(\fn p^n)}
	\sum_{k\in K_\ell/K_0(\ell)}
	f_1(gk)(V_\fl f_2)(\bar{g}\bar{k}\tau_{\ell\fs}^n)\\=
	\sum_{G(\F)\backslash G(\A_f)/K_0(\fn p^n)}
	f_1(g)\sum_{k\in K_\ell/K_0(\ell)}
	f_2(\bar{g}\bar{k}\smat{\varpi_\fl&\\&\varpi_\fl}
	\tau_{\fs}^n)=
	(q_\fl+1)
	\sum_{G(\F)\backslash G(\A_f)/K_0(\fn p^n)} f_1(g)
	(T_\fl^{(2)}f_2)(\bar{g}\tau_{\fs}^n)
	\end{multline*}
	The last equation follows similarly
	from the above computaiton.
\end{proof}


\begin{lem}\label{lem:compare_L_diff_level}
Define 
$P_\ell=(1-\epsilon^{-1}\hat{\chi}_\circ^{-2}(\Fr_\fl)\cdot 
\langle \Fr_\fl\rangle^{-1})
(1-\hat{\chi}_\circ^{-2}(\Fr_\fl)\cdot 
\langle \Fr_\fl\rangle^{-1})\in \Lambda_{\fs}$
when $\ell\nmid \fs$.
Let $\euF_{\fn\ell}\in S_\Lambda^{\ord}(\fn\ell)$
be a Hida family that also satisfies
$\euF_{\fn\ell} \mod I_{\ell}=
(1-\hat{\chi}_\circ(\Fr_\fl)\cdot 
\langle \Fr_\fl\rangle^{-1} V_\fl)
\euF_{\fn}$,
for some $\euF_{\fn}\in S_\Lambda^{\ord}(\fn)$, then 
there exists the equaltiy
\[
	B_{\fn\ell}(\euF_{\fn\ell}, \euF^{\circ}_{\fn\ell})
	\bmod I_\ell=
	(\hat{\chi}_\circ^{-1}(\Fr_\fl)\cdot 
	\langle \Fr_\fl\rangle
	\langle \Fr_{\bar{\fl}}\rangle^{-1})
	P_\ell\cdot B_{\fn}(\euF_{\fn}, \euF^{\circ}_{\fn})
\]
In particular, 
the $p$-adic L-function $\mathcal{L}_\fn$ satisfies 
$\mathcal{L}_{\fn\ell} \bmod I_\ell=
P_\ell\cdot \mathcal{L}_\fn$.
\end{lem}

\begin{proof}
Let $\alpha$ and  $\beta$ denote 
$\epsilon^{-1}\hat{\chi}_\circ(\Fr_\fl)$ and 
$\hat{\chi}_\circ^{-1}(\Fr_\fl)\cdot
\langle \Fr_\fl\rangle$
and apply Proposition \ref{prop:pair_at_deff_level}, thus
\begin{multline*}
B_{\fn\ell}(\euF_{\fn\ell}, \euF^{\circ}_{\fn\ell})
\bmod I_\ell=
B_\fn(\euF, T_\fl^{(1)}\euF^\circ)
-2(q_\ell+1)\beta^{-1}B_\fn(\euF, T_\fl^{(2)}\euF^\circ)
+\beta^{-2}B_\fn(T_\fl^{(1)}\euF, T_\fl^{(2)}\euF^\circ)
\\=
[(\alpha+\beta)-2(q_\fl^{-1}+1)(\alpha)
+q_\fl^{-1}(\alpha+\beta)(\alpha\beta^{-1})]\cdot 
B_\fn(\euF_{\fn}, \euF^\circ_\fn)=
(\alpha-\beta)[q_\fl^{-1}(\alpha\beta^{-1})-1]\cdot 
B_\fn(\euF_{\fn}, \euF^\circ_\fn)
\end{multline*}
Now the lemma follows from
$(\alpha-\beta)(q_\fl^{-1}(\alpha\beta^{-1})-1)
=\beta(1-(\alpha/\beta))(1-(\alpha/\beta)q_\fl^{-1})
=\beta\cdot P_\ell$.
\end{proof}

We can now apply consider the following 
commutative diagram,
where the exactness of the second row
follows from
Proposition \ref{prop:compatibility}
\[
\begin{tikzcd}
&M_{\fn\ell}/S_{\fn\ell}\arrow[r]\arrow[d]&
\fq_{\fn\ell}\otimes_{\TT_{\fn\ell}}
	M_{\fn\ell}/S_{\fn\ell}\arrow[r]\arrow[d,"(1)"]&
\fq_{\fn\ell}^{\red}\otimes_{\TT_{\fn\ell}}
	M_{\fn\ell}/S_{\fn\ell}\arrow[r]\arrow[d]&0\\
0\arrow[r]&
M_\fn/S_\fn \arrow[r]&
\fq_{\fn\ell}M_{\fn\ell}/\fq_{\fn\ell}S_{\fn\ell} \arrow[r]&
\fq_{\fn\ell}M^{\ord}_{\fn\ell}/
\fq_{\fn\ell}S^{\ord}_{\fn\ell} \arrow[r] &0
\end{tikzcd}
\]

\begin{defn}
	By abuse of notation
	let $F_\fn$
	denote the preimage of which in
	$M_{\fn}/S_{\fn}\cong M'_\fn/S'_\fn$.
	$y_\fn(\sigma)=\chi_2^{-1}(\sigma)
	x_{\fn}(\sigma,\tau_0)\otimes
	F_{\fn}\in 
	q_{\fn}\otimes_{\TT_{\fn}}M_{\fn}/S_{\fn}$.
	By definition
	its image under (1) is sent to $0$
	on the lower right corner,
	therefore the image
	comes from lower left 
	by the exactness.
	We let  $c_{\fn}(\sigma)$
	denote the $\Lambda_{\fs}$-valued
	class obtained from composing
	with $\Phi_\fn$.
\end{defn}

\begin{thm}
	The classes $x_{\fn}(\sigma)\in 
	H^1(\K, \Lambda_{\fs})$
	form an Euler system when we vary $\fs$,
	in the sense that 
	\[
		\phi_{\fn\ell}^{\ell}\colon
		H^1(\K,\Lambda_{\fs\ell})\to 
		H^1(\K,\Lambda_{\fs})\quad
		x_{\fn\ell}\mapsto
		P_\ell\cdot x_\fn
	\]
	Moreover, it satisfies the following.
	\begin{enumerate}[label=(\alph*)]
		\item if $\bw\in \Sigma_p\setminus\{w_0\}$,
		then  $res_w(x_\fn)=0$
	\item for the fixed  $w\in \Sigma_p$,
		then  $x_\fn(\sigma_0)=\mathcal{L}_\fn$
	\item unramified at other places.
	\end{enumerate}
\end{thm}

\begin{proof}

Let $x'_{\fn}(\sigma)\in M_{\fn}S_{\fn}$.
Since $F_{\fn\ell}\bmod I_\ell=
\beta^{-1}V_\fl)F_\fn$,
we also have 
$x'_{\fn\ell}(\sigma)\bmod I_\ell
=x'_\fn(\sigma)$.
Thus by Lemma \ref{lem:compare_L_diff_level}
and the diagram below
we have the norm relation.
\begin{equation*}
\begin{tikzcd}[column sep=tiny]
& M_{\fn\ell}/S_{\fn\ell}\arrow{dd} \arrow[rd,"\sim"]\arrow[rr]
&& \fq_{\fn\ell}\otimes M_{\fn\ell}/S_{\fn\ell}
	\arrow[rd,twoheadrightarrow]\arrow[rr]\arrow[dd]
&& \fq_{\fn\ell}^{\red}\otimes M_{\fn\ell}/S_{\fn\ell}\arrow[dd]\arrow[rd]\\
0 \arrow[crossing over]{rr} 
&& \Lambda_{\fn\ell}
	\arrow[crossing over]{dd} \arrow[crossing over]{rr} 
&& \fq_{\fn\ell}M_{\fn\ell}/\fq_{\fn\ell}S_{\fn\ell}
	\arrow[crossing over]{dd}\arrow[crossing over]{rr} 
&& \fq_{\fn\ell}M_{\fn\ell}^{\ord}/\fq_{\fn\ell}S^{\ord}_{\fn\ell}
	\arrow{dd} \arrow[rr] && 0\\
& M_{\fn}/S_{\fn}\arrow{rr}\arrow[rd,"\sim"]
&& \fq_{\fn}\otimes M_{\fn}/S_{\fn}
	\arrow{rr} \arrow[rd,twoheadrightarrow]
&& \fq_{\fn}^{\red}\otimes M_{\fn}/S_{\fn} \arrow[rd]& & & \\
0 \arrow[crossing over]{rr} 
&& \Lambda_{\fn} \arrow[crossing over]{rr} 
&& \fq_{\fn}M_{\fn}/\fq_{\fn}S_{\fn}\arrow[crossing over]{rr} 
&& \fq_{\fn}M_{\fn}^{\ord}/\fq_{\fn}S_{\fn}^{\ord} \arrow[rr] && 0
\arrow[from=2-3,to=4-3,crossing over]
\arrow[from=2-5,to=4-5,crossing over]
\end{tikzcd}
\end{equation*}

	
\end{proof}


\section{Iwasawa main conjecture}


\begin{defn}
Let $\Delta_\fc$ denote
the maximal pro-$p$ quotient of 
$U/V\cong (\oo_K/\fc)^\times$,
write  $\Delta_\fc=U/\tilde{V}$.
Define $\tilde{V}_n=\tilde{V}^Ip\times \tilde{U}_n$
where $\tilde{U}_n$ is the product of $U_n$
and the roots of unity in  $\oo_{K,p}^\times$.
Consider the exact sequence
\[
	1\to \oo_\K^\times\backslash U/\tilde{V}_n
	\to \textnormal{Cl}(\tilde{V}_n)\to H_\K\to 1
\]
Let $C_\K=H_\K/T$ be the maximal pro-$p$ quotient
of $H_K$. There exists  
$ \tilde{T}\subset \textnormal{Cl}(\tilde{V}_\infty)$
whose image is $T$.
Let $G_n$ denote the quotient of  $R_n$
by the image of  $ \tilde{T}$ in $R_n$, we have
\[
	1\to \oo_\K^\times\backslash U/\tilde{V}_n
	\to G_n\to C_\K\to 1
\]
We define $G_*$ and  $G_*^{\pm}$ as above.
We also define $C_K^{\pm}$.
Since $p$ is odd, we have
$G_*^{-}=\tilde{G}_*^-$ and 
\[
	1\to (\oo_\K^\times\backslash U/\tilde{V}_n)^-
	\to G_n^-\to C_\K^-\to 1
\]
Since $U/\tilde{V}_n$ is pro-$p$, the map
\[
	 (1+c,1-c)\colon 
	 U/\tilde{V}_n\to 
	 (U/\tilde{V}_n)^+\times
	 (U/\tilde{V}_n)^-
\]
is an isomorhisms.
The image of $\oo_\K^\times$ through  $1-c$
factors through  $W_p$,
the group of $p$-th powers roots of unity in  $\K$.
Thus we have
\[
	1\to W_p\backslash (\Delta_\fc^-\times (U_1/U_n)^-)
	\to G_n^-\to C_\K^-\to 1
\]
When $\fc=\ell$ is a prime ideal that splits in  $\K$
we put  $G(\ell)=G_0^-$, thus
\[
	1\to W_p\backslash \Delta_\ell^-
	\to G(\ell)\to C_\K^-\to 1
\]
We put $H(\ell)=W_p\backslash \Delta_\ell^-$.
When  $\fc=\oo_\F$, $v\in S_p$,
we let $G(v^n)$ be the quotient 
of $G_n^-$ by the image of 
$\prod_{w\nmid \ell}\oo_w^\times$ in which.
Let $v=w\bw$ and 
$(U_1/U_n)_{w,\bw}$ be the component
of which above $w$ and  $\bw$, thus
\[
	1\to W_p\backslash (U_1/U_n)_{w,\bw}^-
	\to G(v^n)\to C_\K^-\to 1
\]
we similarly denote the kernel by $H(v^n)$.
At last
when  $\fc=\ell_1\cdots\ell_r$
We define $K(\fc v^n)$
as the product of the groups
$K(\ell_1),\cdots, K(\ell_r)$ and $K(v^n)$,
where the extensions corresponds
to the groups  $G(\ell_1),\cdots,G(\ell_r)$ and $G(v^n)$.
We then have 
\[
	H(\fc v^n)\coloneqq 
	\Gal(K(\fc v^n)/K(1))\cong 
	H(\ell_1)\times\cdots\times H(\ell_r)\times H(v^n)
\]
where $K(1)$ is the extension corresponding
to  $C_\K^-$.
\end{defn}


When $\chi\colon\K^\times\backslash\A^\times/V_n\to \C^\times$
is an algebraic Hecke character 
with infinity type  $\chi_\infty(z)=z^\eta$
for $\eta=\sum n_\sigma\cdot\sigma\in \Z[I_\K]$.
We then define 
$\hat{\chi}\colon \K^\times\backslash\A_f^\times/V_\infty
\to \oo^\times$ by 
$ \hat{\chi}(z_f)=\chi(z_f)\cdot z_p^\eta$.
Let $\psi\colon R_\infty\to \oo^\times$
be the induced character from the reciprocity map.
We say  $\psi$ is anticyclotomic if
$\psi$ factors through $R_\infty^-$.

\begin{lem}\label{lem:estimate}
Fix a place $w\in p$ and
define $u_m=(1+p^m,1-p^m)\in \oo_w\times \oo_{\bw}^\times$,
$\tau_m=\rec(u_m)\in R_\infty^-$.
When $\psi\colon R_\infty^-\to \C_p^\times$
is anticyclotomic,
given $m>0$,
there exists  $M$ sufficently large
and a constant  $s$ independent of  $m$
such that 
\[
	\ord_p(q_\fp-1)\geq 2m,\quad
	\ord_p(\psi(\Fr_\fp))=m+s
\]
whenver $\fp$ belongs to 
$\mathcal{L}_m=
\{\fp\nmid \fc p\mid \Fr_\fp=\tau_m \text{ in } R_M\}$.
\end{lem}
\begin{proof}
Let $u_w=1+p\in \oo_w^\times$
and pick $t$ sufficently large so that
$s'=\ord_p(\psi(\rec(u_w))^{p^t}-1)\geq 1$.
This is possible since
$\psi(\rec(u_w))^{p^t}=(1+x)^{p^t}\equiv 
1+x^{p^t}\bmod p$ if
$\psi(\rec(u_w))=1+x$.

Let $M$ be a sufficently large number such that
\[
	\ord_p(\epsilon(\rec(u))-1),\,
	\ord_p(\psi(\rec(u))-1)\geq 2m\quad
	\text{ for } u\in V_{M}
\]
Since 
$\epsilon(\tau_m)=\epsilon(\rec(u_m))=1-p^{2m}$
and $\Fr_\fp=\tau_m\red(u)$ for some $u\in V_M$
when $\fp\in \mathcal{L}_m$, we have 
\[
\ord_p(q_\fp^{-1}-1)=
\ord_p(\epsilon(\Fr_\fp)-1)\geq 2m.
\]

On the other hand,
since $\psi$ is anticyclotomic, the character
$\psi\circ \rec$ on $\oo_w^\times\times\oo_{\bw}^\times$
factors through the homomorphism $(a,b)\mapsto a/b$,
under which
$u_w\mapsto 1+p$ and
$u_m\mapsto 1+p^m/1-p^m=(1+p)^k$
for some $k\in \Zp$ with $p^{m-1}\parallel k$. 
Write $\psi(\rec(u_w))^{p^t}=(1+x)$, then
\[
	(1+x)^{p^{m-1}}=\sum_{i=0}^{p^{m-1}}
	\binom{p^{m-1}}{i}y^i,\quad
	\ord_p(\binom{p^m}{i}x^i)=m-\ord_p(i)+is'\geq
	\begin{cases}
		m-1+s'& i\geq 1\\
		m-1+2s' & i\geq 2
	\end{cases}
\]
which is greater or equal to $m-1+2s'$ when  $i\geq 2$
since  $is'-\ord_p(i)\geq s'(i-\ord_p(i))\geq 2s'$.
Thus 
\[
\ord_p(\psi(\tau_m)-1)=\ord_p(\psi(\rec(u_w)^{p^t})^{p^{m-1-t}}-1)
=m-1-t+s'<2m
\]
when $m>s\coloneqq s'-1-t$
and therefore
$\ord_p(\psi(\Fr_\fp)-1)=m+s$ when  $\fp\in\mathcal{L}_m$.
\end{proof}





\subsection{machinery}

Let $\mathscr{V}\colon \Lambda^a\to \Lambda$
be as in \cite{HT94}.
Our $L$-function is
\[
L(1,\chi^{-2}\tilde{\eta})\prod_{w\in\Sigma}
(1-\chi^{-2}\tilde{\eta}(\bw))
(1-\chi^{-2}\tilde{\eta}(\bw)q_w^{-1})
\]
Let $\psi=\epsilon\hat{\chi}_0^{-2}$ be associated to 
$\chi^{-2}$, which is of type $\Sigma-\Sigma^c$
Then above is 
\[
	L(0,\epsilon\psi)\prod_{w\in \Sigma}
	(1-(\epsilon\psi)^*(\Fr_{\bw}q_w^{-1})
	(1-\epsilon\psi(\Fr_{\bw})
\]
Here  $\lambda^*\coloneqq \lambda^{-c}\epsilon$ 
and $(\epsilon\psi)^*=\psi$ since  $\psi$
is anticyclotomic.
In fact, should view above as
\[
	L(0,\epsilon\psi^{-1})\prod_{w\in\Sigma^c}
	(1-(\epsilon\psi^{-1})^*(\Fr_{\bw}q_w^{-1})
	(1-\epsilon\psi^{-1}(\Fr_{\bw})
\]
Let $\Psi=\psi\langle\rangle$
then this  $L$-function
should be $L(\Psi^D,\Sigma^c)=L(\Psi,\Sigma)$
by functional equation.
The Euler system we constructed vanishes at  $\Sigma^c$,
thus the Selmer group of  $X(\Psi^D,\Sigma^c)$
is bounded. 


\cite{Och05}
\cite{Och08}
\cite{Hsieh2010}
\cite{HT93}
\cite{Hida06}
\cite{Hida06b}
\cite{Rubin}

\subsection{Settings}

Let $\fc=\ell_1\cdots\ell_r$
be a square-free product
of primes $\ell_i$ of  $\F$
that are split in  $\K$
and fix a decomposition
$\ell_i=\fl_i\flw_i$
for each  $\ell_i$.
By our construction, we have 
a class of classes
$z_\fc\in H^1(\rk{\fc},\Lambda(\Psi))$
for  $\Psi=\psi\langle*\rangle$ anticyclotomic 
satisfying 
\[
	N_{\rk{\fc\ell}/\rk{\fc}}z_{\fc\ell}=
	P_\ell(\Fr_{\bar{\fl}})\cdot z_\fc,\quad
	P_\ell(X)\coloneqq(1-\psi(\Fr_{\flw})X)
	(1-\psi(\Fr_{\flw})q_\ell^{-1}X)
\]
Note that here the action of $\Fr_{\flw}$
on $z_{\fc}$ is the action of $\Gal_\K$
on  $H^1(\rk{\fc},\Lambda(\Psi))$,
which factors through  $\rp{\fc p^\infty}$.

In the rest of the subsection,
we demonstrate how the Euler system above
bounds the corresponding Selmer group 
under the following assumption
that $\psi$ is residually nontrivial
at some $w\in p$.
\begin{equation}\label{cond:distinct}\tag{dist}
	\psi\vert_{I_w}\not\equiv1\quad
	\text{ for some } w\mid p
\end{equation}

We consider $H^1(\K, \Lambda^*(\Psi^D))$. 
If  $J\subset \Lambda$ is an ideal 
such that  $\Lambda/J\cong \Lambda^{m}$ for some $m$
we consider  $H^1(\K,(\Lambda/J)^*(\Psi^D))$ as well.
We prove inductively on  $m$ that 
\begin{enumerate}[label=(\alph*)]
	\item $H^1(\K,(\Lambda/J)^*(\Psi^D))^\vee$ is $\Lambda/J$-torsion
	\item the characteristic ideal in $\Lambda/J$
		is bounded by the image of the $L$-function in $\Lambda/J$.
\end{enumerate}


\subsubsection{zero-dimension}

When $\Lambda=\Lambda^{(n)}$ for $n=0$,
let  $T=\oo(\psi)$ and  $T_m=T/p^mT$.

Let $\fc$ be as above.
For each $\ell\mid\fc$,
the subgroup  $\rs{\ell}\cong W_p\backslash \Delta_\ell^-$
is cyclic.
Let  $\sigma_\ell\in \rs{\ell}$ be a generator
and put 
$D_\ell=\sum_{i=0}^{p^{n_\ell}-1}i\cdot \sigma_\ell^i
\in \Z[\rs{\ell}]$,
where $p^{n_\ell}\coloneqq \#\rs{\ell}$.
It is easily seen that 
$(\sigma_\ell-1)D_\ell=p^{n_\ell}-N_{\rk{\ell}/\rk{\id}}$.

\begin{lem}
Let $m$ be a sufficently large integer and
suppose $\fc=\ell_1\cdots\ell_r$
with each $\ell_i\in\mathcal{L}_m$.
Pick $m'=min(n_\ell, 2(m+s))$ then
$D_{\fc}z_{\fc}\bmod p^{m'}\in 
H^1(\rk{\fc},T_{m'})^{\rk{\id}}$.
\end{lem}
\begin{proof}
We prove this by induction on $r$.
The case $r=0$, which corresponds to 
$\fc=\id$, is trivial
since $z_{\id}\in H^1(\rk{\id},T)$.
In general
we show that 
$(\sigma_\ell-1)D_{\fc\ell}z_{\fc\ell}\equiv 0$
modulo  $p^{m'}$ for each $\ell\mid \fc$.
Write $\fc=\fc'\ell$, then
\[
	 (\sigma_\ell-1)D_{\fc}z_{\fc}
	 \overset{\bmod  p^{n_\ell}}{\equiv}
	 -P_\ell(\Fr_{\flw})D_{\fc'}z_{\fc'}
	 \overset{\bmod  p^{m'}}{\equiv}
	 -P_\ell(1)D_{\fc'}z_{\fc'}
\]
where the second equivalence follows by induction.
The lemma thus follows
since $\ord_p(P_\ell(1))=2(m+s)$.
\end{proof}
Since the field $\rk{\fc}$
is unramified at  $p$,
the assumption \eqref{cond:distinct}
implies that $T_{m'}^{\rs{\fc}}=0$
for any integer $m'$. 
It then follows from the exact sequence below
that $H^1(\K, T_{m'})\cong 
H^1(\rk{\fc}, T_{m'})^{\Gal_\K}$.
\[
	0\to H^1(\rk{\fc}/\K, T_{m'}^{\rp{\fc}})\to
	H^1(\K, T_{m'})\to
	H^1(\rk{\fc}, T_{m'})^{\Gal_\K}\to
	H^2(\rk{\fc}/\K, T_{m'}^{\rp{\fc}})
\]

\begin{defn}
For $m$ and  $m'$ as above, 
we let $\kappa_{\fc,m'}\in H^1(\K,T_{m'})$
denote the classes corresponding to 
\[
	N_{\rk{\id}/\K}D_{\fc}z_{\fc}
	\bmod p^{m'} \in 
	H^1(\rk{\fc}, T_{m'})^{\Gal_\K}
\]
\end{defn}
When $\ell\in \mathcal{L}_m$
let $H^1_f(\K_{\flw},T_{m'})$ and
$H^1_s(\K_{\flw},T_{m'})$ be defined respecitvely
as the kernel and image of $H^1(\K_{\flw},T_{m'})$ 
in the exact sequence 
$0\to H^1(D_{\flw}/I_{\flw},T_{m'})
\to H^1(D_{\flw},T_{m'})
\to H^1(I_{\flw},T_{m'})^{D_{\flw}}$.
By \cite[Lem 1.4.7]{Rubin},
evaluating the classes at
$\Fr_{\flw}$ and $\sigma_\ell$ induces isomorphisms
$H^1_f(\K_{\flw},T_{m'})\cong T_{m'}/(\Fr_{\flw}-1)T_{m'}$
and
$H^1_s(\K_{\flw},T_{m'})\cong T_{m'}^{\Fr_{\flw}=1}$.
We define the finite-singular comparison map
$\phi_{\flw}^{fs}$ by the commutative diagram
\[
	\begin{tikzcd}
		H^1_s(\K_{\flw}, T_{m'})^{K_{\flw}} \arrow[r]&
		T_{m'}^{\Fr_{\flw}=1}\\
		H^1_{f}(\K_{\flw}, T_{m'}) \arrow[r]
		\arrow[u,"\phi_{\flw}^{fs}"]&
		T_{m'}/(\Fr_{\flw}-1)T_{m'}
		\arrow[u,"Q_\ell(\Fr_{\flw})",swap]
	\end{tikzcd}
\]
where $Q_\ell(X)$ is the linear factor  such that 
$P_\ell(X)=(1-X)Q_\ell(X)+P_\ell(1)$.
Note that since
$Q_\ell(1)=
(1-\psi(\Fr_{\flw}))+(1-\psi(\Fr_{\flw})q_\ell^{-1})
\equiv 2(1-\psi(\Fr_{\flw})) \bmod p^{m'}$,
we have $\ord_p(Q_\ell(1))=\ord_p(\psi(\Fr_{\flw}-1))
=m+s$ with 
$m+s< m'\leq 2(m+s)$.
Therefore
the order of $\ker(\phi_{\flw}^{fs})$
is fixed independent of $\ell$.
\begin{rem}
Let $p^{w_p}=\#\mu_{p^\infty}(\K)$.
If we further assume that the prime divisors
$\ell\mid\fc$ are choosen such that
$\mu_{p^\infty}(\K)$ has distinc images in $\Delta_\ell^-$,
then we have  $n_\ell\geq 2m-w_p$.
The kernel above is nontrivial only when 
$n_\ell=m'<2(m+s)$, in which case the order 
is bounded by the constant $p^{2s+w_p}$.
\end{rem}


\begin{prop}
The Kolyvagin derivatives
$\kappa_{\fc,m'}$ has the following properties.
\begin{enumerate}[label=(\alph*)]
\item When $w\nmid p\fc$ is a finite place of  $\K$,
then  $loc_w(z_{\fc,m'})\in H^1_f(\K_w,T_{m'})$.
\item When $\ell\mid\fc$, we have
$\phi_{\flw}^{fs}(\kappa_{\fc,m'})=\kappa_{\fs\ell,m'}$.
\end{enumerate}
\end{prop}

\begin{proof}
The first statement is clear when 
$\psi$ is unramified at $w$.
In general, since we assume
that the condcutor of $\psi$ consists
only of split primes, the 
image of $D_w$ has infinite image in  $\rp{p^\infty}$
and the claim follows from
\cite[Cor 4.6.5]{Rubin},
if we further assume that
each divisor $\ell$ of $\fc$
belongs to  $\mathcal{L}_{mM}$ 
for some integer $M$ that only depends on  $\psi$.

For the second statement, 
let $\Gal_\K$ acts on $\tilde{T}=\text{Map}^{\cts}(\Gal_\K,T)$
via $(g\cdot f)(x)=f(xg)$ and 
let define $T\to\tilde{T}$ by
$t\mapsto [x\mapsto x\cdot t]$.
The assumption \eqref{cond:distinct} implies that
we have the $\Gal_\K$-equivariant 
exact sequence
\[
	0\to\tilde{T}^{\rk{\fc}}\to (\tilde{T}/T)^{\rk{\fc}}\to 
	H^1(\rk{\fc},T)\to 0
\]
In particular, for each $\fc$
there exists $ \hat{z}_\fc\in (\tilde{T}/T)^{\rk{\fc}}$
such that $z_\fc(g)=(g-1)\cdot \hat{z}_\fc$.
We claim that we can pick the lifts such that
$N_{\rk{\ell}/\rk{\id}}\hat{z}_{\fc\ell}-P_\ell(\Fr_{\flw})
\hat{z}_{\fc}\in p^{m'}\tilde{T}$.
The statement then follows from 
\begin{align*}
	D_{\fc\ell}z_{\fc\ell}(\sigma_\ell)=
	(\sigma_\ell-1)D_{\fc\ell}\hat{z}_{\fc\ell}
	\overset{p^{m'}}{\equiv}&
	-P_\ell(\Fr_{\flw}) D_\fc \hat{z}_\fc\\
	Q_\ell(\Fr_{\flw})\cdot
	D_\fc z_\fc(\Fr_{\flw})=
	Q_\ell(\Fr_{\flw})(\Fr_{\flw}-1)
	\cdot D_\fc \hat{z}_\fc
	\overset{p^{m'}}{\equiv}&
	-P_\ell(\Fr_{\flw}) D_\fc \hat{z}_\fc
\end{align*}

To prove the claim,
pick $k>0$ such that 
$\Fr_{\flw}$ acts trivially on $T_{m'}$.
Since $\Fr_{\flw}$ has infinite order
in $\rp{p^\infty}$,
there exists  $v\in S_p$ and  $a>0$
such that
$p^{m'}\mid \ord(\Fr_{\flw}^k, \rs{v^a})$.
Choose lifts
$\hat{z}_{v^a\fc\ell}$ and $ \hat{z}_{v^a\fc}$
as above, then
\[
	N_{\rk{\ell}/\rk{\id}}
	\hat{z}_{v^a\fc\ell}-P_\ell(\Fr_{\flw})
	\hat{z}_{v^a\fc}\in T+\tilde{T}^{\rk{v^a\fc}}
\]
In fact, since 
$N_{\rk{\ell}/\rk{\id}}\colon\tilde{T}^{\rk{v^a\fc\ell}}\to
\tilde{T}^{\rk{v^a\fc}}$ 
is surjective,
we can pick the lifts so that the difference
lies in $T$.
Now let 
$\hat{z}_{\fc\ell}=N_{\rk{v^a}/\rk{\id}}
\hat{z}_{v^a\fc\ell}$ and
$\hat{z}_{\fc}=N_{\rk{v^a}/\rk{\id}}\hat{z}_{v^a\fc}$,
which are lifts of $z_{\fc\ell}$ and $z_{\fc}$.
Since
\[
	N_{\rk{v^a}/\rk{\id}}t=
	\sum_{\rk{id}/\langle \Fr_\ell^k\rangle}g
	\sum_{i=1}^{\ord}\Fr_\ell^{k\cdot i} t\equiv 0
	\mod p^{m'}\quad
	\text{ for } t\in T
\]
we have 
$N_{\rk{\ell}\rk{\id}}\hat{z}_{\fc\ell}-P_\ell(\Fr_{\flw})
\hat{z}_{\fc}\in N_{\rk{v^a}/\rk{\id}}T\subset p^{m'}\tilde{T}$
\end{proof}


Assume we are given a collection of classes
$C=\{\eta_1,\cdots,\eta_k\}\subset H^1(K,T_{m'}^*)$.
Let $R_M$ be the class group corresponding to $m$ 
in Lemma \ref{lem:estimate}
and let $L$ be the corresponding field extension over  $\K$.
By the defining property of $R_M$,
we see  $G_L$ acts trivially on both  $T_{m'}$ and $T_{m'}^*$.
Apply \cite[Lem 5.2.1]{Rubin}
to the classes $\kappa_{\id}=\bar{z}_{\id}\in H^1(\K,T_{m'})$
and $\eta_1$, 
there exists  $\gamma\in G_L$ such that
\begin{align*}
	\ord(\kappa(\gamma\tau_m),T_{m'}/(\psi(\tau_m)-1)T_{m'})&\geq
	\ord( (\kappa)_L, H^1(L,T_{m'}) )\\
	\ord(\eta(\gamma\tau_m),T_{m'}^*/(\psi(\tau_m)-1)T_{m'}^*)&\geq
	\ord( (\eta)_L, H^1(L,T_{m'}^*) ) 
\end{align*}
Let $L'=\ker(\kappa_1)_L\cap \ker(\eta_1)_L$.
We then pick $\ell_1\in \mathcal{L}_m$ such that
$\eta_i\in H^1_{ur}(K_{\flw_1},T_{m'}^*)$ for all $i$
and $\Fr_{\flw_1}$ is conjugate to $\gamma\tau_m$ in  $\Gal(L'/\K)$.
We thus have
\begin{align*}
	\ord( (\kappa_1)_{\flw_1}, H^1_{ur}(K_{\flw_1}, T_{m'})\geq
	\ord( (\kappa_1)_{L}, H^1(L, T_{m'})\\
	\ord( (\eta_1)_{\flw_1}, H^1_{ur}(K_{\flw_1}, T_{m'}^*)\geq
	\ord( (\eta_1)_{L}, H^1(L, T_{m'}^*)
\end{align*}
We apply the same procedure then to
$\kappa_{\ell_1}=D_{\ell_1}\bar{z}_{\ell_1}\in H^1(\K,T_{m'})$
and $\eta_2$,
we can pick $\ell_2\in \mathcal{L}_m$ such that
$\eta_i\in H^1_{ur}(K_{\flw_2},T_{m'}^*)$ for all $i$ and
\begin{align*}
	\ord( (\kappa_{\ell_1})_{\flw_2}, H^1_{ur}(K_{\flw_2}, T_{m'})\geq
	\ord( (\kappa_{\ell_1})_{L}, H^1(L, T_{m'})\\
	\ord( (\eta_2)_{\flw_2}, H^1_{ur}(K_{\flw_2}, T_{m'}^*)\geq
	\ord( (\eta_2)_{L}, H^1(L, T_{m'}^*)
\end{align*}
We proceed further and get a collection 
$\Sigma=\{\ell_1,\cdots,\ell_k\}$ of primes in $\mathcal{L}_m$
such that the classes  $\eta_i$ are unramified
at all places  $\flw_i$ and that
\begin{align*}
	\ord( (\kappa_{\fc_{i-1}})_{\flw_i}, H^1_{ur}(K_{\flw_i}, T_{m'})\geq
	\ord( (\kappa_{\fc_{i-1}})_{L}, H^1(L, T_{m'})\\
	\ord( (\eta_i)_{\flw_i}, H^1_{ur}(K_{\flw_i}, T_{m'}^*)\geq
	\ord( (\eta_i)_{L}, H^1(L, T_{m'}^*)
\end{align*}
where $\fc_{i-1}\coloneqq \ell_1\cdots\ell_{i-1}$, for all $1\leq i\leq k$.

\begin{lem}
	Let $\Omega/\K$ be the extension corresponding
	to $R_{\infty}$, then the cohomology groups
	$W^\K$, $H^1(\Omega/\K, W)$ and $H^1(\Omega/\K, W^*)$ 
	are all trivial under the assumption \eqref{cond:distinct}.
\end{lem}
\begin{proof}
	Consider the finite extension $L=\ker(\fF(\psi))$
	and let $\Delta=\Gal(L/\K)$.
	Then  $p\nmid \#\Delta$ 
	and $\fF^\Delta=0$ by the assumption  \ref{cond:distinct}.
	As the conjugation action of $\Delta$ acts trivially
	on the abelian group $\Gal(\Omega/\K)$, we have
	\[
		H^1(\Omega/\K,\fF)=
		\Hom(\Gal(\Omega/L),\fF)^\Delta=
		\Hom(\Gal(\Omega/L),\fF^\Delta)=0
	\]
	By the exact sequence
	$0\to\fF\to W\xrightarrow{\varpi}W\to 0$
	this implies that $H^1(\Omega/K,W)$
	has no nontrivial $\varpi$-torsion, 
	which is possible only if  $H^1(\Omega,W)=0$.
	The same argument works also for $H^1(\Omega,W^*)$.
\end{proof}

Now, given a postive integer $n$,
we pick  $m$ sufficiently large so that
\[
p^{m'}> p^n+(k+1)p^{w_p+2s}+\textnormal{ind}(z_{\id})
\]
By the lemma above,dd
if we define $d_i=\ord(\kappa_{\fc_{i},m'}, H^1(\K,T_{m'}))$, then
\[
	d_i=\ord(\kappa_{\fc_{i},m'}, H^1(\K,T_{m'}))=
	\ord( (\kappa_{\fc_{i},m'})_\Omega, H^1(\Omega,T_{m'}))
\]
We have $d_0=p^{m'}-\textnormal{ind}(z_{\id})>p^n+(k+1)p^{w_p+2s}$.
And for $0\leq i\leq k$ we have
\[
	d_i\geq 
	\ord( (\kappa_{\fc_i,m'})_{\flw_i}, H^1_s(\K_{\flw_i}, T_{m'}))\geq
	\ord( (\kappa_{\fc_{i-1},m'})_{\flw_i}, H^1_f(\K_{\flw_i}, T_{m'}))-p^{w_p+2s}\geq 
	d_{i-1}-p^{w_p+2s}.
\]
In particular $d_i\geq p^n$ for all  $i$.

Pick  $\kappa_i\in H^1(\K, T_{n})$ whose image 
is $p^{d_i-n}\kappa_{\fc_i,m'}$
and let $A^{(i)}\subset H^1(\K,T_{n})$
be the submodule generated by $\kappa_0,\cdots,\kappa_i$.
Then 
\[
	\ord(\textnormal{loc}_\Sigma(A^{(i)})/\textnormal{loc}_\Sigma(A^{(i-1)}))
	\geq p^n+d_{i-1}-d_{i}
\]
and therefore
$\ord(\textnormal{loc}_\Sigma(S^{\Sigma\cup \Sigma_p}(\K, T_{n})) 
\geq kn+d_0-d_{k}\geq kn-\textnormal{ind}(z)$.
Therefore $\ord(\coker(\textnormal{loc}_\Sigma))\leq \textnormal{ind}(z)$
as desired.


\bibliographystyle{amsalpha}
\bibliography{biblio}
\end{document}
