\documentclass[leqno]{amsart}
\usepackage{amssymb}
\usepackage{amsmath} 
\usepackage{enumitem}
\usepackage{hyperref}
\usepackage{mathrsfs}
\usepackage{color}
\usepackage{mathtools,caption,bbm,euscript}
\usepackage[table,dvipsnames]{xcolor}
\usepackage{tikz-cd}
\usepackage[utf8]{inputenc}
\usepackage[OT2,T1]{fontenc}
\hypersetup{
 colorlinks=true,
 linkcolor=DarkOrchid,
 filecolor=blue,
 citecolor=olive,
 urlcolor=orange,
 pdftitle={Paskunas' theory},
 %pdfpagemode=FullScreen,
 }
\usepackage{booktabs}

%[label=(\alph*)]
%[label=(\Alph*)]
%[label=(\roman*)]
%[label={(\bfseries R\arabic*)}]


\setlength{\textwidth}{\paperwidth}
\addtolength{\textwidth}{-2in}
\calclayout


\newcommand{\smat}[1]{\left( \begin{smallmatrix} #1 \end{smallmatrix} \right)}
\newcommand{\mat}[1]{\left( \begin{smallmatrix} #1 \end{smallmatrix} \right)}
\newcommand{\dBr}[1]{\llbracket{#1}\rrbracket}
\newcommand{\leg}[2]{\left(\frac{#1}{#2}\right)}

% double bracket
\makeatletter
\newsavebox{\@brx}
\newcommand{\llangle}[1][]{\savebox{\@brx}{\(\m@th{#1\langle}\)}%
  \mathopen{\copy\@brx\kern-0.5\wd\@brx\usebox{\@brx}}}
\newcommand{\rrangle}[1][]{\savebox{\@brx}{\(\m@th{#1\rangle}\)}%
  \mathclose{\copy\@brx\kern-0.5\wd\@brx\usebox{\@brx}}}
  \newcommand{\llbracket}[1][]{\savebox{\@brx}{\(\m@th{#1[}\)}%
  \mathopen{\copy\@brx\kern-0.5\wd\@brx\usebox{\@brx}}}
\newcommand{\rrbracket}[1][]{\savebox{\@brx}{\(\m@th{#1]}\)}%
  \mathclose{\copy\@brx\kern-0.5\wd\@brx\usebox{\@brx}}}
\makeatother


\newcommand{\Gp}{\mathcal{G}_{\Qp}} %Galois group over \Qp
\newcommand{\laMod}{\textnormal{Mod}^{\textnormal{ladm}}}
\newcommand{\lfMod}{\textnormal{Mod}^{\textnormal{lfin}}}
\newcommand{\Ord}{\textnormal{Ord}}
\newcommand{\V}{\check{\mathbf{V}}} %Colmez


\newcommand{\phk}{\mathcal{H}_{\wt{k}}} %Hecke algebra
\newcommand{\ordp}{e}


\newcommand{\vk}{v_{-\wt{k}}} % lowest weight vector
\newcommand{\lk}{\mathnormal{l}_{\wt{k}}} % projection to lowest weight
\newcommand{\blk}{\mathnormal{l}_{\bwt{k}}} % projection to lowest weight
\newcommand{\Lk}{L_{\wt{k}}} 
\newcommand{\pf}{\hat{f}} % algebraic modular form
\newcommand{\pfk}{\hat{f}_{\wt{k}}}
\newcommand{\euF}{\EuScript{F}} % Hida family
\newcommand{\B}{\mathbf{B}} % pairing of algebraic modular form


\newcommand{\ww}{\boldsymbol{\omega}}
\DeclareMathOperator{\an}{an}
\DeclareMathOperator{\ord}{ord}
\DeclareMathOperator{\cts}{cts}
\DeclareMathOperator{\FJ}{FJ}


%%% Linear algebraic groups
\DeclareMathOperator{\GL}{GL}
\DeclareMathOperator{\SL}{SL}
\DeclareMathOperator{\Sp}{Sp}
\DeclareMathOperator{\Mp}{Mp}
\DeclareMathOperator{\GSp}{GSp}
\DeclareMathOperator{\UU}{U}
\DeclareMathOperator{\GUU}{GU}
\DeclareMathOperator{\gl}{\mathfrak{gl}}
\DeclareMathOperator{\mtr}{tr}
\DeclareMathOperator{\diag}{diag}
\DeclareMathOperator{\Ad}{Ad}
\DeclareMathOperator{\vol}{vol}

\DeclareMathOperator{\val}{val}
\DeclareMathOperator{\Lie}{Lie}
\DeclareMathOperator{\Pol}{Pol_p}

%%% Adelic rings
\newcommand{\Q}{{\mathbf{Q}}}
\newcommand{\Z}{{\mathbf{Z}}}
\newcommand{\Qp}{\mathbf{Q}_p}
\newcommand{\Zp}{\mathbf{Z}_p}
\newcommand{\Ql}{\mathbf{Q}_\ell}
\newcommand{\Zl}{\mathbf{Z}_\ell}
\newcommand{\R}{\mathbf R}
\newcommand{\C}{\mathbf C}
\newcommand{\A}{\mathbf A}
\newcommand{\hZ}{{\hat{\mathbf{Z}}}}
\newcommand{\dd}{\mathfrak{d}} %different
\newcommand{\DD}{\mathcal{D}}  %discriminant

\newcommand{\arch}{\mathbf{a}}
\newcommand{\fin}{\mathbf{h}}

\newcommand{\F}{{\mathcal{F}}} %global 
\newcommand{\OF}{{\mathcal{O}_{\F}}}
\newcommand{\K}{{\mathcal{K}}} %global quadratic
\newcommand{\OK}{\mathcal{O}_{\K}}
\newcommand{\kk}{F} %local
\newcommand{\E}{E} %local quadratic


\DeclareMathOperator{\Sel}{Sel}
\DeclareMathOperator{\Gal}{Gal}
\DeclareMathOperator{\Nr}{\mathsf{N}}
\DeclareMathOperator{\Tr}{Tr}
\newcommand{\qch}{\epsilon} % quadratic character of K/F


%%% Fonts
\newcommand{\oeu}{\EuScript{O}}
\newcommand{\eeu}{\EuScript{E}}
\newcommand{\feu}{\EuScript{F}}
\newcommand{\geu}{\EuScript{G}}
\newcommand{\keu}{\EuScript{K}}

\newcommand{\oo}{\mathcal O}
\newcommand{\bs}{\mathcal S}
\newcommand{\id}{\mathbf{1}}

\newcommand{\1}{\mathbf{1}} 
\newcommand{\bfe}{\mathbf e}
\newcommand{\bff}{\mathbf f}

\newcommand{\bX}{\mathbb{X}}
\newcommand{\bY}{\mathbb{Y}}
\newcommand{\bV}{\mathbb{V}}
\newcommand{\bW}{\mathbb{W}}

\newcommand{\fa}{\mathfrak a}
\newcommand{\fg}{\mathfrak g}
\newcommand{\fc}{\mathfrak c}
\newcommand{\fC}{\mathfrak C}
\newcommand{\fs}{\mathfrak s}
\newcommand{\fm}{\mathfrak m}
\newcommand{\fn}{\mathfrak n}
\newcommand{\fl}{\mathfrak l}
\newcommand{\fp}{\mathfrak p}
\newcommand{\bfp}{\overline{\mathfrak p}}
\newcommand{\fq}{\mathfrak q}
\newcommand{\bfq}{\overline{\mathfrak q}}

\newcommand{\btheta}{\boldsymbol{\theta}}
\newcommand{\bdelta}{\boldsymbol{\delta}}


\newcommand{\fG}{\mathfrak{G}}
\newcommand{\fX}{\mathfrak{X}}
\newcommand{\euW}{\EuScript{W}}


%%% Categorical
\DeclareMathOperator{\Ext}{Ext}
\DeclareMathOperator{\End}{End}
\DeclareMathOperator{\Hom}{Hom}
\DeclareMathOperator{\Inj}{Inj}
\DeclareMathOperator{\Isom}{Isom}
\DeclareMathOperator{\Aut}{Aut}
\DeclareMathOperator{\Ind}{Ind}
\DeclareMathOperator{\coker}{coker}
\DeclareMathOperator{\rank}{rank}
\DeclareMathOperator{\corank}{corank}


\DeclareMathOperator{\Res}{Res}
\DeclareMathOperator{\rec}{rec}



\newcommand{\bw}{{w^c}}
\newcommand{\ee}{\mathbf e}


\newtheorem*{theorem*}{Theorem}
\newtheorem{thm}{Theorem}[section]
\newtheorem{lem}[thm]{Lemma}
\newtheorem{prop}[thm]{Proposition}
\newtheorem{cor}[thm]{Corollary}


\theoremstyle{definition}
\newtheorem{definition}[thm]{Definition}
\newtheorem{defn}[thm]{Definition}
\theoremstyle{remark}
\newtheorem{rem}[thm]{Remark}
\newtheorem*{Remark*}{Remark}
\newtheorem{ack}{Acknowledgement}

\newcommand{\red}[1]{\textcolor{Red}{#1}}



\begin{document}
\title{Paskunas' theory}
\author[Y-S.~Lee]{Yu-Sheng Lee}
\address{Department of Mathematics, University  of Michigan, Ann Arbor, MI 48109, USA}
\email{yushglee@umich.edu}
\date{\today}

\maketitle
\setcounter{tocdepth}{1}
\tableofcontents




\section{Paskunas theory}

\subsection{generically reducible deformation}

Let $\chi_1,\chi_2\colon \Gp\to k^\times$ be continuous characters
such that $\chi_1\chi_2^{-1}\neq \id,\omega^{\pm1}$.
We recall 
the structure of the universal deformation ring $R$
of the $2$-dimensional pseudo-representation $\chi=\chi_1+\chi_2$ from \cite[\S B.1]{pask}.

The generic assumption that $\chi_1\chi_2^{-1}\neq \id,\omega^{\pm1}$
implies that there exists non-split extensions
\[
    0\to \chi_1\to \rho_{12}\to \chi_2\to 0\quad
    0\to \chi_2\to \rho_{21}\to \chi_1\to 0
\]
which are unique up to isomorphisms;
and that the universal deformation rings
$R_{ij}$ of the Galois representations $\rho_{ij}$
are formally smooth of relative dimension $5$ over $\oo$.

Denote by $\tilde{\rho_{ij}}$ the universal deformation,
one may choose bases and think of which as group homomorphisms
$\tilde{\rho_{ij}}\colon \Gp\to \GL_2(R_{ij})$
so that 
$\rho_{12}=\smat{\chi_1&*\\&\chi_2}$ and
$\rho_{21}=\smat{\chi_1&\\*&\chi_2}$.
Then trace induces $\theta\colon R\cong R_{ij}$ by \cite[Prop B.17]{pask}.
Since $R^{red}$ is formally smooth of relative dimension $4$ over  $\oo$,
the reducibility ideal  $\tau\subset R$ is a principal ideal generated by 
an element in $c\in\fm_R\setminus \fm_R^2$. 
Moreover, let $\tau_{ij}\subset R_{ij} $ be the ideal 
generated by the $(j,i)$-entry of  $ \tilde{\rho_{ij}}(g)$
for all $g\in \Gp$,
then  $\theta$ maps  $\tau$ to  $\tau_{ij}$ by \cite[Prop B.23]{pask}

Let $\tilde{\rho}_{12}^c\colon \Gp\to \GL_2(R_{12})$ be the representation defined by
\[
	\tilde{\rho}_{12}^c(g)\coloneqq 
	\smat{\theta(c)&\\&1}
	\tilde{\rho}_{12}(g)
	\smat{\theta(c)^{-1}&\\&1}
\]
which is a deformation of $\rho_{21}$ to $R_{12}$
and induces an isomorphism $\alpha\colon R_{21}\to R_{12}$,
for which the diagram
\[
	\begin{tikzcd}
		R_{21} \arrow[r,"\alpha"] &
		R_{12}\\
		R \arrow[u,"\theta"] \arrow[r,equal] &
		R \arrow[u,"\theta"]
	\end{tikzcd}
\]
commutes by \cite[Prop B.24]{pask}.

Identify $\tilde{\rho}_{21}$ and $\tilde{\rho}_{12}^c$,
it can be shown that 
$\Hom_{\Gp}(\tilde{\rho}_{12}, \tilde{\rho}_{21})$ and
$\Hom_{\Gp}(\tilde{\rho}_{21}, \tilde{\rho}_{12})$
are free modules over $R$ generated respectively by
\[
	\Phi_{12}=\smat{\theta(c)&\\&1} \text{ and }
	\Phi_{21}=\smat{1&\\&\theta(c)}.
\]
Consequently, the ring $\End_{\Gp}(\tilde{\rho}_{12}\oplus \tilde{\rho}_{21})$
is isomorphic to 
$\smat{R& R\Phi_{12}\\ R\Phi_{21}& R}$,
a free $R$-module of rank  $4$,
with the center isomorphic to  $R$
by \cite[Prop B.26]{pask}.

\subsection{Ordinary parts}

Let $\chi_1,\chi_2\colon \Qp^\times\to k^\times$
be smooth characters satisfying the generic assumption
$\chi_1\chi_2^{-1}\neq \id,\omega^{\pm1}$.
Let $\chi\colon T\to k^\times$
be the character  $\chi=\chi_1\otimes\chi_2\omega^{-1}$
and  $\chi^s\alpha=\chi_2\otimes \chi_1\omega^{-1}$, and
\[
	\pi_1\coloneqq \Ind_{P}^G\chi\cong\Ind_{P}^G\chi_1\otimes\chi_2\omega^{-1}\text{ and }
	\pi_2\coloneqq \Ind_{P}^G\chi^s\alpha\cong \Ind_{P}^G\chi_2\otimes\chi_1\omega^{-1}
\]
be the induced irreducible representations.
There exists the unique non-split extension
\[
	0\to \pi_1\to \kappa\to \pi_2\to 0
\]

Denote by $\tilde{P}_i$
the projective envelope of $\pi_i^\vee$
in  $\fC_{G,\zeta}(\oo)$;
any let $\tilde{M}_1, \tilde{M}_2$
be the Pontryagin dual of 
$\Ind_P^G\tilde{J}_\chi$ and $\Ind_P^G\tilde{J}_{\chi^s\alpha}$,
where $ \tilde{J}_\chi$ and $\tilde{J}_{\chi^s\alpha}$
are respectively the injective envelopes
of $\chi$ and  $\chi^s\alpha$ in  $\lfMod_{T,zeta}(\oo)$.
By \cite[Cor 7.7]{pask} there exists exact sequences 
\[
	0\to \tilde{P}_{2}\xrightarrow{\Phi_{12}} \tilde{P}_{1}\to \tilde{M}_1\to 0 \text{ and }
	0\to \tilde{P}_{1}\xrightarrow{\Phi_{21}} \tilde{P}_{2}\to \tilde{M}_2\to 0
\]
Furthermore, 
by \cite[Cor 7.2]{pask}
the composition of the projection $\tilde{P}_i\to \tilde{M}_i$
with any endomorphism of  $\tilde{P}_i$
factors through $\tilde{M}_i$, inducing a ring homomorphism
\[
	\End_{\fC_{G,\zeta}(\oo)}(\tilde{P}_i)\twoheadrightarrow
	\End_{\fC_{G,\zeta}(\oo)}(\tilde{M}_i)\cong \oo\llbracket x,y\rrbracket
\]




\begin{lem}
	Let $\fa=\ker(\tilde{E}_1\to \tilde{E})$,
	then $\fa$ is isomorphic to the reducibility ideal  $\tau$
	under the isomorphism  $ \tilde{E}_1\cong R$.
	Moreover, the 
\[
	\begin{tikzcd}
		0 &
		\tilde{M}_2 \arrow[l] \arrow[d] &
		\tilde{P}_2 \arrow[l] \arrow[d,"\Phi_{12}"] &
		\tilde{P}_1 \arrow[l,"\Phi_{21}"] \arrow[d,equal] & 0 \arrow[l]\\
		0 &
		\tilde{P}_1/\fa\tilde{P}_1 \arrow[l] \arrow[d] &
		\tilde{P}_1 \arrow[l] \arrow[d] &
		\tilde{P}_1 \arrow[l] & 0 \arrow[l]\\
				      & \tilde{M}_1 
				      & \tilde{M}_1 \arrow[l,equal] & & 
	\end{tikzcd}
\]
\end{lem}

Emerton has defined a functor
$\Ord_P\colon \laMod_{G,\zeta}(A)\to \laMod_{T,\zeta}(A)$
where $A$ is local complete Noetherian  $\oo$-algebra
with a finite residue field satisfying 
\[
	\Hom_{A[G]}(\Ind_{\bar{P}}^GU,V)\cong \Hom_{A[T]}(U, \Ord_P(V))
\]

Let $\chi\colon T\to k^\times$ be a smooth character
such that  $\chi|_Z=\zeta$.
Let  $\iota\colon \Ind_{\bar{P}}^G\chi\hookrightarrow J$ 
be an injective envelope of $\Ind_{\bar{P}}^G\chi$
in $\laMod_{G,\zeta}(A)$, 
by \cite[Prop 7.1]{pask} the following hold
\begin{enumerate}[label=(\alph*)]
	\item $\Ord_P(\Ind_{\bar{P}}^G\chi)\hookrightarrow \Ord_P(J)$
		is an injective envelope of $\chi$ in  $\laMod_{T,\zeta}(A)$;
	\item the adjoint map $\Ind_{\bar{P}}^G(\Ord_P(J))\to J$ is injective;
	\item there exists a surjective ring homomorphism
		\[
			\End_{A[G]}(J)\twoheadrightarrow \End_{A[T]}(\Ord_P(J))
		\]
\end{enumerate}





\bibliographystyle{amsalpha}
\bibliography{biblio}
\end{document}
