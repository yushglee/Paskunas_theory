\documentclass[leqno]{amsart}
\usepackage{amssymb,stmaryrd,amsfonts}
\usepackage{amsmath} 
\usepackage{enumitem}
\usepackage{hyperref}
\usepackage{mathrsfs}
\usepackage{color}
\usepackage{mathtools,caption,euscript}
\usepackage[table,dvipsnames]{xcolor}
\usepackage{tikz-cd}
\usepackage[utf8]{inputenc}
\usepackage[OT2,T1]{fontenc}
\hypersetup{
 colorlinks=true,
 linkcolor=DarkOrchid,
 filecolor=blue,
 citecolor=olive,
 urlcolor=orange,
 pdftitle={Pask\={u}nas' theory},
 %pdfpagemode=FullScreen,
 }
\usepackage{booktabs}
%[label=(\alph*)]
%[label=(\Alph*)]
%[label=(\roman*)]
%[label={(\bfseries R\arabic*)}]


\setlength{\textwidth}{\paperwidth}
\addtolength{\textwidth}{-2in}
\calclayout

\newcommand{\smat}[1]{\left( \begin{smallmatrix} #1 \end{smallmatrix} \right)}
\newcommand{\euF}{\EuScript{F}} %Hida family
\newcommand{\fF}{\mathbb{F}} % fintie field
\newcommand{\M}{\mathbf{M}} % modular form
\newcommand{\bnu}{\mathbb{\nu}}
\newcommand{\wt}[1]{\underline{ #1 }}
\newcommand{\bwt}[1]{\underline{\mathbb { #1 }}}
\newcommand{\TT}{\mathbb{T}} % Hecke 
\newcommand{\fG}{\mathfrak{G}}
\newcommand{\fX}{\mathfrak{X}}
\newcommand{\GG}{\mathbf G} 
\newcommand{\Iw}{\textnormal{Iw}} 

\DeclareMathOperator{\Spec}{Spec}

\newcommand{\bw}{\overline{w}}
\newcommand{\flw}{\bar{\fl}}





%%% Block theory

\newcommand{\aMod}{\textnormal{Mod}^{\textnormal{adm}}}
\newcommand{\laMod}{\textnormal{Mod}^{\textnormal{l.adm}}}
\newcommand{\lfMod}{\textnormal{Mod}^{\textnormal{lfin}}}
\newcommand{\fgMod}{\textnormal{Mod}^{\textnormal{fg.aug}}}
\newcommand{\Ban}{\textnormal{Ban}^{\textnormal{adm}}}
\DeclareMathOperator{\Mod}{\textnormal{Mod}}
\DeclareMathOperator{\Rep}{Rep}
\newcommand{\B}{\mathfrak B} 
\newcommand{\fC}{\mathfrak{C}}
\DeclareMathOperator{\soc}{soc}
\DeclareMathOperator{\V}{\check{\mathbf{V}}} %Colmez


%%% p_adic Hodge

\newcommand{\Gp}{\mathcal{G}_{\Qp}} %Galois group over \Qp
\newcommand{\Fr}{\textnormal{Fr}} %geometric Frobenius
\newcommand{\frob}{\textnormal{frob}} %arithmetic Frobenius
\newcommand{\dR}{\textnormal{dR}}
\newcommand{\pst}{\textnormal{pst}}
\newcommand{\cris}{\textnormal{cris}}

\DeclareMathOperator{\Gal}{Gal}

\DeclareMathOperator{\Ord}{Ord}
\DeclareMathOperator{\Irr}{Irr}
\DeclareMathOperator{\WD}{WD}
\DeclareMathOperator{\rec}{rec}
\DeclareMathOperator{\Rec}{Rec}
\DeclareMathOperator{\Art}{Art}

\newcommand{\cont}{\textnormal{cont}}
\newcommand{\cts}{\textnormal{cts}}
\newcommand{\alg}{\textnormal{alg}}
\newcommand{\sm}{\textnormal{sm}}
\newcommand{\adm}{\textnormal{adm}}
\newcommand{\ps}{\textnormal{ps}}
\newcommand{\red}{\textnormal{red}}
\newcommand{\fin}{\textnormal{fin}}
\newcommand{\an}{\textnormal{an}}
\newcommand{\ord}{\textnormal{ord}}


%%% Linear algebraic groups
\DeclareMathOperator{\GL}{GL}
\DeclareMathOperator{\SL}{SL}
\DeclareMathOperator{\gl}{\mathfrak{gl}}
\DeclareMathOperator{\mtr}{tr}
\DeclareMathOperator{\diag}{diag}
\DeclareMathOperator{\Ad}{Ad}
\DeclareMathOperator{\vol}{vol}
\DeclareMathOperator{\Sym}{Sym}

\DeclareMathOperator{\Lie}{Lie}

\newcommand{\bs}{\mathcal{S}}
\newcommand{\id}{\mathbf{1}}

%%% Adelic rings
\newcommand{\Q}{{\mathbf{Q}}}
\newcommand{\Z}{{\mathbf{Z}}}
\newcommand{\Qp}{\mathbf{Q}_p}
\newcommand{\Zp}{\mathbf{Z}_p}
\newcommand{\Ql}{\mathbf{Q}_\ell}
\newcommand{\Zl}{\mathbf{Z}_\ell}
\newcommand{\R}{\mathbf R}
\newcommand{\C}{\mathbf C}
\newcommand{\A}{\mathbf A}
\newcommand{\dd}{\mathfrak{d}} %different
\newcommand{\DD}{\mathcal{D}}  %discriminant
\DeclareMathOperator{\Nr}{\mathsf{N}} %norm
\DeclareMathOperator{\Tr}{Tr} %trace

\newcommand{\arch}{\mathbf{a}}
\newcommand{\finite}{\mathbf{h}}

\newcommand{\F}{{\mathbf{F}}} %global field
\newcommand{\K}{{\mathbf{K}}} %global quadratic
\newcommand{\kk}{F} %local field
\newcommand{\E}{E} %local quadratic
\newcommand{\qch}{\epsilon} % quadratic character of K/F

\newcommand{\oo}{\mathcal{O}} %ring of integer
\DeclareMathOperator{\val}{val}

%%% Class Groups

\newcommand{\rg}[1]{\textnormal{Cl}_{#1}} %ray class gp
\newcommand{\rp}[1]{\mathfrak{C}_{#1}} %pro-p quot
\newcommand{\rs}[1]{H_{#1}} % sub of prop-p quot
\newcommand{\rk}[1]{\K({#1})} % ray class field

%%% Fonts

\newcommand{\oeu}{\EuScript{O}}
\newcommand{\eeu}{\EuScript{E}}
\newcommand{\feu}{\EuScript{F}}
\newcommand{\geu}{\EuScript{G}}
\newcommand{\keu}{\EuScript{K}}

\newcommand{\fa}{\mathfrak{a}}
\newcommand{\fc}{\mathfrak{c}}
\newcommand{\fg}{\mathfrak{g}}
\newcommand{\fk}{\mathfrak{k}}
\newcommand{\fs}{\mathfrak{s}}
\newcommand{\fm}{\mathfrak{m}}
\newcommand{\fn}{\mathfrak{n}}
\newcommand{\fl}{\mathfrak{l}}
\newcommand{\ff}{\mathfrak{f}}
\newcommand{\fp}{\mathfrak{p}}
\newcommand{\fq}{\mathfrak{q}}
\newcommand{\bfp}{\overlin{\mathfrak p}}
\newcommand{\bfq}{\overline{\mathfrak q}}

\newcommand{\btheta}{\mathbb{\theta}}
\newcommand{\bdelta}{\mathbb{\delta}}


%%% Categorical
\DeclareMathOperator{\End}{End}
\DeclareMathOperator{\Aut}{Aut}
\DeclareMathOperator{\Hom}{Hom}
\DeclareMathOperator{\Ext}{Ext}
\DeclareMathOperator{\Tor}{Tor}
\DeclareMathOperator{\Ind}{Ind}
\DeclareMathOperator{\cInd}{c-Ind}
\DeclareMathOperator{\nInd}{n-Ind}
\DeclareMathOperator{\coker}{coker}
\DeclareMathOperator{\Image}{Im}
\DeclareMathOperator{\rank}{rank}
\DeclareMathOperator{\corank}{corank}
\DeclareMathOperator{\Res}{Res}
\DeclareMathOperator{\res}{res}




\newtheorem{thm}{Theorem}[section]
\newtheorem{lem}[thm]{Lemma}
\newtheorem{prop}[thm]{Proposition}
\newtheorem{cor}[thm]{Corollary}


\theoremstyle{definition}
\newtheorem{defn}[thm]{Definition}


\theoremstyle{remark}
\newtheorem{rem}[thm]{Remark}
\newtheorem{ack}{Acknowledgement}




\begin{document}
\title{Pask\={u}nas' theory}
\author[Y-S.~Lee]{Yu-Sheng Lee}
\address{Department of Mathematics, University  of Michigan, Ann Arbor, MI 48109, USA}
\email{yushglee@umich.edu}
\date{\today}

\maketitle
\setcounter{tocdepth}{1}
\tableofcontents

\section{Notations}

Throughout the article, $\F$ is a totally real field
and $\K$ is a totally imaginary quadratic extension over $\F$.
Denote by $\arch=\Hom(\F, \C)$ 
the set of archimedean places of $\F$,
and by $\fin$ the set of finite places of $\F$.
Let $\dd_\K$ and $\dd_{\K/\F}$ denote respectively 
the absolute and the relative ideals of different in $\K$,
recall that $\dd_\K=\dd_{\K/\F}\dd_\F$,
where $\dd_\F$ denotes 
the absolute ideals of different in $\F$.

We fix an odd prime $p$ throughout the article
and assume that $p$ is prime to the class number $h_\K$,
the number of roots of unity in $\K$,
and satisfies the following ordinary condition.
\begin{equation}\label{cond:ord}\tag{ord}
\text{Every finite place of $\F$ above $p$ is split in $\K$}.
\end{equation}
We fix an embedding $\iota_\infty:\bar{\Q}\to \C$
and an isomorphism $\iota:\C\cong \C_p$,
and write $\iota_p=\iota\circ\iota_\infty:\bar{\Q}\to \C_p$.


Given a place $v$ of $\F$, archimedean or finite,
let $w\mid v$ denote a place $w$ of $\K$ above $v$.
Then $\K_w$ and $\F_v$ are respectively
the completions of the fields $\K$ and $\F$ at $w$ and $v$.
When $v\in \fin$ we denote by $\oo_w$ and $\oo_v$ 
the rings of integers of $\K_w$ and $\F_v$.
Let $|\cdot|_v$ be the norm on $\F_v$,
which is the usual absolute value when $v\in \arch$
and $q_v=|\varpi_v|_v^{-1}$,
for any choice of uniformizer $\varpi_v$ in $\oo_v$,
is the cardinality of the residue field $\oo_v/(\varpi_v)$
when $v\in \fin$.
For $w\mid v$, define $|a|_w=|\Nr_{\K_w/\F_v}(a)|_v$.


Denote by $\A=\A_{\F}$ the ring of adeles over $\F$,
by $\A_{\infty}$ and $\A_{f}$ respectively
the archimedean and the finite components of $\A$.
Let $\qch_{\K/\F}$ denote 
the quadratic character on $\A_\F^\times/\F^\times$
associated to $\K/\F$ by the global class field theory,
$\qch_v$ denote the component on $\F_v^\times$ 
when $v\in \fin$.
If $\eta$ is a character of $\A_\K^1/\K^1$, 
we denote
by $\tilde{\eta}(\alpha)\coloneqq \eta(\alpha/\alpha^c)$
the Hecke character which is the base change of $\eta$ 
to $\A_\K^\times/\K^\times$.

\subsection{CM types}

Denote respectively by $S_p$ and $S_p^\K$ the set of places above $p$
of $\F$ and $\K$.
Identify $I_\K=\Hom(\K,\bar{\Q})$ with
$\Hom(\K,\C)$ and $\Hom(\K,\C_p)$ by compositions with $\iota_\infty$ and $\iota_p$.
Given $\sigma\in I_\K$,
let $w_\sigma\in S_p^\K$ be the place induced by
$\sigma_p\coloneqq \iota_p\circ \sigma\in\Hom(\K,\C_p)$.
For $w\in S_p^\K$, define
\[
    I_w=\{\sigma\in I_\K\mid w=w_\sigma \}=\Hom(\K_w,\C_p)
\]
and decompose $I_\K=\sqcup_{w\mid p}I_w$.
For a subset $\Sigma\subset I_\K$
define $\Sigma_p=\{w_\sigma\mid \sigma\in \Sigma\}$.
We write
$\Sigma^c=\{\sigma c\mid \sigma\in \Sigma\}$ and 
$\Sigma_p^c=\{cw\mid w\in \Sigma_p\}$.
We fix throughout the article a $p$-ordinary CM type,
which is a subset $\Sigma\subset I_\K$ such that
\[
    \Sigma\sqcup \Sigma^c=I_\K,\quad
    \Sigma_p\sqcup \Sigma_p^c=S_p^\K.
\]
The $p$-ordinary CM type $\Sigma$
always exists by the assumption \eqref{cond:ord},
and is identified with $\arch=\Hom(\F,\C)$ by restrictions.
When $v\in S_p$ decomposes into $v=w\bw$,
we understand always that $w\in \Sigma_p$.

\subsection{Characters}

Given 
$\kappa=\sum_{\sigma\in \Sigma} a_\sigma\sigma+b_\sigma\sigma c\in \Z[I_\K]$,
an algebraic Hecke character 
$\chi\colon \A_\K^\times/\K^\times\to \C^\times$ 
has type $\kappa$ if
\[
    \chi_\infty(\alpha)=
    \iota_\infty \left(\prod_{\sigma\in \Sigma} 
    \sigma(\alpha)^{a_\sigma}\sigma(c \alpha)^{b_\sigma}\right),\quad
    \alpha\in \K^\times.
\]
For $\alpha_\infty=(\alpha_\sigma)\in \A_{\K,\infty}^\times$
and $\alpha_p=(\alpha_w,\alpha_{\bw})\in \prod_{v\in S_p}\K_v^\times$, 
define
\[
    \alpha_\infty^\kappa=
    \prod_{\sigma\in \Sigma} 
    (\alpha_\sigma)^{a_\sigma}(\bar{\alpha}_\sigma)^{b_\sigma}\in \C^\times,\quad
    \alpha_p^\kappa=
    \prod_{w\in \Sigma_p}
    \prod_{\sigma\in I_w}
    \sigma_p(\alpha_w)^{a_\sigma}\sigma_p(\alpha_{\bw})^{b_\sigma}\in \C_p^\times,
\]
for example,
$(2\pi)^\Sigma=(2\pi)^{[\F:\Q]}$
when $2\pi$ is embedded diagonally in $\A_{\K,\infty}^\times$.
Define the $p$-adic avatar of $\chi$ by
\[
    \hat{\chi}\colon \A_\K^\times\to \bar{\Z}_p^\times,\quad
    \hat{\chi}(\alpha)=\iota(\chi(\alpha)\alpha_\infty^{-\kappa})\alpha_p^{\kappa}
\]
where $\alpha_\infty$ and $\alpha_p$ are respectively 
the archimedean component and the components above $p$ of $\alpha\in \A_\K^\times$.


\subsection{Matrices}
When $R$ is an $\F$-algebra and 
$m=(m_{ij})\in \text{M}_{r,s}(\K\otimes_\F R)$,
we denote by 
$m^\intercal=(m_{ji}), 
m^c=(m^c_{ij})$, and
$m^*=(m^c_{ji})$
respectively the transpose, conjugate, and conjugate-transpose of $m$.

When $r=s$ and $g\in \GL_r(\K\otimes_\F R)$ is invertible, we write
$g^{-\intercal}=(g^{-1})^\intercal$ and $g^{-*}=(g^{-1})^*$.
We write $\mtr(m)$ for the trace of a square matrix $m$,
and reserve $\Tr$ for the traces between fields extensions.

When $v=w\bw$ is a place that is split in $\K$,
identify $\K_w=\F_v=\K_{\bw}$ and 
write $\K_v=\F_v^2$, 
where the first component corresponds to $\K_w$.
Then $m=(m_w,m_{\bw})\in M_n(\K\otimes_\F\F_v)=M_n(\F_v)\times M_n(\F_v)$ 
denotes an element in $m\in M_n(\K\otimes_\F\F_v)$ and its components.

\subsection{Representations of $p$-adic groups}

Let $\oo$ be the ring of integers of a finite extension $E$
over  $\Qp$.
When $G$ is a $p$-adic analytic group,
let $\Mod_G(\oo)$ be the category
of all $\oo[G]$-modules.
We refer the readers to \cite[\S 2]{emeI} and \cite[\S 2]{pask}
for the notations and definitions of the following subcategories.
\[
\begin{tikzcd}
	\fgMod_{G}(\oo) \arrow[r,leftrightarrow] &
	\laMod_{G}(\oo) \arrow[r,hookrightarrow] &
	\aMod_{G}(\oo) \arrow[r,hookrightarrow] &
	\Mod^{\sm}_{G}(\oo) \\
					       &&
	\lfMod_{G}(\oo) \arrow[ru,hookrightarrow] &
\end{tikzcd}
\]



\section{Modular forms on definite unitary groups}

Let $G$ be the definite unitary group over $\F$,
such that for any $\F$-algebra $R$
\[
    G(R)=\{g\in \GL_{n}(\K\otimes_\F R) \mid gg^*=\id_n\}.
\]
In this section we recall from \cite{ger}
the notion of algebraic modular forms on $G$
and results on the associated Galois representation.
Some results from \textit{loc.cit}
regarding Hida theory of ordinary forms
are generalized
to that of $P$-ordinary forms,
for a more general parabolic subgroup $P$,
after incorporating Emerton's functor in \cite{emeI}
We then introduce 
the big Hecke algebra acting on 
the completed cohomology of $P$-ordinary forms,
which admits a Galois pseudo-representation
of $\Gal_\K$.
Borrowing from the idea in \cite{pan},
we can deduce a density result
of crystalline points in the Hecke algebra,
which is crucial for checking 
the local-global compatibility in the next section.


\subsection{Algebraic modular forms}

When $v=w\bw$ is place of $F$ that is split in $\K$,
the group $G(\F_v)$ is a subgroup 
of  $\GL_n(\F_v\otimes_\F\K)\cong \GL_n(\K_w)\times\GL_n(\K_{\bw})$.
Write $g_v=(g_w,g_{\bw})\in G(\F_v)$,
the map $g_v\mapsto g_w$ then defines an isomorphism
$\iota_w\colon G(\F_v)\cong \GL_n(\F_v)$
such that 
$\iota_w(g_v)=\iota_{\bw}(g_v)^{-\intercal}$.
In particular,
for each $v\in S_p$
let  $w\mid v$ be such that  $w\in \Sigma_p$
and write $G_{w}\coloneqq\GL_n(\K_w)$,
we identify $G(\F_v)$ with $G_w$ 
via $\iota_w$ and define 
\[
	G_p\coloneqq\prod_{w\in \Sigma_p}G_w,\quad
	K_p\coloneqq\prod_{w\in \Sigma_p}K_w,\quad
	K_w\coloneqq\GL_n(\oo_w).
\]

Let $B_n\subset \GL_n$ be the subgroup of
upper trigiangular matrices
and $B_n=T_nN_n$ be the Levi decomposition,
where $T_n$ is the diagonal torus.
We identify the set of algebraic characters $X^*(T_n)$
with  $\Z^n$.
The Weyl group $W_n$ of $\GL_n$
then acts on  $\Z^n$ by
$(wk)(t)=k(w^{-1}tw)$.
Let $w_0\in W_n$ denote the longest element.

Following \cite[Def 2.3]{ger},
we say  $k=(k_1,\cdots,k_n)\in \Z^n$
is a dominant weight if $k_1\geq \cdots\geq k_n$.
In which case
$\xi_k\coloneqq \Ind_{B_n}^{\GL_n}(w_0k)$
is the algebraic representation 
of $\GL_n$ of highest weight $k$.
More generally,
we say $\wt{k}=(k_\sigma)\in (\Z^n)^{\Sigma}$
is a dominant 
if $k_\sigma=(k_{\sigma,1},\cdots,k_{\sigma,n})$
is a dominant weight for each $\sigma\in \Sigma$.
When this is the case,
and $\oo$ is the ring of integers  
of a finite extension $E$ over  $\Qp$
that contains $\iota_p(\sigma(\K))$
for all  $\sigma\in I_\K$,
we let $\xi_{\wt{k}}$ denote 
the algebraic $G_p$-representation over $\oo$ given by
\begin{equation}\label{def:algrep}
	\xi_{\wt{k}}=\bigotimes_{\sigma\in \Sigma}
	\Ind_{B_n}^{\GL_n}(w_0k_{\sigma}),\quad
	\xi_{\wt{k}}(g)=
	\otimes_{w\in \Sigma_p}
	\otimes_{\sigma\in I_w}\xi_{k_\sigma}(g_w)\,
	\text{ for } g=(g_w)\in G_p.
\end{equation}

\begin{rem}
	In previous work, 
	we have defined $\rho_k$ as the 
	algebraic representation of lowest weight  $-k$.
	Thus $\xi_k$ is isomorphic to the representation 
	$\rho^k(g)\coloneqq \rho_k(g^{-\intercal})$.
\end{rem}




\begin{defn}\label{def:algform}
Let $M_{\wt{k}}$ be the finite free $\oo$-module
on which $K_p$ acts by $\xi_{\wt{k}}$
if $\wt{k}\in (\Z^n)^{\Sigma}$ is dominant. 
For any $\oo$-module $M$ and any function
$f\colon G(\F)\backslash G(\A_f)\to M\otimes_{\oo}M_{\wt{k}}$,
let $k=(k^p,k_p)\in G(\A_f^p)\times K_p$ acts on $f$ by 
$(k\cdot f)(g)=\xi_{\wt{k}}(k_p)\cdot f(gk)$.
We say $f$ is an algebraic modular form of 
weight $\wt{k}$ and coefficients in $M$
if $f$ is invariant by some open compact subgroup
$U\subset G(\A_f^p)\times K_p$.
Let $S_{\wt{k}}(M)$
denote the space of such modular forms.
Then for $U$ as above we 
denote the subspace of forms
of level $U$ by
\begin{equation}
S_{\wt{k}}(U,M)=
S_{\wt{k}}(M)^U=
\left\{ f: G(\F)\backslash G(\A_f)/U^p 
\rightarrow M\otimes_{\oo}M_{\wt{k}}
\mid f(gu)=\xi_{\wt{k}}(u_p)^{-1}\cdot f(g), u\in U\right\} 
\end{equation}
Note that when $M$ is an  $E$-module,
the above action can be extended 
to  $G(\A_f)$
and the notation  $S_{\wt{k}}(U,M)$
makes sense for any open compact subgroup
$U\subset G(\A_f)$.

We will omit $\wt{k}$ 
when $\xi_{\wt{k}}$ is the trivial representation
and simply write
$S(M)$ and  $S(U,M)$.
\end{defn}


Since $\GG(\F)\backslash \GG(\A_f)/U$ is a finite set
for any open compact subgroup $U\subset G(\A_f)$,
any modular form $f\in S_{\wt{k}}(U,M)$ 
is determined by its values on a finite set of points.
Throughout the section,
we fix an open compact subgroup 
$U^p\subset G(\A_f^p)$ satisfying 
\begin{equation}\label{cond:small}\tag{$U^p$-\text{small}}
	\GG(\A_f)=\bigsqcup_{i\in I}
	\GG(\F)t_i U,\quad
	\GG(\F)\cap t_iUt_i^{-1}=\{1\} \text{ for all } i\in I
	\text{ for } U=U^pK_p
\end{equation}
Then the space $S_{\wt{k}}(U^pU_p,M)$ is
isomorphic to a finite direct sum of 
$M\otimes_{\oo}M_{\wt{k}}$
for any open compact subgroup $U^p\subset K_p$,
where the isomorphism
is given by 
evaluating at a set of representatives
for  $G(\F)\backslash G(\A_f)/U^pU_p$.

\subsection{Emerton's functor of ordinary parts}

Only in this subsection,
let $G$ denote a  $p$-adic reductive group,
$P$ be a parabolic subgroup,
with Levi decomposition $P=QU$.
We briefly recall the functor
of ordinary parts 
$\Ord_P\colon \Mod_G^{\sm}(\oo)\to \Mod_Q^{\sm}(\oo)$
defined in \cite{emeI}.
Fix an open compact subgroup $P_0\subset P$.
Let  $Q_0=P_0\cap Q$ and $U_0=P_0\cap U$
and define $Z_Q^+=Z_Q\cap Q^+$,
where  $Z_Q$ is the center of $Q$ and
\[
	Q^+=\{m\in Q\mid mU_0m^{-1}\subset U_0\}.
\]
If  $V$ is a $P$-representation of over $\oo$
and  $m\in Q^+$,
as in \cite[Def 3.1.3]{emeI} we define
\begin{equation}\label{def:hUm}
	 h_{U}(m)\colon V^{U_0}\to V^{U_0}\quad
	 h_{U}(m)(v)=\sum_{u\in N_0/m U_0 m^{-1}}um\cdot v
\end{equation}
We now recall the definition of the functor
and refer to \cite[Def 3.1.3]{emeI}
for the notations in below.
\begin{equation}\label{def:OrdP}
	\Ord_P\colon \Mod_G^{\sm}(\oo)\to \Mod_Q^{\sm}(\oo)
	\Ord_P(V)=\Hom_{\oo[Z_Q^+]}(\oo[Z_Q], V^{U_0})_{Z_Q-\fin}.
\end{equation}
Here $Z_Q^+$ acts by translation on the left; 
by $h_U$ on the right.
And the action of $Q=Z_Q\cdot Q^+$ is induced by 
havig $Z_Q$ act by translation on the left and 
$Q^+$ act by $h_U$ on the right.

\subsection{Hecke operators and $P$-ordinary forms}

We now return to the previous settings,
so $G$ is the definite unitary group over $\F$
and  $U^p$ satisfies \eqref{cond:small}.
Let $R=R(U^p)$ be a finite set of
finite places of $\F$ containing $S_p$ such that
$\iota_w^{-1}(\GL_n(\oo_w))\subset U^p$
when  $v\notin R$
and  $v=w\bw$ is split in  $\K$.

Recall that $B_n=T_nN_n$ is the Levi decomposition
for the group of upper triangular matrices in  $\GL_n$.
We put $B=\prod_{w\in \Sigma_p}B_w$
for $B_w=B_n(\K_w)$,
which admits decompositions 
$B=TN$ and  $B_w=T_wN_w$
where the subgroup  $T, N, T_w, N_w$ are defined similarly.
Given integers $c\geq b\geq 0$ with  $c>0$, 
we define 
the open compact subgroup 
$\Iw(p^{b,c})=\prod_{w\in \Sigma_p}\Iw(w^{b,c})$, where 
for each $w\in \Sigma_p$
\begin{equation}\label{def:Iwahori}
	\Iw(w^{b,c})=\{
	k\in K_w\mid 
	\iota_w(k) \text{ mod } \varpi_w^c \in B_n(\oo/\varpi_w^c)
	\text{ and }
	\iota_w(k) \text{ mod } \varpi_w^b \in N_n(\oo/\varpi_w^b)
	\}.
\end{equation}
In \cite{ger},
the Hecke operators on $S_{\wt{k}}(U^p\Iw(p^{b,c}),M)$
are defined as the following double-coset operators.

If $v=w\bw$ is split in  $\K$ and $v\neq R$,
for $1\leq j\leq n$ let 
\begin{equation}\label{def:hecke_away_p}
	T_w^{(j)}=
	\left[\iota_w^{-1}\left(
	\GL_n(\oo_v)
	\begin{pmatrix}
		\varpi_v\id_{j}&\\&\id_{n-j}
	\end{pmatrix}
	\GL_n(\oo_v)
	\right)\right],
	\text{ note that }
	T_{\bw}^{(j)}=(T_{w}^{{n}})^{-1}T_w^{(n-j)}.
\end{equation}

If $w\in \Sigma_p$, let  
$\alpha_w^{(j)}=\iota_w^{-1}
\left(\begin{smallmatrix}
\varpi_v\id_{j}&\\&\id_{n-j} 
\end{smallmatrix}\right)$ for $1\leq j\leq n$,
and $u\in \iota_w^{-1}(T_n(\oo_w))$, define
\begin{equation}\label{def:hecke_at_p}
	U_{\wt{k},w}^{(j)}=
	(w_0\wt{k})^{-1}(\alpha_{w}^{(j)})\cdot
	[\Iw(p^{b,c})\alpha_w^{(j)}\Iw(p^{b,c})]
	\text{ and }
	\langle u\rangle= (w_0\wt{k})^{-1}(u)\cdot 
	[\Iw(p^{b,c})u\Iw(p^{b,c})].
\end{equation}
Here $w_0\wt{k}$ is viewed as an algebraic character of $T$ 
using the same recipe as in \eqref{def:algrep}.
\begin{rem}
	Our definition of $\langle u\rangle$
	is different from that of \cite{ger}
	by the factor $(w_0\wt{k})^{-1}(u)$.
\end{rem}
Put $B_0=B\cap K_p$,
and $T^+, N_0$ be as in last subsection.
Then $N_0T^+$ is a monoid
and we can define the following action of which
on  $S_{\wt{k}}(U^p\Iw(p^{b,c}),M)$
extending the trivial action of $N_0$
\begin{equation}\label{def:T_act}
	(nt\cdot f)(g)=(w_0\wt{k})^{-1}(t)\xi_{\wt{k}}(nt)\cdot f(gnt).
\end{equation}
The same formula \eqref{def:hUm}
then defines an endomorphism $h_N(t)$
for $S_{\wt{k}}(U^p\Iw(p^{b,c}),M)$
for $t\in T^+$.
In particular we have
$U_{\wt{k},w}^{(j)}=h_N(\alpha_{w}^{(j)})$ and $\langle u\rangle= h_N(u)$
for $\alpha_w^{(j)}$ and $u$ as in \eqref{def:hecke_at_p}. 


More generally, 
we may pick a standard parabolic subgroup
$P_w\supset B_w$ for each  $w\in \Sigma_p$
and let  $P_w=Q_wU_w$ be the Levi decomposition.
We then put $P=\prod_{w\in \Sigma_p}P_w, P_0=P\cap K_p$,
and define the subgroups $Q,U, Q_0, U_0$ similarly.
For $w\in \Sigma_p$,
let  $\Iw^P(w^{b,c})$ 
be defined as in \eqref{def:Iwahori}
upon replacing $B_n$ and $N_n$
with $P_w$ and  $U_w$.

\begin{defn}\label{def:hecke}
Let $\Iw^P(p^{b,c})=\prod_{w\in \Sigma_p}\Iw^P(w^{b,c})$
for integers  $c\geq b\geq 0$ with $c>0$
and define the action of $U_0T^+$ 
on $S_{\wt{k}}(U^p\Iw^P(p^{b,c}),M)$
extending the trivial action of $U^0$ as in \eqref{def:T_act}.
We define the Hecke operators $T_w^{(j)}$
as in \eqref{def:hecke_away_p}; and 
$U_{\wt{k},w}^{(j)}=h_U(\alpha_w^{(j)}),\langle u\rangle=h_U(u)$,
for $\alpha_w^{(j)}$ and $u$ as in \eqref{def:hecke_at_p}.
We also define $U_P$
as the product of all  $U_{\wt{k},w}^{(j)}$
for which $\alpha_w^{(j)}\in Z_Q$.
\end{defn}


\begin{lem}
The Hecke operators defined above commutes with each other
and are equivariant with respect to the inclusions
$ S_{\wt{k}}(U^p\Iw^P(p^{b,c}),M)\hookrightarrow
S_{\wt{k}}(U^p\Iw^P(p^{b',c'}),M)$
if $b'\geq b$ and $c'\geq c$.
\end{lem}
\begin{proof}
That each $T_w^{(j)}$ commutes with other Hecke operators is classical,
and the equivariance is clear.
For the Hecke operators at  $w\in \Sigma_p$,
the commutivity follows from \cite[Lem 3.1.4]{emeI},
and the equivariance follows from 
that of the action \eqref{def:T_act}.
See also \cite[Lem 2.10]{ger} for the proof when $P=B$.
\end{proof}


When $M$ is either a finite  $\oo$-module
or the Pontryagin dual of which,
the operator $e_P\coloneqq\lim_{n\to \infty}(U_P)^{n!}$
converges to an idempotent 
on $S_{\wt{k}}(U^p\Iw^P(p^{b,c}),M)$.
We then define the space of $P$-ordinary forms by
\[
	S_{\wt{k}}^{P-\ord}(U^p\Iw^P(p^{b,c}),M)\coloneqq
	e_PS_{\wt{k}}(U^p\Iw^P(p^{b,c}),M)
\]
and $S_{\wt{k}}^{P-\ord}(U^p\Iw^P(p^{b,c}),E)\coloneqq 
S_{\wt{k}}^{P-\ord}(U^p\Iw^P(p^{b,c}),\oo)\otimes_{\oo}E$.
Alternatively,
$S_{\wt{k}}^{P-\ord}(U^p\Iw^P(p^{b,c}),M)$
can be defined as the subspace on which the action of
any  $U_{\wt{k},w}^{(j)}$ such that 
$\alpha_w^{(j)}\in Z_Q$ is invertible.
Note that when $P=B$ the definition coincides with 
that of ordinary forms in \cite[Def 2.13]{ger}.

\begin{defn}\label{def:ord_hecke}
	We let $\TT^P_{\wt{k}}(U^p\Iw^P(p^{b,c}),M)$
	be the $\oo$-subalgebra in 
	$\End_{\oo}S_{\wt{k}}(U^p\Iw^P(p^{b,c}),M)$
	generated by all
	$T_w^{(j)}$, for $1\leq j\leq n$,
	and $(T_w^{(n)})^{-1}$ at $w\mid v\notin R$;
	and all $U_{\wt{k},w}^{(j)}$ 
	for which $\alpha_w^{(j)}$  belongs to  $Z_Q$
	and all $\langle u\rangle$.
	We also define 
	$\TT^P_{\wt{k}}(U^p\Iw^P(p^{b,c}),E)\cong
	\TT^P_{\wt{k}}(U^p\Iw^P(p^{b,c}),\oo)\otimes_{\oo}E$,
	which acts faithfully on 
	$S_{\wt{k}}^{P-\ord}(U^p\Iw^P(p^{b,c}),E)$.
\end{defn}

\begin{lem}\label{lem:control}
	For $M$ as in the definition
	the inclusions below are isomorphisms.
	\begin{align*}
	&S_{\wt{k}}^{P-\ord}(U^p\Iw^P(p^{b,b}),M)\hookrightarrow	
	S_{\wt{k}}^{P-\ord}(U^p\Iw^P(p^{b,c}),M)\quad 
	\text{ for } c\geq b\geq 1\\
	&S_{\wt{k}}^{P-\ord}(U^p\Iw^P(p^{0,1}),M)\hookrightarrow	
	S_{\wt{k}}^{P-\ord}(U^p\Iw^P(p^{0,c}),M)\quad \text{ for } c\geq 1
	\end{align*}
\end{lem}
\begin{proof}
	It suffices to show that 
	$(U_P)^{n!}S_{\wt{k}}(U^p\Iw^P(p^{b,c}),M)
	\subset S_{\wt{k}}(U^p\Iw^P(p^{b,b}),M)$
	for $n$ sufficiently large. 
	Since $\Iw^P(p^{b,c})$ admits Iwahori decompositions,
	this follows from \cite[Lem 3.3.2]{emeI}.
	The same argument also applies to 
	$S_{\wt{k}}(U^p\Iw^P(p^{0,c}),M)$.
	See also \cite[Lem 2.19]{ger} for the proof when $P=B$.
\end{proof}

\begin{lem}\label{lem:PtoB}
	For any $b\geq 1$
	the inclusion 
	$S_{\wt{k}}^{B-\ord}(U^p\Iw(p^{b,b}),M)\subset
	S_{\wt{k}}^{P-\ord}(U^p\Iw^P(p^{b,b}),M)$
	is equivariant
	for Hecke operators in 
	Definition \ref{def:ord_hecke}.
	This induces a homomorphism of $\oo$-algebras
	\[
		\TT^P_{\wt{k}}(U^p\Iw^P(p^{b,b}),M)\to
		\TT^B_{\wt{k}}(U^p\Iw(p^{b,b}),M)
	\]
\end{lem}
\begin{proof}
	Since $\Iw^P(p^{b,b})\subset \Iw(p^{b,b})$,
	it suffices to show that the Hecke operators 
	are equivariant with respect to the natural inclusions
	$S_{\wt{k}}(U^p\Iw(p^{b,b}),M)\subset 
	S_{\wt{k}}(U^p\Iw^P(p^{b,b}),M)$.
	This is clear for $T_w^{(j)}$ and $\langle u\rangle$.
	And for  $U_{\wt{k},w}^{(j)}$ such that 
	$\alpha=\alpha_w^{(j)}\in Z_Q$, this follows from that 
	the following set of representatives for 
	$N_0/\alpha N_0\alpha^{-1}$
	\[
	\begin{pmatrix}
		\id_j&X\\&\id_{n-j}
	\end{pmatrix},\quad
	X \text{ runs through a set of representatives of }
	M_{j,n-j}(\oo_w/\varpi_w)
	\]
	as given in \cite[Lem 2.10]{ger}, is also
	a set of representatives for 
	$U_0/\alpha U_0\alpha^{-1}$.
\end{proof}

\subsection{Weights independence}

Let $P\subset G_p$ be a parabolic subgroup 
as in last subsection.
When $\wt{k}=(k_\sigma)\in (\Z^n)^{\Sigma}$ is dominant,
let $\pi_{k_{\sigma}}$
denote the algebraic $Q_w$-representation
$\Ind_{B_w\cap Q_w}^{Q_w}(\omega_0 k_\sigma)$
when $\sigma\in I_w$
and let  $\pi_{\wt{k}}$
be the $Q$-representation
as defined by \eqref{def:algrep},
which we extend to $P$ via the projection  $P=QU\to Q$.
The following proposition
is a generalization of \cite[Prop 2.22]{ger}
to $P$-ordinary forms.

\begin{lem}
	Let $\pi_{\wt{k}}^*$ be the contragredient
	representation and
	$\varpi$ be a uniformizer of $\oo$,
	For $b\geq 1$ let $A=\varpi^{-b}\oo/\oo$,
	then there exists an isomorphism
	\[
		\epsilon_{\wt{k}} \colon 
		S_{\wt{k}}^{P-\ord}(U^p\Iw^P(p^{b,b}),A)\cong 
		\Hom_{\oo}(\pi_{\wt{k}}^*(\oo),
		S^{P-\ord}(U^p\Iw^P(p^{b,b}),A)).
	\]
	Moreover, the isomorphism is equivariant 
	with respect to all the Hecke operators
	in Definition \ref{def:ord_hecke}
	and the following action of $u\in Q_0$
	\begin{align*}
	&u\cdot F(g)=\xi_{\wt{k}}(u)\cdot F(gu),\quad
	F(g)\in S_{\wt{k}}^{P-\ord}(U^p\Iw^P(p^{b,b}),A)\\
	&u\cdot \phi(v^*)(g)=
	\phi(\pi^*_{\wt{k}}(u^{-1})\cdot v^*)(gu),\quad
	\phi\in \Hom_{\oo}(\pi_{\wt{k}}^*(\oo),
	S^{P-\ord}(U^p\Iw^P(p^{b,b}),A))
	\end{align*}
\end{lem}

\begin{proof}
	By inductions in steps
	we can fix an isomorphism 
	$\xi_{\wt{k}}\cong \Ind_{P}^{G_p}\pi_{\wt{k}}$.
	Let $ev\colon \xi_{\wt{k}}\to \pi_{\wt{k}}$
	be the evaluation at the identity.
	For $F(g)\in S_{\wt{k}}(U^p\Iw^P(p^{b,b}),A)$,
	we define 
	$\epsilon_{\wt{k}}(F)$ as 
	\begin{equation}\label{eq:wt_indep}
	\epsilon_{\wt{k}}(F)\colon 
	\pi^*_{\wt{k}}(\oo)\rightarrow
	S(U^p\Iw^P(p^{b,b}),A)\quad
	v^*\mapsto [g\mapsto v^*(ev(F(g)))].
	\end{equation}
	By the assumption on $b$,
	the action of $\Iw^P(p^{b,b})$ on 
	$A\otimes_{\oo}\pi_{\wt{k}}(\oo)$
	is trivial.
	Thus the function defined above is indeed 
	a modular form
	of trivial weight.
	It is also straightforward to verify
	that the map is equivariant with respect
	to the Hecke operators and the $Q_0$-action.


	To construct the reversed map,
	note that if $\mu$ is a weight character of $T$ in  
	$\pi_{\wt{k}}$, then it is also a weight character 
	of $T$ in $\xi_{\wt{k}}$.
	We fix weight vectors $v_\mu\in \xi_{\wt{k}}$
	and $v^*_\mu\in \pi_{\wt{k}}^*$
	such that $v^*_{\mu}(ev(v_\mu))=1$.
	Now, let $\alpha_P\in Z_Q^+$ be the product
	of all $\alpha_w^{(j)}\in Z_Q$, $\alpha=\alpha_P^r$,
	and $\{x_i\}_{i\in I}$
	be a set of represntatives 
	for $U_0/\alpha U_0\alpha^{-1}$,
	we put 
	\begin{align*}
		\varphi\colon 
		\Hom_{\oo}(\pi_{\wt{k}}^*(\oo),&
		S(U^p\Iw^P(p^{b,b}),A))\longrightarrow
		S_{\wt{k}}(U^p\Iw^P(p^{b,b}),A)\\
		\phi&\mapsto 
		F_\phi(g)=\sum_{i\in I} \sum_{\mu}
		\xi_{\wt{k}}(x_i)\cdot 
		\phi(v^*_\mu)(gx_i\alpha)v_\mu
	\end{align*}
	where $\mu$ runs through the weight characters in 
	$\pi_{\wt{k}}$.
	To show that the resulting function 
	defines a modular form,
	let $u\in \Iw^P(p^{b,b})$, 
	then as explained in \cite[Prop 2.22]{ger}
	there exists a bijection $i\mapsto i'$ of $I$
	such that 
	 \[
		ux_i=x_{i'}v_i,\quad
		v_i\in\alpha\Iw^P(p^{b,b})\alpha^{-1} 
		\cap \Iw^P(p^{b,b})
	\]
	Since each $v_i$ is reduced to the identity matrix 
	modulo $\varpi^r$ and thus acts trivially on 
	$\xi_{\wt{k}}(A)$,
	\[
		\xi_{\wt{k}}(u)\cdot F_\phi(gu)=
		\sum_{i\in I}\sum_{\mu}
		\xi_{\wt{k}}(x_i'v_i)\cdot 
		\phi(v^*_\mu)(gx_i'v_i\alpha)v_\mu=
		\sum_{i\in I}\sum_{\mu}
		\xi_{\wt{k}}(x_i')\cdot 
		\phi(v^*_\mu)(gx_i'\alpha)v_\mu=F_\phi(g)
	\]
	and indeed $F_\varphi(g)\in 
	S_{\wt{k}}(U^p\Iw^P(p^{b,b}),A)$.

	At last, we observe that for each $\mu$ 
	the composition
	$\epsilon_{\wt{k}}(F_\phi)$ is the homomorphism
	\[
		v_\mu^*\mapsto \sum_{i\in I}\phi(v_\mu^*)
		(gx_i\alpha) =U_P^r\phi(v_\mu^*)(g)
	\]
	On the other hand 
	if we decompose $F$ with respect to a choice of 
	weight vectors
	$F(g)=\sum_\mu F_\mu(g)v_\mu+
	\sum_{\mu'}F_{\mu'}(g)v_{\mu'}$, 
	with $\mu$ goes through weight vectors 
	that also appears in $\pi_{\wt{k}}$
	and $\mu'$ goes through the complement,
	then we have
	$\mu(\alpha)=(w_0\wt{k})(\alpha)$ for all $\mu$
	and  $\varpi^r(w_0\wt{k})(\alpha)\mid \mu'(\alpha)$
	for all $\mu'$.
	Therefore
	\begin{multline*}
	U_P^rF(g)=
	\sum_{i\in I}
	\sum_\mu \xi_{\wt{k}}(x_i)\cdot F_\mu(gx_i\alpha)v_\mu+
	\sum_{i\in I}
	\sum_{\mu'}\frac{\mu'(\alpha)}{(w_0\wt{k})(\alpha)}
	\xi_{\wt{k}}(x_i)\cdot F_{\mu'}(gx_i\alpha)v_{\mu'}\\=
	\sum_{i\in I}
	\sum_\mu \xi_{\wt{k}}(x_i)\cdot F_\mu(gx_i\alpha)v_\mu=
	\sum_{i\in I}
	\sum_\mu \xi_{\wt{k}}(x_i)\cdot
	\epsilon_{\wt{k}}(F)(v^*_\mu)(gx_i\alpha)
	=F_{\epsilon_{\wt{k}}(F)}(g).
	\end{multline*}

	We thus have the following commutative diagram,
	from which the proposition follows.
	\[
	\begin{tikzcd}
		S_{\wt{k}}(U^p\Iw^P(p^{b,b}),A)
		\arrow[r,"\epsilon_{\wt{k}}"]
		\arrow[d,"U_P^r"]
		& \Hom_\oo(\pi^*_{\wt{k}}(\oo), S(U^p\Iw^P(p^{b,b}),A))
		\arrow[d,"U_P^r"]
		\arrow[dl,"\varphi"]\\
		S_{\wt{k}}(U^p\Iw^P(p^{b,b}),A)
		\arrow[r,"\epsilon_{\wt{k}}"]
		& \Hom_\oo(\pi^*_{\wt{k}}(\oo), S(U^p\Iw^P(p^{b,b}),A))
	\end{tikzcd}	
	\]
\end{proof}


Let $S_{\wt{k}}^{P-\ord}(U^p,E/\oo)=
\varinjlim_{b}
S_{\wt{k}}^{P-\ord}(U^p\Iw^P(p^{b,b}),E/\oo)$
be the injective limit under inclusions
and define 
\[
	\TT^P_{\wt{k}}(U^p,E/\oo)=
	\varprojlim_{b}
	\TT^P_{\wt{k}}(U^p\Iw^P(p^{b,b}),E/\oo)
\]
Since $S_{\wt{k}}^{P-\ord}(U^p,E/\oo)$
is also the injective limit of 
$S_{\wt{k}}^{P-\ord}(U^p\Iw^P(p^{b,b}),\varpi^{-b}\oo/\oo)$
and $\pi_{\wt{k}}^*(\oo)$ is finite over $\oo$,
the following proposition
follows immediately from the previous lemma.

\begin{prop}\label{prop:wt_indep}
	There exists the following isomorphism
	which is equivariant with respect to the 
	Hecke operators and the $Q_0$-action 
	defined in previous lemma.
	\[
		\epsilon_{\wt{k}} \colon 
		S_{\wt{k}}^{P-\ord}(U^p,E/\oo)\cong 
		\Hom_{\oo}(\pi_{\wt{k}}^*(\oo),
		S^{P-\ord}(U^p,E/\oo)).
	\]
	In particular, this isomorphism 
	induces the following surjective homomorphism
	between the Hecke algebras
	\[
		\varphi_{\wt{k}}\colon 
		\TT^P(U^p,E/\oo)\twoheadrightarrow
		\TT^P_{\wt{k}}(U^p,E/\oo).
	\]
\end{prop}


\subsection{Completed homology and cohomology}

Recall that when $\wt{k}$ is the trivial weight,
the action of $K_p$
on  $S(M)$ as in Definition \ref{def:algrep} 
for any  $\oo$-module $M$
is simply the right translation,
which extends to $G_p$
and coincides with that of $N_0T^+$ in \eqref{def:T_act}.
In particular,
when $U^p$ satisfies \eqref{cond:small}
and $\{U_p\}$ is the filtered system of 
all the compact open subgroups in $K_p$
and for $A=E/\oo$ or  $\oo/\varpi^{r}$, we have
\begin{equation}\label{eq:complete}
	S(U^p,A)\coloneqq
	\varinjlim_{U_p}S(U^pU_p,A)\in 
	\Mod^{\adm}_{G_p}(\oo).
\end{equation}
Moreover, Let $P=QU$ be a parabolic subgroup 
as in the previous subsection.
Since 
\[
	S(U^p,E/\oo)^{U_0}=
	\varinjlim_{b}
	S(U^p\Iw^P(p^{b,b}),\varpi^{-b}\oo/\oo)
\]
is the injective limits of finite $\oo$-modules,
It follows from \cite[Lem 3.1.5]{emeI} and \cite[Prop 3.2.4]{emeI}
that $S^{P-\ord}(U^p,E/\oo)\cong \Ord_P(S(U^p,E/\oo))$,
where the latter is an object in $\aMod_Q(\oo)$
by \cite[Thm 3.3.3]{emeI}.
We define the $P$-ordinary completed homology and cohomology by
\begin{align}\label{eq:completed_coh}
	M(U^p)&=
	\Ord_P(S(U^p,E/\oo))^\vee
	\coloneqq \Hom_\oo(\Ord_P(S(U^p,E/\oo)),E/\oo)\\
	S(U^p)&=\Hom_\oo(E/\oo, \Ord_P(S(U^p,E/\oo)))
	\cong \varprojlim_r \Ord_P(S(U^p,\oo/\varpi^{r}))
\end{align}
Let $\Hom_\oo^{\cts}(M(U^p),\oo)$
be the set of
$\Phi\in \Hom_\oo(M(U^p),\oo)$ 
such that for any positive integer $r$,
there exists $b$ sufficiently large so that 
the reduction of $\Phi$ modulo $\varpi^r$
factors through
the Pontryagin dual of 
$S^{P-\ord}(U^p\Iw^P(p^{b,b}),E/\oo)\subset \Ord_P(S(U^p,E/\oo)$. 
It can be verified that 
\[
	M(U^p)\cong \Hom_\oo(S(U^p),\oo),\qquad
	S(U^p)\cong \Hom_\oo^{\cts}(M(U^p),\oo).
\]
From the above isomorhisms we see that
$\TT^P(U^p,E/\oo)$ acts faithfully
on  $M(U^p)$, and  $S(U^p)$.
In fact, 
let $S_{\wt{k}}^{P-\ord}(U^p,\oo)
=\varinjlim_{b}S_{\wt{k}}^{P-\ord}(U^p\Iw^P(p^{b,b}),\oo)$
and 
$\TT^P_{\wt{k}}(U^p,\oo)=\varprojlim_bS_{\wt{k}}^{P-\ord}(U^p\Iw^P(p^{b,b}),\oo)$.
Then there exists an isomorphism of $\oo$-algebras 
$\TT^P(U^p,\oo)\cong \TT^P(U^p,E/\oo)$
as in \cite[Lem 2.17]{ger},
which also coincides with the fact that
$S^{P-\ord}(U^p,\oo)$ is dense in $S(U^p)$.

\begin{defn}\label{def:big_hecke}
	From now on, let $\TT^P(U^p,\oo)$
	be the big $P$-ordinary Hecke algebra 
	acting faithfully on each of 
	$S^{P-\ord}(U^p,E/\oo), \Ord_P(S(U^p,E/\oo)), M(U^p)$
	and $S(U^p)$. 
	We still denote 
	\[
		\varphi_{\wt{k}}\colon \TT^P(U^p,\oo)\twoheadrightarrow
		\TT^P_{\wt{k}}(U^p,\oo)
	\]
	for the surjective homomorphism of $\oo$-algebras
	induced by Proposition \ref{prop:wt_indep}.
	Moreover, 
	let $S(U^p)_E\coloneqq S(U^p)\otimes_{\oo}E$
	be the $E$-Banach space $G_p$-representation with 
	the unit ball $S(U^p)$. 
	We also define $\TT^P(U^p,E)=\TT^P(U^p,\oo)\otimes_{\oo}E$,
	which acts faithfully on $S(U^p)_E$.
\end{defn}


\begin{lem}\label{lem:inj}
	The restriction of
	$\Ord_P(S(U^p,E/\oo))$ to $Q_0$ 
	is an injective object
	in $\Mod^{\sm}_{Q_0}(\oo)$.
\end{lem}
\begin{proof}
	Following the strategy of the proof of 
	\cite[Prop 3.2.4]{pan}, 
	it suffices to show the surjectivity of
	\[
		\Hom_{\oo[Q_0]}(\pi,\Ord_P(S(U^p,E/\oo)))\to 
		\Hom_{\oo[Q_0]}(\pi_1,\Ord_P(S(U^p,E/\oo)))
	\]
	when $\pi_{1}\hookrightarrow \pi$ 
	is an injective morphism between admissible $Q_0$
	representations that are finite $\oo$-modules.
	We first note that since $\pi_1$ is admissible
	and $\oo$-finite,
	any homomorphism on the right factors 
	through 
	\begin{multline*}
		\Hom_{\oo[Q_0]}(\pi_1,
		\Hom_{\oo[Z_Q^+]}
		(\oo[Z_Q], S(U^p\Iw^P(p^{b,b}),
		\varpi^{-r}\oo/\oo)))\\=
		\Hom_{\oo[Z_Q^+]}(\oo[Z_Q],
		\Hom_{\oo[Q_0]}(\pi_1, 
		S(U^p\Iw^P(p^{b,b}),\varpi^{-r}\oo/\oo)))
	\end{multline*}
	for some $b$ and  $r$ sufficiently large,
	on which 
	the $Z_Q$-fintieness condition is automatic
	by \cite[Lem 3.1.5]{emeI}.

	Now, enlarging $b$ if necessary,
	we may assume that 
	the $Q_0$-actions on 
	$\pi$ and $\pi_1$ is trivial on
	$Q_0\cap \Iw(p^{b,b})$. 
	Let $U_0$ acts trivially,
	we may then extend $\pi$ and  $\pi_1$
	to of $\Iw(p^{0,b})$.
	Let $\pi^\vee$ be the Pontryagin dual.
	Define
	\[
		S_{\pi^\vee}(U^p\Iw(p^{0,b}))=
		\{
			F\colon G(\F)\backslash G(\A_f)\to 
			\pi^\vee\mid 
			F(gu)=\pi^\vee(u_p)\cdot F(g),\,
			u\in \Iw(p^{0,b})
		\}
	\]
	then there exists an isomorphism
	\[
		\epsilon\colon 
		S_{\pi^\vee}(U^p\Iw(p^{0,b}))\cong 
		\Hom_{\oo[Q_0]}(\pi,
		S(U^p\Iw^P(p^{b,b}),\varpi^{-r}\oo/\oo)))
		\quad \epsilon(F)\colon
		v\mapsto [g\mapsto v(F(g))]
	\]
	and similarly for $\pi_1$.
	But the smallness assumption implies
	that $S_{\pi^\vee}(U^p\Iw(p^{0,b}))$
	are $S_{\pi_1^\vee}(U^p\Iw(p^{0,b}))$
	are direct sums of 
	$\pi^\vee$ and  $\pi_1^\vee$ 
	on the same indexing set
	$G(\F)\backslash G(\A_f)/U^p\Iw(p^{0,b})$.
	Therefore 
	$S_{\pi^\vee}(U^p\Iw(p^{0,b}))\to 
	S_{\pi_1^\vee}(U^p\Iw(p^{0,b}))$ is 
	surjective
	as $\pi^\vee\to \pi_1^\vee$ is surjective.
	Apply localization to the $P$-ordinary parts
	and apply \cite[Lem 3.1.5]{emeI} again,
	we see that
	\[
		\Hom_{\oo[Z_Q^+]}(\oo[Z_Q],
		S_{\pi^\vee}(U^p\Iw(p^{0,b})))\to 
		\Hom_{\oo[Z_Q^+]}(\oo[Z_Q],
		S_{\pi^\vee_1}(U^p\Iw(p^{0,b})))
	\]
	is also surjective, from which 
	the proposition follows.
\end{proof}


\begin{prop}\label{prop:density}
The subspace $S^{\alg}(U^p)_E$ 
of $Q_0$-algebraic vectors, defined as
\[
\Image\left(\bigoplus_{\wt{k}}\Hom_{E[Q_0]}(\pi_{\wt{k}}^*(\oo), S(U^p)_E)
\otimes_E \pi_{\wt{k}}^*(E)\rightarrow S(U^p)_E\right)
\]
where $\wt{k}$ ranges through all dominant weights,
is dense in the $E$-Banach space $S(U^p)_E$.
\end{prop}
\begin{proof}
	Since $\Ord_P(S(U^p,E/\oo))$ is an injective object
	in $\Mod_{Q_0}^{\sm}(\oo)$
	by Lemma \ref{lem:inj},
	we may follow the strategy of 
	\cite[Prop 3.2.9]{pan}
	and use \cite[Cor 3.2.6]{pan}
	to reduce the statement to that of
	$\mathcal{C}(Q_0,E)$,
	the space of continuous  $E$-valued
	functions on $Q_0$.
	The density result then follows from
	\cite[Prop 6.A.17]{Pask14}.
\end{proof}


\begin{prop}\label{prop:wt_space}
	There exists a Hecke-equivariant isomorphism
	\[
	S_{\wt{k}}^{P-\ord}(U^p\Iw^P(p^{0,1}),E)\cong 
	\Hom_{\oo[Q_0]}(\pi_{\wt{k}}^*(\oo), S(U^p)_E)
	\]
\end{prop}
\begin{proof}
	By Proposition \ref{prop:wt_indep},
	there exists an isomorphism
	\[
		\Hom_\oo(E/\oo, S_{\wt{k}}^{P-\ord}(U^p,E/\oo))\cong 
		\Hom_\oo(E/\oo,
		\Hom_{\oo}(\pi_{\wt{k}}^*(\oo),
		S^{P-\ord}(U^p,E/\oo)))=
		\Hom_{\oo}(\pi_{\wt{k}}^*(\oo), S(U^p))
	\]
	that is equivariant with respect to 
	the Hecke operators and 
	the $Q_0$-actions defined.
	Taking the subspaces of $Q_0$-invariant subspaces.
	This gives
	$\Hom_{\oo[Q_0]}(\pi_{\wt{k}}^*(\oo), S(U^p))$
	on the right hand side.

	On the other hand, since
	$\Hom_\oo(E/\oo, S_{\wt{k}}^{P-\ord}(U^p,E/\oo))\cong
	\varprojlim_r S_{\wt{k}}^{P-\ord}(U^p,\oo/(\varpi^r))=
	S_{\wt{k}}^{P-\ord}(U^p,\oo)$,
	the $Q_0$-invariant subspace 
	on the left hand side
	is $S_{\wt{k}}^{P-\ord}(U^p\Iw^P(p^{0,1}),\oo)$
	by Lemma \ref{lem:control}.
	The claimed result now follows by
	tensoring both subspaces with $E$.
\end{proof}

\begin{rem}
	The above results generalize
	\cite[Prop 3.2.9]{pan} and 
	\cite[\S 3.2.10]{pan},
	which deals with the case 
	when $n=2$ and $P=G_p$,
	in which the $P$-ordinary condition is empty.
\end{rem}


\subsection{Hecke algebras and Galois representations}


Recall taht for a dominant $\wt{k}\in (\Z^n)^{\Sigma}$,
the space $S_{\wt{k}}(\bar{\Q}_p)$
admits the $G(\A_f)$-action defined in 
Definition \ref{def:algform}.
Let $\mathcal{A}$ be
the space of automorphic forms on $G(\A)$, 
and $\xi_{\wt{k}}^*(\C)$ be the
$G(\A_\infty)$-representation over $\C$
defined by the inclusions
$G(\F_\sigma)\subset \GL_n(\F_\sigma\otimes \K)=\GL_n(\C)$
for each $\sigma\in \Sigma$.
By \cite[Prop 3.3.2]{CHT},
there exists an $G(\A_f)$-equivariant isomorphism
\begin{equation}\label{eq:p_to_infty}
	\iota\colon S_{\wt{k}}(\bar{\Q}_p)\otimes_{\iota,\bar{Q}_p}\C
	\rightarrow \Hom_{G(\A_\infty)} (\xi_{\wt{k}}^*(\C), \mathcal{A})\quad
	\iota(F)\colon v^*\mapsto 
	[g\mapsto \xi_{\wt{k}}(g_\infty)\xi_{\wt{k}}(g_p)\cdot F(g_f)].
\end{equation}


\begin{prop}\cite[Prop.2.27]{ger}
	Let $\pi$ be an irreducible constituent of the
	$G(\A_f)$-representation $S_{\wt{k}}(\bar{\Q}_p)$,
	then there exist a unique 
	continuous semisimple representation
	\[
	r_\pi: \Gal_\K \rightarrow \GL_n(\bar{\Q}_p)\quad
	\text{ satisfying }
	r_\pi^c \cong r_\pi^{\vee} \epsilon^{1-n}
	\]
	where $\epsilon$ is the $p$-th cyclotomic character,
	with the following properties.
\begin{enumerate}[label=(\alph*)]
\item Let $v=w\bw$ be a prime-to-$p$ place that is split in $\K$
and $\pi_w$ be the $\GL_n(\K_w)$-representation
induced by $\iota_w\colon G(\F_v)\cong \GL_n(\K_w)$, then
\[
\WD\left(\left.r_\pi\right|_{D_w}\right)^{\mathrm{ss}} \cong
\Rec(\pi_w|\cdot|^{\frac{1-n}{2}})^{\mathrm{ss}}.
\]
Moreover, $r_\pi$ is unramified at $w$ if $\pi_v$ is unramified.
\item Let $v=w\bw$ with $v\in S_p$ and define $\pi_w$ as above.
The representation $r_\pi$ is potentially semistable at $w$ and  $\bw$.
Moreover $r_\pi$ is crystalline at $w$ 
if $\pi_v$ is unramified,
in which case 
the characteristic polynomial of the geometric Frobenius $\Fr_w$
on $\WD\left(D_{\mathrm{cris }}\left(\left.r_\pi\right|_{D_w}\right)\right)$
coincides with that of $\Rec(\pi_w|\cdot|^{\frac{1-n}{2}})^{\mathrm{ss}}$.
\item 
Let $k_{\sigma,j}=-k_{\sigma c, n-j+1}$
for $\sigma\notin \Sigma$.
If $w\mid p$ and  $\sigma\in I_w$, then 
$\dim_{\bar{\Q}_p}\operatorname{gr}^i
\left(r_\pi \otimes_{\sigma, \K_w} B_{\dR}\right)^{D_w}=1$
exactly when $i=k_{\sigma, j}+n-j$ 
for $j=1, \ldots, n$ and is equal to 0 otherwise.
\end{enumerate}
\end{prop}

From now on,
we restrict ourselves to the following situation
\begin{equation}\label{cond:parabolic}\tag{P}
	n=2,\, 
	P_w=G_w \text{ for a fixed }w\in \Sigma,\,
	P_{w'}=B_{w'} \text{ for } w'\neq w.
\end{equation}
By \cite[Lem 2.14]{ger}, 
the Hecke algebras
$\TT^P_{\wt{k}}(U^p\Iw^P(p^{b,b}),\oo)$
are finite flat reduced $\oo$-algebras.
\begin{defn}\label{def:rep_prime}
	Let $\fp\subset \TT^P_{\wt{k}}(U^p\Iw^P(p^{b,b}),\oo)$
	be a height-one prime ideal.
	By abuse of notation, 
	we also let $\fp$ denote the induced minimal prime
	in $\TT^P_{\wt{k}}(U^p\Iw^P(p^{b,b}),E)$.
	The quotient $\TT^P_{\wt{k}}(U^p\Iw^P(p^{b,b}),E)/\fp$
	is isomorphic to a finite extension $E_{\fp}$ of $E$.
	Let 
	$\lambda_\pi\colon \TT^P_{\wt{k}}(U^p\Iw^P(p^{b,b}),E)\to E_\pi$
	denote the associated homomorphism.

	Let $S_{\wt{k}}^{P-\ord}(U^p\Iw^P(p^{b,b}),E)_{\fp}$
	be the localization at $\fp$.
	We say an irreducible component $\pi$ as above
	belongs to  $\fp$ if 
	$\pi\cap S_{\wt{k}}^{P-\ord}(U^p\Iw^P(p^{b,b}),E)_{\fp}\neq 0$.
	The proposition then implies that
	$r_\pi$ satisfies
	\begin{equation}\label{eq:Gal_hecke_away_p}
		\mtr(r_\pi(\Fr_w))=\lambda_\fp(T_w^{(1)}),\quad
		\det(r_\pi(\Fr_w))=q_w\lambda_\fp(T_w^{(2)}),\,
	\end{equation}
	for $v=w\bw$ that is split in  $\K$ and $v\notin R$. 
	By Chebotarev's density theorem,
	this implies that $r_\pi$ is defined over  $E_{\fp}$
	and independent of the choice of $\pi$.
	Thus we also write $r_\fp=r_\pi$.
\end{defn}




Following \cite{ger},
we say a dominant weight $\wt{k}$ is sufficiently regular
(for $w'\neq w$)
if for each $w'\in \Sigma_p$  and $w'\neq w$,
there exists  $\sigma\in I_{w'}$
such that  $k_{\sigma,1}>k_{\sigma,2}$.

\begin{lem}\label{lem:galois_at_p}
	Let $r_{\fp}$ be the Galois representation
	associated to a minimal prime
	$\fp\subset \TT^P_{\wt{k}}(U^p\Iw^P(p^{0,1}),E)$.
	\begin{enumerate}[label=(\alph*)]
	\item The representation $r_\fp$ is crystalline at $w'\neq w$
	if $\wt{k}$ is sufficiently regular.
	Moreover, $r_\fp\vert_{D_{w'}}$ 
	fits into the exact sequence
	$0\to \psi_1\to r_{\fp}\vert_{D_{w'}} \to \epsilon^{-1}\psi_2\to 0$
	for characters $\psi_i\colon D_{w'}\to E_{\fp}^{\times}$
	\begin{equation}\label{eq:Gal_hecke_at_p}
	\begin{aligned}
		\psi_1\circ \Art_{w'}(\varpi_{w'})&=
		\lambda_{\fp}(U_{\wt{k},w'}^{(1)}) &
		\psi_1\circ \Art_{w'}(x)&=
		\lambda_{\fp}
		(\langle 
		\iota_{w'}^{-1}
		(\begin{smallmatrix}
			x&\\&1
		\end{smallmatrix})
		\rangle)\, \text{ for }x\in \oo_{w'}^{\times}\\
		\psi_2\circ \Art_{w'}(\varpi_{w'})&=
		\lambda_{\fp}(U_{\wt{k},w'}^{(2)})/
		\lambda_{\fp}(U_{\wt{k},w'}^{(1)}) &
		\psi_1\circ \Art_{w'}(x)&=
		\lambda_{\fp}
		(\langle 
		\iota_{w'}^{-1}
		(\begin{smallmatrix}
			1&\\&x
		\end{smallmatrix})
		\rangle)\, \text{ for }x\in \oo_{w'}^{\times}
	\end{aligned}
	\end{equation}
	\item The representation $r_\fp$ is 
	crystalline at $w$, with 
	\[
	\det r_\fp\circ \Art_w(\varpi_w)=
	\lambda_{\fp}(U_{\wt{k},w}^{(2)}),\quad
	\det r_\fp\circ \Art_w(x)=
	\lambda_{\fp}
	(\langle 
	\iota_{w}^{-1}
	(\begin{smallmatrix}
		x&\\&x
	\end{smallmatrix})
	\rangle)\, \text{ for }x\in \oo_{w}^{\times}
	\]
	If furthermore $\fp$ extends
	to a prime ideal in the $B$-ordinary Hecke algebra
	$\TT^P(U^p\Iw(p^{0,1}),E)$,
	then $r_\pi\vert_{D_w}$ 
	admits an exact sequence as above.
	\end{enumerate}
\end{lem}
\begin{proof}
This is a restatement of \cite[Cor 2.33]{ger}
for $w'\neq w$.
For the fixed place  $w\in \Sigma$,
note that $\pi_w$ is unramified 
by definition
for any $\pi$ belonging to $\fp$.
We then combine the corollary with \cite[Lem 2.31]{ger} to obtain 
the result.
\end{proof}


Since the Hecke algebra 
$\TT^P_{\wt{k}}(U^p\Iw^P(p^{b,b}),\oo)$
is reduced the homomorphisms
\[
	\prod_{\fp}\lambda_{\fp}\colon 
	\TT^P_{\wt{k}}(U^p\Iw^P(p^{b,b}),\oo)\to 
	\TT^P_{\wt{k}}(U^p\Iw^P(p^{b,b}),E)\to  
	\prod_{\fp}E_{\fp}
\]
where $\fp$ goes through all height-one primes,
is injective.
In particular $\TT^P_{\wt{k}}(U^p\Iw^P(p^{b,b}),\oo)$
can be identified with 
a closed subring of $\prod_{\fp}E_{\fp}$.
By the relation \eqref{eq:Gal_hecke_away_p}
and Chebotarev's density theorem,
the pseudo-representations  $r_{\fp}$
lift to a pseudo-representation
\begin{equation}\label{eq:pseudo_rep_finite}
	\Psi_{\wt{k}}\colon \Gal_{\K}
	\to \TT^P_{\wt{k}}(U^p\Iw^P(p^{b,b}),\oo)\quad
	\Psi_{\wt{k}}(\Fr_w)=T_w^{(1)}
\end{equation}
that are compatible among different levels of $\Iw^P(p^{b,b})$.
By abuse of notation,
we let 
$\Psi_{\wt{k}}\colon \Gal_{\K} \to \TT^P_{\wt{k}}(U^p,\oo)$
denote the big Galois pseudo-representation 
obtained through the inverse limit.

\begin{defn}\label{def:big_Gal}
Let $\Psi\colon \Gal_{\K}\to \TT^P(U^p,\oo)$	
denote the big Galois pseudo-representation
when $\wt{k}$ is the trivial weight.
Note that by the relation \eqref{eq:Gal_hecke_away_p}
and Chebotarev's density theorem again,
the composition of $\Psi$
with the surjection
$\varphi_{\wt{k}}\colon 
\TT^P(U^p,\oo)\to \TT^P_{\wt{k}}(U^p,\oo)$
is precisely  $\Psi_{\wt{k}}$.
\end{defn}


\section{Results on p-adic local Langlands}

We continue with the assumptions
in the previous section.
So $U^p\subset G(\A_f^p)$ is an
open compact subgroup satisfying \eqref{cond:small},
$P\subset G_p$ is a parabolic subgroup
as \eqref{cond:parabolic},
$\TT^P(U^p,\oo)$
is the big  $P$-ordinary Hecke algebra
acting on the completed homology 
and cohomology as defined in Definition \ref{def:big_hecke}.
We further assume that 
the fixed place $w\in \Sigma_p$
in \eqref{cond:parabolic}
is such that  $\K_w\cong \Qp$.
So that the decomposition group
$D_w$ is identified with
$\Gp$, the absolute Galois group of  $\Qp$.



Fix a maximal ideal $\fm\subset \TT^P(U^p,\oo)$
and let $\fF$ denote the residue field of the localization
$\TT^P(U^p,\oo)_{\fm}$.
We note that the same argument
as \cite[Prop 3.3.6]{pan} shows that
$\TT^P(U^p,\oo)\to \TT^P(U^p\Iw^P(p^{0,1}),\oo/\varpi)$ 
induces a bijection of maximal ideals
and thus $\fF$ is a finite field.
Enlarging the finite extension  $E$ over  $\Qp$
if necessary,
we may assume that  $\fF$ is also 
the residue field of the ring of integers  $\oo$ of  $E$.


Let  $\Psi_{\fm}$ denote 
the localization of the big Galois pseudo-representation.
We say $\Psi_{\fm}$ is 
residually reducible and locally generic at $w$ if
there exists characters
$\bar{\delta}_1, \bar{\delta}_2\colon \Gal_{\K}\to \fF$
such that 
\begin{equation}\tag{red.gen}\label{cond:red_gen}
	\Psi_\fm\equiv \bar{\delta}_1+\bar{\delta}_2
	\mod \fm,\quad
	\bar{\delta}_1\bar{\delta}_2^{-1} \vert_{D_w}
	\neq \id,\omega^{\pm}
\end{equation}

Let $P=QU$ be the Levi decomposition.
Our assumption on $w$ implies that
$Q$ is the product of  $\GL_2(\Qp)$
and a torus.
In this section,
we follow the idea in \cite{urban}
and show that the localization
$M(U^p)_{\fm}$ belongs to a block
of $Q$-representations in the sense of  \cite{pask}
when  $\fm$ satisfies \eqref{cond:red_gen}.
As a consequence,
we show that $\TT^P(U^p,\oo)_{\fm}$
is Noetherian
and relate the ``reducible'' part of 
$M(U^p)_{\fm}$ to the space of 
$B$-ordinary modular forms.
We deduce from which
the fundamental exact sequence
that is critical for our
construction of the Euler systems
in next section.

\subsection{Generically reducible deformation}



We first recall from \cite[\S B.1]{pask}
the structure of the universal deformation ring 
$R$ of the $2$-dimensional pseudo-representation 
$\chi_1+\chi_2$,
where $\chi_1,\chi_2\colon \Gp\to \fF^\times$ 
are continuous characters satisfying the 
the following generic assumption
\begin{equation}\label{cond:generic}\tag{\text{gen}}
	\chi_1\chi_2^{-1}\neq \id,\omega^{\pm1}.
\end{equation}

By \textit{loc.cit}, the assumption
implies the existence of non-split extensions
\begin{equation*}
    0\to \chi_1\to \rho_{12}\to \chi_2\to 0\quad
    0\to \chi_2\to \rho_{21}\to \chi_1\to 0
\end{equation*}
which are unique up to isomorphisms;
and that the universal deformation rings
$R_{ij}$ of the Galois representations $\rho_{ij}$
are formally smooth of relative dimension $5$ over $\oo$.

Denote by $\tilde{\rho}_{ij}$ the universal deformation,
one may choose bases and think of which as group homomorphisms
$\tilde{\rho}_{ij}\colon \Gp\to \GL_2(R_{ij})$,
whose reduction modulo the maximal ideals are
$\rho_{12}=\smat{\chi_1&*\\&\chi_2}$ and
$\rho_{21}=\smat{\chi_1&\\ * &\chi_2}$.
The trace then induces 
$\theta\colon R\cong R_{ij}$
by \cite[Prop B.17]{pask}.
Since the reducible universal deformation ring
$R_{\red}$ is formally smooth of 
relative dimension $4$ over $\oo$,
the reducibility ideal  $\tau\subset R$ 
is a principal ideal generated by 
an element $x\in\fm_R\setminus \fm_R^2$,
where $\fm_R\subset R$ is the maximal ideal. 
Let $\tau_{ij}\subset R_{ij} $ be the ideal 
generated by the $(j,i)$-entry of  $ \tilde{\rho}_{ij}(g)$
for all $g\in \Gp$,
then  $\theta$ maps  
$\tau$ to  $\tau_{ij}$ by \cite[Prop B.23]{pask}

Let $\tilde{\rho}_{12}^x\colon \Gp\to \GL_2(R_{12})$ be the representation defined by
\begin{equation*}
	\tilde{\rho}_{12}^x(g)\coloneqq 
	\smat{\theta(x)&\\&1}
	\tilde{\rho}_{12}(g)
	\smat{\theta(x)&\\&1}^{-1}.
\end{equation*}
Then $\tilde{\rho}_{12}^x$
is a deformation of $\rho_{21}$ to $R_{12}$
and induces an isomorphism 
$\alpha\colon R_{21}\to R_{12}$,
for which the diagram
\begin{equation*}
	\begin{tikzcd}
		R_{21} \arrow[r,"\alpha"] &
		R_{12}\\
		R\arrow[u,"\theta"] \arrow[r,equal] &
		R\arrow[u,"\theta"]
	\end{tikzcd}
\end{equation*}
commutes by \cite[Prop B.24]{pask}.
We identify $\tilde{\rho}_{21}$ with 
$\tilde{\rho}_{12}^x$.

\begin{lem}\cite[Prop B.26]{pask}.
The modules
$\Hom_{\Gp}(\tilde{\rho}_{12}, \tilde{\rho}_{21})$ and
$\Hom_{\Gp}(\tilde{\rho}_{21}, \tilde{\rho}_{12})$
are free over $R\cong R_{12}$ and
generated respectively by
\begin{equation}\label{eq:Phi_ij}
	\Phi_{12}=\smat{\theta(x)&\\&1} \text{ and }
	\Phi_{21}=\smat{1&\\&\theta(x)}.
\end{equation}
And the algebra
$\End_{\Gp}(\tilde{\rho}_{12}\oplus \tilde{\rho}_{21})$
is isomorphic to the generalized matrix algebra
$\smat{R& R\Phi_{12}\\ R\Phi_{21}& R}$,
which is a free $R$-module of rank  $4$,
with the center isomorphic to  $R$.
\end{lem}

\subsection{Generically reducible block}

In the next two subsections,
we let $G$ and  $K$
denote  $\GL_2(\Qp)$ and $\GL_2(\Zp)$
respecitvely.
We also let $B=TN$
denote the subgroup of upper triangular matrices
and its Levi decomposition,
$\bar{B}$ be
the subgroup of lower triangular matrices,
and identify the center $Z\subset T$ with $\Qp^\times$.

We briefly replace the deformation rings
$R, R_{\red}, R_{12}$ and $R_{21}$
introduced in the previous subsection
by the variants with 
fixed determinant $\epsilon\zeta$,
where $\zeta\colon \Gp\to \oo^\times$
is a continuous character 
such that  $\epsilon\zeta\equiv \chi_1\chi_2$ 
modulo $\varpi$.
All the results there
still hold true, 
except that the relative dimensions
are decreased by  $2$.

Let $\Art\colon \Qp^\times\to \Gp^{\textnormal{ab}}$
denote the reciprocity map
so that $\Fr\coloneqq \Art(p)$
is a geometric Frobenius.
We identify characters
of $\Gp$ with that of $\Qp^\times$
through the composition with $\Art$.
For any character $\chi$ of 
of $T\cong \Qp^\times\times\Qp^\times$,
let $\chi^s$
denote the composition
with the involution
$(a,b)\mapsto (b,a)$.
Define the characters
$\chi=\chi_1\otimes\chi_2\omega^{-1}$ and
$\chi^s\alpha=\chi_2\otimes \chi_1\omega^{-1}$,
where  $\alpha=\omega\otimes\omega^{-1}$.
By \cite[Thm 30]{barthel},
\[
\pi_1\coloneqq \Ind_{B}^G\chi\cong
\Ind_{B}^G\chi_1\otimes\chi_2\omega^{-1}\quad
\pi_2\coloneqq \Ind_{B}^G\chi^s\alpha\cong 
\Ind_{B}^G\chi_2\otimes\chi_1\omega^{-1} 
\]
are irreducible representations
in $\Mod^{\sm}_{G,\zeta}(\oo)$.

\begin{defn}\label{def:block}
Let $\lfMod_{G,\zeta}(\oo), \lfMod_{T,\zeta}(\oo)$
be the subcategories
of representations that are 
locally of finite length 
and with central character $\zeta$;
and let $\fC_G(\oo), \fC_T(\oo)$
be the dual categories
of the Pontryagin duals.
For $\pi_1,\pi_2$ as before
the set $\B\coloneqq\{\pi_1,\pi_2\}$ 
defines a block 
in the sense of \cite[\S 5]{pask}.
Then $\lfMod_{G,\zeta}(\oo)^\B$
is the subcategory
of representations whose subquotients
all belong to $\B$;
and $\fC_G(\oo)^\B$
is the dual category.
\end{defn}

Let $\Ord\colon \Mod_{G,\zeta}^{\sm}(\oo)
\to \Mod_{T,\zeta}^{\sm}(\oo)$
denote the functor of $B$-ordinary
parts as recalled in \eqref{def:OrdP}.
The adjunction formula
in \cite[Thm 4.4.6]{emeI} states that
\begin{equation}\label{eq:adjunct}
	\Hom_{\oo[G]}(\Ind_{\bar{B}}^GU,V)
	\xrightarrow{\Ord}
	\Hom_{\oo[T]}(\Ord(\Ind_{\bar{B}}^GU),\Ord V)
	\cong
	\Hom_{\oo[T]}(U,\Ord V)
\end{equation}
is an isomorphism, where the last isomorphism
is induced by $\Ord(\Ind_{\bar{B}}^GU)\cong U$.


By \cite[Prop 7.1]{pask},
if $\iota\colon \pi_1\hookrightarrow \tilde{J}_1$
is the injective envelope of $\pi_1$
in $\lfMod_{G,\zeta}(\oo)$,
We have $\Ord\pi_1=\Ord(\Ind_B^G\chi)=\chi^s$
and $\Ord(\iota)\colon \chi^s \to \Ord(\tilde{J}_1)$
is isomorphic to an injective envelope
$\tilde{J}_{\chi^s}$ of $\chi^s$
in $\lfMod_{T,\zeta}(\oo)$.
Furthermore, the morphism 
$\iota_1\colon \Ind_{\bar{B}}^G(\tilde{J}_{\chi^s})\to
\tilde{J}_1$
induced through the adjunction formula \eqref{eq:adjunct}
by a fixed isomorphism
$\tilde{J}_{\chi^s}\to \Ord_B(\tilde{J}_1)$
is injective.
To simplify notations,
we identify $\lfMod_{T,\zeta}(\oo)$
with $\lfMod_{\Qp^\times}(\oo)$ through 
the map $\Qp^\times\cong \{\smat{1&\\&*}\}\subset T$
and write $\tilde{J}_{\chi_1}=\tilde{J}_{\chi^s}$.
Then for $ \tilde{J}_{\chi_2}$
and $ \tilde{J}_2$ defined as above
we also have the injective morphisms
$\iota_2\colon \Ind_{\bar{B}}^G(\tilde{J}_{\chi_2})\to
\tilde{J}_2$.


\begin{lem}\cite[Cor 7.7]{pask}
Let $\tilde{P}_{\chi_i^\vee}\coloneqq \tilde{J}_{\chi_i}^\vee
\in\fC_T(\oo)$ and
$\tilde{M}_i\coloneqq 
\Ind_{\bar{B}}^G(\tilde{J}_{\chi_i})^\vee,
\tilde{P}_i\coloneqq \tilde{J}_i^\vee\in\fC_G(\oo)$
for $i=1,2$.
The morphisms
$p_i\colon \tilde{P}_i\twoheadrightarrow \tilde{M}_i$
that are dual to
$\iota_i\colon 
\Ind_{\bar{B}}^G(\tilde{J}_{\chi_i})\hookrightarrow 
\tilde{J}_i$ 
extend to the exact sequences
\begin{equation}\label{eq:exact_PPM}
	0\to \tilde{P}_{2}\xrightarrow{\phi_{12}} 
	\tilde{P}_{1}\xrightarrow{p_1} \tilde{M}_1\to 0 
	\text{ and }
	0\to \tilde{P}_{1}\xrightarrow{\phi_{21}} 
	\tilde{P}_{2}\xrightarrow{p_2} \tilde{M}_2\to 0
\end{equation}
\end{lem}

Let $\Rep_{\Gp}(\oo)$
be the category of compact $\oo$-modules with
continuous actions of $\Gp$,
and $\V\colon \fC_G(\oo)\to \Rep_{\Gp}(\oo)$
be the Colmez functor introduced 
in \cite[\S 5.7]{pask},
which is exact and covariant.
By \cite[Cor 8.7]{pask},
for $(i,j)=(1,2)$ or  $(2,1)$,
there exists unique non-split extensions
in $\Mod^{\sm}_{G,\zeta}(\oo)$ 
\[
	0\to \pi_2\to \kappa_{12}\to \pi_1\to 0,\quad
	0\to \pi_1\to \kappa_{21}\to \pi_2\to 0
\]
such that
$\V(\pi_i^\vee)=\chi_i$, $\V(\kappa_{ij}^\vee)=\rho_{ij}$,
and $\V(\tilde{P}_j)=\tilde{\rho}_{ij}$
are the universal deformations
with determinant $\zeta\varepsilon$.

It then follows from \cite[Lem 8.10]{pask} that 
taking the Colmez functor 
$\V$ induces the isomorphisms below.
\begin{equation}\label{eq:end_deform}
\begin{split}
	\End_{\fC_{G}(\oo)}(\tilde{P_2})\cong R_{12}\cong R,\quad
	\Hom_{\fC_G(\oo)}(\tilde{P}_2, \tilde{P}_1)\cong R\Phi_{12}\\
	\Hom_{\fC_G(\oo)}(\tilde{P}_1, \tilde{P}_2)\cong R\Phi_{21},\quad
	\End_{\fC_{G}(\oo)}(\tilde{P_1})\cong R_{21}\cong R
\end{split}
\end{equation}
Write $ \tilde{P}_\B=\tilde{P}_1\oplus \tilde{P}_2$,
then $\tilde{E}_\B\coloneqq
\End_{\fC_G(\oo)}(\tilde{P}_\B)$
is isomorphic to 
$\End_{\Gp}(\tilde{\rho}_{12}\oplus \tilde{\rho}_{21})$.



By \cite[Prop 7.1]{pask}
any morphism 
$\End_{\fC_G(\oo)}(\tilde{P}_i, \tilde{M}_i)$
factors through
$\End_{\fC_G(\oo)}(\tilde{M}_i)\cong
\End_{\fC_T(\oo)}(\tilde{P}_{\chi_i^\vee})$,
which are isomorphic to
a formal power series ring
$ \oo\llbracket x,y\rrbracket$
of two variables
by \cite[Prop 3.34]{pask}.

\begin{lem}\label{lem:ker_red}
	The kernel of the following 
	surjective homomorphism 
	is the reducibility ideal $\tau\subset R$.
	\begin{equation}
	R\cong 
	\End_{\fC_G(\oo)}(\tilde{P}_i)
	\overset{p_i}{\twoheadrightarrow}
	\End_{\fC_G(\oo)}(\tilde{P}_i, \tilde{M}_i)=
	\oo\llbracket x,y\rrbracket
	\end{equation}
\end{lem}
\begin{proof}
It suffices to show that 
the image of $\tau$ consists of 
$\phi\in \End_{\fC_G(\oo)}(\tilde{P}_i)$
such that $p_i\circ \phi$ is trivial.
Let
\[
\begin{tikzcd}
	0 \arrow[r]&
	\tilde{P}_1  \arrow[r,"\phi_{21}"]  &
	\tilde{P}_2 \arrow[r,"p_2"] \arrow[dr,equal] &
	\tilde{M}_2  \arrow[r] & 0 \\
	0 & 
	\tilde{M}_1 \arrow[l]&
	\tilde{P}_1 \arrow[l,"p_1",swap]  &
	\tilde{P}_2  \arrow[l,"\phi_{12}",swap]  & 
	0  \arrow[l] 
\end{tikzcd}
\]
be the exact sequences in \eqref{eq:exact_PPM}.
Take $\Hom(\tilde{P}_j,\cdot)$.
The adjunction formula implies that
$\Hom_{\fC_G(\oo)}(\tilde{P}_j,\tilde{M}_i)\cong
\Hom_{\fC_T(\oo)}
(\tilde{P}_{\chi_j^\vee},\tilde{P}_{\chi_i^\vee})$,
which is trivial for $i\neq j$
since blocks in $\fC_T(\oo)$ are singletons
by the discussion in \cite[\S 7.2]{pask}.
Consequently there are isomorphisms
\[
	\phi_{12}\circ\colon
	\End_{\fC_G(\oo)}(\tilde{P}_2)\cong
	\Hom_{\fC_G(\oo)}(\tilde{P}_2, \tilde{P}_1)\quad
	\phi_{21}\circ\colon
	\End_{\fC_G(\oo)}(\tilde{P}_1)\cong
	\Hom_{\fC_G(\oo)}(\tilde{P}_1, \tilde{P}_2)
\]
Combined with the isomorphisms in \eqref{eq:end_deform},
we may assume that $\V(\phi_{ij})$ agree with 
$\Phi_{ij}$ in \eqref{eq:Phi_ij}.
Thus the image of 
$\phi_{ij}\circ\phi_{ji}\in End_{\fC_G(\oo)}(\tilde{P}_i)$,
which belongs to the kernel in question,
generates $\tau\subset R$,
and the lemma follows 
since  $R$ is formally smooth of relative dimension  $3$.
\end{proof}

\subsection{Univeral unitary completion}

Let $\Ban_{G,\zeta}(E)$
be the category of admissible 
$G$-representation with central character $\zeta$
on $E$-Banach spaces
as defined in \cite{pask}
and  $\B$ be the block defined earlier.
Identify the center of $\C_G(\oo)^\B$ with  $R$,
then  $R[\frac{1}{p}]$ acts on objects of $\Ban_{G,\zeta}(E)^{\B}$.
If $\fn\subset R[\frac{1}{p}]$ is a maximal ideal,
let $\Irr(\fn)$ be the set of
irreducible representation in  $\Ban_{G,\zeta}(E)^{\B}$
on which the action of $R[\frac{1}{p}]$ 
factors through $\fn$.

On the other hand,
let $T\colon \Gp\to R$ 
and $T_\fn\colon \Gp\to R[\frac{1}{p}]/\fn$ 
denote the universal deformation and 
the reduction.
Enlarge $E$ if necessary,
we assume $R[\frac{1}{p}]/\fn\cong E$.
Let $r_{\fn}\colon \Gp\to \GL_2(E)$
be the semisimple Galoi representation
such that $\mtr r_{\fn}=T_{\fn}$.
When $r_\fn$ is crystalline,
the associated Weil-Deligne representation,
defined by having $\Fr$ acts as $\varphi$
on  $D(r_\fn)\coloneqq (r_\fn\otimes_{\Qp}B_{\cris})^{\Gp}$,
is the sum of 
unramified characters $\chi_1$ and  $\chi_2$.
We then define the unramified
$\GL_2(\Qp)$ representation
$\pi_\fn$
as the un-normalized induction
$\Ind_B^G(\chi_1\otimes\chi_2|\cdot|^{-1})$
(the so-called Hecke correspondence in \cite{pan}).

\begin{rem}
	The representation 
	$\Ind_B^G(\chi_1\otimes\chi_2|\cdot|^{-1})$
	is irreducible under the assumption \eqref{cond:generic}.
	For otherwise
	$T_\fn$ would be of the form
	$\eta+\eta\epsilon$ for some character  $\eta$.
\end{rem}

\begin{lem}\label{lem:uni_completion}
	When $r_\fn$ is crystalline 
	and with 
	Hodge-Tate weights $\{-l,-l-k\}$,
	the universal unitary completion of 
	$\pi_\fn\otimes \pi_{-l,1-l-k}^*(E)$ belongs to 
	$\Irr(\fn)$.
\end{lem}
\begin{proof}
	When  $r_\fn$ is irreducible,
	this follows from \cite[Thm. 1.3]{CDP}
	since 
	$\pi_{-l,1-l-k}^*$,
	which is the algebraic representation of $\GL_2$
	of highest weight  $(l+k-1,l)$,
	is isomorphic to 
	$W_{l,k}\coloneqq \Sym^{k-1}\otimes\det^l$.
	In fact the universal unitary completion
	is the only elements in $\Irr(\fn)$
	in this case by \cite[Cor 8.14]{pask}.
	
	When  $r_\fn$ is reducible
	and $T_\fn=\psi_1+\psi_2$
	for crystalline characters
	$\psi_i\colon \Gp\to E^\times$,
	say of Hodge-Tate weights
	$1-l-k$ and  $-l$ respectively,
	then $\pi=\Ind_B^G(\chi_1\otimes\chi_2|\cdot|^{-1})$
	for the charaters
	\[
	\chi_1=\psi_1(\epsilon|\cdot|^{-1})^{-l-k},\quad
	\chi_2=\psi_2(\epsilon|\cdot|^{-1})^{-l}	
	\]
	By \cite[Thm 12.3]{pask} the universal unitary completion
	of $\pi\otimes W_{l,k}$
	is then $\Ind_B^G(\psi)_{cont}$ for
	\[
		\psi( (\begin{smallmatrix}
			a&b\\&d
		\end{smallmatrix}))
		=\chi_2(a)a^l\chi_1(d)|d|^{-1}d^{l+k-1}
		=\psi_2(a)\psi_1\epsilon^{-1}(d)
	\]
	since $\val_p(\chi_2(p))=-l$ and $\varepsilon(a)=a|a|$.
	And indeed by \cite[Cor 8.15]{pask} we have
	\[
	\Irr(\fn)=\{(\Ind_P^G\psi_1\otimes\psi_2\varepsilon^{-1})_{cont},
	(\Ind_P^G\psi_2\otimes\psi_1\varepsilon^{-1})_{cont}\}.
	\]
\end{proof}

\subsection{Local-global compatibility}
\label{sub:compatible}

We now return to the setting in 
the beginning of the section,
in which we have assumed 
the existence of
a maximal ideal $\fm\subset \TT^P(U^p,\oo)$ 
and characters $\bar{\delta}_i$
such that \eqref{cond:red_gen} holds.
Thus the characters
$\chi_i\coloneqq\bar{\delta}_i\omega$
of $D_w\cong \Gp$
satisfies \eqref{cond:generic} and
\[
	\epsilon\Psi_\fm\vert_{D_w}\equiv
	\chi_1+\chi_2 \mod \fm.
\]
Fix a crystalline character 
$\zeta\colon \Gp\to \oo^\times$ 
such that $\epsilon\zeta\equiv \chi_1\chi_2$
as in the previous subsection.
We let $\B=\{\pi_1,\pi_2\}$
be the block of $\GL_2(\Qp)$
representations
in $\lfMod_{\GL_2(\Qp),\zeta}(\oo)$
as introduced in Definition \ref{def:block}.


To apply the results in the previous subsections,
we follow \cite{urban} and make the following twist.
Let $\Psi\colon \Gp\to R=R_{\fm}$ be 
the universal pseudo-deformation of $\chi_1+\chi_2$.
Since $\det\Psi\equiv \epsilon\zeta$ modulo the 
maximal ideal and the residue characteristic is odd,
the character
\[
	\xi=\epsilon\zeta(\det \Psi^u)^{-1}\colon
	\Gp\to 1+\fm_R
\]
admits a character $\xi^{1/2}$ which is the square root of $\xi$.
Then  $\Psi'\coloneqq \xi^{1/2}\Psi$ has determinant $\epsilon\zeta$
and induces $R_\fm^{\epsilon\zeta}\to R_\fm$,
where $R_\fm^{\epsilon\zeta}$ is the universal 
pseudo-deformation with fixed determinant $\epsilon\zeta$.

By the universal property,
the pseudo-representations
$\epsilon\Psi_{\fm}$
induces homomorphisms of $R_\fm$ 
to the Noetherian local $\oo$-algebras
$\TT^P(U^p,\Iw^P(p^{b,b}),\oo)_{\fm}$.
Let $R_{\fm}\to \TT(U^p,\oo)_{\fm}$
be the homomorphism
coming from the inverse limit.
Note that we cannot define the homomorphism
from the universal property
since we haven't shown that
$\TT(U^p,\oo)_{\fm}$ is Noetherian.
Let $\xi_\fm^{1/2}$ be the  composition 
of $\xi^{1/2}$ with the homomorphism,
so that 
\[
	\xi_\fm\coloneqq(\xi_\fm^{1/2})^2=
	\epsilon\zeta(\det \epsilon\Psi_\fm)^{-1}
	=\zeta(\epsilon\det \Psi_\fm)^{-1}\colon
	\Gp\to 1+\fm\subset \TT(U^p,\oo)_\fm.
\]
It follows from the density of crystalline points
that 
\[
	\epsilon\det\Psi_\fm\circ \Art\colon
	\Qp^\times\to \TT(U^p,\oo)_{\fm}\quad
	p\mapsto U_w^{(2)},\quad
	x\mapsto h_U(x)\,x\in\Zp^\times
\]
When $\Qp^\times$
is identified with the center of  $\GL_2(\Qp)$,
its action on $\Ord_P(S(U^p,E/\oo))_{\fm}$
through the above
is simply the usual right translation.
Let $(\Qp^\times)^\wedge$ be the pro-$p$ completion. 
We may thus assume that 
$\xi_{\fm}^{1/2}\circ \Art$
factors through $\llbracket (\Qp^\times)^\wedge\rrbracket$,
which acts on any smooth $\Qp^\times$-representation.


\begin{defn}\label{def:twist}
	For any smooth $\GL_2(\Qp)$-representation
	$V$,
	we let $V'=V(\xi_{\fm}^{1/2})$
	denote the twist 
	by $(\xi_{\fm}^{1/2}\circ\Art)\circ \det$.
	In particular, 
	$\Ord_P(S(U^p,E/\oo))_{\fm}'$
	has central character $\zeta$.

	Define $Q'=\GL_2(\Qp)\times Z_Q$
	and note that the natural 
	homomorphism  $Q'\to Q$
	is surjective with the kernel
	isomorphic to the center  $\Qp^\times$
	of  $\GL_2(\Qp)$.
	We also view
	$\Ord_P(S(U^p,E/\oo))_{\fm}'$
	as an admissible $Q'$-representation,
	with the usual $Z_Q$-action
	and twisted $\GL_2(\Qp)$-action.
	The restriction of which to $\GL_2(\Qp)$
	is locally admissible,
	and thus by \cite[Thm 2.3.8]{emeI}
	\[
		\Ord_P(S(U^p,E/\oo))_{\fm}'\in \lfMod_{\GL_2(\Qp),\zeta}(\oo).
	\]
\end{defn}


\begin{prop}\label{prop:compatibility}
	Let $\fC(\oo)$ be the dual category of 
	$\lfMod_{\GL_2(\Qp),\zeta}(\oo)$, then 
	$M(U^p)_{\fm}'\in \fC(\oo)^{\B}$
	and the $R_\fm^{\epsilon\zeta}$-action on which
	as the center of the category
	factors through $\TT(U^p,\oo)_\fm$.
\end{prop}

\begin{proof}
	For $\fp\subset \TT_{\wt{k}}^P(U^p\Iw^P(p^{0,1}),E)_{\fm}$,
	the associated representation $r_\pi$ defined 
	in Definition \ref{def:rep_prime}
	is crystalline at $D_w\cong \Gp$.
	Let $\pi_{\fp}$ denote the unramified
	$\GL_2(\Qp)$-representation 
	corresponding to the Weil-Deligne representation
	as in \eqref{eq:Gal_hecke_at_p}.
	By Proposition \ref{prop:wt_space} there exists
	a homomorphism between  $\GL_2(\Qp)$-representations
	\begin{equation}\label{eq:hom_wt}
		\pi_\fp\otimes \pi_{\wt{k}}^*(E)\to S(U^p)_\fm\otimes_{\oo}\E
	\end{equation}
	Use the same argument as \cite[Thm 3.5.5]{pan}
	and the density result in Proposition \ref{prop:density},
	it suffices to show that
	the universal unitary completion of
	the twisted $\GL_2(\Qp)$-representation
	of $\pi_{\fp}\otimes\pi_{\wt{k}}^*(E)$
	belongs to $\Irr(\fn)$.
	Here $\fn$ is the maximal ideal 
	of  $R_\fm^{\epsilon\zeta}[\frac{1}{p}]$ 
	which is the pull-back of $\fp$
	through the homomorphism induced by
	the pseudo-representation  $\xi_{\fm}^{1/2}\epsilon\Psi_\fm$.

	Recall that $r_\fp$
	has Hodge-Tate weights  $k_1+1,k_2$,
	where we write 
	$k_1=k_{\sigma,1}$ and $k_2=k_{\sigma,2}$
	for the unique $\sigma\in I_w$.
	Define $\xi_\fp^{1/2}=\lambda_\fp\circ \xi_\fm^{1/2}$
	and $\xi_\fp=(\xi_fp^{1/2})^2=\zeta(\epsilon\det r_\fp)^{-1}$.
	Suppose $\zeta$ has Hodge-Tate weight  $w_0$,
	then  $w_0-(k_1+k_2)$ is even as  $\xi_\fp\equiv \id$.
	Therefore  $\xi_{\fp}^{1/2}$
	is also crystalline,
	with Hodge-Tate weight $w_{\fp}\coloneqq \frac{1}{2}(w_0-k_1-k_2)$.
	Let $T\colon \Gp\to R_\fm^{\epsilon\zeta}$ 
	and $T_\fn\colon \Gp\to E$ be as last subsection.
	By definition  $T_\fn$ is the trace of the
	representation
	$\xi_{\fp}^{1/2}\epsilon r_{\fp}$,
	which has Hodge-Tate weights  $\{k_1',k_2'-1\}$ 
	for $k_1'=k_1+w_\fp, k_2'=k_2+w_\fp$.

	Since $\pi_\fp$ is also 
	the representation associated to  
	$D(\epsilon r_\fp)$ as in last subsection,
	the representation
	associated to  $D(\xi_\fp^{1/2}\epsilon r_\fp)$ is
	$\pi_\fp(\xi_\fp^{1/2}(\epsilon|\cdot|^{-1})^{w_\fp})$.
	Now, 
	twisting \eqref{eq:hom_wt} by $\xi_\fm^{1/2}$ gives
	\[
		(\pi_{\fp}\otimes \pi_{\wt{k}}^*(E))(\xi_\fp^{1/2})=
		\pi_\fp(\xi_\fp^{1/2}(\epsilon|\cdot|^{-1})^{w_\fp})
		\otimes (\pi_{\wt{k}}(\epsilon|\cdot|^{-1})^{w_{\fp}})^* \to 
		S(U^p)_\fm'\otimes_{\oo}E
	\]
	As $(\pi_{\wt{k}}(\epsilon|\cdot|^{-1})^{w_{\fp}})^*$
	is isomorphic to $\pi_{k_1,k_2}^*(E)$,
	the claim follows from Lemma \ref{lem:uni_completion}.
\end{proof}

On the other hand,
let $Z_Q\to \TT^p(U^p,\oo)_{\fm}^{\times}$
be the character
sending $x\in Z_Q^+$ to $h_U(x)$
and let 
$\chi\colon Z_Q\to \TT^P(U^p,\oo)/\fm=\fF^\times$
denote the residue character modulo $\fm$.
Let $Z_Q$ acts on $\pi_i\in \B$ by $\chi$.
Then the irreducible subquotients
of  $\Ord_P(S(U^p,E/\oo))_{\fm}'$
are all isomorphic to either $\pi_1$ or  $\pi_2$
by previus proposition.


\begin{defn}

Let $\lfMod_{Z_Q}(\oo)$
be the category of $\oo[Z_Q]$-modules
that are locally of finite length
and  $\fC_{Z_Q}(\oo)$ be the dual category.
We write 
$\tilde{J}_\chi\in \lfMod_{Z_Q}(\oo)$ and 
$\tilde{P}_\chi\in \fC_{Z_Q}(\oo)$
for the injective and projective envelopes
for $\chi$ and its Pontryagin dual.
The same argument in \cite[Prop 3.34]{pask}
shows that 
$\tilde{E}_\chi\coloneqq 
\End_{\fC_{Z_Q}(\oo)}(\tilde{P}_\chi)$
is the formal power series ring over $\oo$ 
of relative dimension $\dim\Hom^{\cont}(Z_Q,\fF)$
and $\tilde{P}_\chi$ is free of rank one over 
$\tilde{E}_\chi$.

\end{defn}


\begin{prop}\label{prop:envelope}
	Let $\tilde{P}_{i,\chi}=
	\tilde{P}_i\hat{\otimes}_{\oo}\tilde{P}_\chi$,
	and $R_{\fm, \chi}=R_\fm^{\epsilon\zeta}
	\hat{\otimes}_{\oo}\tilde{E}_\chi$.
	Then $\tilde{P}_{i,\chi}$
	is the projective envelope
	of $\pi_i$ and 
	$\End_{Q'}(R_{\fm, \chi}) \cong R_{\fm,\chi}$.
\end{prop}
\begin{proof}
	When $V\in \Mod_{Q'}(\oo)$
	is locally admissible and $v\in V$.
	Then  vector 
	$v$ is  $Z_Q$-finite by \cite[Lem 2.3.5]{emeI}
	and the $\oo[Q']-module$
	generated by $v$ is also 
	finitely-generated over  $\GL_2(\Qp)$
	and admissible.
	Thus it is also of finite length 
	by \cite[Thm 2.3.8]{emeI}.
	In other word,
	any locally admissible representation $V$
	is also locally of finite length.
	We may now apply \cite[Lem B.6]{GN}
	and use the same argument in \cite[Lem B.8]{GN}
	to prove the above lemma.
\end{proof}

\begin{cor}\label{cor:Hecke_Noetherian}
	Let $\tilde{P}_{\B,\chi}
	=\tilde{P}_\B\hat{\otimes}_{\oo}\tilde{P}_\chi$.
	Then $\mathbf{m}\coloneqq
	\Hom_{Q'}(\tilde{P}_{\B,\chi},M(U^p)_{\fm}')$
	is a finite faithful $\TT(U^p,\oo)_{\fm}$-module
	and the big Hecke algebra
	$\TT(U^p,\oo)$ is Noetherian.
\end{cor}
\begin{proof}
	Let $\tilde{E}_{\B,\chi}
	=\tilde{E}_\B\hat{\otimes}_{\oo}\tilde{E}_\chi$.
	Since $\Ord_P(S(U^p,E/\oo))_\fm'
	\in \fC_{Q'}(\oo)^{\B}$
	for $\B=\{\pi_1,\pi_2\}$,
	on which $M\to \Hom_{Q'}(\tilde{P}_{\B,\chi},M)$
	is an anti-equivalence,
	the action of $\TT(U^p,\oo)_\fm$
	on $\mathbf{m}$ is still faithful.

	Moreover, $\fm$ is finitely-generated
	over  $ \tilde{E}_{\B,\chi}$
	by \cite[Prop 4.17]{pask}.
	Since $ \tilde{E}_{\B,\chi}$
	is also finitely-generated
	over $R_{\fm,\chi}$
	and the action of which on $\mathbf{m}$
	factors through $\TT(U^p,\oo)_{\fm}$,
	we conclude that $\mathbf{m}$ 
	is fintie faithul over $\TT(U^p,\oo)_{\fm}$.
	As this induces an injective homomorphism
	of $R_{\fm,\chi}$-algebra
	$\TT(U^p,\oo)_{\fm}\to 
	\End_{R_{\fm,\chi}}(\mathbf{m})$,
	the Hecke algebra
	is also finite over $R_{\fm}$ and hence Noetherian.
\end{proof}


\subsection{Reducible part of completed homology}

Recall that the reducibility ideal 
$\tau\subset R^{\epsilon\zeta}=R^{\epsilon\zeta}_{\fm}$
is a principal ideal.
Let $R^{\epsilon\zeta,\red}$ be the quotient 
and denote $R_{\fm,\chi}^{\red}=R^{\epsilon\zeta,\red}
\hat{\otimes}_{\oo}\tilde{E}_\chi$.
If $M$ is a  $R_{\fm,\chi}$-module,
we define the reducible part $M^{\red}$ 
as the $R_{\fm,\chi}^{\red}$-module
$M/\tau M$.





\begin{lem}
	For some non-negative integer $r$
	there exists a projective resolution 
	\begin{equation}\label{eq:resolution}
	0\to \tilde{P}_{\B,\chi}^r\to 
	\tilde{P}_{\B,\chi}^r\to 
	M(U^p)_{\fm}'\to 0
	\end{equation}
\end{lem}
\begin{proof}
Identify $\Qp^\times$ with the center of $\GL_2(\Qp)$.
By \cite[Thm 33]{barthel} and \cite[Thm 19]{barthel}, 
there exists
smooth irreducible 
$\GL_2(\Zp)\Qp^\times$-representations $\sigma_i$
and exact sequences
\begin{equation}
	0\to 
	\cInd_{\GL_2(\Zp)\Qp^\times}^{\GL_2(\Qp)}\sigma_i\to
	\cInd_{\GL_2(\Zp)\Qp^\times}^{\GL_2(\Qp)}\sigma_i\to
	\pi_i\to 0
\end{equation}
Let $Z_Q$ acts on each 
of the above by $\chi$
and apply $\Ext^i_{Q'}(*,S')$
for $S'=\Ord_P(S(U^p,E/\oo))_{\fm}'$.
We obtain
\begin{equation*}
    \begin{tikzcd}[row sep=2ex]
	    \Ext^{i-1}_{Q'}
	    (\pi, S')\arrow[r] &
	    \Ext^{i}_{Q'}(\text{c-ind}_{KZ}^G\sigma, S')
	    \arrow[r] \arrow[d,equal] &
	    \Ext^{i}_{Q'}(\text{c-ind}_{KZ}^G\sigma, S')
	    \arrow[r] \arrow[d,equal] &
	    \Ext^{i}_{Q'}(\pi, S')\\ 
	 & \Ext^i_{\GL_2(\Zp)\times Z_Q}(\sigma ,S') &
	    \Ext^i_{\GL_2(\Zp)\times Z_Q}(\sigma ,S') &
    \end{tikzcd}
\end{equation*}
The same argument in Lemma \ref{lem:inj}
shows that $S'$ is still 
an injective $\GL_2(\Zp)$-representation
after the twist in Definition \ref{def:twist}.
Therefore the long exact sequence reduces to 
the following, in which 
all the terms 
are finite-dimensional over $\fF$
since $S'$ is $Q'$-admissible.
\begin{equation*}
	0 \to \Hom_{Q'}(\pi_i,S')\to 
	\Hom_{\GL_2(\Zp)\times Z_Q}(\sigma_i,S')\to 
	\Hom_{\GL_2(\Zp)\times Z_Q}(\sigma_i,S')\to 
	\Ext^1_{Q'}(\pi_i,S')\to 0
\end{equation*}



Write $a_i\coloneqq \dim_{\fF} \Hom(\pi_i,S')=
\dim_{\fF} \Ext^1(\pi_i,S')$
so that $\soc(S')=\pi_1^{a_1}\oplus \pi_2^{a_2}$.
Let $\tilde{J}_{i,\chi}$ denote 
the injective envelope of $\pi_i$.
Then the injective envelope 
$\soc(S')\hookrightarrow \tilde{J}=\tilde{J}_1^{'a_1}\oplus \tilde{J}_2^{'a_2}$
factors through an injective morphism 
$\phi_0\colon S'\to \tilde{J}$
since the inclusion $\soc(S')\hookrightarrow S'$
is essential.
Apply the same construction 
to $\soc(\coker(\phi_0))$
and use $\dim_{\fF}\Hom_{Q'}(\pi_i, \coker(\phi_0))=
\dim_{\fF}\Ext^1_{Q'}(\pi_i, S')=a_i$
gives another injective morphism
$\phi_1\colon \coker(\phi_0)\to \tilde{J}$,
which is also surjective as 
$\Hom(\pi_i,\coker(\phi_1))
\cong \Ext^1(\phi_i,\coker(\phi_0))
\cong \Ext^2(\pi, S')=0$.
Now the lemma follows by picking
$r=\max\{a_1,a_2\}$ and taking the Pontryagin dual.
\end{proof}

Let $A\colon \tilde{P}_\fm^r\to \tilde{P}_\fm^r$ 
denote the morphism
in the  resolution \eqref{eq:resolution}.
By the isomorphisms \eqref{eq:end_deform}
and Proposition \ref{prop:envelope},
the morphism 
can be represented by a matrix
$A=\smat{A_{11} & A_{12}\Phi_{12}\\A_{21}\Phi_{21} & A_{22}}$,
where $A_{ij}\in M_r(R_{\fm,\chi})$.
Let $\Ord$ be the functor of ordinary parts
on the upper triangular Borel subgroup of  $\GL_2(\Qp)$.
Then  
$\Ord(\tilde{P}_{i,\chi}^\vee)^\vee\cong 
\tilde{P}_{\chi_i^\vee,\chi}\coloneqq 
\tilde{P}_{\chi_i^\vee}\hat{\otimes}_{\oo}
\tilde{P}_\chi$.
And $\Ord(A)$
is represented by the matrix
$\smat{\tilde{A}_{11} & \\& \tilde{A}_{22}}$,
where $\bar{A}_{ij}\in M_r(R^{\red}_{\fm,\chi})$ denote 
the reduction of the matrices by Lemma \ref{lem:ker_red}.
We apply the right exact $\Ord$ to 
\eqref{eq:resolution} and obtain
\begin{equation}\label{eq:exact_ord}
	\tilde{P}_{\chi_1^\vee,\chi}^r\oplus 
	\tilde{P}_{\chi_2^\vee,\chi}^r
	\xrightarrow{\overline{A}_{11}\oplus\overline{A}_{22}}
	\tilde{P}_{\chi_1^\vee,\chi}^r\oplus 
	\tilde{P}_{\chi_2^\vee,\chi}^r
	\to (\Ord S')^\vee\to 0
\end{equation}


\begin{prop}    
	The matrices 
	$\bar{A}_{ii}\in M_r(R^{\red}_{\fm,\chi})$
	are injective and the sequence above 
	is also left exaxt.
\end{prop}
\begin{proof}
	We first note that 
	$\Ord(\Ord_P(S(U^p,E/\oo))=\Ord_B(S(U^p,E/\oo))$
	is an admissible $T$-representation
	for the diagonal torus $T\subset G_p$,
	thus so is $\Ord(S')$.
	But then the Pontryagin dual 
	$\Ord(S')^\vee$ is finitely-generated
	over  $\Lambda_T\coloneqq
	\oo\llbracket T(\oo)^{\wedge}\rrbracket$
	by \cite[Lem 2.2.11]{emeI}.
	Apply $\Hom_{T}(\tilde{P}_{\chi_i^\vee,\chi},*)$
	to the \eqref{eq:exact_ord} shows
\begin{equation*}
	\End_{T}(\tilde{P}_{\chi_i^\vee,\chi}^r)\xrightarrow{\bar{A}_{ii}}
	\End_{T}(\tilde{P}_{\chi_i^\vee,\chi}^r)\to 
	\Hom_{T}(\tilde{P}_{\chi_i^\vee,\chi}, ,(\Ord S')^\vee)
	\to 0
\end{equation*}
	The two terms on the left 
	are finite free over $R_{\fm,\chi}^{\red}$
	while the last term is torsion over which
	since $\oo\llbracket T(\oo)^{\wedge}\rrbracket$
	is a power series ring of relative dimension
	strictly smaller than that of $R_{\fm,\chi}^{\red}$.
	Passing to the field of fraction of $R_{\fm,\chi}^{\red}$,
	we see that $\bar{A}_ii$ has to be invertible
	and the result follows.
\end{proof}

Let $M^{\ord}(U^p)_\fm$
denote the Pontryagin dual
of $\Ord(\Ord_P(S(U^p,E/\oo))_{\fm})=\Ord_B(S(U^p,E/\oo))_{\fm}$,
on which $\smat{p&\\&1}$ acts by  $U_{w}^{(1)}$.
Then \eqref{eq:exact_ord}
implies that 
$M^{\ord}(U^p)_{\fm}$ is the direct sum of 
$M^{\ord}(U^p)_{\fm}^{U_{w}^{(1)}\equiv \chi_i(p)}$,
the subspace where $U_{w}^{(1)}$ acts residually by $\chi_i(p)$.

\begin{cor}\label{cor:no_torsion}
	Let $x$ be a generator
	of $\tau$, then $M(U^p)_{\fm}'$
	has no $x$-torsion.
\end{cor}
\begin{proof}
	It suffices to show that
	$\Hom_{Q'}(\tilde{P}_{j,\chi}, M(U^p)_{\fm}'[x])=0$
	for $j=1,2$.
	Apply the snake lemma to
    \begin{equation*}
    \begin{tikzcd}
        0 \arrow[r] & \tilde{P}_{\B,\chi}^{\oplus r} 
	\arrow[d,"x",hookrightarrow] \arrow[r,"A"] & 
	\tilde{P}_{\B,\chi}^{\oplus r} 
	\arrow[d,"x",hookrightarrow] \arrow[r] & 
	M(U^p)_{\fm}'\arrow[d,"x"] \arrow[r] & 0 \\ 
        0 \arrow[r] & \tilde{P}_{\B,\chi}^{\oplus r}
	\arrow[r,"A"] & \tilde{P}_{\B,\chi}^{\oplus r}
	\arrow[r] &M(U^p)_{\fm}'  \arrow[r] & 0 
    \end{tikzcd}
\end{equation*}
Since the $R$-action
on $\tilde{P}_{i,\chi}$ are faithful for both $i=1,2$,
we see that $\Hom_{Q'}(\tilde{P}_{j,\chi}, M(U^p)_{\fm}'[x])$
is isomorphic to the kernel below,
which is trivial by the previous proposition.
\[
	\ker\left(
	\Hom_{Q'}(\tilde{P}_{j,\chi}, 
	\tilde{P}_{\B,\chi}^{\red})^{\oplus r}
	\xrightarrow{\smat{\bar{A}_{ii}& \bar{A}_{ij}\Phi_{i}\\& \bar{A}_{jj}}}
	\Hom_{Q'}(\tilde{P}_{j,\chi}, 
	\tilde{P}_{\B,\chi}^{\red})^{\oplus r}\right)
\]
\end{proof}


\begin{cor}\label{cor:fil_by_ord}
	We have the following exact sequence of
	$R_{\fm,\chi}^{\red}$-modules
\begin{equation}
	0\to M^{\ord}(U^p)_{\fm}^{U_{w}^{(1)}\equiv \chi_1(p)}\to
	\Hom_{Q'}(\tilde{P}_{2,\chi},M(U^p)_{\fm}')^{\red}\to
	M^{\ord}(U^p)_{\fm}^{U_{w}^{(1)}\equiv \chi_2(p)}\to0.
\end{equation}
A similarly exact sequence also exsits
for $\Hom_{Q'}(\tilde{P}_{1,\chi},M(U^p)_{\fm}')^{\red}$.
\end{cor}
\begin{proof}
	The last step in the proof of the previous corollary
	gives the following commutative diagram
\begin{equation*}
    \begin{tikzcd}
	    0 \arrow[r]& 
	    \Hom_{Q'}(\tilde{P}_{2,\chi},\tilde{P}_{1,\chi}^{\red})^{\oplus r}
	    \arrow[r,"\bar{A}_{11}"] \arrow[d]&
	    \Hom_{Q'}(\tilde{P}_{2,\chi}, \tilde{P}_{1,\chi}^{\red})^{\oplus r}
	    \arrow[d] &&\\
	    0\arrow[r] & 
	    \Hom_{Q'}(\tilde{P}_{2,\chi},\tilde{P}_{\B,\chi}^{\red})^{\oplus r}
	    \arrow[r] \arrow[d] &
	    \Hom_{Q'}(\tilde{P}_{2,\chi}, \tilde{P}_{\B,\chi}^{\red})^{\oplus r}
	    \arrow[d] \arrow[r]&
	    \Hom_{Q'}(\tilde{P}_{2,\chi}, M(U^p)_{\fm}')^{\red}\arrow[r]&0\\
	    0\arrow[r] & 
	    \Hom_{Q'}(\tilde{P}_{2,\chi}, \tilde{P}_{2,\chi}^{\red})^{\oplus r}
	    \arrow[r,"\bar{A}_{22}"] &
	    \Hom_{Q'}(\tilde{P}_{2,\chi},\tilde{P}_{2,\chi}^{\red})^{\oplus r}&&
    \end{tikzcd}
\end{equation*}
Since $\tilde{P}_{i,\chi}\cong \tilde{M}_i\hat{\otimes}_{\oo}\tilde{P}_{\chi}$,
by Lemma \ref{lem:ker_red}
all the terms 
$\Hom_{Q'}(\tilde{P}_{2,\chi},\tilde{P}_{2,\chi}^{\red})$
are free of rank one over  $R_{\fm,\chi}^{\red}$.
Compare the cokernels of the first and last rows
with \eqref{eq:exact_ord}
and note that $\tilde{P}_{\chi_i^\vee,\chi}$
are also free of rank one $R_{\fm,\chi}^{\red}$,
we obtain the desired result.
\end{proof}  

\begin{cor}\label{cor:Hecke_finite}
The Hecke algebra 
$\TT(U^p,\oo)_{\fm}$ 
is finite over
$\oo\llbracket x\rrbracket \hat{\otimes}_{\oo}\Lambda_T
=\Lambda_T\llbracket x\rrbracket$.
\end{cor}
\begin{proof}
	Consider the natural
	$\Lambda_T\llbracket x\rrbracket$-action on
	$\mathbf{m}=\Hom_{Q'}(\tilde{P}_{\B,\chi},
	M(U^p)_{\fm}')$.
	Since $M^{\ord}(U^p)_{\fm}$ 
	is fintie over $\Lambda_T$,
	by the corollary above 
	we see that 
	$\mathbf{m}^{\red}=\mathbf{m}/x\mathbf{m}$ 
	is finitely-generated over $\Lambda_T$.
	Moreover $\mathbf{m}$ is a compact 
	topological space 
	since it is finite over  
	the complete Noetherian local ring 
	$\TT(U^p,\oo)_{\fm}$ 
	by Corollary \ref{cor:Hecke_Noetherian}.
	Therefore $\mathbf{m}$
	is finite over $\Lambda_T\llbracket x\rrbracket$
	by the topological Nakayama lemma.
	As $\TT(U^p,\oo)_{\fm}$ acts faithfully
	on $\mathbf{m}$,
	we have an injective homomorphism
	$\TT(U^p,\oo)_{\fm}\hookrightarrow 
	\End_{\Lambda_T\llbracket x\rrbracket}(\mathbf{m})$,
	which proves the claim.
\end{proof}


\subsection{Fundamental exact sequence}
\label{sub:fund_exact_sequence}

We are now ready to 
formulate the fundamental
exact sequence 
that generalize \cite[Prop 6.3.5]{urban}.
To simplify notations,
we write 
$\Lambda=\Lambda_T$,
$R=R_{\fm,\chi}$,
$\TT=\TT(U^p,\oo)_{\fm}$, and
$M=\Hom_{Q'}(\tilde{P}_{2,\chi},M(U^p)_{\fm}')$.
We fix a generator $x$ of the reducibility
ideal in $R$.
Note that 
while $R^{\red}$ and $\TT$
are  $\Lambda$-algebras,
the deformation ring $R$ 
has no natural $\Lambda$-algebra structure.
We also write
$M_1=M^{\ord}(U^p)_{\fm}^{U_{w}^{(1)}\equiv \chi_1(p)}$
and
$M_2=M^{\ord}(U^p)_{\fm}^{U_{w}^{(1)}\equiv \chi_2(p)}$.
Note that $M^{\red}$ is finite over $\Lambda$
by Corollary \ref{cor:fil_by_ord}.


Let $\lambda\colon \TT\to \Lambda'$
be a nonzero homomorphism of $\Lambda$-algebras
to an integral domain $\Lambda'$
which is a quotient of  $\Lambda$.
We make the following assumptions
on $M_2'=M_2\coloneqq_{\Lambda}\Lambda'$.
\begin{enumerate}[label=(C\roman*)]
\item The image of $x$ in  $\TT$ belongs to  $\ker(\lambda)$.
	\label{cond:C1}
\item There exists a nonzero 
$\Lambda$-modules homomorphism
$\Theta\colon M_2'\to \Lambda'$  such that 
$\Theta(T\cdot m)=\lambda(T)\cdot \Theta(m)$
for $T\in \TT_{\fm}, m\in M_2'$.
	\label{cond:C2}
\item There exists an ideal 
$\fq\subset \ker(\TT\to \End(M_1))$
containing the image of $x$ under $R\to \TT$
such that $\fq M_2/\fq S_2$
is torsion over $\Lambda'$ for 
$S_2\coloneqq\ker(M_2\to M_2'\xrightarrow{\Theta}\Lambda')$.
Here the $\Lambda'$-structure comes from the surjective
map  $\fq\otimes_{\TT}(M_2/S_2)\to \fq M_2/\fq S_2$.
	\label{cond:C3}

\end{enumerate}

\begin{lem}
Let $S=\ker(M\to M^{\red}\to M_2/S_2)$,
then there exists an exact sequence
\begin{equation}\label{eq:fund}
	M/S\xrightarrow{x} \fq M/\fq S\to 
	\fq M_2/\fq S_2 \to 0
\end{equation}
\end{lem}
\begin{proof}

Applying the snake lemma to 
the commutative diagram below
\[
\begin{tikzcd}
	S\arrow[r,"x"] \arrow[d]
	& \fq S \arrow[r] \arrow[d]
	& \fq S^{\red} \arrow[r] \arrow[d] & 0\\
	M\arrow[r,"x"]
	& \fq M \arrow[r]
	& \fq M^{\red} \arrow[r] & 0
\end{tikzcd}
\]
It then suffices to show that 
$\fq M^{\red}/\fq S^{\red}=\fq M_2/\fq S_2$.
This follws from that the natural map from
$\fq M^{\red}=\fq M/xM$ to $M^{\red}=M/xM$
is injective and
\[
	\Image(\fq M^{\red}\to M^{\red})=\fq M/xM=\fq M_2,\quad
	\Image(\fq S^{\red}\to M^{\red})=\fq S+xM/xM=\fq S_2.
\]

\end{proof}

Let $\wp=\ker(\lambda)$,
which is a prime ideal of $\TT$
since $\Lambda'$ is an integral domain.
Let $\mathbb{K}=(\Lambda')_{\wp}$ 
be the field of fraction
and apply 
$\otimes_{\TT}\TT_{\wp}=
\otimes_{\Lambda'}(\Lambda'\otimes_{\TT}\TT_\wp)=
\otimes_{\Lambda'}\mathbb{K}$ 
to the sequence \eqref{eq:fund}.
As $M/S\hookrightarrow \Lambda'$
and $\fq M_2/\fq S_2$ is $\Lambda'$-torsion, 
we obtain a surjective homomorphism 
of $\mathbb{K}$-vector spaces
\begin{equation}\label{eq:fund_frac}
	\mathbb{K}\twoheadrightarrow (\fq M/\fq S)_{\wp}
	= \fq M_\wp/\fq S_\wp\to 0
\end{equation}


\begin{lem}
	There exists a minimal prime ideal
	$\fp\subset \TT_{\wp}$
	such that $\TT_{\wp}/\fp$
	is an finite ring extension
	is over $\mathbb{K}\llbracket x\rrbracket$.
	In particular,
	the integral closure $\tilde{\TT}$
	of $\mathbb{K}\llbracket x\rrbracket$
	in $\textnormal{Frac}(\TT/\fp)$
	is a discrete valuation ring.
\end{lem}
\begin{proof}
	It suffices to show that 
	the image of $x$ in  $\TT_\wp$
	is not a nilpotent.
	Since $\TT$ is reduced 
	by \cite[Lem 2.14]{ger}, 
	so is $\TT_\wp$,
	therefore the result follows from that
	$x$ acts through  $\TT$ nontrivially
	on  $M$ by Corollary \ref{cor:no_torsion}
\end{proof}

\begin{prop}
	The sequence \eqref{eq:fund} is also left exaxt.
\end{prop}
\begin{proof}
	Since $M/S$ is isomorphic to a nonzero submodule 
	in  $\Lambda'$,
	by \eqref{eq:fund_frac} it suffices to show that 
	$\fq M_\wp/\fq S_\wp\neq 0$.
	Let $\mathbb{M}$ and  $\mathbb{S}$
	denote  $M_\wp$ and $S_\wp$.
	As $\fq\tilde{\TT}$ is a principal ideal
	in the d.v.r. $\tilde{\TT}$,
	we have $\fq\tilde{\TT}\cong \tilde{\TT}$ and 
\[
	(\fq \mathbb{M}/\fq \mathbb{S})\otimes_\TT
	\tilde{\TT}\cong
	\fq\mathbb{M}\otimes_\TT\tilde{\TT}/
	\fq\mathbb{S}\otimes_\TT\tilde{\TT}=
	\mathbb{M}\otimes_\TT\fq\tilde{\TT}/
	\mathbb{S}\otimes_\TT\fq\tilde{\TT} \cong 
	\mathbb{M}\otimes_\TT\tilde{\TT}/
	\mathbb{S}\otimes_\TT\tilde{\TT}\cong
	(\mathbb{M}/\mathbb{S})\otimes_\TT\tilde{\TT}=
	\mathbb{K}\otimes_{\lambda,\TT}\tilde{\TT}.
\]
The last ring is nonzero since 
$\tilde{\TT}$ is integral over $\TT$
and the maximal ideals of  $ \tilde{\TT}$ 
restricts to that of $\TT$.
\end{proof}

\begin{rem}
$\TT$ is etale over  $\Lambda\llbracket x\rrbracket$.
\end{rem}

\section{Euler systems}

Let $\chi_\circ$ be an algebraic Hecke
character of $\A_{\K}^\times/\K^\times$
of infinity type $\Sigma^c$ such that 
$\chi_\circ\mid_{\A_\F^\times}=\qch_{\K/\F}|\cdot|_\F$.
In this section,
we use the fundamental exact sequence
and the Hida family of theta lifts in \cite{lee}
to construct an anticyclotomic Euler system
for (the $p$-adic avatar of) $\chi_\circ^{-2}|\cdot|_\K$,
which has the infinity type $\Sigma-\Sigma^c$.

\subsection{Hida family}

We first recall the space of Hida families
used in \textit{loc.cit}.
Let $\fc$ be the conductor of  $\chi_\circ$
and let $\rk{\fc}$ be the maximal compact subgroup
of $G(\A_f)$ defined in the beginning of \cite[\S 6]{lee}.
In particular, the group $\rk{\fc}$ contains $K_p$
and satisfies \eqref{cond:small}.
We will vary $\fn=\fc\fs$,
where $\fs=\bar{\fs}$ is a
square-free, prime-to-$p\fc$
ideal of  $\K$ consists only of split primes
such that  $\rk{\fc}$
contains  $\iota_w^{-1}(\GL_2(\oo_w))$
whenever $w\mid \fs$.
For each prime $\ell$ of $\F$
such that  $\ell\mid \fs$ we 
fix  $\ell=\fl\bar{\fl}$ and define
\begin{align*}
	K(\fn)&=
	\{
	k\in K(\fc)\mid
	\iota_{\fl}(k_\ell)\equiv
	(\begin{smallmatrix}
		1&*\\&*
	\end{smallmatrix})\mod \fl
	\text{ for } \ell\mid \fs
	\}\\
	K_1^n(\fn)&=
	\{
	k\in K(\fn)\mid
	\iota_{w}(k_v)\equiv
	(\begin{smallmatrix}
		1&*\\&*
	\end{smallmatrix})\mod \varpi_w^n
	\text{ for } w\in \Sigma_p
	\text{ and }w\mid v
	\}
\end{align*}
Note that if we decompose $K(\fn)=U^pU_p$,
then $K^n_1(\fn)$ is generated by $U^p\Iw(p^{n,n})$ 
and $\iota_{w}^{-1}(\smat{1&\\&\oo_w^\times})$
for all $w\in \Sigma_p$.
We may then define 
$S_{\wt{k}}(K^n_1(\fn),M)$
as the subspace of 
$S_{\wt{k}}(U^p\Iw(p^{n,n}),M)$
that are invariant by 
$\iota_{w}^{-1}(\smat{1&\\&\oo_w^\times})$
for all $w\in \Sigma_p$
with respect to the natural action
defined in Definition \ref{def:algform}.
The restriction of $U_{\wt{k},w}^{(j)}$
on $S_{\wt{k}}(U^p\Iw(p^{n,n}),M)$
is simply the double-coset operator 
defined as \eqref{def:hecke_at_p}.
We would now also include
Hecke operators defined at $\ell\mid \fs$.
For the fixed decomposition $\ell=\fl\bar{\fl}$,
we define $U_{\fl}^{(j)}$ and $\langle u\rangle$
as in \eqref{def:hecke_at_p},
after identifying $G(\F_\ell)$
and  $\GL_2(\oo_\fl)$ with  $\iota_\fl$.
Note that the Hecke operators 
do not depend on the wieght  $\wt{k}$.

\begin{defn}\label{def:ord_space_auxlevel}
Let $U_\fs$ be the product of all $  U_{\fl}$
for $\ell\mid \fs$ and $U_p=U_B$ be defined 
as in Definition\ref{def:Iwahori}.
When $M$ is a finite  $\oo$-module or the 
Pontryagin dual of which,
we define 
\[
	S_{\wt{k}}^{\ord}(K^n_1(\fn),M)\coloneqq
	e_pe_{\fs} S_{\wt{k}}(K^n_1(\fn),M)\quad
	e_p=\lim_{n\to \infty}U_p^{n!},\,
	e_\fs=\lim_{n\to \infty}U_\fs^{n!}.
\]
We define $\TT^{\ord}_{\wt{k}}(K^n_1(\fn),M)$
to be the subalgebra of 
$\End_{\oo}(S_{\wt{k}}^{\ord}(K^n_1(\fn),M)$
generated by $T_w^{(j)}, (T_{w}^{(2)})^{-1}$
and the Hecke operators at $p$ and  $\fs$ defined above.
\end{defn}
By the definition,
the big ordinary Hecke algebra
$\TT^{\ord}(\fn,\oo)=\varprojlim_n
\TT^{\ord}(K^n_1(\fn),\oo)$
acts faithfully on the spaces
$S^{\ord}(\fn,E/\oo)=\varinjlim_{n\to \infty}
S^{\ord}(K^n_1(\fn),E/\oo)$,
$M^{\ord}(\fn)=S^{\ord}(\fn,E/\oo)^\vee$, and
\begin{equation*}
	S^{\ord}(\fn)\coloneqq
	\Hom_\oo(E/\oo, S^{\ord}(\fn,E/\oo))
	\cong \varprojlim_r\varprojlim_n 
	S^{\ord}(K^n_1(\fn),\oo/\varpi^r).
\end{equation*}

\begin{defn}\label{def:lambda_rings}
Let $\oo_{\Sigma_p}=\prod_{w\in\Sigma_p}\oo_w$
and $\Lambda\coloneqq
\oo\llbracket(\oo_{\Sigma_p}^\times)^\wedge\rrbracket$
be the completed group ring on the pro-$p$ completion.
We let $\Lambda$ acts on the above spaces
through the Hecke operators  $\langle\cdot\rangle$,
where we identify each of  $\oo_w^\times$
with the center of $\GL_2(\oo_w)$.
Similarly, let $\Delta_\fl$ be the 
pro-$p$ subgroup in $\oo_\fl^\times$
and $\Delta_{\fs}=\prod_{\ell\mid \fs}\Delta_{\fl}$.
We apply the same procedure 
and extends the action of $\Lambda$
to $\Lambda_{\fs}=\Lambda[\Delta_{\fs}]$.
\end{defn}

Recall that in previus section,
we defined $\Lambda_T$
to be the completed group ring 
on the pro-$p$ completion of 
the diagonal torus of $K_p$.
Then $\Lambda$ is a quotient of  $\Lambda_T$
through the projection to the upper-left entry.
\begin{lem}\label{lem:coh_to_ord}
	There exists a Hecke-equivariant
	surjective homomorphism 
	$M(K(\fn)^p)\to M^{\ord}(\fn)$. 
\end{lem}
\begin{proof}
By the definition above and Lemma \ref{lem:PtoB}
we have the following chain of 
inclusions
\[
	S_{\wt{k}}^{\ord}(K^n_1(\fn),M)\subset
	S_{\wt{k}}^{B-\ord}(K(\fn)^p\Iw(p^{n,n}),M)\subset
	S_{\wt{k}}^{P-\ord}(K(\fn)^p\Iw^P(p^{n,n}),M)
\]
which are equivariant with the Hecke operators
that are defined at both spaces.
Taking the injective limit and the Pontryagin dual 
gives the desired homomorphism.
We also note that the $\Lambda_T$-action
factors through  the $\Lambda$-action defined in
Definition \ref{def:lambda_rings}.
\end{proof}


Now, let $\K_{p^n\fs}$ be the ray class field of conductor 
$p^n\fs$ over $\K$ and 
$\K_{p^\infty\fs}=\bigcup_{n}\K_{p^n\fs}$.
We let $\fG_{\fs}$ denote
the Galois group of the maximal pro-$p$ 
subextension in $\K_{p^\infty\fs}$.
As $p$ is odd, the complex conjugation
decompose  $\fG_{\fs}$ into
$\fG_{\fs}^+\times\fG_{\fs}^-$.
Since we assume that the class number 
$h_\K$ is prime to  $p$,
the group $\fG_{\fs}^-$
is isomorphic to 
$\Delta_{\fs}\times \oo_{\Sigma_p}^\times$ 
by the reciprocity map of class field theory.

\begin{defn}
We define a Hida family as a 
$S^{\ord}(\fn)$-valued measure $\euF$ on $\fG_{\fs}^-$
such that 
\begin{equation}\label{def:Hida_family}
	\langle u\rangle \cdot 
	\int_{\fG_{\fs}^-}\alpha\,d\euF=
	\alpha(u)\cdot \int_{\fG_{\fs}^-}\alpha\,d\euF
\end{equation}
for continuous functions $\alpha$ on  $\fG_{\fs}^-$
and $u\in \fG_{\fs}^{-}$, 
identified with elements in $\Lambda_{\fs}$
via the reciprocity map.
Let $S^\ord_\Lambda(\fn)$
denote the space of Hida family
and let the Hecke algebra
$\TT^{\ord}(\fn,\oo)$
act on which through the target.
\end{defn}

\begin{prop}\label{prop:ord_to_dual}
	There eixsts a Hecke-equivariant surjective 
	homomorphism 
	\[
		\Phi\colon M^{\ord}(\fn)\to 
		\Hom_{\Lambda_{\fs}}
		(S_\Lambda^{\ord}(\fn),\Lambda_{\fs})
	\]
\end{prop}
\begin{proof}
Since $\Hom_\oo(S^{\ord}(\fn),\oo)\cong M^{\ord}(\fn)$,
given $m\in M^{\ord}(\fn)$ and
$\euF\in S^{\ord}(\fn,\Lambda_{\fs})$,
we define $\Phi(m)(\euF)$ as the 
$\oo$-valued measure which is the composition
of $m$ with  $\euF$.
In other word, 
\begin{equation}
	\Phi(m)\colon \euF\mapsto
	[\alpha\mapsto m(f_\alpha)]
	\in \Hom_{\Lambda_{\fs}}
	(S_\Lambda^{\ord}(\fn),\Lambda_{\fs})\quad
	m\in \Hom_\oo(S^{\ord}(\fn),\oo)\cong M^{\ord}(\fn)
\end{equation}
The Hecke-equivalence of $\Phi$
is clear from the definition,
and that $\Phi(m)$ is $\Lambda_{\fs}$
is equivariant to that 
$\Phi(m)$ sends $\langle u\rangle \euF$
to $[\alpha\mapsto \alpha(u)\cdot m(f_\alpha)]$,
which is the defining property \eqref{def:Hida_family}.

Now, let $I_\fs\subset\Lambda_{\fs}$ be the augmentation ideal
and define the open compact subgroup
\begin{align*}
	K_0(\fn)&=
	\{
	k\in K(\fc)\mid
	\iota_{w}(k_v)\equiv
	(\begin{smallmatrix}
		*&*\\&*
	\end{smallmatrix})\mod \varpi_\fl
	\text{ for } \fl\mid \fs
	\}\\
	K_0(\fn p^n)&=
	\{
	k\in K_0(\fn)\mid
	\iota_{w}(k_v)\equiv
	(\begin{smallmatrix}
		*&*\\&*
	\end{smallmatrix})\mod \varpi_w^n
	\text{ for } w\in \Sigma_p
	\}
\end{align*}
Then $M^{\ord}(\fn)/I_\fs M^{\ord}(\fn)$
is isomorphic to the Pontryagin dual
of $S^{\ord}(K_0(\fn p),E/\oo)$;
while $S^{\ord}_\Lambda(\fn)/I_\fs S^{\ord}_\Lambda(\fn)$
is isomorphic to the subspace 
of modular forms in $S^{\ord}(K_0(\fn p))$ that belong
to a Hida family. Therefore
\[
	M/I_{\fs}M\cong \Hom_{\oo}(S^{\ord}(K_0(\fn p),\oo)
	,\oo) \twoheadrightarrow \Hom_{\oo} 
	(S_\Lambda^{\ord}(\fn)/I_\fs S_\Lambda^{\ord}(\fn)
	,\oo)
	=\Hom_{\Lambda_{\fs}} 
	(S_\Lambda^{\ord}(\fn),\Lambda_{\fs})
	\otimes_{\Lambda_{\fs}}(\Lambda_{\fs}/I_{\fs}).
\]
Since the two spaces in question
are finitely-genreated over $\Lambda_{\fs}$
by Lemma \ref{lem:coh_to_ord},
the surjectivity follows from the Nakayama lemma.
\end{proof}

Let $B_\fn\coloneqq S_\Lambda^{\ord}(\fn)\times
S_\Lambda^{\ord}(\fn)\to \Lambda_{\fs}$ 
by the $\Lambda_{\fs}$-bilinear pairing
on Hida families defined in \cite[\S 6.4.1]{lee}.
To compare the pairing between different
auxiliary levels $\fn=\fc\fs$,
define $I_{\ell}=\ker(\Lambda_{\fs\ell}\to \Lambda_{\fs})$
when $\ell\nmid \fs$.
The reduction 
$B_{\fn\ell}^{\ell}\coloneqq B_{\fn\ell}\text{ mod }I_\ell$
is a $\Lambda_{\fs}$-valued pairing on 
$S_\Lambda^{\ord}(\fn\ell)/I_\fs S_\Lambda^{\ord}(\fn\ell)$.
This last space is by definition
the space of Hida families that are valued 
in $\varprojlim_r\varprojlim_n 
S^{\ord}((K^n_1(\fn))^{\fs}K_0(\fs),\oo/\varpi^r)$,
where the componets above $\fs$ in the subgroup $K^n_1(\fn)$
are replaced by that of in $K_0(\fn p)$.
The inclusion map 
$\id\colon S^{\ord}(K^n_1(\fn),M)\subset
S^{\ord}((K^n_1(\fn))^{\fs}K_0(\fs),M)$
and the map
$V_{\fl}\colon S^{\ord}(K^n_1(\fn),M)\to
S^{\ord}((K^n_1(\fn))^{\fs}K_0(\fs),M)$
given by $V_\fl\cdot f(g)=
f(g\iota_\fl^{-1}(\smat{1&\\&\varpi_\fl})$ induces
\[
\id, V_\fl\colon S^{\ord}_{\Lambda}(\fn)\to
S_\Lambda^{\ord}(\fn\ell)/I_\fs S_\Lambda^{\ord}(\fn\ell)
\]

\begin{prop}\label{prop:pair_at_deff_level}
	With notations above we have the following
	relation between $B_{\fn}$
	and $B_{\fn\ell}^\ell$.
	\begin{align*}
	%&B_{\fn}(T_{w}^{(1)}\cdot\euF_1,\euF_2)=
	%B_{\fn}(\euF_1,T_{w}^{(1)}\cdot\euF_2)\\
	&B_{\fn\ell}^{\ell}(\euF_1,\euF_2)=
	B_{\fn}(\euF_1,T_{\fl}^{(1)}\cdot \euF_2)\\
	&B_{\fn\ell}^{\ell}(\euF_1,V_{\fl}\cdot\euF_2)=
	(q_\ell+1) B_{\fn}(\euF_1,
	T_{\fl}^{(2)}\cdot\euF_2)\\
	&B_{\fn\ell}^\ell
	(V_{\fl}\cdot \euF_1,V_{\fl}\cdot\euF_2)=
	B_{\fn} (T_{\fl}^{(1)}\cdot\euF_1,
	T_{\fl}^{(2)}\cdot \euF_2)
	\end{align*}
\end{prop}
\begin{proof}
	By definition,
	the pairing $B_{\fn\ell}^\ell$
	and $B_{\fn}$ are defined by 
	interpolating the pairings 
	on modular forms
	$f_1,f_2\in S^{\ord}(K_0(\fn p^n),\oo)$
	given by 
	\[
	\sum_{G(\F)\backslash G(\A_f)/K_0(\fn\ell p^n)}
	f_1(g)f_2(\bar{g}\tau_{\fs\ell}^n)\text{ and }\quad
	\sum_{G(\F)\backslash G(\A_f)/K_0(\fn p^n)}
	f_1(g)f_2(\bar{g}\tau_{\fs}^n)
	\]
	Here $\tau_{\fs}$ is the product 
	$\tau_\fl\coloneqq
	\iota_{\fl}^{-1}(\smat{\varpi_\fl&\\&1})$
	for $\fl\mid \fs$
	and  $\tau_{\fs}^n$ is the product
	of $\tau_{\fs}$ and 
	$\iota_{w}^{-1}(\smat{\varpi_w^n&\\&1})$
	for all $w\in \Sigma_p$.
	Identify $G(\F_\ell)$ with  $\GL_2(\K_\fl)$
	via  $\iota_\fl$, and 
	write $K_\ell=\GL_2(\oo_\fl)$, we see that 
	\begin{multline*}
	\sum_{G(\F)\backslash G(\A_f)/K_0(\fn\ell p^n)}
	f_1(g)f_2(\bar{g}\tau_{\fs\ell}^n)=
	\sum_{G(\F)\backslash G(\A_f)/K_0(\fn p^n)}
	\sum_{k\in K_\ell/K_0(\ell)}
	f_1(gk)f_2(\bar{g}\bar{k}\tau_{\fs\ell}^n)\\=
	\sum_{G(\F)\backslash G(\A_f)/K_0(\fn p^n)} f_1(g)
	\sum_{k\in K_\ell/K_0(\ell)}
	f_2(\bar{g}\bar{k}\smat{1&\\&\varpi_\fl}
	\tau_{\fs}^n)=
	\sum_{G(\F)\backslash G(\A_f)/K_0(\fn p^n)} f_1(g)
	(T_\fl^{(1)}f_2)(\bar{g}\tau_{\fs}^n).
	\end{multline*}
	%That the pairing is symmetric 
	%follows from that 
	%$g\mapsto \bar{g}\smat{\varpi_\fl&\\&1}$
	%is an involution on 
	%$G(\F)\backslash G(\A_f)/K^n_0(\fn)$.
	%Now, we identify
	%$G(\F_\ell)$ and  $\GL_2(\oo_\fl)$
	%via  $\iota_{\fl}$, then
	%\begin{multline*}
	%B_{\fn}(T_{\fl}^{(1)}f_1, f_2)=
	%\sum_{G(\F)\backslash G(\A_f)/K^n_0(\fn)}
	%\left[\sum_b f_1(g\smat{\varpi_\fl&b\\&1})
	%+f_1(g\smat{1&\\&\varpi_\fl}) \right]
	%f_2(\bar{g}\tau_{\fs}(n))\\=
	%\sum_{G(\F)\backslash G(\A_f)/K^n_0(\fn)K^0(\ell)}
	%f_1(g\smat{\varpi_\fl&\\&1})
	%f_2(\bar{g}\tau_{\fs}(n))=
	%\sum_{G(\F)\backslash G(\A_f)/K^n_0(\fn)K^0(\ell)}
	%f_1(\bar{g}\tau_{\fs}(n))
	%f_2(g\smat{\varpi_\fl&\\&1})\\=
	%B_{\fn}(T_\fl^{(1)}\cdot f_2, f_1)=
	%B_{\fn}(f_1, T_\fl^{(1)}\cdot f_2)
	%\end{multline*}
	%The third equality follow from 
	%that $g\mapsto 
	%\bar{g}\smat{\varpi_\fl^{-1}&\\&1}\tau_\fs(n)$
	%is an involution on the underlying set.
	And the first equality follows.
	For the second equality, we have
	\begin{multline*}
	\sum_{G(\F)\backslash G(\A_f)/K_0(\fn\ell p^n)}
	f_1(g)(V_\fl f_2)(\bar{g}\tau_{\ell\fs}^n)=
	\sum_{G(\F)\backslash G(\A_f)/K_0(\fn p^n)}
	\sum_{k\in K_\ell/K_0(\ell)}
	f_1(gk)(V_\fl f_2)(\bar{g}\bar{k}\tau_{\ell\fs}^n)\\=
	\sum_{G(\F)\backslash G(\A_f)/K_0(\fn p^n)}
	f_1(g)\sum_{k\in K_\ell/K_0(\ell)}
	f_2(\bar{g}\bar{k}\smat{\varpi_\fl&\\&\varpi_\fl}
	\tau_{\fs}^n)=
	(q_\fl+1)
	\sum_{G(\F)\backslash G(\A_f)/K_0(\fn p^n)} f_1(g)
	(T_\fl^{(2)}f_2)(\bar{g}\tau_{\fs}^n)
	\end{multline*}
	The last equation follows similarly
	from the above computaiton.
\end{proof}

\subsection{Theta lifts and L-function}

We now recall the main result in \cite{lee}.
Let $\chi\coloneqq \chi_\circ|\cdot|_\K^{-1/2}$
for $\chi_\circ$ as in the beginning of the section.
Fix $\delta=-\bar{\delta}\in \K$,
then $\delta$ and $\chi$
defines a Weil representation
on the dual reductive pair $G\times \mathrm{U}(1)$.
In \cite[\S 4]{lee},
for algebraic Hecke characters $\eta$ such that 
\begin{enumerate}
    \item $\eta$ has infinity type $k\geq 0$.
    \item $\eta$ is ramified only at places dividing $p\fs$.
\end{enumerate}
we have chosen Schwartz functions $\varphi$ 
and define the theta lift 
$\theta(\eta)$ via the pull-back
$\theta^\square_\varphi(g,\bnu)$.
By the integral structure of Shimura varieties
of $\mathrm{U}(2,2)$,
we also showed that 
there exists  $f^\circ(\eta)\in 
S_{\wt{k}}^{\ord}(K^n_1(\fn),\oo)$
for $\wt{k}=(0,-k)$, $n$ large, and 
$\oo$ a sufficiently large ring of integers in $ \bar{\Q}_p$,
such that the homomorphism
$\iota(f^\circ(\eta))$ as defined in \eqref{eq:p_to_infty}
sends the highest weight vector in  $\xi_{\wt{k}}^*$
to $\theta(\eta)$ up to explicit factors.
\begin{prop}\cite[Thm 7.6]{lee}
	Let $\fX_{\fs}^-$ be the set of characters
	on $\fG_{\fs}^-$ consists of 
	the restrictions to $\fG_{\fs}^-$ of characters 
	$\hat{\eta}$ of $\fG_{\fs}$
	that are $p$-adic avatars of 
	algebraic  Hecke characters  $\eta$ as above.
	There exists a Hida family 
	$\euF^\circ_{\fn}\in S^{\ord}_\Lambda(\fn)$
	such that 
	\[
		\int_{\fG_{\fs}^-}\hat{\eta}\,d\euF^\circ_\fn
		=f^\circ(\eta)\quad
		\text{ for } \hat{\eta}\in \fX_{\fs}^-
	\]
\end{prop}
\begin{rem}
	We slightly abuse the notation
	and use the same notation 
	for an algebraic modular form of weight $\wt{k}$
	 and its image in $S^{\ord}(\fn)$.
\end{rem}

Moreover, by the local computations at splits
places, we have $\hat{\chi}^{-1}_\circ(\varpi_w)\hat{\eta}(\varpi_w,\varpi_{\bw}^{-1})$
\begin{align*}
	T_w^{(1)}\cdot &\euF^\circ_{\fn}=
	(\epsilon^{-1}\hat{\chi}_\circ(\Fr_w)
	+\hat{\chi}^{-1}_\circ(\Fr_w)\cdot 
	\langle \Fr_w\rangle )
	\euF^\circ_{\fn},\\
	U_{\wt{k}',w}^{(1)}\cdot &\euF^{\circ}_{\fn}=
	\hat{\chi}_\circ^{-1}(\Fr_w)\cdot 
	\langle \Fr_w\rangle
	\euF^{\circ}_{\fn},\quad w\in \Sigma_p\\
	 &\euF^\circ_{\fn\ell} \mod I_{\ell}=
	(1-\hat{\chi}_\circ(\Fr_\fl)\cdot 
	\langle \Fr_\fl\rangle^{-1} V_\fl)
	\euF^\circ_{\fn}.
\end{align*}
Note that for $\Fr_w$ depends on the 
choice of the uniformizer  $\varpi_w$ when  $w\in \Sigma_p$.
We let 
$\lambda_{\fn}\colon\TT^{\ord}(\fn,\oo)\to\Lambda_{\fs}$
denote the associated homomorphism.

\begin{prop}\cite[Thm 7.7]{lee}   
Put $\mathcal{L}_\fn=B_\fn
(\euF^\circ_{\fn},U_\fs^{-1}\cdot\euF^\circ_{\fn})
\in \Lambda_{\fs}$. 
When $ \hat{\eta}\in \fX_{\fs}^-$, we have
\begin{equation*}
	\frac{1}{\Omega_p^{2k+4}}
	\int_{\fG_{\fs}^-}\hat{\eta}\,\mathcal{L}_\fn=
	C(\eta,\chi,\K)
	\frac{(2\pi i)^{k}\Gamma(k+2)
	L(1,\chi^{-2}\tilde{\eta})}{\Omega_\infty^{k+2\Sigma}}
	\prod_{w\mid p\fs}
	\varepsilon(1,(\chi^{2}\tilde{\eta}^{-1})_w,\psi_w)
	\frac{1-(\chi^{-2}\tilde{\eta})^{-1}_w(\varpi_v)}
	{1-(\chi^{-2}\tilde{\eta})_w(\varpi_v)q_v^{-1}}
\end{equation*}
up to powers of $2$.
Here $\Omega_p$ and  $\Omega_\infty$
are the CM periods associated to  $\K$,
and $C(\eta,\chi,\K)$ is an explict constant 
which is a $p$-unit when the following
conditions are satisfies.
\begin{gather}
	\frac{L(0,\chi_0)}{\Omega_\infty^{\Sigma}}
	\text{ is prime to }p
	\label{cond:chi}\tag{$\chi$}\\
	p\nmid |G(\F)\backslash G(\A_f)/K(\fc)|,
	\text{ and }p\nmid q_w^2-1\text{ for all }w\mid 2\fc.
	\label{cond:fc}\tag{$\fc$}\\
	p\nmid h_\K, \, w_\K, \, (2\pi)^{-2\Sigma}\zeta_\F(2).
	\label{cond:K}\tag{$\K$}
\end{gather}
\end{prop}

\begin{lem}\label{lem:compare_L_diff_level}
Define 
$P_\ell=(1-\epsilon^{-1}\hat{\chi}_\circ^{-2}(\Fr_\fl)\cdot 
\langle \Fr_\fl\rangle^{-1})
(1-\hat{\chi}_\circ^{-2}(\Fr_\fl)\cdot 
\langle \Fr_\fl\rangle^{-1})\in \Lambda_{\fs}$
when $\ell\nmid \fs$.
Let $\euF_{\fn\ell}\in S_\Lambda^{\ord}(\fn\ell)$
be a Hida family that also satisfies
$\euF_{\fn\ell} \mod I_{\ell}=
(1-\hat{\chi}_\circ(\Fr_\fl)\cdot 
\langle \Fr_\fl\rangle^{-1} V_\fl)
\euF_{\fn}$,
for some $\euF_{\fn}\in S_\Lambda^{\ord}(\fn)$, then 
there exists the equaltiy
\[
	B_{\fn\ell}(\euF_{\fn\ell}, \euF^{\circ}_{\fn\ell})
	\text{ mod }I_\ell=
	(\hat{\chi}_\circ^{-1}(\Fr_\fl)\cdot 
	\langle \Fr_\fl\rangle
	\langle \Fr_{\bar{\fl}}\rangle^{-1})
	P_\ell\cdot B_{\fn}(\euF_{\fn}, \euF^{\circ}_{\fn})
\]
In particular, 
the $p$-adic L-function $\mathcal{L}_\fn$ satisfies 
$\mathcal{L}_{\fn\ell} \text{ mod }I_\ell=
P_\ell\cdot \mathcal{L}_\fn$.
\end{lem}

\begin{proof}
Let $\alpha$ and  $\beta$ denote 
$\epsilon^{-1}\hat{\chi}_\circ(\Fr_\fl)$ and 
$\hat{\chi}_\circ^{-1}(\Fr_\fl)\cdot
\langle \Fr_\fl\rangle$
and apply Proposition \ref{prop:pair_at_deff_level}, thus
\begin{multline*}
B_{\fn\ell}(\euF_{\fn\ell}, \euF^{\circ}_{\fn\ell})
\text{ mod }I_\ell=
B_\fn(\euF, T_\fl^{(1)}\euF^\circ)
-2(q_\ell+1)\beta^{-1}B_\fn(\euF, T_\fl^{(2)}\euF^\circ)
+\beta^{-2}B_\fn(T_\fl^{(1)}\euF, T_\fl^{(2)}\euF^\circ)
\\=
[(\alpha+\beta)-2(q_\fl^{-1}+1)(\alpha)
+q_\fl^{-1}(\alpha+\beta)(\alpha\beta^{-1})]\cdot 
B_\fn(\euF_{\fn}, \euF^\circ_\fn)=
(\alpha-\beta)[q_\fl^{-1}(\alpha\beta^{-1})-1]\cdot 
B_\fn(\euF_{\fn}, \euF^\circ_\fn)
\end{multline*}
Now the lemma follows from
$(\alpha-\beta)(q_\fl^{-1}(\alpha\beta^{-1})-1)
=\beta(1-(\alpha/\beta))(1-(\alpha/\beta)q_\fl^{-1})
=\beta\cdot P_\ell$.
\end{proof}

\subsection{Cocyles from reducible representation}

Let $\mathcal{G}$ be a group and 
$\Psi\colon \mathcal{G}\to A$
be a two-dimensional pseudo-representation
to a Henselian local ring $A$
of odd residual characteristic.
Recall that 
$\Psi$ is residually reducible
when there exists characters
$ \bar{\delta}_i$ of $\mathcal{G}$
to the residue field of  $A$
such that  
$\Psi\equiv \bar{\delta}_1+\bar{\delta}_2\mod \fm_A$.
We further assume that $\bar{\delta}_1\neq \bar{\delta}_2$,
so there exists $z\in \mathcal{G}$
with  $\bar{\delta}_1(z)\neq \bar{\delta}_2(z)$.
Then
\begin{equation}
    P_\Psi(z,X)=
    X^2-\Psi(z)X+\det(\Psi)(z) \equiv 
    (X-\bar{\delta}_1(z))(X-\bar{\delta}_2(z))
    \mod \fm_A
\end{equation}
The distinct roots $\bar{\delta}_i(z)$
lifts to two roots $\alpha,\beta$ of  $P_\Psi(z,X)$
by the Henselian property,
with $\alpha-\beta\in A^\times$.
We then define the functions
\begin{equation}
   a(\sigma)=
   \frac{\Psi(\sigma z)-\beta\Psi(\sigma)}{\alpha-\beta}\quad
   d(\sigma)=
   \frac{\Psi(\sigma z)-\alpha\Psi(\sigma)}{\beta-\alpha}\quad
   x(\sigma,\tau)=a(\sigma\tau)-a(\sigma)a(\tau).
\end{equation}
Then 
$a(\sigma)\equiv \bar{\delta}_1(\sigma)$,
$d(\sigma)\equiv \bar{\delta}_2(\sigma)$,
and $x(\sigma,\tau)$
generate the reducibility ideal
$I_\Psi$ of  $\Psi$.
In particular,
for any ideal $I$ containing $I_\Psi$,
the functions $a(\sigma),d(\sigma)$
defines characters to  $A/I$
lifting the characters
$\bar{\delta}_1$ and $\bar{\delta}_2$.
Now, if we let
$\bar{\delta}=\bar{\delta}_1\bar{\delta}_2^{-1}$, then
\begin{equation*}
    c(\sigma,\tau)\coloneqq \bar{\delta}_1^{-1}(\tau)
    \bar{\delta}_1^{-1}(\sigma)x(\sigma, \tau)\mod I_\Psi^2
\end{equation*}
defines a cocycle in 
$Z^1(\mathcal{G}\times \mathcal{G}, 
(I_\Psi/I_\Psi^2)(\bar{\delta}\boxtimes \bar{\delta}^{-1}))$.


\subsection{Main construction}

We now have all the ingredient needed 
for the construction of our Euler systems.
When $U^p=K(\fn)^p$,
let $M(U^p)$ be defined as in \eqref{eq:completed_coh}
and let $\TT(\fn,\oo)\coloneqq \TT(U^p,\oo)$
be the big Hecke algebra acting on which.
By Lemma \ref{lem:coh_to_ord}
and Proposition \ref{prop:ord_to_dual}
there exists a surjective homomorphism
\[
	M(U^p)\to M^{\ord}(\fn)\to 
	\Hom_{\Lambda_{fs}}(S_\Lambda^{\ord}(\fn),\Lambda_\fs)
\]
Let $\TT(\fn,\oo)\to \TT^{\ord}(\fn,\oo)$
be the induced homomorphism.
By abuse of notation, we write 
$\lambda_{\fn}\colon \TT(\fn,\oo)\to \TT^{\ord}(\fn,\oo)\to
\Lambda_{\fs}$ be the composition.
Let $\Psi\colon \Gal_\K\to \TT(\fn,\oo)$ be the 
big Galois pseudo-representation, then
\[
	\lambda_{\fn}\circ \Psi=
	\epsilon^{-1}\hat{\chi}_\circ+
	\hat{\chi}_\circ^{-1}\langle\cdot\rangle
\]
To apply the results from $p$-adic local Langlands
from \S\ref{sub:compatible},
we make the generic assumption
\begin{equation}\label{cond:chi_gen}\tag{$\chi$-gen}
\epsilon^{-1}\hat{\chi}_\circ^{2}\vert_{D_w}\neq
\id, \omega^{\pm}.
\end{equation}
In particular we let 
$\chi_1$ and  $\chi_2$ be the 
characters obtained from 
modulo 
$\hat{\chi}_\circ$ and $\epsilon\hat{\chi}_\circ$ 
respectively with the maximal ideal.
Note that $\lambda_{\fn}(U_{w}^{(1)})\equiv \chi_2(p)$
since $\epsilon(p)=1$.


\begin{defn}
	Using the notations from 
	\S\ref{sub:compatible}.
	Let $\TT_\fn=\TT(\fn,\oo)_\fm$,
	$M_\fn=\Hom_{Q'}(\tilde{P}_{2,\chi},M(U^p)_{\fm}')$,
	$M_i=M^{\ord}(U^p)_\fm^{U_{w}^{(1)}\equiv\chi_i(p)}$,
	and $M_2'=\Hom_{\Lambda_{\fs}}
	(S^{\ord}_{\Lambda}(\fn),\Lambda_{\fs})_{\fm_2}$,
	here $\fm_2$ is the maximal ideal
	of $\TT^{\ord}(\fn,\oo)$ obtained through 
	$\lambda_\fn$.
	We define 
	$\Theta_\fn\colon M_2'\to \Lambda_{\fs}$
	by $B_\fn(*,U_\fs^{-1}\euF^\circ_\fn)$.
\end{defn}


Let $\Psi_{\fn}\colon \Gal_K\to \TT_\fn$ be the localization
of the pseudo-representation
and let $R_{\fm}$ be the universal 
pseudo-deformation of $\chi_1+\chi_2$.
Note that  $R_\fm$ only depends on  $\chi_\circ$
and does not depends on the level  $\fn$.
In particular the pseudo-representations induces
the commutative diagram
\[
\begin{tikzcd}
	& \TT_{\fn\ell}\arrow[d,"\phi_{\fn\ell}^\ell"]\\
	R_{\fm}\arrow[r]\arrow[ur]
	& \TT_{\fn}
\end{tikzcd}
\]
Since the residue characters $\chi_i$ are distinct,
we may follow the previous subsection,
pick  $z\in \Gp\cong D_W$ 
and lift roots $\alpha,\beta\in R_\fm$
to form functions  $a(\sigma), d(\sigma), x(\sigma,\tau)$
in  $R_\fm$.

\begin{defn}
With the above choice of $z$
we also form the $\TT_\fn$-valued
function  $x_\fn(\sigma,\tau)$ 
from the pseudo-representation $\Psi_\fn$,
using the image of the roots  $\alpha,\beta$ in  $\TT_\fn$.
Thus the functions $x(\sigma,\tau)$
and  $x_\fn(\sigma,\tau)$ are compatible 
among the above diagram.
Moreover,
let $\sigma_0,\tau_0\in D_w$
be such that $ x=x(\sigma_0,\tau_0)$
is a generator of the reducibility ideal.
By abuse of notation we let
$x=x_\fn(\sigma_0,\tau_0)$
denote the image of which in $\TT_\fn$.
\end{defn}

\begin{prop}

The ideal 
$q_{\fn}\coloneqq
\{x_\fn(\sigma,\tau)\mid\sigma\in\Gal_K, \tau\in D_w\}
\subset \TT_\fn$
satisfies \ref{cond:C3}
in \S\ref{sub:fund_exact_sequence}.

\end{prop}

\begin{proof}

It is obvious by definition
that $x=x_\fn(\sigma_0,\tau_0)\in \fq_\fn$.
We also note that 
$\fq_\fn\subset\ker(\lambda_\fn)$
since  $\lambda_\fn\circ\Psi_\fn$ is reducible.
We still have to show that
$\fq_\fn$ acts trivially on  $M_1$
and that 
$\fq_\fn M_{\fn,2}/\fq_\fn S_{\fn,2}$
is torsion over $\Lambda_{\fs}$.

For the first statement,
let $\fp\subset \TT_\fn$
be a prime ideal that 
factors through  $M_1$. 
Enlarge $E$ if necessary,
we may assume  $\textnormal{Frac}(\TT_\fn/\fp)=E$,
then  $\Psi_\fn\text{ mod }\fp$
is the trace of a Galois representation  $r_{\fp}$
as in \eqref{eq:Gal_hecke_at_p}
that is ordinary at $w$ as well and such that
\[
	r_\fp\vert_{D_w}\sim
	\smat{\psi_1&*\\&\psi_2}
\]
where $\epsilon\psi_i\equiv \chi_i$
and $x(\sigma,\tau)=b(\sigma)c(\tau)$.
In particular $q_\fn\subset \fp$.
Since such primes are dense in the image
of $\TT_{\fn}\to \End(M_1)$,
we conclude that $\fq_{\fn}$ acts trivially on $M_1$.

On the other hand,
the bilinear pairing $B_n$ induces
a homomorphism 
$S^{\ord}_\Lambda(\fn)\to\Hom_{\Lambda_{\fs}}
(S^{\ord}_\Lambda(\fn),\Lambda_{\fs})$
between $\Lambda_{\fs}$-modules.
Let $F_\fn\in M_2'$ be the image of which
and $E_{2,\fn}=\Lambda_{\fs}\cdot F_\fn$.
Since  $\Phi_\fn(F_{\fn})=\mathcal{L}_n$ is nonzero 
by definition, we have
\[
	M_{2,\fn}'/S_{2,\fn}'+E_{2,\fn}'\hookrightarrow
	\Lambda_{\fs}/\mathcal{L}_{\fn}
\]
is $\Lambda_{\fs}$-torsion.
Let $E_{\fn}\subset M_{\fn}$ be the 
preimage of $E_{\fn,2}'$,
then we have $\fq(S_\fn+E_\fn)=\fq S_\fn$, 
therefore 
$\fq_{\fn}M_{\fn}\fq_{\fn}S_{\fn}$ is a quotient of
the following, which is $\Lambda_{\fs}$-torsion.
\[
	\fq_\fn\otimes 
	M_\fn/(S_\fn+E_\fn)\cong 
	\fq_fn\otimes 
	M'_\fn/(S'_\fn+E'_\fn)
\]
\end{proof}

We can now apply consider the following 
commutative diagram,
where the exactness of the second row
follows from
Proposition \ref{prop:compatibility}
\[
\begin{tikzcd}
&M_{\fn\ell}/S_{\fn\ell}\arrow[r]\arrow[d]&
\fq_{\fn\ell}\otimes_{\TT_{\fn\ell}}
	M_{\fn\ell}/S_{\fn\ell}\arrow[r]\arrow[d,"(1)"]&
\fq_{\fn\ell}^{\red}\otimes_{\TT_{\fn\ell}}
	M_{\fn\ell}/S_{\fn\ell}\arrow[r]\arrow[d]&0\\
0\arrow[r]&
M_\fn/S_\fn \arrow[r]&
\fq_{\fn\ell}M_{\fn\ell}/\fq_{\fn\ell}S_{\fn\ell} \arrow[r]&
\fq_{\fn\ell}M^{\ord}_{\fn\ell}/
\fq_{\fn\ell}S^{\ord}_{\fn\ell} \arrow[r] &0
\end{tikzcd}
\]

\begin{defn}
	By abuse of notation
	let $F_\fn$
	denote the preimage of which in
	$M_{\fn}/S_{\fn}\cong M'_\fn/S'_\fn$.
	$y_\fn(\sigma)=\chi_2^{-1}(\sigma)
	x_{\fn}(\sigma,\tau_0)\otimes
	F_{\fn}\in 
	q_{\fn}\otimes_{\TT_{\fn}}M_{\fn}/S_{\fn}$.
	By definition
	its image under (1) is sent to $0$
	on the lower right corner,
	therefore the image
	comes from lower left 
	by the exactness.
	We let  $c_{\fn}(\sigma)$
	denote the $\Lambda_{\fs}$-valued
	class obtained from composing
	with $\Phi_\fn$.
\end{defn}

\begin{thm}
	The classes $x_{\fn}(\sigma)\in 
	H^1(\K, \Lambda_{\fs})$
	form an Euler system when we vary $\fs$,
	in the sense that 
	\[
		\phi_{\fn\ell}^{\ell}\colon
		H^1(\K,\Lambda_{\fs\ell})\to 
		H^1(\K,\Lambda_{\fs})\quad
		x_{\fn\ell}\mapsto
		P_\ell\cdot x_\fn
	\]
	Moreover, it satisfies the following.
	\begin{enumerate}[label=(\alph*)]
	\item if $\bw\in \Sigma_p$,
		then  $res_w(x_\fn)=0$
	\item for the fixed  $w\in \Sigma_p$,
		then  $x_\fn(\sigma_0)=\mathcal{L}_\fn$
	\item unramified at other places.
	\end{enumerate}
\end{thm}

\begin{proof}

Let $x'_{\fn}(\sigma)\in M_{\fn}S_{\fn}$.
Since $F_{\fn\ell}\text{ mod }I_\ell=
\beta^{-1}V_\fl)F_\fn$,
we also have 
$x'_{\fn\ell}(\sigma)\text{ mod }I_\ell
=x'_\fn(\sigma)$.
Thus by Lemma \ref{lem:compare_L_diff_level}
and the diagram below
we have the norm relation.
\begin{equation*}
\begin{tikzcd}[column sep=tiny]
& M_{\fn\ell}/S_{\fn\ell}\arrow{dd} \arrow[rd,"\sim"]\arrow[rr]
&& \fq_{\fn\ell}\otimes M_{\fn\ell}/S_{\fn\ell}
	\arrow[rd,twoheadrightarrow]\arrow[rr]\arrow[dd]
&& \fq_{\fn\ell}^{\red}\otimes M_{\fn\ell}/S_{\fn\ell}\arrow[dd]\arrow[rd]\\
0 \arrow[crossing over]{rr} 
&& \Lambda_{\fn\ell}
	\arrow[crossing over]{dd} \arrow[crossing over]{rr} 
&& \fq_{\fn\ell}M_{\fn\ell}/\fq_{\fn\ell}S_{\fn\ell}
	\arrow[crossing over]{dd}\arrow[crossing over]{rr} 
&& \fq_{\fn\ell}M_{\fn\ell}^{\ord}/\fq_{\fn\ell}S^{\ord}_{\fn\ell}
	\arrow{dd} \arrow[rr] && 0\\
& M_{\fn}/S_{\fn}\arrow{rr}\arrow[rd,"\sim"]
&& \fq_{\fn}\otimes M_{\fn}/S_{\fn}
	\arrow{rr} \arrow[rd,twoheadrightarrow]
&& \fq_{\fn}^{\red}\otimes M_{\fn}/S_{\fn} \arrow[rd]& & & \\
0 \arrow[crossing over]{rr} 
&& \Lambda_{\fn} \arrow[crossing over]{rr} 
&& \fq_{\fn}M_{\fn}/\fq_{\fn}S_{\fn}\arrow[crossing over]{rr} 
&& \fq_{\fn}M_{\fn}^{\ord}/\fq_{\fn}S_{\fn}^{\ord} \arrow[rr] && 0
\arrow[from=2-3,to=4-3,crossing over]
\arrow[from=2-5,to=4-5,crossing over]
\end{tikzcd}
\end{equation*}

	
\end{proof}


\section{Iwasawa main conjecture}


Let $\A_{\K,f}^\times=\A_f^\times$
and $U=\hat{\oo}_\K^\times\subset \A_f^\times$.
For $V\subset U$, we have
\[
	\begin{tikzcd}
	1\arrow[r]&
	K^\times\backslash K^\times U/V
	\arrow[r] \arrow[d]&
	K^\times\backslash A_f^\times/V
	\arrow[r] \arrow[d]&
	K^\times\backslash A_f^\times/U
	\arrow[r] \arrow[d]&1\\
	   & \oo_\K^\times\backslash U/V&
	   \textnormal{Cl}(V)&
	   H_K&
	\end{tikzcd}
\]
\begin{defn}
When $\fc$ is an ideal of  $\oo_\F$,
let  $V=(1+\fc\hat{\oo}_\K)^\times$
and $V_n=V^pU_n$, where
$U_n=\prod_{w\mid p}(1+\fp_w^n)$.
We then put
$R=R_0=\textnormal{Cl}(V)$,
$R_n=\textnormal{Cl}(V_n)$,
$\textnormal{Cl}(V_\infty)=\varprojlim_n\textnormal{Cl}(V_n)$.
For $R_*$ with  $*=0,n,\infty$,
define 
$R_*^+=\ker(1-c)$,
$R_*^-=R_*/R_*^+$,
and $\tilde{R}_{*}^-=\ker(1+c)$.
Thus we have 
\[
\begin{tikzcd}
	R_* \arrow[r,"1-c"]\arrow[d,twoheadrightarrow]&
	R_*\\
	R_*^- \arrow[r,hookrightarrow] &
	\tilde{R}_*^-\arrow[u,hookrightarrow]
\end{tikzcd}
\]
Let $\rec\colon K^\times\backslash \A_f^\times/V_*\cong R_*$ 
be the Artin reciprocity map, we define 
the anticyclotomic reciprocity map
$\rec\colon K^1\backslash \A_f^1/V'_*\cong R_*^-$
by the diagram
\[
\begin{tikzcd}
K^\times\backslash \A_f^\times/V_*
\arrow[r,twoheadrightarrow] \arrow[d]
\arrow[rr,bend left,"1-c"] &
K^1\backslash \A_f^1/V'_*
\arrow[r] \arrow[d] &
K^\times\backslash \A_f^\times/V_*
\arrow[d]\\
R_* \arrow[r,twoheadrightarrow] &
R_*^- \arrow[r] & R^*
\end{tikzcd}
\]
Here $V_*'=\A_f^1\cap K^\times V_*$.
\end{defn}

When $\chi\colon\K^\times\backslash\A^\times/V_n\to \C^\times$
is an algebraic Hecke character 
with infinity type  $\chi_\infty(z)=z^\eta$
for $\eta=\sum n_\sigma\cdot\sigma\in \Z[I_\K]$.
We then define 
$\hat{\chi}\colon \K^\times\backslash\A_f^\times/V_\infty
\to \oo^\times$ by 
$ \hat{\chi}(z_f)=\chi(z_f)\cdot z_p^\eta$.
Let $\psi\colon R_\infty\to \oo^\times$
be the induced character from the reciprocity map.
We say  $\psi$ is anticyclotomic if
$\psi$ factors through $R_\infty^-$.

\begin{lem}\label{lem:estimate}
Fix a place $w\in p$ and
define $u_m=(1+p^m,1-p^m)\in \oo_w\times \oo_{\bw}^\times$,
$\tau_m=\rec(u_m)\in R_\infty^-$.
When $\psi\colon R_\infty^-\to \C_p^\times$
is anticyclotomic,
given $m>0$,
there exists  $M$ sufficently large
and a constant  $s$ independent of  $m$
such that 
\[
	\ord_p(q_\fp-1)\geq 2m,\quad
	\ord_p(\psi(\Fr_\fp))=m+s
\]
whenver $\fp$ belongs to 
$\mathcal{L}_m=
\{\fp\nmid \fc p\mid \Fr_\fp=\tau_m \text{ in } R_M\}$.
\end{lem}
\begin{proof}
Let $u_w=1+p\in \oo_w^\times$
and pick $t$ sufficently large so that
$s'=\ord_p(\psi(\rec(u_w))^{p^t}-1)\geq 1$.
This is possible since
$\psi(\rec(u_w))^{p^t}=(1+x)^{p^t}\equiv 
1+x^{p^t}\text{ mod }p$ if
$\psi(\rec(u_w))=1+x$.

Let $M$ be a sufficently large number such that
\[
	\ord_p(\epsilon(\rec(u))-1),\,
	\ord_p(\psi(\rec(u))-1)\geq 2m\quad
	\text{ for } u\in V_{M}
\]
Since 
$\epsilon(\tau_m)=\epsilon(\rec(u_m))=1-p^{2m}$
and $\Fr_\fp=\tau_m\red(u)$ for some $u\in V_M$
when $\fp\in \mathcal{L}_m$, we have 
\[
\ord_p(q_\fp^{-1}-1)=
\ord_p(\epsilon(\Fr_\fp)-1)\geq 2m.
\]

On the other hand,
since $\psi$ is anticyclotomic, the character
$\psi\circ \rec$ on $\oo_w^\times\times\oo_{\bw}^\times$
factors through the homomorphism $(a,b)\mapsto a/b$,
under which
$u_w\mapsto 1+p$ and
$u_m\mapsto 1+p^m/1-p^m=(1+p)^k$
for some $k\in \Zp$ with $p^{m-1}\parallel k$. 
Write $\psi(\rec(u_w))^{p^t}=(1+x)$, then
\[
	(1+x)^{p^{m-1}}=\sum_{i=0}^{p^{m-1}}
	\binom{p^{m-1}}{i}y^i,\quad
	\ord_p(\binom{p^m}{i}x^i)=m-\ord_p(i)+is'\geq
	\begin{cases}
		m-1+s'& i\geq 1\\
		m-1+2s' & i\geq 2
	\end{cases}
\]
which is greater or equal to $m-1+2s'$ when  $i\geq 2$
since  $is'-\ord_p(i)\geq s'(i-\ord_p(i))\geq 2s'$.
Thus 
\[
\ord_p(\psi(\tau_m)-1)=\ord_p(\psi(\rec(u_w)^{p^t})^{p^{m-1-t}}-1)
=m-1-t+s'<2m
\]
when $m>s\coloneqq s'-1-t$
and therefore
$\ord_p(\psi(\Fr_\fp)-1)=m+s$ when  $\fp\in\mathcal{L}_m$.
\end{proof}

\begin{defn}
Let $\Delta_\fc$ denote
the maximal pro-$p$ quotient of 
$U/V\cong (\oo_K/\fc)^\times$,
write  $\Delta_\fc=U/\tilde{V}$.
Define $\tilde{V}_n=\tilde{V}^Ip\times \tilde{U}_n$
where $\tilde{U}_n$ is the product of $U_n$
and the roots of unity in  $\oo_{K,p}^\times$.
Consider the exact sequence
\[
	1\to \oo_\K^\times\backslash U/\tilde{V}_n
	\to \textnormal{Cl}(\tilde{V}_n)\to H_\K\to 1
\]
Let $C_\K=H_\K/T$ be the maximal pro-$p$ quotient
of $H_K$. There exists  
$ \tilde{T}\subset \textnormal{Cl}(\tilde{V}_\infty)$
whose image is $T$.
Let $G_n$ denote the quotient of  $R_n$
by the image of  $ \tilde{T}$ in $R_n$, we have
\[
	1\to \oo_\K^\times\backslash U/\tilde{V}_n
	\to G_n\to C_\K\to 1
\]
We define $G_*$ and  $G_*^{\pm}$ as above.
We also define $C_K^{\pm}$.
Since $p$ is odd, we have
$G_*^{-}=\tilde{G}_*^-$ and 
\[
	1\to (\oo_\K^\times\backslash U/\tilde{V}_n)^-
	\to G_n^-\to C_\K^-\to 1
\]
Since $U/\tilde{V}_n$ is pro-$p$, the map
\[
	 (1+c,1-c)\colon 
	 U/\tilde{V}_n\to 
	 (U/\tilde{V}_n)^+\times
	 (U/\tilde{V}_n)^-
\]
is an isomorhisms.
The image of $\oo_\K^\times$ through  $1-c$
factors through  $W_p$,
the group of $p$-th powers roots of unity in  $\K$.
Thus we have
\[
	1\to W_p\backslash (\Delta_\fc^-\times (U_1/U_n)^-)
	\to G_n^-\to C_\K^-\to 1
\]
When $\fc=\ell$ is a prime ideal that splits in  $\K$
we put  $G(\ell)=G_0^-$, thus
\[
	1\to W_p\backslash \Delta_\ell^-
	\to G(\ell)\to C_\K^-\to 1
\]
We put $H(\ell)=W_p\backslash \Delta_\ell^-$.
When  $\fc=\oo_\F$, $v\in S_p$,
we let $G(v^n)$ be the quotient 
of $G_n^-$ by the image of 
$\prod_{w\nmid \ell}\oo_w^\times$ in which.
Let $v=w\bw$ and 
$(U_1/U_n)_{w,\bw}$ be the component
of which above $w$ and  $\bw$, thus
\[
	1\to W_p\backslash (U_1/U_n)_{w,\bw}^-
	\to G(v^n)\to C_\K^-\to 1
\]
we similarly denote the kernel by $H(v^n)$.
At last
when  $\fc=\ell_1\cdots\ell_r$
We define $K(\fc v^n)$
as the product of the groups
$K(\ell_1),\cdots, K(\ell_r)$ and $K(v^n)$,
where the extensions corresponds
to the groups  $G(\ell_1),\cdots,G(\ell_r)$ and $G(v^n)$.
We then have 
\[
	H(\fc v^n)\coloneqq 
	\Gal(K(\fc v^n)/K(1))\cong 
	H(\ell_1)\times\cdots\times H(\ell_r)\times H(v^n)
\]
where $K(1)$ is the extension corresponding
to  $C_\K^-$.
\end{defn}



\subsection{machinery}

Let $\mathscr{V}\colon \Lambda^a\to \Lambda$
be as in \cite{HT94}.
Our $L$-function is
\[
L(1,\chi^{-2}\tilde{\eta})\prod_{w\in\Sigma}
(1-\chi^{-2}\tilde{\eta}(\bw))
(1-\chi^{-2}\tilde{\eta}(\bw)q_w^{-1})
\]
Let $\psi=\epsilon\hat{\chi}_0^{-2}$ be associated to 
$\chi^{-2}$, which is of type $\Sigma-\Sigma^c$
Then above is 
\[
	L(0,\epsilon\psi)\prod_{w\in \Sigma}
	(1-(\epsilon\psi)^*(\Fr_{\bw}q_w^{-1})
	(1-\epsilon\psi(\Fr_{\bw})
\]
Here  $\lambda^*\coloneqq \lambda^{-c}\epsilon$ 
and $(\epsilon\psi)^*=\psi$ since  $\psi$
is anticyclotomic.
In fact, should view above as
\[
	L(0,\epsilon\psi^{-1})\prod_{w\in\Sigma^c}
	(1-(\epsilon\psi^{-1})^*(\Fr_{\bw}q_w^{-1})
	(1-\epsilon\psi^{-1}(\Fr_{\bw})
\]
Let $\Psi=\psi\langle\rangle$
then this  $L$-function
should be $L(\Psi^D,\Sigma^c)=L(\Psi,\Sigma)$
by functional equation.
The Euler system we constructed vanishes at  $\Sigma^c$,
thus the Selmer group of  $X(\Psi^D,\Sigma^c)$
is bounded. 


\cite{Och05}
\cite{Och08}
\cite{Hsieh2010}
\cite{HT93}
\cite{Hida06}
\cite{Hida06b}
\cite{Rubin}

\subsection{Settings}

Let $\fc=\ell_1\cdots\ell_r$
be a square-free product
of primes $\ell_i$ of  $\F$
that are split in  $\K$
and fix a decomposition
$\ell_i=\fl_i\flw_i$
for each  $\ell_i$.
By our construction, we have 
a class of classes
$z_\fc\in H^1(\rk{\fc},\Lambda(\Psi))$
for  $\Psi=\psi\langle*\rangle$ anticyclotomic 
satisfying 
\[
	N_{\rk{\fc\ell}/\rk{\fc}}z_{\fc\ell}=
	P_\ell(\Fr_{\bar{\fl}})\cdot z_\fc,\quad
	P_\ell(X)\coloneqq(1-\psi(\Fr_{\flw})X)
	(1-\psi(\Fr_{\flw})q_\ell^{-1}X)
\]
Note that here the action of $\Fr_{\flw}$
on $z_{\fc}$ is the action of $\Gal_\K$
on  $H^1(\rk{\fc},\Lambda(\Psi))$,
which factors through  $\rp{\fc p^\infty}$.

In the rest of the subsection,
we demonstrate how the Euler system above
bounds the corresponding Selmer group 
under the following assumption
that $\psi$ is residually nontrivial
at some $w\in p$.
\begin{equation}\label{cond:distinct}\tag{dist}
	\psi\vert_{I_w}\not\equiv1\quad
	\text{ for some } w\mid p
\end{equation}

We consider $H^1(\K, \Lambda^*(\Psi^D))$. 
If  $J\subset \Lambda$ is an ideal 
such that  $\Lambda/J\cong \Lambda^{m}$ for some $m$
we consider  $H^1(\K,(\Lambda/J)^*(\Psi^D))$ as well.
We prove inductively on  $m$ that 
\begin{enumerate}[label=(\alph*)]
	\item $H^1(\K,(\Lambda/J)^*(\Psi^D))^\vee$ is $\Lambda/J$-torsion
	\item the characteristic ideal in $\Lambda/J$
		is bounded by the image of the $L$-function in $\Lambda/J$.
\end{enumerate}


\subsubsection{zero-dimension}

When $\Lambda=\Lambda^{(n)}$ for $n=0$,
let  $T=\oo(\psi)$ and  $T_m=T/p^mT$.

Let $\fc$ be as above.
For each $\ell\mid\fc$,
the subgroup  $\rs{\ell}\cong W_p\backslash \Delta_\ell^-$
is cyclic.
Let  $\sigma_\ell\in \rs{\ell}$ be a generator
and put 
$D_\ell=\sum_{i=0}^{p^{n_\ell}-1}i\cdot \sigma_\ell^i
\in \Z[\rs{\ell}]$,
where $p^{n_\ell}\coloneqq \#\rs{\ell}$.
It is easily seen that 
$(\sigma_\ell-1)D_\ell=p^{n_\ell}-N_{\rk{\ell}/\rk{\id}}$.

\begin{lem}
Let $m$ be a sufficently large integer and
suppose $\fc=\ell_1\cdots\ell_r$
with each $\ell_i\in\mathcal{L}_m$.
Pick $m'=min(n_\ell, 2(m+s))$ then
$D_{\fc}z_{\fc}\text{ mod }p^{m'}\in 
H^1(\rk{\fc},T_{m'})^{\rk{\id}}$.
\end{lem}
\begin{proof}
We prove this by induction on $r$.
The case $r=0$, which corresponds to 
$\fc=\id$, is trivial
since $z_{\id}\in H^1(\rk{\id},T)$.
In general
we show that 
$(\sigma_\ell-1)D_{\fc\ell}z_{\fc\ell}\equiv 0$
modulo  $p^{m'}$ for each $\ell\mid \fc$.
Write $\fc=\fc'\ell$, then
\[
	 (\sigma_\ell-1)D_{\fc}z_{\fc}
	 \overset{\text{mod } p^{n_\ell}}{\equiv}
	 -P_\ell(\Fr_{\flw})D_{\fc'}z_{\fc'}
	 \overset{\text{mod } p^{m'}}{\equiv}
	 -P_\ell(1)D_{\fc'}z_{\fc'}
\]
where the second equivalence follows by induction.
The lemma thus follows
since $\ord_p(P_\ell(1))=2(m+s)$.
\end{proof}
Since the field $\rk{\fc}$
is unramified at  $p$,
the assumption \eqref{cond:distinct}
implies that $T_{m'}^{\rs{\fc}}=0$
for any integer $m'$. 
It then follows from the exact sequence below
that $H^1(\K, T_{m'})\cong 
H^1(\rk{\fc}, T_{m'})^{\Gal_\K}$.
\[
	0\to H^1(\rk{\fc}/\K, T_{m'}^{\rp{\fc}})\to
	H^1(\K, T_{m'})\to
	H^1(\rk{\fc}, T_{m'})^{\Gal_\K}\to
	H^2(\rk{\fc}/\K, T_{m'}^{\rp{\fc}})
\]

\begin{defn}
For $m$ and  $m'$ as above, 
we let $\kappa_{\fc,m'}\in H^1(\K,T_{m'})$
denote the classes corresponding to 
\[
	N_{\rk{\id}/\K}D_{\fc}z_{\fc}
	\text{ mod }p^{m'} \in 
	H^1(\rk{\fc}, T_{m'})^{\Gal_\K}
\]
\end{defn}
When $\ell\in \mathcal{L}_m$
let $H^1_f(\K_{\flw},T_{m'})$ and
$H^1_s(\K_{\flw},T_{m'})$ be defined respecitvely
as the kernel and image of $H^1(\K_{\flw},T_{m'})$ 
in the exact sequence 
$0\to H^1(D_{\flw}/I_{\flw},T_{m'})
\to H^1(D_{\flw},T_{m'})
\to H^1(I_{\flw},T_{m'})^{D_{\flw}}$.
By \cite[Lem 1.4.7]{Rubin},
evaluating the classes at
$\Fr_{\flw}$ and $\sigma_\ell$ induces isomorphisms
$H^1_f(\K_{\flw},T_{m'})\cong T_{m'}/(\Fr_{\flw}-1)T_{m'}$
and
$H^1_s(\K_{\flw},T_{m'})\cong T_{m'}^{\Fr_{\flw}=1}$.
We define the finite-singular comparison map
$\phi_{\flw}^{fs}$ by the commutative diagram
\[
	\begin{tikzcd}
		H^1_s(\K_{\flw}, T_{m'})^{K_{\flw}} \arrow[r]&
		T_{m'}^{\Fr_{\flw}=1}\\
		H^1_{f}(\K_{\flw}, T_{m'}) \arrow[r]
		\arrow[u,"\phi_{\flw}^{fs}"]&
		T_{m'}/(\Fr_{\flw}-1)T_{m'}
		\arrow[u,"Q_\ell(\Fr_{\flw})",swap]
	\end{tikzcd}
\]
where $Q_\ell(X)$ is the linear factor  such that 
$P_\ell(X)=(1-X)Q_\ell(X)+P_\ell(1)$.
Note that since
$Q_\ell(1)=
(1-\psi(\Fr_{\flw}))+(1-\psi(\Fr_{\flw})q_\ell^{-1})
\equiv 2(1-\psi(\Fr_{\flw})) \text{ mod }p^{m'}$,
we have $\ord_p(Q_\ell(1))=\ord_p(\psi(\Fr_{\flw}-1))
=m+s$ with 
$m+s< m'\leq 2(m+s)$.
Therefore
the order of $\ker(\phi_{\flw}^{fs})$
is fixed independent of $\ell$.
\begin{rem}
Let $p^{w_p}=\#\mu_{p^\infty}(\K)$.
If we further assume that the prime divisors
$\ell\mid\fc$ are choosen such that
$\mu_{p^\infty}(\K)$ has distinc images in $\Delta_\ell^-$,
then we have  $n_\ell\geq 2m-w_p$.
The kernel above is nontrivial only when 
$n_\ell=m'<2(m+s)$, in which case the order 
is bounded by the constant $p^{2s+w_p}$.
\end{rem}


\begin{prop}
The Kolyvagin derivatives
$\kappa_{\fc,m'}$ has the following properties.
\begin{enumerate}[label=(\alph*)]
\item When $w\nmid p\fc$ is a finite place of  $\K$,
then  $\res_w(z_{\fc,m'})\in H^1_f(\K_w,T_{m'})$.
\item When $\ell\mid\fc$, we have
$\phi_{\flw}^{fs}(\kappa_{\fc,m'})=\kappa_{\fs\ell,m'}$.
\end{enumerate}
\end{prop}

\begin{proof}
The first statement is clear when 
$\psi$ is unramified at $w$.
In general, since we assume
that the condcutor of $\psi$ consists
only of split primes, the 
image of $D_w$ has infinite image in  $\rp{p^\infty}$
and the claim follows from
\cite[Cor 4.6.5]{Rubin},
if we further assume that
each divisor $\ell$ of $\fc$
belongs to  $\mathcal{L}_{mM}$ 
for some integer $M$ that only depends on  $\psi$.

For the second statement, 
let $\Gal_\K$ acts on $\tilde{T}=\text{Map}^{\cts}(\Gal_\K,T)$
via $(g\cdot f)(x)=f(xg)$ and 
let define $T\to\tilde{T}$ by
$t\mapsto [x\mapsto x\cdot t]$.
The assumption \eqref{cond:distinct} implies that
we have the $\Gal_\K$-equivariant 
exact sequence
\[
	0\to\tilde{T}^{\rk{\fc}}\to (\tilde{T}/T)^{\rk{\fc}}\to 
	H^1(\rk{\fc},T)\to 0
\]
In particular, for each $\fc$
there exists $ \hat{z}_\fc\in (\tilde{T}/T)^{\rk{\fc}}$
such that $z_\fc(g)=(g-1)\cdot \hat{z}_\fc$.
We claim that we can pick the lifts such that
$N_{\rk{\ell}/\rk{\id}}\hat{z}_{\fc\ell}-P_\ell(\Fr_{\flw})
\hat{z}_{\fc}\in p^{m'}\tilde{T}$.
The statement then follows from 
\begin{align*}
	D_{\fc\ell}z_{\fc\ell}(\sigma_\ell)=
	(\sigma_\ell-1)D_{\fc\ell}\hat{z}_{\fc\ell}
	\overset{p^{m'}}{\equiv}&
	-P_\ell(\Fr_{\flw}) D_\fc \hat{z}_\fc\\
	Q_\ell(\Fr_{\flw})\cdot
	D_\fc z_\fc(\Fr_{\flw})=
	Q_\ell(\Fr_{\flw})(\Fr_{\flw}-1)
	\cdot D_\fc \hat{z}_\fc
	\overset{p^{m'}}{\equiv}&
	-P_\ell(\Fr_{\flw}) D_\fc \hat{z}_\fc
\end{align*}

To prove the claim,
pick $k>0$ such that 
$\Fr_{\flw}$ acts trivially on $T_{m'}$.
Since $\Fr_{\flw}$ has infinite order
in $\rp{p^\infty}$,
there exists  $v\in S_p$ and  $a>0$
such that
$p^{m'}\mid \ord(\Fr_{\flw}^k, \rs{v^a})$.
Choose lifts
$\hat{z}_{v^a\fc\ell}$ and $ \hat{z}_{v^a\fc}$
as above, then
\[
	N_{\rk{\ell}/\rk{\id}}
	\hat{z}_{v^a\fc\ell}-P_\ell(\Fr_{\flw})
	\hat{z}_{v^a\fc}\in T+\tilde{T}^{\rk{v^a\fc}}
\]
In fact, since 
$N_{\rk{\ell}/\rk{\id}}\colon\tilde{T}^{\rk{v^a\fc\ell}}\to
\tilde{T}^{\rk{v^a\fc}}$ 
is surjective,
we can pick the lifts so that the difference
lies in $T$.
Now let 
$\hat{z}_{\fc\ell}=N_{\rk{v^a}/\rk{\id}}
\hat{z}_{v^a\fc\ell}$ and
$\hat{z}_{\fc}=N_{\rk{v^a}/\rk{\id}}\hat{z}_{v^a\fc}$,
which are lifts of $z_{\fc\ell}$ and $z_{\fc}$.
Since
\[
	N_{\rk{v^a}/\rk{\id}}t=
	\sum_{\rk{id}/\langle \Fr_\ell^k\rangle}g
	\sum_{i=1}^{\ord}\Fr_\ell^{k\cdot i} t\equiv 0
	\mod p^{m'}\quad
	\text{ for } t\in T
\]
we have 
$N_{\rk{\ell}\rk{\id}}\hat{z}_{\fc\ell}-P_\ell(\Fr_{\flw})
\hat{z}_{\fc}\in N_{\rk{v^a}/\rk{\id}}T\subset p^{m'}\tilde{T}$
\end{proof}


Assume we are given a collection of classes
$C=\{\eta_1,\cdots,\eta_k\}\subset H^1(K,T_{m'}^*)$.
Let $R_M$ be the class group corresponding to $m$ 
in Lemma \ref{lem:estimate}
and let $L$ be the corresponding field extension over  $\K$.
By the defining property of $R_M$,
we see  $G_L$ acts trivially on both  $T_{m'}$ and $T_{m'}^*$.
Apply \cite[Lem 5.2.1]{Rubin}
to the classes $\kappa_{\id}=\bar{z}_{\id}\in H^1(\K,T_{m'})$
and $\eta_1$, 
there exists  $\gamma\in G_L$ such that
\begin{align*}
	\ord(\kappa(\gamma\tau_m),T_{m'}/(\psi(\tau_m)-1)T_{m'})&\geq
	\ord( (\kappa)_L, H^1(L,T_{m'}) )\\
	\ord(\eta(\gamma\tau_m),T_{m'}^*/(\psi(\tau_m)-1)T_{m'}^*)&\geq
	\ord( (\eta)_L, H^1(L,T_{m'}^*) ) 
\end{align*}
Let $L'=\ker(\kappa_1)_L\cap \ker(\eta_1)_L$.
We then pick $\ell_1\in \mathcal{L}_m$ such that
$\eta_i\in H^1_{ur}(K_{\flw_1},T_{m'}^*)$ for all $i$
and $\Fr_{\flw_1}$ is conjugate to $\gamma\tau_m$ in  $\Gal(L'/\K)$.
We thus have
\begin{align*}
	\ord( (\kappa_1)_{\flw_1}, H^1_{ur}(K_{\flw_1}, T_{m'})\geq
	\ord( (\kappa_1)_{L}, H^1(L, T_{m'})\\
	\ord( (\eta_1)_{\flw_1}, H^1_{ur}(K_{\flw_1}, T_{m'}^*)\geq
	\ord( (\eta_1)_{L}, H^1(L, T_{m'}^*)
\end{align*}
We apply the same procedure then to
$\kappa_{\ell_1}=D_{\ell_1}\bar{z}_{\ell_1}\in H^1(\K,T_{m'})$
and $\eta_2$,
we can pick $\ell_2\in \mathcal{L}_m$ such that
$\eta_i\in H^1_{ur}(K_{\flw_2},T_{m'}^*)$ for all $i$ and
\begin{align*}
	\ord( (\kappa_{\ell_1})_{\flw_2}, H^1_{ur}(K_{\flw_2}, T_{m'})\geq
	\ord( (\kappa_{\ell_1})_{L}, H^1(L, T_{m'})\\
	\ord( (\eta_2)_{\flw_2}, H^1_{ur}(K_{\flw_2}, T_{m'}^*)\geq
	\ord( (\eta_2)_{L}, H^1(L, T_{m'}^*)
\end{align*}
We proceed further and get a collection 
$\Sigma=\{\ell_1,\cdots,\ell_k\}$ of primes in $\mathcal{L}_m$
such that the classes  $\eta_i$ are unramified
at all places  $\flw_i$ and that
\begin{align*}
	\ord( (\kappa_{\fc_{i-1}})_{\flw_i}, H^1_{ur}(K_{\flw_i}, T_{m'})\geq
	\ord( (\kappa_{\fc_{i-1}})_{L}, H^1(L, T_{m'})\\
	\ord( (\eta_i)_{\flw_i}, H^1_{ur}(K_{\flw_i}, T_{m'}^*)\geq
	\ord( (\eta_i)_{L}, H^1(L, T_{m'}^*)
\end{align*}
where $\fc_{i-1}\coloneqq \ell_1\cdots\ell_{i-1}$, for all $1\leq i\leq k$.

\begin{lem}
	Let $\Omega/\K$ be the extension corresponding
	to $R_{\infty}$, then the cohomology groups
	$W^\K$, $H^1(\Omega/\K, W)$ and $H^1(\Omega/\K, W^*)$ 
	are all trivial under the assumption \eqref{cond:distinct}.
\end{lem}
\begin{proof}
	Consider the finite extension $L=\ker(\fF(\psi))$
	and let $\Delta=\Gal(L/\K)$.
	Then  $p\nmid \#\Delta$ 
	and $\fF^\Delta=0$ by the assumption  \ref{cond:distinct}.
	As the conjugation action of $\Delta$ acts trivially
	on the abelian group $\Gal(\Omega/\K)$, we have
	\[
		H^1(\Omega/\K,\fF)=
		\Hom(\Gal(\Omega/L),\fF)^\Delta=
		\Hom(\Gal(\Omega/L),\fF^\Delta)=0
	\]
	By the exact sequence
	$0\to\fF\to W\xrightarrow{\varpi}W\to 0$
	this implies that $H^1(\Omega/K,W)$
	has no nontrivial $\varpi$-torsion, 
	which is possible only if  $H^1(\Omega,W)=0$.
	The same argument works also for $H^1(\Omega,W^*)$.
\end{proof}

Now, given a postive integer $n$,
we pick  $m$ sufficiently large so that
\[
p^{m'}> p^n+(k+1)p^{w_p+2s}+\textnormal{ind}(z_{\id})
\]
By the lemma above,dd
if we define $d_i=\ord(\kappa_{\fc_{i},m'}, H^1(\K,T_{m'}))$, then
\[
	d_i=\ord(\kappa_{\fc_{i},m'}, H^1(\K,T_{m'}))=
	\ord( (\kappa_{\fc_{i},m'})_\Omega, H^1(\Omega,T_{m'}))
\]
We have $d_0=p^{m'}-\textnormal{ind}(z_{\id})>p^n+(k+1)p^{w_p+2s}$.
And for $0\leq i\leq k$ we have
\[
	d_i\geq 
	\ord( (\kappa_{\fc_i,m'})_{\flw_i}, H^1_s(\K_{\flw_i}, T_{m'}))\geq
	\ord( (\kappa_{\fc_{i-1},m'})_{\flw_i}, H^1_f(\K_{\flw_i}, T_{m'}))-p^{w_p+2s}\geq 
	d_{i-1}-p^{w_p+2s}.
\]
In particular $d_i\geq p^n$ for all  $i$.

Pick  $\kappa_i\in H^1(\K, T_{n})$ whose image 
is $p^{d_i-n}\kappa_{\fc_i,m'}$
and let $A^{(i)}\subset H^1(\K,T_{n})$
be the submodule generated by $\kappa_0,\cdots,\kappa_i$.
Then 
\[
	\ord(\textnormal{loc}_\Sigma(A^{(i)})/\textnormal{loc}_\Sigma(A^{(i-1)}))
	\geq p^n+d_{i-1}-d_{i}
\]
and therefore
$\ord(\textnormal{loc}_\Sigma(S^{\Sigma\cup \Sigma_p}(\K, T_{n})) 
\geq kn+d_0-d_{k}\geq kn-\textnormal{ind}(z)$.
Therefore $\ord(\coker(\textnormal{loc}_\Sigma))\leq \textnormal{ind}(z)$
as desired.


\bibliographystyle{amsalpha}
\bibliography{biblio}
\end{document}

