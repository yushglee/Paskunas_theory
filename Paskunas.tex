\documentclass[leqno]{amsart}
\usepackage{amssymb}
\usepackage{amsmath} 
\usepackage{enumitem}
\usepackage{hyperref}
\usepackage{mathrsfs}
\usepackage{color}
\usepackage{mathtools,caption,bbm,euscript}
\usepackage[table,dvipsnames]{xcolor}
\usepackage{tikz-cd}
\usepackage[utf8]{inputenc}
\usepackage[OT2,T1]{fontenc}
\hypersetup{
 colorlinks=true,
 linkcolor=DarkOrchid,
 filecolor=blue,
 citecolor=olive,
 urlcolor=orange,
 pdftitle={Paskunas' theory},
 %pdfpagemode=FullScreen,
 }
\usepackage{booktabs}

%[label=(\alph*)]
%[label=(\Alph*)]
%[label=(\roman*)]
%[label={(\bfseries R\arabic*)}]


\setlength{\textwidth}{\paperwidth}
\addtolength{\textwidth}{-2in}
\calclayout


\newcommand{\smat}[1]{\left( \begin{smallmatrix} #1 \end{smallmatrix} \right)}
\newcommand{\mat}[1]{\left( \begin{smallmatrix} #1 \end{smallmatrix} \right)}
\newcommand{\dBr}[1]{\llbracket{#1}\rrbracket}
\newcommand{\leg}[2]{\left(\frac{#1}{#2}\right)}

% double bracket
\makeatletter
\newsavebox{\@brx}
\newcommand{\llangle}[1][]{\savebox{\@brx}{\(\m@th{#1\langle}\)}%
  \mathopen{\copy\@brx\kern-0.5\wd\@brx\usebox{\@brx}}}
\newcommand{\rrangle}[1][]{\savebox{\@brx}{\(\m@th{#1\rangle}\)}%
  \mathclose{\copy\@brx\kern-0.5\wd\@brx\usebox{\@brx}}}
  \newcommand{\llbracket}[1][]{\savebox{\@brx}{\(\m@th{#1[}\)}%
  \mathopen{\copy\@brx\kern-0.5\wd\@brx\usebox{\@brx}}}
\newcommand{\rrbracket}[1][]{\savebox{\@brx}{\(\m@th{#1]}\)}%
  \mathclose{\copy\@brx\kern-0.5\wd\@brx\usebox{\@brx}}}
\makeatother


\newcommand{\Gp}{\mathcal{G}_{\Qp}} %Galois group over \Qp


\newcommand{\phk}{\mathcal{H}_{\wt{k}}} %Hecke algebra
\newcommand{\ordp}{e}


\newcommand{\vk}{v_{-\wt{k}}} % lowest weight vector
\newcommand{\lk}{\mathnormal{l}_{\wt{k}}} % projection to lowest weight
\newcommand{\blk}{\mathnormal{l}_{\bwt{k}}} % projection to lowest weight
\newcommand{\Lk}{L_{\wt{k}}} 
\newcommand{\pf}{\hat{f}} % algebraic modular form
\newcommand{\pfk}{\hat{f}_{\wt{k}}}
\newcommand{\euF}{\EuScript{F}} % Hida family
\newcommand{\B}{\mathbf{B}} % pairing of algebraic modular form


\newcommand{\ww}{\boldsymbol{\omega}}
\DeclareMathOperator{\an}{an}
\DeclareMathOperator{\ord}{ord}
\DeclareMathOperator{\cts}{cts}
\DeclareMathOperator{\FJ}{FJ}


%%% Linear algebraic groups
\DeclareMathOperator{\GL}{GL}
\DeclareMathOperator{\SL}{SL}
\DeclareMathOperator{\Sp}{Sp}
\DeclareMathOperator{\Mp}{Mp}
\DeclareMathOperator{\GSp}{GSp}
\DeclareMathOperator{\UU}{U}
\DeclareMathOperator{\GUU}{GU}
\DeclareMathOperator{\gl}{\mathfrak{gl}}
\DeclareMathOperator{\mtr}{tr}
\DeclareMathOperator{\diag}{diag}
\DeclareMathOperator{\Ad}{Ad}
\DeclareMathOperator{\vol}{vol}

\DeclareMathOperator{\val}{val}
\DeclareMathOperator{\Lie}{Lie}
\DeclareMathOperator{\Pol}{Pol_p}

%%% Adelic rings
\newcommand{\Q}{{\mathbf{Q}}}
\newcommand{\Z}{{\mathbf{Z}}}
\newcommand{\Qp}{\mathbf{Q}_p}
\newcommand{\Zp}{\mathbf{Z}_p}
\newcommand{\Ql}{\mathbf{Q}_\ell}
\newcommand{\Zl}{\mathbf{Z}_\ell}
\newcommand{\R}{\mathbf R}
\newcommand{\C}{\mathbf C}
\newcommand{\A}{\mathbf A}
\newcommand{\hZ}{{\hat{\mathbf{Z}}}}
\newcommand{\dd}{\mathfrak{d}} %different
\newcommand{\DD}{\mathcal{D}}  %discriminant

\newcommand{\arch}{\mathbf{a}}
\newcommand{\fin}{\mathbf{h}}

\newcommand{\F}{{\mathcal{F}}} %global 
\newcommand{\OF}{{\mathcal{O}_{\F}}}
\newcommand{\K}{{\mathcal{K}}} %global quadratic
\newcommand{\OK}{\mathcal{O}_{\K}}
\newcommand{\kk}{F} %local
\newcommand{\E}{E} %local quadratic


\DeclareMathOperator{\Sel}{Sel}
\DeclareMathOperator{\Gal}{Gal}
\DeclareMathOperator{\Nr}{\mathsf{N}}
\DeclareMathOperator{\Tr}{Tr}
\newcommand{\qch}{\epsilon} % quadratic character of K/F


%%% Fonts
\newcommand{\oeu}{\EuScript{O}}
\newcommand{\eeu}{\EuScript{E}}
\newcommand{\feu}{\EuScript{F}}
\newcommand{\geu}{\EuScript{G}}
\newcommand{\keu}{\EuScript{K}}

\newcommand{\oo}{\mathcal O}
\newcommand{\bs}{\mathcal S}
\newcommand{\id}{\mathbf{1}}

\newcommand{\1}{\mathbf{1}} 
\newcommand{\bfe}{\mathbf e}
\newcommand{\bff}{\mathbf f}

\newcommand{\bX}{\mathbb{X}}
\newcommand{\bY}{\mathbb{Y}}
\newcommand{\bV}{\mathbb{V}}
\newcommand{\bW}{\mathbb{W}}

\newcommand{\fa}{\mathfrak a}
\newcommand{\fg}{\mathfrak g}
\newcommand{\fc}{\mathfrak c}
\newcommand{\fs}{\mathfrak s}
\newcommand{\fm}{\mathfrak m}
\newcommand{\fn}{\mathfrak n}
\newcommand{\fl}{\mathfrak l}
\newcommand{\fp}{\mathfrak p}
\newcommand{\bfp}{\overline{\mathfrak p}}
\newcommand{\fq}{\mathfrak q}
\newcommand{\bfq}{\overline{\mathfrak q}}

\newcommand{\btheta}{\boldsymbol{\theta}}
\newcommand{\bdelta}{\boldsymbol{\delta}}


\newcommand{\fG}{\mathfrak{G}}
\newcommand{\fX}{\mathfrak{X}}
\newcommand{\euW}{\EuScript{W}}


%%% Categorical
\DeclareMathOperator{\Ext}{Ext}
\DeclareMathOperator{\End}{End}
\DeclareMathOperator{\Hom}{Hom}
\DeclareMathOperator{\Inj}{Inj}
\DeclareMathOperator{\Isom}{Isom}
\DeclareMathOperator{\Aut}{Aut}
\DeclareMathOperator{\Ind}{Ind}
\DeclareMathOperator{\coker}{coker}
\DeclareMathOperator{\rank}{rank}
\DeclareMathOperator{\corank}{corank}


\DeclareMathOperator{\Res}{Res}
\DeclareMathOperator{\rec}{rec}



\newcommand{\bw}{{w^c}}
\newcommand{\ee}{\mathbf e}


\newtheorem*{theorem*}{Theorem}
\newtheorem{thm}{Theorem}[section]
\newtheorem{lem}[thm]{Lemma}
\newtheorem{prop}[thm]{Proposition}
\newtheorem{cor}[thm]{Corollary}


\theoremstyle{definition}
\newtheorem{definition}[thm]{Definition}
\newtheorem{defn}[thm]{Definition}
\theoremstyle{remark}
\newtheorem{rem}[thm]{Remark}
\newtheorem*{Remark*}{Remark}
\newtheorem{ack}{Acknowledgement}

\newcommand{\red}[1]{\textcolor{Red}{#1}}



\begin{document}
\title{Paskunas' theory}
\author[Y-S.~Lee]{Yu-Sheng Lee}
\address{Department of Mathematics, University  of Michigan, Ann Arbor, MI 48109, USA}
\email{yushglee@umich.edu}
\date{\today}

\maketitle
\setcounter{tocdepth}{1}
\tableofcontents




\section{Paskunas theory}
A

\subsection{generically reducible local deformation}

We recall below the results in \cite{pask} onward
on the structure of the universal pseudo-deformation ring 
of $\chi\colon\Gp\to k$,
where  $\chi=\chi_1+\chi_2$
and  $\chi_1\chi_2^{-}\neq \id,\omega^\pm$.


We recall the results in \cite{pask}
which gives a descripton of the endomorphism algebra in the generic case.
In this case, we have
$\dim_k\Ext_{\mathcal{G}_{\Q_p}}^1(\chi_1,\chi_2)=\dim_k\Ext_{\mathcal{G}_{\Q_p}}^1(\chi_2,\chi_1)=1$
and the universal deformation rings $R_{ij}$ and $R$ are isomorphic to each other 
and formally smooth over $\ir$ of relative dimension $3$.
Since the reducible quotient $R^\red$ is the completed tensor product
of the universal deformations of the characters $\chi_i$, with the fixed determinant,
$R^\red$ is formally smooth of dimension $2$
and the reducibility ideal of $R$ is generated by a regular element.

More explicitly, after fixing a suitable basis, we can realize the universal deformation 
$\tilde{\rho}_{12}$ by the the matrix representation
\begin{equation*}
    \tilde{\rho}_{ij}(\sigma)=\mat{a(\sigma) & b(\sigma)\\ c(\sigma) & d(\sigma)}
\end{equation*}
on $\gl_2(R_{12})$ with the following properties.
\begin{enumerate}[label=(\roman*)]
    \item $\tr\tilde{\rho}_{12}(\sigma)=a(\sigma)+b(\sigma)\in R_{12}$ induces an isomorphism $R\cong R_{12}$,
    \item the $c(\sigma)$'s generate the reducibility ideal of $R$ through the above isomorphism,
    \item let $x\in R_{12}$ be a generator of (the image of) the reducibility ideal, 
    the representation
    \begin{equation*}
        \tilde{\rho}_{12}^x(\sigma)\coloneqq \mat{x&\\&1}\tilde{\rho}_{12}(\sigma)\mat{x&\\&1}^{-1}\in \gl_2(R_{12})
    \end{equation*}
    is a deformation of $\rho_{21}$ to $R_{12}$ and induces an isomorphism $R_{12}\cong R_{21}$,
    \item the reductions of $a(\sigma)$ and $d(\sigma)$ in $R^\red$ are
    the universal deformations of the characters $\chi_1$ and $\chi_2$ respectively.
    \item we have $\End(\tilde{\rho}_{12})=R_{12}=\End(\tilde{\rho}^x_{12})$,
    \item $\Hom(\tilde{\rho}_{12},\tilde{\rho}_{12}^x)$
    and $\Hom(\tilde{\rho}_{12}^x,\tilde{\rho}_{12})$
    are free of rank one over $R$ generated by 
    $\Phi_{12}=\smat{x&\\&1}$ and $\Phi_{21}=\smat{1&\\&x}$ respectively.
\end{enumerate}
From now on we identify $R=R_{12}=R_{21}$ and $\tilde{\rho}_{21}=\tilde{\rho}_{12}^x$. Therefore
\begin{equation}\label{center}
        \En_\B=\End(\Pn_\B)\cong \End(\tilde{\rho}_{12}\oplus \tilde{\rho}_{21})\cong 
        \mat{R & R\Phi_{12}\\ R\Phi_{21} & R}
\end{equation}
is free of rank $4$ over $R$ with the center isomorphic to $R$ and $\Phi_{ji}\circ \Phi_{ij}=x\in R\cong \End(\tilde{\rho}_{ij})$.
Note that if we denote by $\tilde{\chi}_i$ the universal deformations of the characters to $R^\red$
and let $\tilde{\rho}_{ij}^\red$ be the reduction of $\tilde{\rho}_{ij}$ in $R^\red$, then
\begin{equation*}
    \tilde{\rho}_{12}^\red=\mat{\tilde{\chi}_1 & * \\ & \tilde{\chi}_2},\quad
    \tilde{\rho}_{21}^\red=\mat{\tilde{\chi}_1 &  \\ * & \tilde{\chi}_2}.
\end{equation*}






\bibliographystyle{amsalpha}
\bibliography{biblio}
\end{document}
