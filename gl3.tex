\documentclass[leqno]{amsart}
\usepackage{amssymb}
\usepackage{amsmath} 
\usepackage{enumitem}
\usepackage{hyperref}
\usepackage{mathrsfs}
\usepackage{color}
\usepackage{mathtools,caption,bbm,euscript}
\usepackage[table,dvipsnames]{xcolor}
\usepackage{tikz-cd}
\usepackage[utf8]{inputenc}
\usepackage[OT2,T1]{fontenc}
\hypersetup{
 colorlinks=true,
 linkcolor=DarkOrchid,
 filecolor=blue,
 citecolor=olive,
 urlcolor=orange,
 pdftitle={Pask\={u}nas' theory},
 %pdfpagemode=FullScreen,
 }
\usepackage{booktabs}
%[label=(\alph*)]
%[label=(\Alph*)]
%[label=(\roman*)]
%[label={(\bfseries R\arabic*)}]


\setlength{\textwidth}{\paperwidth}
\addtolength{\textwidth}{-2in}
\calclayout

\tikzset{
  symbol/.style={
    draw=none,
    every to/.append style={
      edge node={node [sloped, allow upside down, auto=false]{$#1$}}}
  }
}

\newcommand{\smat}[1]{\left( \begin{smallmatrix} #1 \end{smallmatrix} \right)}

% double bracket
\makeatletter
\newsavebox{\@brx}
\newcommand{\llangle}[1][]{\savebox{\@brx}{\(\m@th{#1\langle}\)}%
  \mathopen{\copy\@brx\kern-0.5\wd\@brx\usebox{\@brx}}}
\newcommand{\rrangle}[1][]{\savebox{\@brx}{\(\m@th{#1\rangle}\)}%
  \mathclose{\copy\@brx\kern-0.5\wd\@brx\usebox{\@brx}}}
  \newcommand{\llbracket}[1][]{\savebox{\@brx}{\(\m@th{#1[}\)}%
  \mathopen{\copy\@brx\kern-0.5\wd\@brx\usebox{\@brx}}}
\newcommand{\rrbracket}[1][]{\savebox{\@brx}{\(\m@th{#1]}\)}%
  \mathclose{\copy\@brx\kern-0.5\wd\@brx\usebox{\@brx}}}
\makeatother




\newcommand{\euF}{\EuScript{F}} %Hida family
\newcommand{\M}{\mathbf{M}} % modular form
\newcommand{\bnu}{\boldsymbol{\nu}}
\newcommand{\wt}[1]{\underline{ #1 }}
\newcommand{\bwt}[1]{\underline{\boldsymbol { #1 }}}
\newcommand{\TT}{\mathbb{T}} % Hecke 
\newcommand{\fG}{\mathfrak{G}}
\newcommand{\fX}{\mathfrak{X}}
\newcommand{\GG}{\mathbf G} 
\newcommand{\Iw}{\textnormal{Iw}} 


\newcommand{\bw}{\overline{w}}
%%% Block theory

\newcommand{\aMod}{\textnormal{Mod}^{\textnormal{adm}}}
\newcommand{\laMod}{\textnormal{Mod}^{\textnormal{l.adm}}}
\newcommand{\lfMod}{\textnormal{Mod}^{\textnormal{lfin}}}
\newcommand{\Ban}{\textnormal{Ban}^{\textnormal{adm.fl}}}
\DeclareMathOperator{\Mod}{\textnormal{Mod}}
\DeclareMathOperator{\Rep}{Rep}
\newcommand{\B}{\mathfrak B} 
\newcommand{\fC}{\mathfrak C}
\DeclareMathOperator{\soc}{soc}
\DeclareMathOperator{\V}{\check{\mathbf{V}}} %Colmez


%%% p_adic Hodge

\newcommand{\Gp}{\mathcal{G}_{\Qp}} %Galois group over \Qp
\newcommand{\Fr}{\textnormal{Fr}} %geometric Frobenius
\newcommand{\frob}{\textnormal{frob}} %arithmetic Frobenius
\newcommand{\dR}{\textnormal{dR}}
\newcommand{\pst}{\textnormal{pst}}
\newcommand{\cris}{\textnormal{cris}}

\DeclareMathOperator{\Gal}{Gal}

\DeclareMathOperator{\Ord}{Ord}
\DeclareMathOperator{\Irr}{Irr}
\DeclareMathOperator{\WD}{WD}
\DeclareMathOperator{\rec}{rec}
\DeclareMathOperator{\Rec}{Rec}

\newcommand{\cont}{\textnormal{cont}}
\newcommand{\cts}{\textnormal{cts}}
\newcommand{\alg}{\textnormal{alg}}
\newcommand{\sm}{\textnormal{sm}}
\newcommand{\adm}{\textnormal{adm}}
\newcommand{\ps}{\textnormal{ps}}
\newcommand{\red}{\textnormal{red}}
\newcommand{\fin}{\textnormal{fin}}
\newcommand{\an}{\textnormal{an}}
\newcommand{\ord}{\textnormal{ord}}


%%% Linear algebraic groups
\DeclareMathOperator{\GL}{GL}
\DeclareMathOperator{\SL}{SL}
\DeclareMathOperator{\gl}{\mathfrak{gl}}
\DeclareMathOperator{\mtr}{tr}
\DeclareMathOperator{\diag}{diag}
\DeclareMathOperator{\Ad}{Ad}
\DeclareMathOperator{\vol}{vol}
\DeclareMathOperator{\Sym}{Sym}

\DeclareMathOperator{\Lie}{Lie}

\newcommand{\bs}{\mathcal{S}}
\newcommand{\id}{\mathbf{1}}

%%% Adelic rings
\newcommand{\Q}{{\mathbf{Q}}}
\newcommand{\Z}{{\mathbf{Z}}}
\newcommand{\Qp}{\mathbf{Q}_p}
\newcommand{\Zp}{\mathbf{Z}_p}
\newcommand{\Ql}{\mathbf{Q}_\ell}
\newcommand{\Zl}{\mathbf{Z}_\ell}
\newcommand{\R}{\mathbf R}
\newcommand{\C}{\mathbf C}
\newcommand{\A}{\mathbf A}
\newcommand{\dd}{\mathfrak{d}} %different
\newcommand{\DD}{\mathcal{D}}  %discriminant
\DeclareMathOperator{\Nr}{\mathsf{N}} %norm
\DeclareMathOperator{\Tr}{Tr} %trace

\newcommand{\arch}{\mathbf{a}}
\newcommand{\finite}{\mathbf{h}}

\newcommand{\F}{{\mathbf{F}}} %global field
\newcommand{\K}{{\mathbf{K}}} %global quadratic
\newcommand{\kk}{F} %local field
\newcommand{\E}{E} %local quadratic
\newcommand{\qch}{\epsilon} % quadratic character of K/F

\newcommand{\oo}{\mathcal{O}} %ring of integer
\DeclareMathOperator{\val}{val}


%%% Fonts

\newcommand{\oeu}{\EuScript{O}}
\newcommand{\eeu}{\EuScript{E}}
\newcommand{\feu}{\EuScript{F}}
\newcommand{\geu}{\EuScript{G}}
\newcommand{\keu}{\EuScript{K}}

\newcommand{\fa}{\mathfrak{a}}
\newcommand{\fc}{\mathfrak{c}}
\newcommand{\fg}{\mathfrak{g}}
\newcommand{\fk}{\mathfrak{k}}
\newcommand{\fs}{\mathfrak{s}}
\newcommand{\fm}{\mathfrak{m}}
\newcommand{\fn}{\mathfrak{n}}
\newcommand{\fl}{\mathfrak{l}}
\newcommand{\fp}{\mathfrak{p}}
\newcommand{\fq}{\mathfrak{q}}
\newcommand{\bfp}{\overlin{\mathfrak p}}
\newcommand{\bfq}{\overline{\mathfrak q}}

\newcommand{\btheta}{\boldsymbol{\theta}}
\newcommand{\bdelta}{\boldsymbol{\delta}}


%%% Categorical
\DeclareMathOperator{\End}{End}
\DeclareMathOperator{\Aut}{Aut}
\DeclareMathOperator{\Hom}{Hom}
\DeclareMathOperator{\Ext}{Ext}
\DeclareMathOperator{\Tor}{Tor}
\DeclareMathOperator{\Ind}{Ind}
\DeclareMathOperator{\cInd}{c-Ind}
\DeclareMathOperator{\nInd}{n-Ind}
\DeclareMathOperator{\coker}{coker}
\DeclareMathOperator{\Image}{Im}
\DeclareMathOperator{\rank}{rank}
\DeclareMathOperator{\corank}{corank}
\DeclareMathOperator{\Res}{Res}




\newtheorem{thm}{Theorem}[section]
\newtheorem{lem}[thm]{Lemma}
\newtheorem{prop}[thm]{Proposition}
\newtheorem{cor}[thm]{Corollary}


\theoremstyle{definition}
\newtheorem{defn}[thm]{Definition}


\theoremstyle{remark}
\newtheorem{rem}[thm]{Remark}
\newtheorem{ack}{Acknowledgement}




\begin{document}
\title{Pask\={u}nas' theory}
\author[Y-S.~Lee]{Yu-Sheng Lee}
\address{Department of Mathematics, University  of Michigan, Ann Arbor, MI 48109, USA}
\email{yushglee@umich.edu}
\date{\today}

\maketitle
\setcounter{tocdepth}{1}
\tableofcontents

\section{Notations}

Throughout the article, $\F$ is a totally real field
and $\K$ is a totally imaginary quadratic extension over $\F$.
Denote by $\arch=\Hom(\F, \C)$ 
the set of archimedean places of $\F$,
and by $\fin$ the set of finite places of $\F$.
Let $\dd_\K$ and $\dd_{\K/\F}$ denote respectively 
the absolute and the relative ideals of different in $\K$,
recall that $\dd_\K=\dd_{\K/\F}\dd_\F$,
where $\dd_\F$ denotes 
the absolute ideals of different in $\F$.

We fix an odd prime $p$ throughout the article
and assume that $p$ is prime to the class number $h_\K$,
the number of roots of unity in $\K$,
and satisfies the following ordinary condition.
\begin{equation}\label{cond:ord}\tag{ord}
\text{Every finite place of $\F$ above $p$ is split in $\K$}.
\end{equation}
We fix an embedding $\iota_\infty:\bar{\Q}\to \C$
and an isomorphism $\iota:\C\cong \C_p$,
and write $\iota_p=\iota\circ\iota_\infty:\bar{\Q}\to \C_p$.


Given a place $v$ of $\F$, archimedean or finite,
let $w\mid v$ denote a place $w$ of $\K$ above $v$.
Then $\K_w$ and $\F_v$ are respectively
the completions of the fields $\K$ and $\F$ at $w$ and $v$.
When $v\in \fin$ we denote by $\oo_w$ and $\oo_v$ 
the rings of integers of $\K_w$ and $\F_v$.
Let $|\cdot|_v$ be the norm on $\F_v$,
which is the usual absolute value when $v\in \arch$
and $q_v=|\varpi_v|_v^{-1}$,
for any choice of uniformizer $\varpi_v$ in $\oo_v$,
is the cardinality of the residue field $\oo_v/(\varpi_v)$
when $v\in \fin$.
For $w\mid v$, define $|a|_w=|\Nr_{\K_w/\F_v}(a)|_v$.


Denote by $\A=\A_{\F}$ the ring of adeles over $\F$,
by $\A_{\infty}$ and $\A_{f}$ respectively
the archimedean and the finite components of $\A$.
Let $\qch_{\K/\F}$ denote 
the quadratic character on $\A_\F^\times/\F^\times$
associated to $\K/\F$ by the global class field theory,
$\qch_v$ denote the component on $\F_v^\times$ 
when $v\in \fin$.
If $\eta$ is a character of $\A_\K^1/\K^1$, 
we denote
by $\tilde{\eta}(\alpha)\coloneqq \eta(\alpha/\alpha^c)$
the Hecke character which is the base change of $\eta$ 
to $\A_\K^\times/\K^\times$.

\subsection{CM types}

Denote respectively by $S_p$ and $S_p^\K$ the set of places above $p$
of $\F$ and $\K$.
Identify $I_\K=\Hom(\K,\bar{\Q})$ with
$\Hom(\K,\C)$ and $\Hom(\K,\C_p)$ by compositions with $\iota_\infty$ and $\iota_p$.
Given $\sigma\in I_\K$,
let $w_\sigma\in S_p^\K$ be the place induced by
$\sigma_p\coloneqq \iota_p\circ \sigma\in\Hom(\K,\C_p)$.
For $w\in S_p^\K$, define
\[
    I_w=\{\sigma\in I_\K\mid w=w_\sigma \}=\Hom(\K_w,\C_p)
\]
and decompose $I_\K=\sqcup_{w\mid p}I_w$.
For a subset $\Sigma\subset I_\K$
define $\Sigma_p=\{w_\sigma\mid \sigma\in \Sigma\}$.
We write
$\Sigma^c=\{\sigma c\mid \sigma\in \Sigma\}$ and 
$\Sigma_p^c=\{cw\mid w\in \Sigma_p\}$.
We fix throughout the article a $p$-ordinary CM type,
which is a subset $\Sigma\subset I_\K$ such that
\[
    \Sigma\sqcup \Sigma^c=I_\K,\quad
    \Sigma_p\sqcup \Sigma_p^c=S_p^\K.
\]
The $p$-ordinary CM type $\Sigma$
always exists by the assumption \eqref{cond:ord},
and is identified with $\arch=\Hom(\F,\C)$ by restrictions.
When $v\in S_p$ decomposes into $v=w\bw$,
we understand always that $w\in \Sigma_p$.

\subsection{Characters}

Given 
$\kappa=\sum_{\sigma\in \Sigma} a_\sigma\sigma+b_\sigma\sigma c\in \Z[I_\K]$,
an algebraic Hecke character 
$\chi\colon \A_\K^\times/\K^\times\to \C^\times$ 
has type $\kappa$ if
\[
    \chi_\infty(\alpha)=
    \iota_\infty \left(\prod_{\sigma\in \Sigma} 
    \sigma(\alpha)^{a_\sigma}\sigma(c \alpha)^{b_\sigma}\right),\quad
    \alpha\in \K^\times.
\]
For $\alpha_\infty=(\alpha_\sigma)\in \A_{\K,\infty}^\times$
and $\alpha_p=(\alpha_w,\alpha_{\bw})\in \prod_{v\in S_p}\K_v^\times$, 
define
\[
    \alpha_\infty^\kappa=
    \prod_{\sigma\in \Sigma} 
    (\alpha_\sigma)^{a_\sigma}(\bar{\alpha}_\sigma)^{b_\sigma}\in \C^\times,\quad
    \alpha_p^\kappa=
    \prod_{w\in \Sigma_p}
    \prod_{\sigma\in I_w}
    \sigma_p(\alpha_w)^{a_\sigma}\sigma_p(\alpha_{\bw})^{b_\sigma}\in \C_p^\times,
\]
for example,
$(2\pi)^\Sigma=(2\pi)^{[\F:\Q]}$
when $2\pi$ is embedded diagonally in $\A_{\K,\infty}^\times$.
Define the $p$-adic avatar of $\chi$ by
\[
    \hat{\chi}\colon \A_\K^\times\to \bar{\Z}_p^\times,\quad
    \hat{\chi}(\alpha)=\iota(\chi(\alpha)\alpha_\infty^{-\kappa})\alpha_p^{\kappa}
\]
where $\alpha_\infty$ and $\alpha_p$ are respectively 
the archimedean component and the components above $p$ of $\alpha\in \A_\K^\times$.


\subsection{Matrices}
When $R$ is an $\F$-algebra and 
$m=(m_{ij})\in \text{M}_{r,s}(\K\otimes_\F R)$,
we denote by 
$m^\intercal=(m_{ji}), 
m^c=(m^c_{ij})$, and
$m^*=(m^c_{ji})$
respectively the transpose, conjugate, and conjugate-transpose of $m$.

When $r=s$ and $g\in \GL_r(\K\otimes_\F R)$ is invertible, we write
$g^{-\intercal}=(g^{-1})^\intercal$ and $g^{-*}=(g^{-1})^*$.
We write $\mtr(m)$ for the trace of a square matrix $m$,
and reserve $\Tr$ for the traces between fields extensions.

When $v=w\bw$ is a place that is split in $\K$,
identify $\K_w=\F_v=\K_{\bw}$ and 
write $\K_v=\F_v^2$, 
where the first component corresponds to $\K_w$.
Then $m=(m_w,m_{\bw})\in M_n(\K\otimes_\F\F_v)=M_n(\F_v)\times M_n(\F_v)$ 
denotes an element in $m\in M_n(\K\otimes_\F\F_v)$ and its components.


\section{Modular forms on definite unitary groups}

Let $G$ be the definite unitary group over $\F$,
such that for any $\F$-algebra $R$
\[
    G(R)=\{g\in \GL_{n}(\K\otimes_\F R) \mid gg^*=\id_n\}.
\]
Let $g_v=(g_w,g_{\bw})\in G(\F_v)$
when $v=w\bw$ is place of $F$ that is split in $\K$.
Then the map $g_v\mapsto g_w$ is an isomorphism
$\iota_w\colon G(\F_v)\cong \GL_n(\F_v)$
such that 
$\iota_w(g_v)=\iota_{\bw}(g_v)^{-\intercal}$.
In particular, when $w\in \Sigma_p$ and  $w\mid v\in S_p$,
we identify $G(\F_v)$ with 
$G_{w}\coloneqq\GL_n(\K_w)=\GL_n(\F_v)$ 
via the isomorphisms $\iota_w$, and define 
\[
	G_p\coloneqq\prod_{w\in \Sigma_p}G_w,\quad
	K_p\coloneqq\prod_{w\in \Sigma_p}K_w,\quad
	K_w\coloneqq\GL_n(\oo_v).
\]

Let $B_n\subset \GL_n$ be the subgroup of
upper trigiangular matrices
and $T_n\subset B_n$ be the diagonal torus.
We identify the set of algebraic characters $X^*(T_n)$
with  $\Z^n$.



Following \cite[Defn 2.3]{ger},
when $\wt{k}=(k_\sigma)\in (\Z^n)^{\Sigma}$ 
is a dominant weight, 
we let $\xi_{\wt{k}}$ be the algebraic representation 
of $G_p$ given by 
\[
	\xi_{\wt{k}}=\bigotimes_{\sigma\in \Sigma}
	\Ind_{B_n}^{\GL_n}(w_0k_{\sigma})\quad
	g=(g_w)\mapsto 
	\otimes_{w\in \Sigma_p}
	\otimes_{\sigma\in I_w}\xi_{k_\sigma}(g_w)
\]
where $w_0$ is the longest element in the Weyl group of
$\GL_n$ and  
$\xi_{k_\sigma}\coloneqq\Ind_{B_n}^{\GL_n}(w_0k_{\sigma})$
is the algebraic representation of highest weight $k_\sigma$.

Let $E$ be a sufficiently large
finite extension over  $\Qp$
that contains $\iota_p(\sigma(\K))$
for all  $\sigma\in I_\K$.
Let  $\oo$ be the ring of integers of  $E$.
Let $W_{\wt{k}}$ and $M_{\wt{k}}$ 
be the points of $\xi_{\wt{k}}$ over
$E$ and over  $\oo$ respecitvely.
Then $W_{\wt{k}}=M_{\wt{k}}\otimes_{\oo}E$
and $K_p$ acts on $M_{\wt{k}}$ via $\xi_{\wt{k}}$.

\begin{rem}
	In previous work, 
	we have defined $\rho_k$ as the 
	algebraic representation with lowest weight  $-k$.
	Thus $\xi_k$ is isomorphic to the representation 
	$\rho^k(g)\coloneqq \rho_k(g^{-\intercal})$.
\end{rem}



When $U=\prod_{v\in\finite}U_v\subset G(\A_f)$
is an open compact subgroup, 
let $U^p=\prod_{v\notin S_p}U_v$ and 
$U_p=\prod_{v\in S_p}U_v$.
\begin{defn}
Let $\wt{k}\in (\Z^n)^{\Sigma}$ be a dominant weight
and $U$ be an open compact subgroup as above,
and let $M$ be an $\oo$-module.
When  $U_p\subset K_p$, we define 
the space of modular forms
of weight $\wt{k}$ and level $U$ as 
\begin{equation}
S_{\wt{k}}(U,M)=
\left\{ f: G(\F)\backslash G(\A_f)/U^p 
\rightarrow M\otimes_{\oo}M_{\wt{k}}
\mid f(gu)=\xi_{\wt{k}}(u_p)^{-1}\cdot f(g), u\in U\right\} 
\end{equation}
We will write $S(U,M)$ for $S_{\wt{k}}(U,M)$
when $\wt{k}$ is the trivial weight.
\end{defn}
Since $\GG(\F)\backslash \GG(\A_f)/U$ is a finite set,
any modular form $f\in S_{\wt{k}}(U,M)$ 
is determined by its values on a finite set of points.
Furthermore, when $U$ satisfies the 
sufficiently small assumption
\begin{equation}\label{cond:small}\tag{\text{small}}
	\GG(\A_f)=\bigsqcup_{i\in I}
	\GG(\F)t_i U,\quad
	\GG(\F)\cap t_iUt_i^{-1}=\{1\} \text{ for all } i\in I
\end{equation}
then a modular form determines and is determined by
its values at the representatives $\{t_i\}$.
In other word  $S_{\wt{k}}(U,M)$ is
isomorphic to a finite direct sum of 
$M\otimes_{\oo}M_{\wt{k}}$.

\subsection{Hecke operators and $P$-ordinary forms}

Let $U$ be an open compact subgroup as above
such that $U_p=K_p$.
For $w\in \Sigma_p$ let $B_w=B_n(\oo_w)$
be the subgroup of upper triangular matrices
and  $B=\prod_{w\in \Sigma_p}B_w$.
For $c\geq b\geq 0$ with  $c>0$, 
let $\Iw(w^{b,c})\subset K_w$
be the subgroup of elements
that are upper triangular modulo $\varpi_w^c$ and 
upper triangular unipotent modulo $\varpi_w^b$.
Let $\Iw(p^{b,c})=\prod_{v\in S_p}\Iw(w^{b,c})$.
We recall 
the Hecke operators 
acting on $S_{\wt{k}}(U^p\Iw(p^{b,c}),M)$,
defined in \cite[\S 2.3]{ger}
as the following double-coset operators.

If $v=w\bw$ is split in  $\K$
and  $\iota_w(U_v)=\GL_n(\oo_v)$, 
for $1\leq j\leq n$ let
\begin{equation}
	T_w^{(j)}=
	\left[\iota_w^{-1}\left(
	\GL_n(\oo_v)
	\begin{pmatrix}
		\varpi_v\id_{j}&\\&\id_{n-j}
	\end{pmatrix}
	\GL_n(\oo_v)
	\right)\right],
	\text{ note that }
	T_{\bw}^{(j)}=(T_{w}^{{n}})^{-1}T_w^{(n-j)}.
\end{equation}

If $w\in \Sigma_p$, let  
$\alpha_w^{(j)}=\iota_w^{-1}
\left(\begin{smallmatrix}
\varpi_v\id_{j}&\\&\id_{n-j} 
\end{smallmatrix}\right)$ for $1\leq j\leq n$,
or let $u\in \iota_w^{-1}(T_n(\oo_w))$, then define
\begin{equation}
	U_{\wt{k},w}^{(j)}=
	(w_0\wt{k})^{-1}(\alpha_{w}^{(j)})\cdot
	[\Iw(p^{b,c})\alpha_w^{(j)}\Iw(p^{b,c})]
	\text{ and }
	\langle u\rangle= [\Iw(p^{b,c})u\Iw(p^{b,c})].
\end{equation}
Note that if $B=TN$ is the Levi decomposition
and  $N_0\coloneqq N\cap K_p$,
then the $\alpha_w^{(j)}$ and $u$ above belong to 
\[
	T^+=\{z\in T\mid zN_0z^{-1}\subset N_0\}
\]
In particular a set of representatives for $N_0/zN_0z^{-1}$ 
is also a set of representatives for $\Iw(p^{b,c})/z\Iw(p^{b,c})z^{-1}$.
If we let $m\in T^+N_0$ acts on  $S_{\wt{k}}(U^p\Iw(p^{b,c}),M)$ via 
\begin{equation}\label{eq:eme_action}
	m\cdot f(g)=(w_0\wt{k})^{-1}(m)\cdot \xi_{\wt{k}}(m)f(gm),\quad
	f\in S_{\wt{k}}(U^p\Iw(p^{b,c}),M)
\end{equation}
then $U_{\wt{k},w}^{(j)}=h_T(\alpha_w^{(j)})$ and $\langle u\rangle=h_T(u)$,
where $h_T(z)$ for $z\in T^+$ are as defined in 
\cite[Def 3.1.3]{emeI}.



More generally, 
for each $w\in \Sigma_p$ let $P_w\subset G_w$
be a standard parabolic subgroup containing $B_w$
and $P=\prod_{w\in \Sigma_p}P_w$.
Let $P=QU$ be the Levi decompositions and define
$\Iw^P(p^{b,c})=\prod_{v\in S_p}\Iw^P(w^{b,c})$, where
\[
	\Iw^P(w^{b,c})=
	\{
	k\in K_w\mid 
	k\text{ mod } \varpi_w^c \in P\,\text{ and }
	k\text{ mod } \varpi_w^c \in U
	\}.
\]
Let $Z_Q$ be
the center of the Levi subgroup $Q$ 
and $Z_Q^+=Z_Q\cap T^+$.
And let $Z_Q^+U_0$ acts on  $S_{\wt{k}}(U^p\Iw^P(p^{b,c}),M)$
as in \eqref{eq:eme_action},
where $U_0=U\cap K_p$,
for those $\alpha_w^{(j)}$ and $u$ belonging to  $Z_Q$, we also define
the Hecke operators 
$U_{\wt{k},w}^{(j)}=h_Q(\alpha_w^{(j)})$ and $\langle u\rangle=h_Q(u)$
acting on $S_{\wt{k}}(U^p\Iw^P(p^{b,c}),M)$.


\begin{lem}
The Hecke operators defined above commutes with each other
and are equivariant with respect to the inclusions
$ S_{\wt{k}}(U^p\Iw^P(p^{b,c}),M)\hookrightarrow
S_{\wt{k}}(U^p\Iw^P(p^{b',c'}),M)$
if $b'\geq b$ and $c'\geq c$.
\end{lem}
\begin{proof}
That each $T_w^{(j)}$ commutes with other Hecke operators is classical,
and the equivariance is clear.
For the Hecke operators at  $w\in \Sigma_p$,
the commutivity follows from \cite[Lem 3.1.4]{emeI},
and the equivariance follows from 
that of the action \eqref{eq:eme_action}.
See also \cite[Lem 2.10]{ger} for the proof when $P=B$.
\end{proof}



\begin{defn}
	Givne $P$ as above,
	let $U_P$ be the product of all  $U_{\wt{k},w}^{(j)}$
	such that $\alpha_w^{(j)}\in Z_Q$.
	When $M$ is either a finite  $\oo$-module
	or the Pontryagin dual of which,
	we define the space of $P$-ordinary modular forms
	\[
	S_{\wt{k}}^{P-\ord}(U^p\Iw^P(p^{b,c}),M)\coloneqq
		e_PS_{\wt{k}}(U^p\Iw^P(p^{b,c}),M),\quad
		e_P\coloneqq\lim_{n\to \infty}(U_P)^{n!}
	\]
	Note that
	$S_{\wt{k}}^{B-\ord}(U^p\Iw(p^{b,c}),M)$ 
	is the space of ordinary forms as defined in 
	\cite[Def 2.13]{ger}.
\end{defn}

\begin{lem}
	For $M$ as in the definition above,
	the inclusions below are isomorphisms.
	\begin{align*}
	&S_{\wt{k}}^{P-\ord}(U^p\Iw^P(p^{b,b}),M)\hookrightarrow	
	S_{\wt{k}}^{P-\ord}(U^p\Iw^P(p^{b,c}),M)\quad c\geq b\geq 1\\
	&S_{\wt{k}}^{P-\ord}(U^p\Iw^P(p^{0,1}),M)\hookrightarrow	
	S_{\wt{k}}^{P-\ord}(U^p\Iw^P(p^{0,c}),M)\quad c\geq 1
	\end{align*}
\end{lem}
\begin{proof}
	It suffices to show that 
	$(U_P)^{n!}S_{\wt{k}}(U^p\Iw^P(p^{b,c}),M)
	\subset S_{\wt{k}}(U^p\Iw^P(p^{b,b}),M)$
	for $n$ sufficiently large. 
	Since $\Iw^P(p^{b,c})$ admits Iwahori decompositions,
	this follows from \cite[Lem 3.3.2]{emeI}.
	The same argument also applies to 
	$S_{\wt{k}}(U^p\Iw^P(p^{0,c}),M)$.
	See also \cite[Lem 2.19]{ger} for the proof when $P=B$.
\end{proof}

\begin{lem}
	For $M$ as above and $b\geq 1$ we have the inclusions 
	\begin{equation*}
		S_{\wt{k}}^{B-\ord}(U^p\Iw(p^{b,b}),M)\subset
		S_{\wt{k}}^{P-\ord}(U^p\Iw^P(p^{b,b}),M)
	\end{equation*}
	which are equivariant with repsect to the 
	Hecke operator that are defined in both spaces.
\end{lem}
\begin{proof}
	Since $\Iw^P(p^{b,b})\subset \Iw(p^{b,b})$,
	it suffices to show that the Hecke operators 
	are equivariant with respect to the natural inclusions
	$S_{\wt{k}}(U^p\Iw(p^{b,b}),M)\subset S_{\wt{k}}(U^p\Iw^P(p^{b,b}),M)$.
	This is clear for $T_w^{(j)}$ and $\langle u\rangle$.
	And for  $U_{\wt{k},w}^{(j)}$ such that 
	$\alpha=\alpha_w^{(j)}\in Z_Q$, this follows from that 
	the following set of representatives for $N_0/\alpha N_0\alpha^{-1}$
	\[
	\begin{pmatrix}
		\id_j&X\\&\id_{n-j}
	\end{pmatrix},\quad
	X \text{ runs through a set of representatives of }
	M_{j,n-j}(\oo_w)\mod \varpi_w.
	\]
	as given in \cite[Lem 2.10]{ger}, is also
	a set of representatives for $U_0/\alpha U_0\alpha^{-1}$.
\end{proof}


\begin{defn}
	Let $P_w=Q_wU_w$ be the Levi decomposition  for each  $w\in \Sigma_p$.
	For a dominant weight 
	$\wt{k}=(k_\sigma)\in (\Z^n)^{\Sigma}$,
	let $\pi_{k_\sigma}=\Ind_{T_n(B_w\cap Q_w)}^{Q_w}(\omega_0 k_\sigma)$
	and $\pi_{\wt{k}}$ be the algebraic representation of $M$ given by
	\[
		\pi_{\wt{k}}=\bigotimes_{\sigma\in \Sigma}\pi_{k_\sigma}\quad
		m=(m_w)\mapsto 
		\otimes_{w\in \Sigma_p}\otimes_{\sigma\in I_w}
		\pi_{k_\sigma}(m_w).
	\]
\end{defn}

We now formulate a weight independence results 
for $P$-ordinary forms
generalizaing \cite[Prop 2.22]{ger}.

\begin{prop}
	Let $A=\varpi^{-r}\oo/\oo$,
	where $\varpi\in\oo$ be a uniformizer,
	and $\pi_{\wt{k}}^*$ be the contragredient of 
	$\pi_{\wt{k}}$.
	When $b$ satisfies
	$\iota_p\circ\sigma(\varpi_w^b)\in (\varpi^r)$
	for all  $w\in \Sigma_p$ and $\sigma\in I_w$,
	there exists an isomorphism
	\[
		\epsilon_{\wt{k}}
		\colon S_{\wt{k}}^{P-\ord}(U^p\Iw^P(p^{b,b}),A)\cong 
		\Hom_{\oo}(\pi_{\wt{k}}^*(\oo),
		S^{P-\ord}(U^p\Iw^P(p^{b,b}),A)).
	\]
\end{prop}

\begin{proof}
	Fix an isomorphism $\xi_{\wt{k}}\cong \Ind_{P}^{G_p}\pi_{\wt{k}}$
	and let $ev\colon \xi_{\wt{k}}\to \pi_{\wt{k}}$
	be the evaluation at the identity.
	For $F(g)\in S_{\wt{k}}^{P-\ord}(U^p\Iw^P(p^{b,b}),A)$, we define 
	$\epsilon_{\wt{k}}(F)$ as 
	\begin{equation}\label{eq:wt_indep}
	\epsilon_{\wt{k}}(F)\colon \pi^*_{\wt{k}}(\oo)\rightarrow
	S(U^p\Iw^P(p^{b,b}),A)\quad
	v^*\mapsto [g\mapsto v^*(ev(F(g)))].
	\end{equation}
	By the assumption on $b$,
	the action of $\Iw^P(p^{b,b})$ on $\pi_{\wt{k}}(A)$,
	which factors through $U(A)$,
	is trivial.
	Thus the function defined above is indeed a modular form
	of trivial weight.


	To construct the reversed map,
	we first note that the weight characters of $T$
	in $\xi_{\wt{k}}$ or $\pi_{\wt{k}}$
	are of the form $w\wt{k}$
	and of multiplicity one,
	where $w$ are elements in Weyl groups of $G_p$ or $M$.
	If $\mu$ is a weight character of  $\pi_{\wt{k}}$, then
	it is also a weight character  of $\xi_{\wt{k}}$.
	We fix weight vectors $v_\mu\in \xi_{\wt{k}}$
	and $v^*_\mu\in \pi_{\wt{k}}^*$
	such that $v^*_{\mu}(ev(v_\mu))=1$.
	Now, let $\alpha_P\in Z_Q^+$ be the product
	of all $\alpha_w^{(j)}\in Z_Q$ and 
	let $\alpha=\alpha_P^r$.
	Pick a set of represntatives $\{x_i\}_{i\in I}$
	for $U_0/\alpha U_0\alpha^{-1}$,
	we put 
	\begin{align*}
		\varphi\colon 
		\Hom_{\oo}(\pi_{\wt{k}}^*(\oo),&
		S(U^p\Iw^P(p^{b,b}),A))\longrightarrow
		S_{\wt{k}}(U^p\Iw^P(p^{b,b}),A)\\
		\phi&\mapsto 
		F_\phi(g)=\sum_{i\in I} \sum_{\mu}
		\xi_{\wt{k}}(x_i)\cdot \phi(v^*_\mu)(gx_i\alpha)v_\mu
	\end{align*}
	where $\mu$ runs through the weight characters in  $\pi_{\wt{k}}$.
	To show that the resulting function defines a modular form,
	let $u\in \Iw^P(p^{b,b})$, then as explained in \cite[Prop 2.22]{ger}
	there exists a bijection $i\mapsto i'$ of $I$ such that 
	 \[
		ux_i=x_{i'}v_i,\quad
		v_i\in\alpha\Iw^P(p^{b,b})\alpha^{-1} \cap \Iw^P(p^{b,b})
	\]
	Since each $v_i$ is reduced to the identity matrix 
	modulo $\varpi^r$ and thus acts trivially on $\xi_{\wt{k}}(A)$,
	\[
		\xi_{\wt{k}}(u)\cdot F_\phi(gu)=
		\sum_{i\in I}\sum_{\mu}
		\xi_{\wt{k}}(x_i'v_i)\cdot 
		\phi(v^*_\mu)(gx_i'v_i\alpha)v_\mu=
		\sum_{i\in I}\sum_{\mu}
		\xi_{\wt{k}}(x_i')\cdot 
		\phi(v^*_\mu)(gx_i'\alpha)v_\mu=F_\phi(g)
	\]
	and indeed $F_\varphi(g)\in S_{\wt{k}}(U^p\Iw^P(p^{b,b}),A)$.

	At last, we observe that for each $\mu$ the composition
	$\epsilon_{\wt{k}}(F_\phi)$ is the homomorphism
	\[
		v_\mu^*\mapsto \sum_{i\in I}\phi(v_\mu^*)(gx_i\alpha)
		=U_P^r\phi(v_\mu^*)(g)
	\]
	On the other hand 
	if we decompose $F$ with respect to a choice of weight vectors
	$F(g)=\sum_\mu F_\mu(g)v_\mu+\sum_{\mu'}F_{\mu'}(g)v_{\mu'}$, 
	with $\mu$ go through weight vectors that also appears in $\pi_{\wt{k}}$
	and $\mu'$ go through the complement,
	then as $\mu(\alpha)=(w_0\wt{k})(\alpha)$ for $\alpha\in Z_Q$,
	\begin{multline*}
	U_P^rF(g)=
	\sum_{i\in I}
	\sum_\mu \xi_{\wt{k}}(x_i)\cdot F_\mu(gx_i\alpha)v_\mu+
	\sum_{i\in I}
	\sum_{\mu'}\frac{\mu'(\alpha)}{(w_0\wt{k})(\alpha)}
	\xi_{\wt{k}}(x_i)\cdot F_{\mu'}(gx_i\alpha)v_{\mu'}\\=
	\sum_{i\in I}
	\sum_\mu \xi_{\wt{k}}(x_i)\cdot F_\mu(gx_i\alpha)v_\mu=
	\sum_{i\in I}
	\sum_\mu \xi_{\wt{k}}(x_i)\cdot\epsilon_{\wt{k}}(F)(v^*_\mu)(gx_i\alpha)
	=F_{\epsilon_{\wt{k}}(F)}(g).
	\end{multline*}

	We thus have the following commutative diagram,
	from which the proposition follows.
	\[
	\begin{tikzcd}
		S_{\wt{k}}(U^p\Iw^P(p^{b,b}),A)
		\arrow[r,"\epsilon_{\wt{k}}"]
		\arrow[d,"U_P^r"]
		& \Hom_\oo(\pi^*_{\wt{k}}(\oo), S(U^p\Iw^P(p^{b,b}),A))
		\arrow[d,"U_P^r"]
		\arrow[dl,"\varphi"]\\
		S_{\wt{k}}(U^p\Iw^P(p^{b,b}),A)
		\arrow[r,"\epsilon_{\wt{k}}"]
		& \Hom_\oo(\pi^*_{\wt{k}}(\oo), S(U^p\Iw^P(p^{b,b}),A))
	\end{tikzcd}	
	\]
\end{proof}

\begin{rem}
	The map \eqref{eq:wt_indep} is equivariant 
	with respect to $Q_0=Q\cap K_p$. 
\end{rem}

\subsection{Hecke algebras and Galois representations}

For a dominant weight $\wt{k}\in (\Z^n)^{\Sigma}$ let
\[
	S_{\wt{k}}(\bar{\Q}_p)=\varinjlim_{U}
	S_{\wt{k}}(U,\bar{\Q}_p)\quad
	U\subset G(\A_f) \text{ open compact subgroup}
\]
endowed with the $G(\A_f)$-action
where $g\cdot f(x)=\xi_{\wt{k}}(g_p)\cdot f(xg)$ for
$f(x)\in S_{\wt{k}}(\bar{Q}_p)$,
so $S_{\wt{k}}(U,\bar{Q}_p)=S_{\wt{k}}(\bar{Q}_p)^U$.
Moreover, by \cite[Prop 3.3.2]{CHT}, 
ther exists the $G(\A_f)$ equivariant isomorphism
\[
	\iota\colon S_{\wt{k}}(\bar{Q}_p)\otimes_{\iota,\bar{Q}_p}\C
	\rightarrow \Hom_{G(\A_\infty)} (\xi_{\wt{k}}^*(\C), \mathcal{A})\quad
	\iota(F)\colon v^*\mapsto 
	[g\mapsto \xi_{\wt{k}}(g_\infty)\xi_{\wt{k}}(g_p)\cdot F(g_f)]
\]


\begin{prop}\cite[Prop.2.27]{ger}
	Let $\pi$ be an irreducible constituent of 
	$G(\A_f)$-representation in $S_{\wt{k}}(\bar{Q}_p)$.
	Then there exist a unique continuous semisimple representation
	\[
	r_\pi: G_\K \rightarrow \GL_n(\bar{\Q}_p)\quad
	r_\pi^c \cong r_\pi^{\vee} \epsilon_p^{1-n}
	\]
	where $\epsilon_p$ is the cyclotomic character,
	with the following properties.
\begin{enumerate}[label=(\alph*)]
\item If $v=w\bw$ is a prime-to-$p$ place that is split in $\K$,
let $\pi_w$ be the $\GL_n(\K_w)$-representation
induced by $\iota_w\colon G(\F_v)\cong \GL_n(\K_w)$, then
\[
\WD\left(\left.r_\pi\right|_{G_{\K_w}}\right)^{\mathrm{ss}} \cong
\Rec(\pi_w|\cdot|^{\frac{1-n}{2}})^{\mathrm{ss}}
\]
Moreover, $r_\pi$ is unramified at $w$ and $\bw$ if $\pi_v$ is unramified.
\item If $v=w\bw$ and  $w\in \Sigma_p$.
The representation $r_\pi$ is potentially semistable at $w$ and  $\bw$;
and crystalline if $\pi_v$ is unramified.
In this latter case, 
and the characteristic polynomial of $\Fr_w$, the geometric Frobenius,
on $\WD\left(D_{\mathrm{cris }}\left(\left.r_\pi\right|_{G_{F_w}}\right)\right)$
coincides with that of $\Rec(\pi_w|\cdot|^{\frac{1-n}{2}})^{\mathrm{ss}}$.
\item 
Let $k_{\sigma,j}=-k_{\sigma c, n-j+1}$
for $\sigma\notin \Sigma$.
If $\sigma\in I_w$ for $w\in S_p^\K$, then
\[
\dim_{\bar{\Q}_p}\operatorname{gr}^i
\left(r_\pi \otimes_{\sigma, \K_w} B_{\dR}\right)^{G_{\K_w}}=1
\]
when $i=k_{\sigma, j}+n-j$ 
for some $j=1, \ldots, n$, and is equal to 0 otherwise.
\end{enumerate}
\end{prop}

From now on,
we fix a place $w\in \Sigma_p$ and
assume we are in one of the following two situations
\begin{enumerate}[label=(\Roman*)]
	\item $n=3$, $P_{w'}=B_{w'}$ for $w'\neq w$ and $P_w$ is
		maximal proper parabolic.
	\item $n=2$, $P_{w'}=B_{w'}$ for $w'\neq w$ and $P_w=G_w$.
\end{enumerate}

Following \cite{ger},
we say a dominant weight $\wt{k}$ is sufficiently regular
if for each $w\in \Sigma_p$ and $1\leq j<n$,
there exists  $\sigma\in I_w$
such that  $k_{\sigma,j}>k_{\sigma,j+1}$.
\begin{lem}
	Let $\wt{k}$ be a sufficiently regular dominant weight
	$\pi$ be an irreducible constituent of 
	$G(\A_f)$-representation in $S_{\wt{k}}(\bar{Q}_p)$
	and $U$ be a compact open subgroup such that 
	\[
		\pi^{U^p\Iw^P(p^{1,0})}\cap 
		S^{P-\ord}_{\wt{k}}(U^p\Iw^P(p^{1,0}),\oo)\neq\{0\},
	\]
	then $\pi_v$ is actually unramified for all $v\in S_p$.
\end{lem}
\begin{proof}
	The statement follows from \cite[Lem 2.30]{ger} when $P_{w'}=B_{w'}$
	and is automatic when $P_w=G_w$.
	When $n=3$ and
	$P=\begin{psmallmatrix}
		*&*&*\\
		*&*&*\\
		 &&*
	\end{psmallmatrix}$,
	by the proof of \cite[Lem 3.1.5]{CHT}
	$\pi_w$ is either unramified or 
	the irreducible quotient of 
	$\Ind_{B_3}^{\GL_3}(\chi_1\otimes \chi_2\otimes\chi_2|\cdot|)$
	for unramified characters  $\chi_1$ and $\chi_2$.
	In this latter case
	the  $\Iw^P(p^{1,0})$-invariant subspace is one-dimensional, with
	the normalized Hecke operators acting as
	\begin{align*}
		(w_0\wt{k})^{-1}(\alpha_w^{(1)})
		[\Iw^P(p^{1,0})\alpha_w^{(1)}\Iw^P(p^{1,0})]&=
		(w_0\wt{k})^{-1}(\alpha_w^{(1)})
		(q_w\chi_1(\varpi_w)+\chi_2(\varpi_2))\\
		(w_0\wt{k})^{-1}(\alpha_w^{(2)})
		[\Iw^P(p^{1,0})\alpha_w^{(2)}\Iw^P(p^{1,0})]&=
		(w_0\wt{k})^{-1}(\alpha_w^{(1)})
		\chi_1\chi_2(\varpi_w)\\
		(w_0\wt{k})^{-1}(\alpha_w^{(3)})
		[\Iw^P(p^{1,0})\alpha_w^{(3)}\Iw^P(p^{1,0})]&=
		(w_0\wt{k})^{-1}(\alpha_w^{(1)})
		q_w^{-1}\chi_1\chi_2\chi_2(\varpi_w)
	\end{align*}
	where the first eigenvalue is $p$-integral
	and the latter two are $p$-unit, which implies that
	\begin{align*}
		\val_w(\chi_2(\varpi_w))&=
		1+\sum_{\sigma\in I_w}k_{\sigma,1}
		&\Longrightarrow
		\val_w((w_0\wt{k})^{-1}(\alpha_w^{(1)})\chi_2(\varpi_2))=
		1+\sum_{\sigma\in I_w}(k_{\sigma,1}-k_{\sigma,3})>0\\
		\val_w(\chi_1(\varpi_w))&=
		-1+\sum_{\sigma\in I_w}
		(k_{\sigma,3}+k_{\sigma,2}-k_{\sigma,1})
		&\Longrightarrow
		\val_w((w_0\wt{k})^{-1}(\alpha_w^{(1)})q_w\chi_1(\varpi_w))=
		\sum_{\sigma\in I_w}(k_{\sigma,2}-k_{\sigma,2})<0
	\end{align*}
	which leads to a contradiction when $\wt{k}$
	is sufficiently regular.
	The case when
	$P=\begin{psmallmatrix}
	*&*&*\\
	&*&*\\
	&*&*
	\end{psmallmatrix}$ is similar.
\end{proof}

\begin{prop}
	Galois representation has subspace.
\end{prop}
\begin{proof}
	Following \cite[Lem 2.30]{ger},
	$\pi$ is a subquotient of 
	$\sigma=\Ind(\chi_1\otimes\chi_2\otimes\chi_3)$.
	which implies 
\end{proof}

\begin{defn}
	Hecke algebra 
	generated by $T$ and $  U_{w'}$
\end{defn}

\begin{prop}
	density of crystalline.
\end{prop}



\subsection{Emerton's Functor of ordinary parts}

On the other hand,
given a parabolic subgroup $P$ as above, 
we may apply 
the functor of $P$-ordinary parts as defined in \cite{emeI}
on the smooth admissible $G_p$-module
\[
	S(U^p,E/\oo)\coloneqq
	\varinjlim_{U_p}S(U^pU_p,E/\oo)\in 
	\Mod^{\adm}_{G_p}(\oo)
\]
where $\{U_p\}$ is the filtered system of 
all the compact open subgroups in $K_p$
and $G_p$ acts through right translation.
To recall the definition,
let $Q^+=\{m\in Q\mid mU_0m^{-1}\subset U_0\}$ 
so that $Z_Q^+=Z_Q\cap Q^+$.
Then the functor of ordinary parts
$ \Ord_P\colon \Mod^{\sm}_{G_p}(\oo)\to \Mod^{\sm}_Q(\oo)$
is defined as
\[
	\Ord_P(V)=\Hom_{\oo[Z_Q^+]}(\oo[Z_Q], V^{U_0})_{Z_Q-\fin}.
\]
Here $Z_Q^+$ acts by translation on the left; by $h_Q$ on the right.
And the action of $Q=Z_Q\cdot Q^+$ is induced by 
havig $Z_Q$ act by translation on the left and 
$Q^+$ act by $h_Q$ on the right.



\begin{prop}
	After restriction, 
	$\Ord_P(S(U^p,E/\oo))$ is injective
	as an object in $\Mod^{\sm}_{M_0}(\oo)$.
\end{prop}
\begin{proof}
	Following the strategy of the proof of 
	\cite[Prop 3.2.4]{pan}, it suffices to show the surjectivity of
	\[
		\Hom_{\oo[M_0]}(\pi,\Ord_P(S(U^p,E/\oo)))\to 
		\Hom_{\oo[M_0]}(\pi',\Ord_P(S(U^p,E/\oo)))
	\]
	when $\pi_{1}\hookrightarrow \pi$ 
	is injective between $M_0$-admissible
	representations that are finite $\oo$-modules.
	By definition
	\begin{multline*}
		\Hom_{\oo[M_0]}(\pi,\Ord_P(S(U^p,E/\oo)))=
		\Hom_{\oo[M_0]}(\pi,
		\Hom_{\oo[Z_M^+]}
		(\oo[Z_M], S(U^p,E/\oo)^{N_0}))\\=
		\Hom_{\oo[Z_M^+]}(\oo[Z_M],
		\Hom_{\oo[M_0]}(\pi, S(U^p,E/\oo)^{N_0}))=
		\Hom_{\oo[Z_M^+]}(\oo[Z_M],
		S_{\pi^\vee}(U^pH_{c,0})
	\end{multline*}
	where the first equality follows from the $\oo$-finiteness of $\pi$,
	as any homomorphism in above factors through some 
	$S(U^pU_p,p^{-s}\oo/\oo)$,
	for which
	the $Z_M$-fintieness condition is automatic
	by \cite[Lem 3.1.5]{emeI};
	and the section equality comes from 
	extending $\pi$ from  $M_0$ to  $H_{c,0}$ via
	\[
		H_{c,0}\to P \mod p^c\to M_0/M_c
	\]
	suppose the $M_0$-actions on  $\pi$ and  $\pi_1$
	factor through  $M_0/M_c$.
\end{proof}




We assume from now on that 
there exists a prime $v\in S_p$
such that  $\F_v\cong \Qp$,
let  $\sigma\in I_\F$
denote the unique embedding such that  $v_\sigma=v$.
We also assume that 
the parabolic subgroup $P\subset G_p$ is given by 
$P=P_v\times\prod_{v'\neq v}B_{v'}$,
where $B_{v'}\subset \GL_n(\oo_{v'})$
are the Borel subgroups
and the Levi component of $P_v\subset GL_n(\oo_v)$
is the product of $\GL_2(\Q_p)$ and a split torus.
In other word, $M$ is isomorphic to the product  
of  $G=\GL_2(\Qp)$ and a split torus  $T$,
and  $\Ord_P(S(U^p,E/\oo))\in \aMod_M(\oo)\cap \laMod_{G}(\oo)$.
\begin{defn}
	We define the $P$-ordinary completed homology and cohomology by
	\begin{equation*}
		M(U^p)=\Hom_\oo(\Ord_P(S(U^p,E/\oo)),E/\oo),\qquad
		S(U^p)=\Hom_\oo(E/\oo, \Ord_P(S(U^p,E/\oo))).
	\end{equation*}
	We also write $S(U^p)_E=S(U^p)\otimes_\oo E$.
	It is immediately that 
	$M(U^p)\in \fC_{G}(\oo)$,
	It is also easily verified that 
	\[
		M(U^p)\cong \Hom_\oo(S(U^p),\oo),\qquad
		S(U^p)\cong \Hom_\oo^{\cts}(M(U^p),\oo),
	\]
	here $\Phi\in \Hom_\oo(M(U^p),\oo)$ 
	is continuous if 
	$\Phi \mod p^n$ factors through
	$S(U^pU_p,E/\oo)^\vee$ for some $U_p$
	for any positive integer $n$
	($U_p$ depedents on  $n$).
\end{defn}




\subsection{compatibility}

\begin{prop}
Assume that $\psi(a_v)=\sigma(a_v)^w$
for  $a_v\in \oo_v^\times$,
then $S_\psi(U^p)_E^{\alg}\subset S_\psi(U^p)_E$
is dense.
\end{prop}
\begin{proof}
	Reduced to $C_\psi(K_p,E)$ use the  $K_p$-injective properties.
\end{proof}
\[
S_\psi(U^p)_E^{\alg}=
\Image\left(
	\bigoplus_{\lambda}\Hom_{E[K\times T_0]}(W_{\lambda,E}^*, 
	S_\psi(U^p)_E)
	\otimes_EW_{\lambda,E}^*\xrightarrow{ev}
	S_\psi(U^p)_E
\right)
\]

\begin{lem}
	Let $S_{\lambda,\psi}^{P-\ord}(U^pK_p)_E
	\coloneqq \Hom_{\oo[Z_M^+]}
	(\oo[Z_M], S_{\lambda,\psi}(U^pK_p,\oo))\otimes_\oo E$,
	then
	\[
		\Hom_{\oo[M_0]}(\pi_\lambda^*, S_\psi(U^p)_E)\cong
		S_{\lambda,\psi}^{P-\ord}(U^pK_p,E)
	\]
\end{lem}
\begin{proof}
	We first note that 
	$S_\psi(U^p)\colon \varprojlim_{n}\Ord_PS_\psi(U^p, \oo/p^n)$,
	thus 
	\[
		\Hom_{\oo[M_0]}(\pi_\lambda^\vee(\oo),S_\psi(U^p)\cong
		\varprojlim_{n}\Hom_{\oo[M_0]}(\pi^\vee(\oo/p^n),
		\Ord_PS_\psi(U^p,\oo/p^n))
	\]
	which induces an element in 
	$\varprojlim_n S_{\psi,\lambda}^{P-\ord}(U^pH_{n,n},\oo/p^n)^{M_0}
	=\varprojlim_n S_{\psi,\lambda}^{P-\ord}(U^pH_{n,0},\oo/p^n)$,
	which gives rises to a $P$-ordinary form in 
	$S_{\psi,\lambda}^{P-\ord}(U^pH_{n,0},\oo)$.
\end{proof}


\begin{rem}
	Will prove the compatibility on 
	$\Ord_P(S_\psi(U^p,E/\oo))^{T_0}$ first.
	On this we have the density of 
	algebraic vectors for sure.
	On which prove compatibility. 
\end{rem}



\section{Pask\={u}nas' theory}

In this section,
$G$ denotes  $\GL_2(\Qp)$, 
$K$ denotes  $\GL_2(\Zp)$,  
$P$ and  $\bar{P}$ 
be the upper and lower triangular subgroups,
$T\subset G$ denotes the diagonal torus,
and  $Z\subset T$ is the center of  $G$.
Let  $\Gp$ be the absolute Galois group of  $\Qp$,
we normalize the reciprocity map  geometrically
so that  $p\in \Qp^\times$
is sent to a geometric Frobenius  $\Fr$.
We identify characters of  $\Qp^\times$
and characters of  $\Gp$ through the reciprocity.
In particular, 
the $p$-adic cyclotomic character $\varepsilon$ 
and the Teichmuller character $\omega$
are viewed as characters of both  $\Gp$ and  $\Qp^\times$.

Addtionally,
we denote by $L$ a sufficiently large 
finite extension of  $\Qp$,
with the ring of integers  $\oo$,
a uniformizer  $\varpi$,
and the residue field $k$.
We fix 
two continuous characters
$\chi_1,\chi_2\colon \Gp\to k^\times$ 
throughout the section satisfying
the following generic assumption
\begin{equation}\label{cond:generic}\tag{\text{gen}}
	\chi_1\chi_2^{-1}\neq \id,\omega^{\pm1}.
\end{equation}
And let $\zeta\colon \Gp\to L^\times$
be the Teichmuller lift of the character  $\chi_1\chi_2\omega^{-1}$.



\subsection{generically reducible deformation}

We recall from \cite[\S B.1]{pask}
the structure of the universal deformation ring $R_{\Psi}^{\ps}$
of the $2$-dimensional pseudo-representation $\Psi=\chi_1+\chi_2$. 

The assumption \eqref{cond:generic}
implies the existence of non-split extensions
\begin{equation*}
    0\to \chi_1\to \rho_{12}\to \chi_2\to 0\quad
    0\to \chi_2\to \rho_{21}\to \chi_1\to 0
\end{equation*}
which are unique up to isomorphisms;
and that the universal deformation rings
$R_{\rho_{ij}}$ of the Galois representations $\rho_{ij}$
are formally smooth of relative dimension $5$ over $\oo$.

Denote by $\tilde{\rho}_{ij}$ the universal deformation,
one may choose bases and think of which as group homomorphisms
$\tilde{\rho}_{ij}\colon \Gp\to \GL_2(R_{\rho_{ij}})$
so that 
$\rho_{12}=\smat{\chi_1&*\\&\chi_2}$ and
$\rho_{21}=\smat{\chi_1&\\ * &\chi_2}$.
Then trace induces $\theta\colon R_{\Psi}^{\ps}\cong R_{\rho_{ij}}$ by \cite[Prop B.17]{pask}.
Since $R^{\red}_{\Psi}$ is formally smooth of relative dimension $4$ over $\oo$,
the reducibility ideal  $\tau\subset R_{\Psi}^{\ps}$ is a principal ideal generated by 
an element in $c\in\fm\setminus \fm^2$,
where $\fm\subset R_{\Psi}^{\ps}$ is the maximal ideal. 
Moreover, let $\tau_{ij}\subset R_{\rho_{ij}} $ be the ideal 
generated by the $(j,i)$-entry of  $ \tilde{\rho}_{ij}(g)$
for all $g\in \Gp$,
then  $\theta$ maps  $\tau$ to  $\tau_{ij}$ by \cite[Prop B.23]{pask}

Let $\tilde{\rho}_{12}^c\colon \Gp\to \GL_2(R_{\rho_{ij}})$ be the representation defined by
\begin{equation*}
	\tilde{\rho}_{12}^c(g)\coloneqq 
	\smat{\theta(c)&\\&1}
	\tilde{\rho}_{12}(g)
	\smat{\theta(c)&\\&1}^{-1}.
\end{equation*}
Then $ \tilde{\rho}_{12}^c$ is a deformation of $\rho_{21}$ to $R_{\rho_{12}}$
and induces an isomorphism $\alpha\colon R_{\rho_{21}}\to R_{\rho_{12}}$,
for which the diagram
\begin{equation*}
	\begin{tikzcd}
		R_{\rho_{21}} \arrow[r,"\alpha"] &
		R_{\rho_{12}}\\
		R_{\Psi}^{\ps} \arrow[u,"\theta"] \arrow[r,equal] &
		R_{\Psi}^{\ps} \arrow[u,"\theta"]
	\end{tikzcd}
\end{equation*}
commutes by \cite[Prop B.24]{pask}.

Identify $\tilde{\rho}_{21}$ with $\tilde{\rho}_{12}^c$,
then
$\Hom_{\Gp}(\tilde{\rho}_{12}, \tilde{\rho}_{21})$ and
$\Hom_{\Gp}(\tilde{\rho}_{21}, \tilde{\rho}_{12})$
are free modules over $R_{\Psi}^{\ps}\cong R_{\rho_{12}}$ 
generated respectively by
\begin{equation}\label{eq:Phi_ij}
	\Phi_{12}=\smat{\theta(c)&\\&1} \text{ and }
	\Phi_{21}=\smat{1&\\&\theta(c)},
\end{equation}
and, the ring $\End_{\Gp}(\tilde{\rho}_{12}\oplus \tilde{\rho}_{21})$
is isomorphic to the generalized matrix algebra
$\smat{R_{\Psi}^{\ps}& R_{\Psi}^{\ps}\Phi_{12}\\ R_{\Psi}^{\ps}\Phi_{21}& R_{\Psi}^{\ps}}$,
which is a free $R_{\Psi}^{\ps}$-module of rank  $4$,
with the center isomorphic to  $R_{\Psi}^{\ps}$
by \cite[Prop B.26]{pask}.

In below, we denote by 
$R$,  $R_{12}$, and $R_{21}$
the deformations rings of 
$\Psi$,  $\rho_{12}$, and $\rho_{21}$
of fixed determinant $\det=\zeta\varepsilon$.
By a slight abuse of notations,
we still denote by $\tilde{\rho}_{12}$
and $\tilde{\rho}_{21}$ the universal deformations
of fixed determinant.
The above properties among the deformation rings
remain unchanged,
except that the relative dimensions are reduced by $2$.

\subsection{Ordinary parts}

In this subsection,
we view the characters introduced 
in the previous subsection as characters
on $\Qp^\times$ by the reciprocity map,
as remarked in the begining of the section.
We start by recalling the results from
\cite[\S 7 \S 8]{pask}
on the generic blocks of the categories
$\laMod_{G,\zeta}(\oo), \laMod_{T,\zeta}(\oo)$,
and their duals $\fC_G(\oo), \fC_T(\oo)$.

Let $\chi, \chi^s\alpha\colon T\to k^\times$
denote the character  
$\chi=\chi_1\otimes\chi_2\omega^{-1}$
and  $\chi^s\alpha=\chi_2\otimes \chi_1\omega^{-1}$, and
\[
\pi_1\coloneqq \Ind_{P}^G\chi\cong
\Ind_{P}^G\chi_1\otimes\chi_2\omega^{-1}\quad
\pi_2\coloneqq \Ind_{P}^G\chi^s\alpha\cong 
\Ind_{P}^G\chi_2\otimes\chi_1\omega^{-1} \in \laMod_{G,\zeta}(\oo).
\]
By \cite[Thm 30]{barthel},
$\pi_i$ are irreducible for  $i=1,2$;
and by \cite[Thm 33]{barthel}, there exists
smooth irreducible $KZ$-representations $\sigma_i$
and exact sequences
\begin{equation}
	0\to \cInd_{KZ}^G\sigma_i\to
	\cInd_{KZ}^G\sigma_i\to \pi_i\to 0
\end{equation}
where the injectiveness follows from \cite[Thm 19]{barthel}.


Let
$\Ord_P\colon \laMod_{G,\zeta}(\oo)\to \laMod_{T,\zeta}(\oo)$
be the functor of ordinary parts
defined by Emerton in \cite{emeI}.
By \cite[Thm 4.4.6]{emeI},
the functor satisfies the adjunction formula
\begin{equation}\label{eq:adj}
	\Hom_{A[G]}(\Ind_{\bar{P}}^GU,V)\cong
	\Hom_{A[T]}(U,\Ord_PV)
\end{equation}
by passage to ordinary parts and apply the isomorphism 
$\Ord_P(\Ind_{\bar{P}}^GU)\cong U$.

By \cite[Prop 7.1]{pask},
when $\iota\colon \pi_1\hookrightarrow \tilde{J}_1$
is the injective envelope of $\pi_1$
in $\laMod_{G,\zeta}(\oo)$,
its passage to the ordinary parts
$\Ord_P(\iota)\colon \Ord_P(\pi_1)\to \Ord_P(\tilde{J}_1)$
is an injective envelope of $\chi^s=\Ord_P(\pi_1)$
in $\laMod_{T,\zeta}(\oo)$.
Furthermore, 
let $\tilde{J}_{\chi^s}$
be the injective envelope of $\chi^s$
in $\laMod_{T,\zeta}(\oo)$,
then a morphism
\begin{equation}\label{eq:inj_envelope}
	\iota_1\colon \Ind_{\bar{P}}^G(\tilde{J}_{\chi^s})\to \tilde{J}_1
\end{equation}
that induces an isomorphism 
$\tilde{J}_{\chi^s}\to \Ord_P(\tilde{J}_1)$
through the adjunction formula \eqref{eq:adj}
is injective.

To simplify notations,
we identify $\laMod_{T,\zeta}(\oo)$
with $\laMod_{\Qp^\times}(\oo)$ through 
the map $\Qp^\times\cong \{\smat{1&\\&*}\}\subset T$;
and identify $\tilde{J}_{\chi^s}$
with $\tilde{J}_{\chi_1}$,
the injective envelope of $\chi_1$
in $\laMod_{\Qp^\times}(\oo)$.
The above results holds similarly for
the injective envelope $\tilde{J}_2$ of $\pi_2$
in $\laMod_{G,\zeta}(\oo)$ and
the injective envelope $\tilde{J}_{\chi_2}$ of $\chi_2$
in $\laMod_{\Qp^\times}(\oo)$.

Dually, let
$\tilde{P}_{\chi_i^\vee}\coloneqq \tilde{J}_{\chi_i}^\vee\in\fC_T(\oo)$ and
$\tilde{M}_i\coloneqq \Ind_{\bar{P}}^G(\tilde{J}_{\chi_i})^\vee,
\tilde{P}_i\coloneqq \tilde{J}_i^\vee\in\fC_G(\oo)$  for $i=1,2$.
Fix injections
$\iota_i\colon \Ind_{\bar{P}}^G(\tilde{J}_{\chi_i})\hookrightarrow \tilde{J}_i$
as in \eqref{eq:inj_envelope}, 
which induces isomorphisms after passing
to the ordinary parts.
Then the surjections 
$p_i\colon \tilde{P}_i\twoheadrightarrow \tilde{M}_i$
induced by taking the Pontryagin duals
extend to the exact sequences
\begin{equation}\label{eq:exact}
	0\to \tilde{P}_{2}\xrightarrow{\phi_{12}} 
	\tilde{P}_{1}\xrightarrow{p_1} \tilde{M}_1\to 0 \text{ and }
	0\to \tilde{P}_{1}\xrightarrow{\phi_{21}} 
	\tilde{P}_{2}\xrightarrow{p_2} \tilde{M}_2\to 0
\end{equation}
by \cite[Cor 7.7]{pask}.
Moreover
composition with $p_i$
induces the surjective ring homomorphisms
\begin{equation}\label{eq:end_surj}
	\End_{\fC_G(\oo)}(\tilde{P}_i)\twoheadrightarrow
\End_{\fC_G(\oo)}(\tilde{P}_i, \tilde{M}_i)=
\End_{\fC_G(\oo)}(\tilde{M}_i)\cong
	\End_{\fC_T(\oo)}(\tilde{P}_{\chi_i^\vee})\cong
	\oo\llbracket x,y\rrbracket
\end{equation}
by \cite[Cor 7.2]{pask}.

Let $\V\colon \C_G(\oo)\to \Rep_{\Gp}(\oo)$
be the Colmez functor introduced by Pask\={u}nas
in \cite[\S 5.7]{pask},
where 
$\Rep_{\Gp}(\oo)$
is the category of continuous $\Gp$-representations
on compact $\oo$-modules.
The functor is exact and covariant;
and, with notations introduced 
in the previous subsection, satisfies that
$\V(\pi_i^\vee)=\chi_i$,
$\V(\kappa_{ij}^\vee)=\rho_{ij}$,
and $\V(\tilde{P}_j)=\tilde{\rho}_{ij}$
is the universal deformation
with fixed determinant $\det=\zeta\varepsilon$
by \cite[Cor 8.7]{pask}.
Here
$\kappa_{ij}\in \laMod_{G,\zeta}(\oo)$,
for $(i,j)=(1,2)$ or  $(2,1)$,
are the unique non-split extensions
\[
	0\to \pi_2\to \kappa_{12}\to \pi_1\to 0,\quad
	0\to \pi_1\to \kappa_{21}\to \pi_2\to 0
\]
It then follows from \cite[Lem 8.10]{pask} that 
taking the Colmez functor 
$\V$ induces the isomorphisms below.
\begin{equation}\label{eq:end_def}
\begin{split}
	\End_{\fC_{G}(\oo)}(\tilde{P_2})\cong R_{12}\cong R,\quad
	\Hom_{\fC_G(\oo)}(\tilde{P}_2, \tilde{P}_1)\cong R\Phi_{12}\\
	\Hom_{\fC_G(\oo)}(\tilde{P}_1, \tilde{P}_2)\cong R\Phi_{21},\quad
	\End_{\fC_{G}(\oo)}(\tilde{P_1})\cong R_{21}\cong R
\end{split}
\end{equation}
Write $\B=\{\pi_1,\pi_2\}$ 
and $ \tilde{P}_\B=\tilde{P}_1\oplus \tilde{P}_2$,
then $\End_{\fC_G(\oo)}(\tilde{P}_\B)\cong 
\End_{\Gp}(\tilde{\rho}_{12}\oplus \tilde{\rho}_{21})$;
whose center $R$ is also identified with 
the center of the category $\fC_G(\oo)^\B$.

\begin{lem}\label{lem:ker_ord}
	The kernel of the ring homomorphisms
	on $\End_{\fC_G(\oo)}(\tilde{P}_i)\cong R$
	induced by passage to the ordinary parts
	is the image under the above isomorhisms
	of the reducibility ideal $\tau\subset R$.
\end{lem}
\begin{proof}

We first note that the kernel in question 
with that of the surjection in \eqref{eq:end_surj},
since the isomorphism in the middle of which
is also induced by passing to the ordinary parts
and $\Ord_P(\iota_i)$ are isomorphisms.
Thus it suffices to show that 
the image of $\tau\subset$
consists of endomorphisms  in  $\End_{\fC_G(\oo)}(\tilde{P}_i)$
whose compositions with $p_i$ are trivial.
Consider now the diagram
\[
	\begin{tikzcd}
		0 \arrow[r]&
		\tilde{P}_1  \arrow[r,"\phi_{21}"] \arrow[ddr,"0"] &
		\tilde{P}_2 \arrow[r,"p_2"] \arrow[d,"\phi_{12}"] &
		\tilde{M}_2  \arrow[r] & 0 \\
		 &
		%\tilde{P}_1/\tau\tilde{P}_1 \arrow[l] \arrow[d] 
		 &
		\tilde{P}_1  \arrow[d,"p_1"] &
		 & \\
				      &  
				      & \tilde{M}_1  & & 
	\end{tikzcd}
\]
By the adjunction formula \eqref{eq:adj}, 
$\Hom_{\fC_G(\oo)}(\tilde{P}_j,\tilde{M}_i)\cong
\Hom_{\fC_T(\oo)}(\tilde{P}_{\chi_j^\vee}, \tilde{P}_{\chi_i^\vee})=0$
for $i\neq j$
since blocks in $\fC_T(\oo)$ are singletons
(see the discussion in the begining of \cite[\S 7.2]{pask}).
Therefore taking $\Hom(\tilde{P}_j,\cdot)$ to the exact sequences in
\eqref{eq:exact} gives
\[
	\phi_{12}\circ\colon
	\End_{\fC_G(\oo)}(\tilde{P}_2)\cong
	\Hom_{\fC_G(\oo)}(\tilde{P}_2, \tilde{P}_1)\quad
	\phi_{21}\circ\colon
	\End_{\fC_G(\oo)}(\tilde{P}_1)\cong
	\Hom_{\fC_G(\oo)}(\tilde{P}_1, \tilde{P}_2)
\]
Combined with the isomorphisms in \eqref{eq:end_def},
we may assume that $\V(\phi_{ij})$ agree with 
$\Phi_{ij}$ in \eqref{eq:Phi_ij}.

Since $\phi_{12}\circ\phi_{21}\in \End_{\fC_G(\oo)}$
belongs to the kernel of the surjection in \eqref{eq:end_surj}
and is the image of a generator 
of $\tau\subset R$,
it generates the kernel 
since  $R$ is formally smooth of relative dimension  $3$.
\end{proof}




\subsection{injective objects}

In this subsection 
let $N\in\laMod_{G,\zeta}(\oo)$ be an object
satisfying the assumption
\begin{equation}\label{cond:adm_inj}\tag{\text{adm-inj}}
	N\in \aMod_{G,\zeta}(\oo)\cap \laMod_{G,\zeta}(\oo)^\B
	\text{ and is injective as a $\GL_2(\Zp)$-representation}.
\end{equation}
Here $\B=\{\pi_1,\pi_2\}$ 
is the block consisting of the two irreducible representations
introduced previously.



\begin{lem}
	When $N$ satisfies \eqref{cond:adm_inj},
	there exists a projective resolution
	for $N^\vee\in \fC_G(\oo)^\B$ 
	as follows 
	for $\tilde{P}_\B\coloneqq \tilde{P}_1\oplus \tilde{P_2}$ and
	some non-negative integer $r$.
\begin{equation}\label{eq:resolution}
0\to \tilde{P}_\B^r\to \tilde{P}_\B^r\to N^\vee\to 0
\end{equation}
\end{lem}
\begin{proof}
Denote $\Ext^i_{\laMod_{G,\zeta}(\oo)}$
by $\Ext^i_G$ for short 
and apply which
to the exact sequences in \eqref{eq:exact} gives
\begin{equation*}
    \begin{tikzcd}[row sep=2ex]
        \Ext^{i-1}_{G}(\pi, N)\arrow[r] &
        \Ext^{i}_{G}(\text{c-ind}_{KZ}^G\sigma, N)\arrow[r] \arrow[d,symbol={=}] &
        \Ext^{i}_{G}(\text{c-ind}_{KZ}^G\sigma, N)\arrow[r] \arrow[d,symbol={=}] &
        \Ext^{i}_{G}(\pi, N)\\ 
        & \Ext^i_K(\sigma ,N) &
         \Ext^i_K(\sigma ,N) &
    \end{tikzcd}
\end{equation*}
Since $N$ is injective as a $K$-representation, 
the long exact sequence reduces to 
\begin{equation*}
    0 \to \Hom_G(\pi_i,N)\to \Hom_K(\sigma_i,N)\to \Hom_K(\sigma_i,N)\to \Ext^1_G(\pi_i,N)\to 0
\end{equation*}
It follows that $a_i\coloneqq \dim_k\Hom_G(\pi_i,N)=\dim_k \Ext^1_G(\pi_i,N)$,
which are finite since $N$ is admissible.

Let $\soc(N)=\pi_1^{a_1}\oplus \pi_2^{a_2}$
be the socle  of $N$,
the injective envelope 
$\soc(N)\hookrightarrow \tilde{J}=\tilde{J}_1^{a_1}\oplus \tilde{J}_2^{a_2}$
in $\laMod_{G,\zeta}(\oo)$
factors through an injective morphism 
$\phi_0\colon N\to \tilde{J}$
since the inclusion $\soc(N)\hookrightarrow N$
is essential.
Apply the same construction 
to $\soc(\coker(\phi_0))$
and use $\dim_k\Hom_G(\pi_i, \coker(\phi_0))=\dim_k\Ext^1_G(\pi_i, N)=a_i$
gives another injective morphism
$\phi_1\colon \coker(\phi_0)\to \tilde{J}$,
which is also surjective as
\[
	\Hom_G(\pi_i,\coker(\phi_1))
	\cong \Ext^1_G(\phi_i,\coker(\phi_0))
	\cong \Ext^2_G(\pi, N)=0
\]
Now the lemma follows by picking
$r=\max\{a_1,a_2\}$ and taking the Pontryagin dual.
\end{proof}

Let $A\colon \tilde{P}_\B^r\to \tilde{P}_\B^r$ denote the morphism
in the  resolution \eqref{eq:resolution},
which can be represented by the matrix
$A=\smat{A_{11} & A_{12}\Phi_{12}\\A_{21}\Phi_{21} & A_{22}}$,
where $A_{ij}\in M_r(R)$,
through the isomorphisms in \eqref{eq:end_def}.
Denote by $\bar{A}_{ij}\in M_r(R^{\red})$
the reduction of the matrices modulo $\tau$.
Lemma \ref{lem:ker_ord} shows that 
the resolution induces the right exact sequence
\begin{equation}\label{eq:exact_ord}
	\tilde{P}_{\chi_2^\vee}^r\oplus \tilde{P}_{\chi_1^\vee}^r
	\xrightarrow{\overline{A}_{11}\oplus\overline{A}_{22}}
	\tilde{P}_{\chi_2^\vee}^r\oplus \tilde{P}_{\chi_1^\vee}^r
	\to (\Ord_PN)^\vee\to 0
\end{equation}


\begin{prop}    
	The sequence \eqref{eq:exact_ord}
	is also left exaxt;
	and the matrices $ \bar{A}_{ii}\in M_r(R^{\red})$
	are injective.
\end{prop}
\begin{proof}
	The representation $(\Ord_PN)^\vee\in \fC_T(\oo)$
	is finitely generated over 
	$\oo\llbracket (1+p\Z_p)\rrbracket\cong \oo\llbracket X\rrbracket$
	since $\Ord_P$ preserves admissibility.
	Taking $\Hom_{\fC_T(\oo)}(\tilde{P}_{\chi_i^\vee},\cdot)$
	to \eqref{eq:exact_ord}
	gives the exaxt sequence of $\oo\llbracket x,y\rrbracket$-modules
\begin{equation*}
	\End_{\fC_T(\oo)}(\tilde{P}_{\chi_i^\vee}^r)\to 
	\End_{\fC_T(\oo)}(\tilde{P}_{\chi_i^\vee}^r)\to 
	\Hom_{\fC_T(\oo)}(\tilde{P}_{\chi_i^\vee}, (\Ord_PN)^\vee)\to 0
\end{equation*}
	We note that the first two terms in which are 
	isomorphic to $\oo\llbracket x,y\rrbracket^{\oplus r}$;
	and the last term is torsion over $\oo\llbracket x,y\rrbracket$,
	since there is no injective ring homomorphism from
	$\oo\llbracket x,y\rrbracket$ to $\oo\llbracket X\rrbracket$.
	It is now clear that $\bar{A}_{ii}$
	must be injective,
	for otherwise a contradiction occurs after 
	tensor with the field of fraction of 
	$\oo\llbracket x,y\rrbracket$.
\end{proof}

\begin{cor}
	When $N$ satisfies \eqref{cond:adm_inj},
	the $\End_{\fC_{G}(\oo)}(\tilde{P}_i)$-module
	$\Hom_{\fC_G(\oo)}(\tilde{P}_2, N^\vee)$
	has no $\tau$-torsion 
	under the isomorphism $\End_{\fC_{G}(\oo)}(\tilde{P}_2)\cong R$
	in \eqref{eq:end_def}.
\end{cor}
\begin{proof}
	Identify the center of $\End_{\fC_{G}(\oo)}(\tilde{P}_\B)$ 
	with $R$ by the isomorphisms
	in \eqref{eq:end_def}
	and fix a generator $x$ of the reducibility ideal  $\tau\subset R$.
	The generator acts by, up to an invertible elements,
	$\phi_{ij}\circ\phi_{ji}$ on $\tilde{P}_i$.
	Thus $\tilde{P}_\B[\tau]=0$,
	since by definition $\varphi_{ij}\circ\varphi_{ji}$ are injective.
	Then apply 
	$\Hom_{\fC_G(\oo)}(\tilde{P}_2,\cdot)$
	and the snake lemma to the diagram
    \begin{equation*}
    \begin{tikzcd}
        0 \arrow[r] & \tilde{P}_\B^{\oplus r} 
	\arrow[d,"x",hookrightarrow] \arrow[r,"A"] & 
	\tilde{P}_\B^{\oplus r} 
	\arrow[d,"x",hookrightarrow] \arrow[r] & 
	N^\vee \arrow[d,"x"] \arrow[r] & 0 \\ 
        0 \arrow[r] & \tilde{P}_\B^{\oplus r}
	\arrow[r,"A"] & \tilde{P}_\B^{\oplus r}
	\arrow[r] &N^\vee  \arrow[r] & 0 
    \end{tikzcd}
\end{equation*}
results in the following isomorphism,
whose right-hand-side
is trivial by the previous proposition.
\[
\Hom_{\fC_G(\oo)}(\tilde{P}_2, N^\vee)[\tau]\cong 
\ker\left(
	\Hom_{\fC_G(\oo)}(\tilde{P}_2, \tilde{P}_\B^{\red})^{\oplus r}
\xrightarrow{\smat{\bar{A}_{11}& \bar{A}_{12}\Phi_{12}\\& \bar{A}_{22}}}
\Hom_{\fC_G(\oo)}(\tilde{P}_2, \tilde{P}_\B^{\red})^{\oplus r} \right)
\]\qedhere
\end{proof}

\begin{cor}
	When $N$ satisfies \eqref{cond:adm_inj}, 
	there exists an isomorphism between $R^{\red}$-modules
    \begin{align}
	    \Hom_{\fC_{G}(\oo)}(\tilde{P}_2,(N^\vee)^{\red})
	    &\cong
	    \Hom_{\fC_{T}(\oo)}(\tilde{P}_{\chi_1^\vee}, (\Ord_PN)^\vee)\oplus
	    \Hom_{\fC_{T}(\oo)}(\tilde{P}_{\chi_2^\vee},(\Ord_PN)^\vee)\\
	    &\cong
	    ((\Ord_PN)_{U_p\equiv \chi_2\omega^{-1}})^\vee\oplus
	    ((\Ord_PN)_{U_p\equiv \chi_1\omega^{-1}})^\vee
    \end{align}
\end{cor}
\begin{proof}
	Denote $\Hom_{\fC_{G}(\oo)}$ by $\Hom_G$ for short.
	We decompse the resulting sequence of cokernels
	from the proof of the previous corollary into the following diagram
\begin{equation*}
    \begin{tikzcd}
	    0 \arrow[r]& \Hom_{G}(\tilde{P}_2, \tilde{P}_2^{\red})^{\oplus r}
	    \arrow[r,"\bar{A}_{11}"] \arrow[d]&
	    \Hom_{G}(\tilde{P}_2, \tilde{P}_2^{\red})^{\oplus r}
	    \arrow[d] &&\\
	    0\arrow[r] & \Hom_{G}(\tilde{P}_2, \tilde{P}_\B^{\red})^{\oplus r}
	    \arrow[r] 
	    \arrow[d] &
	    \Hom_{G}(\tilde{P}_2, \tilde{P}_\B^{\red})^{\oplus r}
	    \arrow[d] \arrow[r]&
	    \Hom_{G}(\tilde{P}_2, (N^\vee)^{\red})\arrow[r]&0\\
	    0\arrow[r] & \Hom_{G}(\tilde{P}_2, \tilde{P}_1^{\red})^{\oplus r}
	    \arrow[r,"\bar{A}_{22}"] &
	    \Hom_{G}(\tilde{P}_2, \tilde{P}_1^{\red})^{\oplus r}&&
    \end{tikzcd}
\end{equation*}
But by the exactness of \eqref{eq:exact_ord}
and the proof of the proposition thereafter,
the cokernels on the first and last rows
are isomorphic, as  $R^{\red}$-modules,
to $\Hom_{\fC_T(\oo)}(\tilde{P}_{\chi_2^\vee}, (\Ord_PN)^\vee)$ and
$\Hom_{\fC_T(\oo)}(\tilde{P}_{\chi_1^\vee}, (\Ord_PN)^\vee)$ respectively.
\end{proof}  

\subsection{unitary completion}

Recall that $R$ is the universal deformation ring
of the pseudo-representation
$\chi_1+\chi_2$, with fixed determinant  $\zeta\varepsilon$;
and that  $R$
is also isomorphic to the center
of the category  $\fC_G(\oo)^\B$,
for  $\B=\{\pi_1,\pi_2\}$ 
introduced in the previous subsections.

Let $\fn$ be a maximal ideal of  $R[\frac{1}{p}]$
with residue field  $L$ and 
$T_\fn\colon \Gp\to L$ 
be the corresponding pseudo-representation.
Let  $\Irr(\fn)$
be the set of irreducible objects in  $\Ban_{G,\zeta}(L)^\B_{\fn}$.

By \cite[Cor 8.15]{pask}, 
if $T_\fn=\psi_1+\psi_2$,
where  $\psi_i\colon \Gp\to L^\times$
are continuous homomorphisms, then  
\begin{equation}\label{eq:comple1}
	\Irr(\fn)=\{(\Ind_P^G\psi_1\otimes\psi_2\varepsilon^{-1})_{cont},
	(\Ind_P^G\psi_2\otimes\psi_1\varepsilon^{-1})_{cont}\}
\end{equation}

And by \cite[Cor 8.14]{pask}, 
if $T_\fn$ is irreducible over the residue field $L$, 
then $\Irr(\fn)=\{\Pi\}$ contains only one irreducible object,
which satisfies that $\mtr\mathbf{V}(\Pi)=T_\fn$.
Moreover, by \cite[Thm. 1.3]{CDP},
when $\mathbf{V}(\Pi)$ is crystalline with Hodge-Tate weights $-a,-a-b$,
and $\pi$ is the smooth admissible irreducible
$\GL_2(\Qp)$-representation
corresponding to the Weil-Deligne representation
associated to  $D_{\cris}(\mathbf{V}(\Pi)\vert_{\Gp})$
under the Hecke correspondence
(if the Weil-Deligne representation is $\psi_1+\psi_2$,
then $\pi$ is isomorphic to 
the (un-normalized) induction $\Ind_B^G(\psi_1\otimes\psi_2||^{-1})$,
then $\Pi$ is the universal unitary completion
of  $\pi\otimes(\Sym^{b-1}\otimes \det^a)$


\subsection{computation}

Let $\rho$ be the Galois representation
associated to  $\pi$, that satisfies
\[
	\WD(\rho_v)=\Rec(\pi_v||^{-1/2})
\]
and let $\B$ be the block associated to the reduction
of  $\rho\otimes\varepsilon$.

If $\mtr\rho_v=\psi_1+\psi_2$ is reducible,
where $\psi_i$ are respectively of weight
$w$ and  $w+k-1$
then 
\[
	\WD(\rho)=\psi_1\varepsilon^w||^{-w}+
	\psi_2\varepsilon^{w+k-1}||^{1-k-2}
\]
Then $\pi_v$ is isomorphic to the un-normalized induction
$\Ind(\mu_1\otimes\mu_2||^{-1})_{\sm}$, where
\begin{align*}
	\mu_1&=\psi_1\varepsilon^{w}||^{1-w} &
	\val_p(\mu_1(p))&=w-1=(w+k-2)-(k-1)\\
	\mu_2&=\psi_2\varepsilon^{w+k-1}||^{2-k-w} &
	\val_p(\mu_2(p))&=w+k-2
\end{align*}
Since $(\Sym^{k-2}\otimes\det{}^w)^*\cong \Sym^{(k-1)-1}\otimes\det{}^{2-k-w}$
and $\varepsilon(x)=x|x|$,
by \cite[Thm 12.3]{pask}, 
the universal unitary completion
of $\pi_v\otimes(\Sym^{k-2}\otimes\det{}^w)^*$
is isomorphic to $\Ind_B^G(\psi)_{\cont}$, where
\[
	\psi(\smat{a&*\\&d})=\mu_2(a)a^{2-k-w}\mu_1(d)|d|^{-1}d^{(k-1)+(2-k-w)-1}
	=\psi_2\varepsilon(a)\psi_1(d)
\]
Note that indeed 
$\Ind(\psi_2\varepsilon\otimes\psi_1)_{cont}=
\Ind((\psi_2\varepsilon)\otimes(\psi_1\varepsilon)\varepsilon^{-1})$.

\subsection{finite condition}

Let $G$ be a  $p$-adic analytic group
and  $V\in Mod_G(A)$, define 
 \[
	V_{Z-fin}=
	\{v\in V\mid \text{$A[Z]$-submodule generated by  $v$
	is finitely generated over  $A$}\}
\]
\begin{lem}
	For $V\in Mod_G^{sm}(A)$
	\begin{enumerate}[label=(\alph*)]
		\item finite length and admissible
		\item finitely generated over $A[G]$
			and admissible, 
		\item finite length and $Z$-finite
	\end{enumerate}
	Then $(i)$ implies $(ii), (iii)$
	and $(ii)$ implies  $Z$-finite.
\end{lem}

\begin{rem}
	If $G=G_1\times G_2$ and  $G_2\subset Z=Z_G$,
	then  $(A[G]-fg + G-adm)$ 
	implies  $(A[G_1]-fg + G_1-adm)$ 
	Thus when
	$G=\prod\GL_2(\Qp)\times T$,
	then locally admissible 
	implies locally of finite length
\end{rem}

\section{pseudo-compact}

\begin{defn}
	A pseudocompact ring $\Lambda$
	is a complete Hausdorff topological ring 
	admitting systems of open neighborhood of  $0$
	consisting of two-sided ideas  $I$
	for which  $\Lambda/I$ is an Artin ring.
\end{defn}

\begin{defn}
	A pseudocompact $\Lambda$-module $M$
	is a complete Hausdorff topological $\Lambda$-module
	admitting systems of open neighborhood of  $0$
	consisting of submodules $N$
	for which  $M/N$ is of finite length.
	Which is equivalent to that 
	$M$ is the inverse limit of 
	 $\Lambda$-modules of finite length.
\end{defn}

\begin{defn}
	Fix a commutative pseudocompact ring $\Omega$.
	A pseudocompact $\Omega$-algebra  $\Lambda$
	is a complete Hausdorff topological ring
	admitting systems of open neighborhood of  $0$
	consisting of two-sided ideals $I$
	for which  $\Lambda/I$ is $\Omega$-modules of finite length.
\end{defn}
For example, $\Lambda=\oo\llbracket G\rrbracket$
for profinite group  $G$.

If $A$ is a f.g. right  $\Lambda$-modules, then 
 \[
	A\hat{\otimes}_\Lambda B\coloneqq
	\varprojlim(A/U\otimes_{\Lambda}B/V)\cong A\otimes_{\Lambda}B
\]
And also $\Omega\llbracket G\rrbracket\hat{\otimes}\Omega\llbracket H\rrbracket
\cong\Omega\llbracket G\times H\rrbracket$.

Let $E$ be the injective envelope of  $\Omega$ in the category of diescrete 
 $\Omega$-modules, then 
  \[
 	\{\text{right pseudocpt $\Lambda$-mod}\}\leftrightarrow
	\{\text{left discrete $\Lambda$-mod}\}\quad
	A\to \Hom_{\Omega}(A,E), 
	\Hom_{\Omega}(C,E)\rightarrow C
 \]

\begin{lem}
	Let $ \{\Lambda_i,\lambda_{ij}\}$
	be a  directed partially ordered inverse system of pseudocompact 
	$\Omega$-algebras. Similarly
	Let $ \{A_i,\alpha_{ij}\}, \{B_i,\beta_{ij}\}$
	and define 
	\[
		\Lambda=\varprojlim_{i}\Lambda_i,
		A=\varprojlim_{i}A_i,
		B=\varprojlim_{i}B_i,
	\]
	Then $A \hat{\otimes}_{\Lambda}B\cong 
	\varprojlim A_i\hat{\otimes}_{\Lambda_i}B_i$
	if $\lambda_{ij}, \alpha_{ij}, \beta_{ij}$
	are epimorphisms.
\end{lem}

For example, let $\Lambda=\Lambda_i=\oo$,
a valuationring of a finite extension  $E/\Qp$,
$A=A_i$ a local complete Noetherian  $\oo$-algebra
with finite residue,
and  $B=\oo\llbracket G\rrbracket =\varprojlim \oo[G/N_i]$
and  $B_i=\oo[G/N_i]$, then
\[
	A\llbracket G\rrbracket \coloneqq 
	A\hat{\otimes}_\oo\oo\llbracket G\rrbracket 
	\cong \varprojlim
	A\hat{\otimes}_{\oo}\oo[G/N_i]
	\cong \varprojlim
	A[G/N_i]
\]


\bibliographystyle{amsalpha}
\bibliography{biblio}

\newpage



We start by recalling some facts about 
anticyclotomic extensions and their Galois groups.
For a prime-to-$p$ ideal $\fs=\fs^c$ as above
let $V=V_\fs=(1+\fs\widehat{\oo}_\K)^\times$,
by class field theory
$\Cl(V)\coloneqq \K^\times\backslash\A_{\K,f}^\times/V$
is isomorphic to the ray class group of conductor $\fs$
and fits into the exact sequence
\[
	1\to \oo_\K^\times\backslash U/V
	\to \Cl(V)\to \Cl_\K\to 1
\]
for $U=\widehat{\oo}_\K^\times$.
Let $\tilde{V}\subset V$
be the open compact subgroup
such that $U/\tilde{V}$
is isomorphic to the maximal pro-$p$ quotient
$\Delta_\fs$ of $(\oo_\K/\fs)^\times$.
Then the maximal pro-$p$ quotient $G_\fs$ of 
$\Cl(\tilde{V})$ fits into an exact sequence
\[
	1\to \oo_\K^\times\backslash \Delta_\fs
	\to G_\fs\to C_\K\to 1
\]
where $C_\K$ is the maximal pro-$p$ quotient
of the class group $\Cl_\K$.
As $\fs=\fs^c$,
each group above admits the action 
of the complex conjugation.
Let the supscript $\pm$
denote the subgroups on which 
the complex conjugation 
takes a group element
to itself or the inverse,
and let the subscript $a$
(which stands for anticyclotomic)
denote the quotient by the $+$part.
As $p$ is odd,
we have $1-c\colon G_\fs^a\cong G_\fs^-$
and similary for $\oo_\K^\times\backslash \Delta_\fs$
and $C_\K$.
If particular,
if we fix a splitting $\fs=\ff\ff^c$
with $\ff+\ff^c=\oo_\K$,
then $(1-c)\Delta_\fs^-$ is isomorphic to $\Delta_\ff$,
the maximal pro-$p$ quotient of $(\oo_\K/\ff)^\times$,
and the $(1-c)\oo_\K^\times$-action on which 
factors through the usual action of $W_p$,
the group of $p$-power roots of unity in $\K$.
Therefore we have the exact sequence
\[
    1\to W_p\backslash \Delta_\ff\to G_\fs^a\to C_\K^a\to 1.
\]

Now, for any ideal 
$\fa=\prod_{w\in\Sigma_p}\fp_w^{n_w}$
divisible only by primes in $\Sigma_p$
and $\fs=\fs^c$ as above
we write $\tilde{V}_\fa=\tilde{V}^p\times U_\fa$
for $U_\fa=(1+\fa\fa^c\oo_{\K,p})^\times$.
Let $G_{\fs\fa}$
be the maximal pro-$p$ quotient of $\Cl(\tilde{V}_\fa)$,
then the same procedure as above gives
the exact sequence
\[
    1\to W_p\backslash(\Delta_\ff\times\Delta_\fa)\to 
    G_{\fs\fa}^a\to C_\K^a\to 1
\]
where $\Delta_\fa$ is the product over $w\in\Sigma_p$
of $(1+\varpi_w\oo_w)/(1+\varpi_w^{n'_w}\oo_w)$
for $n'_w=\max\{n_w,1\}$.
Then $\fG_\fs^a$ is isomorphic to $\varprojlim G_{\fs\fa}^a$,
where $\fa$ goes through all ideals as above,
and fits into the exact sequence
\[
    1\to W_p\backslash(\Delta_\ff\times U_1)\to 
    \fG_{\fs}^a\to C_\K^a\to 1
\]
where $U_1\coloneq\prod_{w\in\Sigma_p}(1+\varpi_w\oo_w)$.

\end{document}

