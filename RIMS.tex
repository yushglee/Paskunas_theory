\documentclass[leqno]{amsart}
\usepackage{amsmath} 
\usepackage{amssymb,mathtools,stmaryrd}
\usepackage{mathrsfs,euscript}
\usepackage[table,dvipsnames]{xcolor}
\usepackage{hyperref,tikz-cd,enumitem}
\usepackage[utf8]{inputenc}
\hypersetup{
 colorlinks=true,
 linkcolor=DarkOrchid,
 filecolor=blue,
 citecolor=olive,
 urlcolor=orange,
 pdftitle={Pask\={u}nas' theory},
}
\pagestyle{plain}

\setlength{\textwidth}{\paperwidth}
\addtolength{\textwidth}{-2in}
\calclayout

\newtheorem{thm}{Theorem}[section]
\newtheorem{lem}[thm]{Lemma}
\newtheorem{prop}[thm]{Proposition}
\newtheorem{cor}[thm]{Corollary}
\newtheorem*{conj}{Conjecture}



\theoremstyle{definition}
\newtheorem{defn}[thm]{Definition}

\theoremstyle{remark}
\newtheorem{rem}[thm]{Remark}
\newtheorem{ack}{Acknowledgement}

%%%%%%%%%% COMMONLY USED COMMAND %%%%%%%%%%%%%%%%

\newcommand{\smat}[1]{\left(\begin{smallmatrix} #1 \end{smallmatrix}\right)}
\newcommand{\id}{\mathbf{1}}
\newcommand{\oo}{\mathcal{O}} 
\newcommand{\eo}{\EuScript{O}}
\newcommand{\fF}{\mathbb{F}} % residue field

\newcommand{\Q}{{\mathbf{Q}}}
\newcommand{\Z}{{\mathbf{Z}}}
\newcommand{\Qp}{\mathbf{Q}_p}
\newcommand{\Zp}{\mathbf{Z}_p}
\newcommand{\Ql}{\mathbf{Q}_\ell}
\newcommand{\Zl}{\mathbf{Z}_\ell}
\newcommand{\R}{\mathbf R}
\newcommand{\C}{\mathbf C}
\newcommand{\A}{\mathbf A}
\newcommand{\dd}{\mathfrak{d}} %different
\newcommand{\DD}{\mathcal{D}}  %discriminant
\newcommand{\Cl}{Cl} %class_group
\newcommand{\arch}{\mathbf{a}}
\newcommand{\finite}{\mathbf{h}}
\DeclareMathOperator{\Nr}{N}
\DeclareMathOperator{\Tr}{Tr}

\DeclareMathOperator{\End}{End}
\DeclareMathOperator{\Aut}{Aut}
\DeclareMathOperator{\Hom}{Hom}
\DeclareMathOperator{\Ext}{Ext}
\DeclareMathOperator{\Tor}{Tor}
\DeclareMathOperator{\Ind}{Ind}
\DeclareMathOperator{\cInd}{c-Ind}
\DeclareMathOperator{\nInd}{n-Ind}
\DeclareMathOperator{\Res}{Res}
\DeclareMathOperator{\Cor}{Cor}
\DeclareMathOperator{\Image}{Im}
\DeclareMathOperator{\coker}{coker}
\DeclareMathOperator{\rank}{rank}
\DeclareMathOperator{\corank}{corank}

\DeclareMathOperator{\Sym}{Sym}
\DeclareMathOperator{\Ad}{Ad}

\DeclareMathOperator{\Lie}{Lie}
\DeclareMathOperator{\GL}{GL}
\DeclareMathOperator{\SL}{SL}
\DeclareMathOperator{\UU}{U}
\DeclareMathOperator{\gl}{\mathfrak{gl}}
\DeclareMathOperator{\mtr}{tr}
\DeclareMathOperator{\diag}{diag}
\DeclareMathOperator{\chr}{char} 

\DeclareMathOperator{\vol}{vol} %volume
\DeclareMathOperator{\val}{val} %valuation
\DeclareMathOperator{\ind}{ind} %index
\DeclareMathOperator{\odr}{ord} %order

\DeclareMathOperator{\Spec}{Spec}
\DeclareMathOperator{\Supp}{Supp}
\DeclareMathOperator{\Ass}{Ass}
\DeclareMathOperator{\Ann}{Ann}
\DeclareMathOperator{\Der}{Der}
\DeclareMathOperator{\Fitt}{Fitt}
\DeclareMathOperator{\depth}{depth}
\DeclareMathOperator{\length}{length}
\DeclareMathOperator{\Inj}{Inj}
\DeclareMathOperator{\Isom}{Isom}

\DeclareMathOperator{\Gal}{\mathcal{G}}
\DeclareMathOperator{\WD}{WD}
\DeclareMathOperator{\Rec}{Rec}
\DeclareMathOperator{\rec}{rec}
\DeclareMathOperator{\Art}{Art}
\newcommand{\Fr}{\textnormal{Fr}} %geometric Frobenius
\newcommand{\frob}{\textnormal{frob}} %arithmetic Frobenius
\newcommand{\dR}{\textnormal{dR}}
\newcommand{\pst}{\textnormal{pst}}
\newcommand{\st}{\textnormal{st}}
\newcommand{\cris}{\textnormal{cris}}

\newcommand{\cont}{\textnormal{cont}}
\newcommand{\cts}{\textnormal{cts}}

\newcommand{\fa}{\mathfrak{a}}
\newcommand{\fc}{\mathfrak{c}}
\newcommand{\ff}{\mathfrak{f}}
\newcommand{\fg}{\mathfrak{g}}
\newcommand{\fk}{\mathfrak{k}}
\newcommand{\fl}{\mathfrak{l}}
\newcommand{\fm}{\mathfrak{m}}
\newcommand{\fn}{\mathfrak{n}}
\newcommand{\fp}{\mathfrak{p}}
\newcommand{\fq}{\mathfrak{q}}
\newcommand{\fs}{\mathfrak{s}}
\newcommand{\ft}{\mathfrak{t}}

%%%%%%%%%% MORE SPECIFIC COMMAND %%%%%%%%%%%%%%%%

\newcommand{\bs}{\mathcal{S}} %Bruhat-Shwartz

%%% p-adic local Langlands

\DeclareMathOperator{\Mod}{\textnormal{Mod}}
\DeclareMathOperator{\fC}{\mathfrak{C}} %dual category
\DeclareMathOperator{\Ban}{\textnormal{Ban}_{G,\zeta}^{\adm}}
\DeclareMathOperator{\Rep}{Rep}
\DeclareMathOperator{\V}{\check{\mathbf{V}}} %Colmez's functor
\DeclareMathOperator{\Ord}{Ord} %Emerton's functor
\DeclareMathOperator{\Irr}{Irr}
\DeclareMathOperator{\soc}{soc}

\newcommand{\Gp}{\mathcal{G}_{\Qp}} %Galois group over \Qp
\newcommand{\B}{\mathfrak B} %Paskunas' Block

\newcommand{\sm}{\textnormal{sm}}
\newcommand{\adm}{\textnormal{adm}}
\newcommand{\ladm}{\textnormal{ladm}}
\newcommand{\lfin}{\textnormal{lfin}}
\newcommand{\ps}{\textnormal{ps}}
\newcommand{\red}{\textnormal{red}}

\newcommand{\xx}{x_\textnormal{red}}

%%% Number Fields

\newcommand{\F}{{\mathcal{F}}} %global totally real
\newcommand{\K}{{\mathcal{K}}} %global CM
\newcommand{\qch}{\epsilon} % quadratic character of K/F
\newcommand{\bw}{{\overline{w}}}
\newcommand{\bl}{{\bar{\fl}}}

\newcommand{\cG}{\mathcal{G}}
\newcommand{\fG}{\mathfrak{G}}
\newcommand{\fX}{\mathfrak{X}}


%%% Modular Forms

\newcommand{\wt}[1]{\underline{ #1 }}
\newcommand{\Iw}{\textnormal{Iw}} %Iwahori subgroup
\newcommand{\TT}{\mathbb{T}} % Hecke algebra
\newcommand{\euF}{\EuScript{F}} %Hida family
\newcommand{\I}{\mathcal{I}} %lCN flat over Lambda
\newcommand{\ord}{\textnormal{ord}} %ordinary


%%% Galois cohomology

\DeclareMathOperator{\loc}{loc} 
\DeclareMathOperator{\Sel}{Sel} 
\DeclareMathOperator{\car}{char} 
\newcommand{\lin}[1]{\mathcal{L}^{(#1)}} 
\newcommand{\Lda}[1]{\Lambda^{(#1)}} 

\begin{document}
\title{Anticyclotomic Euler systems for CM fields}
\author[Y-S.~Lee]{Yu-Sheng Lee\\
Department of Mathematics, University  of Michigan, Ann Arbor}
\address{Department of Mathematics, University  of Michigan, Ann Arbor, MI 48109, USA}
\email{yushglee@umich.edu}
\date{}

\begin{abstract}
    The article surveys the author's preprint \cite{lee2} on
    a construction of an anticyclotomic Euler systems
    for characters of CM fields and its application 
    toward the Iwasawa main conjecture.
\end{abstract}

\maketitle
\setcounter{tocdepth}{1}

\section*{Introduction}

Let $\K$ be a CM field with maximal totally real subfield $\F$.
Fix an odd prime $p>0$ which is unramified in $\F$.
We assume that $\K/\F$ is ordinary at $p$ in the sense that
every prime of $\F$ above $p$ splits in $\K$.
Under this assumption we can then fix a CM type,
which is a subset $\Sigma\subset \Hom(\K,\C)$
such that if $S_p^\K$ denote the set of primes of $\K$ above $p$
and $\Sigma_p\subset S_p^\K$ is the subset of primes
induced by embeddings in $\Sigma$,
through a fixed choice of embeddings
$\iota_\infty\colon \K\to \C$ and
$\iota_p\colon \K\to \C_p$, then we have
\[
    \Hom(\K,\C)=\Sigma\sqcup \Sigma^c,\quad
    S_p^\K=\Sigma_p\sqcup \Sigma_p^c
\]
where $\Sigma^c=c\circ \Sigma$ and $\Sigma_p^c=c\circ \Sigma_p^c$
for the complex multiplication $c\in Gal(\K/\F)$.

Given a finite order character $\psi$
of the absolute Galois group $\Gal_\K$ of $\K$.
The Iwasawa-Greenberg main conjecture 
relates the following two objects associated to $\psi$.
On the one hand, 
by class field theory, the Galois group of
the maximal pro-$p$ abelian extension which is unramified 
away $p$ has a free part $W$ of $\Zp$-rank $d+1-\delta$,
where $d=[\F:\Q]$ and $\delta$ is the Leopoldt defect.
We let $\widetilde{K}/\K$ denote the extension such that 
$Gal(\widetilde{K}/\K)\cong W$.
Define $\Lambda_W=\oo\llbracket W\rrbracket$,
where $\oo$ is the ring of integers of a sufficiently large
field extension $E$ over $\Qp$,
and let $\Gal_\K$ acts on 
the Pontryagin dual $\Lambda_W^*$ of which through
$\psi_\Lambda=\psi\langle*\rangle$, where 
$\langle*\rangle\colon \Gal_\K\to W\to \Lambda^\times$
is the tautological character. 
Consider the Selmer group
\[
    \Sel(\psi_\Lambda,\Sigma_p)=\ker
    \big\{
    H^1(\K, \Lambda_W^*(\psi_\Lambda))\to \prod_{w\notin \Sigma_p}
    H^1(I_w, \Lambda_W^*(\psi_\Lambda))
    \big\}
\]
where $w$ ranges through all finite primes of $\K$
outside $\Sigma_p$
and $I_w$ is the inertia group at $w$.
Then the Pontryagin dual 
$X(\Lambda_W)\coloneqq \Sel(\psi,\Sigma_p)^\vee$
is a finitely generated torsion $\Lambda$-module,
for which we can consider the characteristic ideal
$F(\Lambda_W)\coloneqq \car_{\Lambda_W}(X(\Lambda_W))$.

On the other hand, fix a CM type $\Sigma$ as above.
By the work Katz \cite{Katz1978} 
and Hida-Tilouine \cite{HT93} there exists 
a $p$-adic L-function
$L_p(\psi,\Sigma_p)\in \Lambda_W$ 
that interpolates the algebraic part
of the $L$-value $L(0,\psi\alpha)$,
when $\alpha$ is a character of $W$ such that
the $L$-value is critical in the sense of Deligne's conjecture.

\begin{conj}
The characteristic ideal $F(\Lambda_W)$
is generated by $L(\psi,\Sigma_p)$ in $\Lambda_W$.
\end{conj}


The conjecture can also be formulated for
subextensions in $\widetilde{K}/\K$.
But unless the subextension contains the 
cyclotomic $\Zp$-extension,
the Pontryagin dual of the Selmer group defined above
could fail to be torsion.

In particular, the case when 
$\psi$ is anticyclotomic and
$\widetilde{\K}^a\subset \widetilde{K}$ is the maximal 
anticyclotomic subextension is considered
in \cite{HT93} and \cite{HT94}.
Here $\widetilde{\K}^a$ is anticyclotomic in the sense that 
$c\in Gal(\K/\F)$ acts by the inversion on $W^-=\Gal(\widetilde{K}^a/\K)$.
In this case $W^-$ has the $\Zp$-rank $d=[\F:\Q]$.
And we say $\psi$ is anticyclotomic if 
$\psi(\gamma^c)=\psi^{-1}(\gamma)$ for $\gamma\in\Gal_\K$.
Using the CM congruences between modulars forms,
it is shown in \textit{loc.cit.}
that the characteristic ideal $F(\Lambda_{W}^-)$
belongs to the ideal generated by $L_p(\psi,\Sigma_p)$,
the projection of the Katz-Hida-Tilouine $p$-adic $L$-function
to the anticyclotomic space $\oo\llbracket W^-\rrbracket$.
Furthermore, assume that
\begin{itemize}
    \item The order of $\psi$ is coprime to $p$.
    \item The prime-to-$p$ part of the conductor of $\psi$
    is a product of primes that splits in $\K$.
    \item The restriction $\psi\vert_{D_w}$ is nontrivial
    at the decomposition group $D_w$ of each $w\in\Sigma_p$.
    \item The restriction of $\psi$ to 
    the absolute Galois group of $\K(\sqrt{(-1)^{(p-1)/2}p})$
    is nontrivial.
\end{itemize}
Then the crucial torsion property is obtained in
\cite[Thm.5.33]{Hida06b} by relating the Selmer group to 
Galois deformation problems
and the full main conjecture 
is proved in \cite{Hida06}.

In \cite{lee2}, we consider the same anticyclotomic main conjecture under
a different set of assumptions on $\psi$.
One of the main result is as follows.

\begin{thm}\label{thm:main}

Assume there exists a place $w_0\in\Sigma_p$ of degree one and that
\begin{equation}\label{cond:K2in}\tag{$\K$}
\K/\F \text{ is generic in the sense of \cite{Rohrlich}
and every prime of $\F$ above $2$ splits in $\K$}.
\end{equation}
Identify the decomposition group $D_{w_0}$
with the absolute Galois group $\Gp$ of $\Qp$
and let $\omega\colon \Gp\to (\Z/p\Z)^\times$ be the Teichmuller character.
Suppose $\psi$ is ramified only at primes that split in $\K$ and
\begin{enumerate}[label=($\psi$\arabic*)]
\item  The restriction $\psi\vert_{\Gp}$ is not congruent
to the trivial character $\id$ or $\omega^{\pm1}$.
\label{cond:psi1in}
\item The restriction $\psi\vert_{D_w}$ is nontrivial
at the decomposition group $D_w$ of each $w\in\Sigma_p$.
\label{cond:psi2in}
\end{enumerate}
Then $X(\Lambda^-)$ is torsion
and we have $(L_p(\psi,\Sigma_p))\subset F(\Lambda_W^-)$
in $E\llbracket W\rrbracket$.

\end{thm}

Our proof does not rely on global Galois deformation problems
and instead follows from the construction of 
an anticyclotomic Euler system associated to $\psi$.
Then the general machinery from \cite{Rubin}
allows us to bound the size
of the Selmer group from above and obtain the theorem.
This is the reason that the inclusion relation
we obtained is opposite to that of \cite{HT94},



\subsection*{Congruences and Euler systems}

Our construction of the Euler system
follows from Urban's approach in \cite{urban},
which uses Eisenstein congruences 
to reproduce the Euler system of cyclotomic units.
The idea of using congruences relations to construct cohomology classes
dates back to the proof of the converse of
Herbrand-Ribet theorem in \cite{Ribet1976}
and is also the main ingredient of \cite{HT93}.

The usual strategy goes as follows.
Let $\lambda\coloneqq \TT^{\ord}\to \Lambda_W$
be a Hecke eigensystem
of the big ordinary Hecke algebra $\TT^{\ord}$
such that the associated Galois representation is reducible.
In favorable cases, one can find a different eigensystem
$\lambda'$ with irreducible Galois representation and such that
$\lambda$ is congruent to $\lambda'$ modulo some ideal $J$.
Then one can construct 
a nontrivial cohomology class over the quotient $\Lambda_W/J$.


While the strategy has succeeded in many cases,
we cannot construct Euler systems,
which are integral classes that value in $\Lambda_W$.
The insight of Urban's method is that we should add more variables
to the Hecke algebra by consider a larger space of $p$-adic modular forms.
In our case, this is done by allowing modular forms
that are not ordinary at $w_0$.
Then under our assumptions the resulting Hecke algebra
has an extra variable coming from the reducibility ideal
of the local deformation at $w_0$.

\subsection{Outlines of the article}

In the first section we will review 
the theory of Hecke algebras and the associated Galois representations 
for the space of modular forms over a definite unitary group.
In particular, we introduce the larger Hecke algebra $\TT$
acting on modular forms that are possibly non-ordinary at $w_0$.

In the second section, we explain 
that the Hecke algebra $\TT$ has the expected dimension
by borrowing techniques from $p$-adic local Langlands.
We then formulate an exact sequence that will be fundamental 
when we apply a generalization of Ribet's lemma in our setting.

The actual construction is done in the last section,
using the eigensystem constructed by the author in previous work.
We then explain the properties of the resulting Euler
system and  sketch the proof of our main theorem.



\subsection{Notations}

Let $\K$, $\F$, and $p$ be as in the introduction
and let $\epsilon=\epsilon_{cyc}$ denote 
the $p$-th cyclotomic character.
We assume that $\K/\F$ is ordinary 
and fix a CM type $\Sigma$.
Let $\arch$ and $\finite$ denote respectively
the set of archimedean and finite places of $\F$.
Write $I_\K=\Hom(\K,\bar{\Q})$ and 
let $c\in \Gal(\K/\F)$ be the complex conjugation.
For $\sigma\in I_\K$
we define $\sigma_p\coloneqq\iota_p\circ \sigma$,
which determines a place of $\K$ above $p$.
Conversely, if $w\in S_p^\K$,
we let $I_w$ denote the set of $\sigma\in I_\K$
such that $\sigma_p$ induces $w$.
If $L$ is a non-archimedean local field,
we normalize the Artin reciprocity map $\Art_L$
so that a uniformizer is sent to the geometric Frobenius.



We let $\A=\A_\F$ and $\A_\K$ be the ring of adeles
and let $|\cdot|_\F$ and $|\cdot|_\K$ denote the norm on which.
If $\kappa=\sum_{\sigma\in \Sigma} 
a_\sigma\sigma+b_\sigma\sigma c\in \Z[\I]$,
for $\alpha_\infty=(\alpha_\sigma)_{\sigma\in\Sigma}
\in \A_{\K,\infty}^\times$
and $\alpha_p=(\alpha_w,\alpha_{\bw})_{w\in\Sigma_p}\in 
\prod_{w\mid p}\K_w^\times$ we define 
\[
    \alpha_\infty^\kappa=
    \prod_{\sigma\in \Sigma} 
    (\alpha_\sigma)^{a_\sigma}(\bar{\alpha}_\sigma)^{b_\sigma}\in \C^\times,\quad
    \alpha_p^\kappa=
    \prod_{w\in \Sigma_p}
    \prod_{\sigma\in I_w}
    \sigma_p(\alpha_w)^{a_\sigma}\sigma_p(\alpha_{\bw})^{b_\sigma}
    \in \C_p^\times.
\]
where we identity $\sigma\in\I$ 
with embeddings into $\C$ and $\C_p^\times$ 
by composition with the fixed choice of embeddings 
$\iota_\infty$ and $\iota_p$.
We will sometimes
let $\Sigma$ denote the formal sum $\sum_{\sigma\in\Sigma}\sigma$
and similarly for $\Sigma^c$


We say a Hecke character $\chi$ of $\K^\times$ is algebraic of
the infinity type $\kappa$ if
$\chi_\infty(\alpha)=\alpha^\kappa$.
When this is the case we define the $p$-adic avatar 
$\hat{\chi}\colon \A_\K^\times\to \bar{\Z}_p^\times$  of $\chi$ by
$\hat{\chi}(\alpha)=
\chi(\alpha)\alpha_\infty^{-\kappa}\cdot \alpha_p^{\kappa}$.
 

When $R$ is an $\F$-algebra and 
$m=(m_{ij})\in \text{M}_{r,s}(\K\otimes_\F R)$,
we denote by 
$m^\intercal=(m_{ji}), 
m^c=(m^c_{ij})$, and
$m^*=(m^c_{ji})$
respectively the transpose, conjugate, and conjugate-transpose of $m$.




\section{Modular forms on definite unitary groups}

Let $G$ be the definite unitary group over $\F$
such that for any $\F$-algebra $R$
\begin{equation*}
    G(R)=\{g\in \GL_{2}(\K\otimes_\F R) \mid gg^*=\id_n\}.
\end{equation*}
In this section we recall the definition of algebraic modular forms on $G$
and the big Hecke algebra acting on which. We also recall the properties
of the Galois pseudo-representation of $\Gal_\K$ to the big Hecke algebra.


\subsection{Algebraic modular forms}

Let $B=TN\subset \GL_2$ be the subgroup of
upper triangular matrices and its Levi decomposition.
Following \cite[Def 2.3]{ger},
we identify $X^*(T)$ with $\Z^2$
and $k=(k_1,k_2)\in X^*(T)$
is said to be dominant if $k_1\geq k_2$.
Then $\xi_k(R)\coloneqq \Sym^{k_1-k_2}R^2\otimes\det^{k_2}$
defines an algebraic representation of $\GL_2$
of highest weight $k$.


Suppose $v\in\finite$ and $v=w\bw$ splits in $\K$,
then $\GL_n(\F_v\otimes_\F\K)
\cong \GL_n(\K_w)\times\GL_n(\K_{\bw})$.
Let $\iota_w$ denote the projection to the component
at $w$, then $\iota_w\colon G(\F_v)\to \GL_n(\K_w)$
is an isomorphism and 
$\iota_w(g_v)=\iota_{\bw}(g_v)^{-\intercal}$
if we identify $\K_w=\F_v=\K_\bw$.
Fix a choice of $w\mid v$ for each $v\in\finite$ that splits,
we will identify $\GL_2(\K_w)$ with $G(\F_v)$ via $\iota_w$,
so $B(\K_w), N(\K_w)$ and $T(\K_w)$ are viewed as 
subgroups of $G(\F_v)$.
In particular, pick $w\in \Sigma_p$ when $v\in S_p$, then we view
$K_p\coloneqq\prod_{w\in \Sigma_p}\GL_2(\oo_w)$
as an open compact subgroup of $\prod_{v\mid p}G(\F_v)$.

Fix a finite extension $E$ over $\Qp$
that contains $\iota_p(\sigma(\K))$
for all $\sigma\in I_\K$.
Let $\oo=\oo_E$ be the ring of integers in $E$ and
$\varpi\in\oo$ be a uniformizer.
When $\wt{k}=(k_\sigma)\in (\Z^2)^{\Sigma}$
is dominant in the sense that
$k_\sigma=(k_{\sigma,1},k_{\sigma,2})$
is dominant for each $\sigma\in \Sigma$,
let $\xi_{\wt{k}}$ be
the algebraic $K_p$-representation over $\oo$ given by
\begin{equation*}
	\xi_{\wt{k}}=\bigotimes_{\sigma\in \Sigma}
	\xi_{k_\sigma},\quad
	\xi_{\wt{k}}(g)=
	\otimes_{w\in \Sigma_p}
	\otimes_{\sigma\in I_w}\xi_{k_\sigma}(g_w)\,
	\text{ for } g=(g_w)\in K_p.
\end{equation*}

\begin{defn}\label{def:algform}
Let $A$ be an $\oo$-module,
$\wt{k}$ be a dominant weight,
and $U\subset G(\A_f^p)\times K_p$ be an open compact subgroup.
We define the space of algebraic modular forms
of weight $\wt{k}$, coefficients in $A$,
and level $U$ by
\begin{equation*}
S_{\wt{k}}(U,A)=
\left\{ f: G(\F)\backslash G(\A_f)/
\rightarrow A\otimes_{\oo}\xi_{\wt{k}}(\oo)
\mid f(gu)=\xi_{\wt{k}}(u_p)^{-1}\cdot f(g), u\in U\right\}.
\end{equation*}
\end{defn}

Let $U^p=\prod_{v\in\finite\setminus{S_p}}U_v$
be an open compact subgroup of $G(\A_f^p)$.
For integers $c\geq b\geq 0$ and $c>0$ we define the open subgroup
$\Iw(w^{b,c})=\{k\in \GL_2(\oo_w)\mid 
k \bmod \varpi_w^c \in B(\oo/\varpi_w^c)
\text{ and } k \bmod \varpi_w^b \in N(\oo/\varpi_w^b)\}$
for each $w\in \Sigma_p$ and put
$\Iw(p^{b,c})=\prod_{w\in \Sigma_p}\Iw(w^{b,c})\subset K_p$.
In \cite{ger}, the Hecke operators acting on 
$S_{\wt{k}}(U^p\Iw(p^{b,c}),A)$
are defined by the following double cosets operators.

\begin{itemize}
\item If $v=w\bw$ splits in $\K$ and $U_v=\GL_2(\oo_w)$, we define
\begin{equation*}
	T_w^{(1)}=
	\left[
	\GL_2(\oo_w)
	\begin{pmatrix}
		\varpi_w&\\&1
	\end{pmatrix}
	\GL_2(\oo_w)
	\right],\quad
	T_w^{(2)}=
	\left[
	\GL_2(\oo_w)
	\begin{pmatrix}
		\varpi_w&\\&\varpi_w
	\end{pmatrix}
	\GL_2(\oo_w)
	\right],
\end{equation*}
We can similarly define $T^{(j)}_\bw$, but then 
$T_{\bw}^{(1)}=(T_{w}^{(2)})^{-1}T_w^{(1)}$ and
$T_{\bw}^{(2)}=(T_{w}^{(2)})^{-1}$.
\item
If $w\in \Sigma_p$, for 
$\alpha_w^{(1)}=\smat{\varpi_w&\\&1}$ and
$\alpha_w^{(2)}=\smat{\varpi_w&\\&\varpi_w}\in T(\K_w)$
and $u\in T(\oo_w)$
we define
\begin{equation*}
	U_{\wt{k},w}^{(j)}=
	(w_0\wt{k})^{-1}(\alpha_{w}^{(j)})\cdot
	[\Iw(w^{b,c})\alpha_w^{(j)}\Iw(w^{b,c})]\quad
    \langle u\rangle_{\wt{k}}=
	(w_0\wt{k})^{-1}(u)\cdot
	[\Iw(w^{b,c})u\Iw(w^{b,c})].
\end{equation*}
Here $w_0\wt{k}\coloneqq(k_{2,\sigma},k_{1,\sigma})_\sigma$ defines a character
of $T(\K_w)$ by the same recipe defining $\xi_{\wt{k}}$.
\end{itemize}

The Hecke operators are compatible with the inclusion
$S_{\wt{k}}(U^p\Iw(p^{b,c}),A)\subset S_{\wt{k}}(U^p\Iw(p^{b',c'}),A)$
when $b'\geq b$ and $c'\geq c$. 
We define the ordinary projector 
$e\coloneqq\lim_{n\to \infty}\prod_{w\in\Sigma_p}
(U_{\wt{k},w}^{(1)}U_{\wt{k},w}^{(2)})^{n!}$
when $A$ is equal to $\oo$ or $E/\oo$.
Omit the subscript $\wt{k}$
when $\xi_{\wt{k}}$ is the trivial representation, we then define
\[
    S^{\ord}(U^p,E/\oo)=\varinjlim_{b\geq 1}\,eS(U^p\Iw(p^{b,b}),E/\oo)
\]
and let $\TT^{\ord}(U^p)$ be the Hecke algebra acting on which 
generated by the Hecke operators defined above.


Put $\Iw'(p^{b,c})=\prod_{w\neq w_0}\Iw(w^{b,c})$ 
and let $U_0\subset \GL_2(\oo_{w_0})$ be an open subgroup.
We can also consider the projector
$e'\coloneqq\lim_{n\to \infty}\prod_{w\neq w_0}
(U_{\wt{k},w}^{(1)}U_{\wt{k},w}^{(2)})^{n!}$
on $S_{\wt{k}}(U^pU_0\Iw'(p),A)$ and define
\[
    S(U^p,E/\oo)=
    \varinjlim_{U_0}\varinjlim_{b\geq 1}\,e'S(U^pU_0\Iw'(p^{b,b}),E/\oo),
\]
where $U_0$ goes through open compact subgroups of $\GL_2(\oo_{w_0})$.
Let $\TT(U^p)$ denote the Hecke algebra
generated by the Hecke operators except for $U_{\wt{k},w_0}^{(1)}$.

For $b\geq 0$ let
$T(w^b)=\{u\in T(\oo_w)\mid u\equiv \id \mod\varpi_w^b\}$
for each $w\in \Sigma_p$ and put $T(p^b)=\prod_{w\in\Sigma_p}T(w^b)$.
Define $\Lambda^+\coloneqq\Lambda\llbracket T(p^0)\rrbracket$.
Then the diamond operators $\langle u\rangle$ 
define structures of $\Lambda^+$-algebras on the Hecke algebras
and the clear inclusion $S^{\ord}(U^p,E/\oo)\subset S(U^p,E/\oo)$
induces an $\Lambda^+$-algebra homomorphism
\[
\TT(U^p)\to \TT^{\ord}(U^p).
\]



\subsection{Completed homology and cohomology}

Let $N^\vee\coloneqq \Hom_\oo(N,E/\oo)$ denote 
the Pontryagin dual of a locally compact $\oo$-module $N$. We then define
the (ordinary-)completed homology of tame level $U^p$ by
\[
    M(U^p)=S(U^p,E/\oo)^\vee\text{ and }
    M^{\ord}(U^p)=S^{\ord}(U^p,E/\oo)^\vee.
\]
Note that $S^{\ord}(U^p,E/\oo)\subset S(U^p,E/\oo)$ 
induces a Hecke equivariant homomorphism
$M(U^p)\to M^{\ord}(U^p)$.
Furthermore, if we define $S(U^p)=\Hom_\oo(E/\oo, S(U^p,E/\oo))$,
it can be verified that 
\begin{equation*}
	M(U^p)\cong \Hom_\oo(S(U^p),\oo),\qquad
	S(U^p)\cong \Hom_\oo^{\cts}(M(U^p),\oo)
\end{equation*}
and similar statement holds for 
$S^{\ord}(U^p)=\Hom_\oo(E/\oo, S^{\ord}(U^p,E/\oo))$ as well.
Therefore it makes sense to call $S(U^p)$ and $S^{\ord}(U^p)$
the (ordinary-)completed cohomology of tame level $U^p$.
We also note that the space $S(U^p)_E\coloneqq S(U^p)\otimes_\oo E$
is equipped with a structure of $E$-Banach space
for which $S(U^p)$ is open.

Let $T_p'=\prod_{w\neq w_0}T(\K_w)$ and identify $\K_{w_0}=\Qp$,
we can define a $\GL_2(\Qp)\times T'_p$-representation on
$S(U^p,E/\oo)$ by right translation.
Furthermore, let $T'(p^b)=\prod_{w\neq w_0}T(w^b)$, 
then the restriction of the representation to $\GL_2(\Zp)\times T'(p^0)$
is an injective object following the same argument in \cite[\S 3]{pan}.
Write $\wt{k}'=(k_\sigma)_{w\neq w_0}$ and let 
$\pi_{\wt{k}}=\xi_{k_{\sigma_0}}\otimes \wt{k}'$ be
the $\GL_2(\Zp)\times T'(p^0)$ representation
defined by the same recipe defining $\xi_{\wt{k}}$.
The injectivity then yields the following proposition.

\begin{prop}\label{prop:density}
The subspace $S(U^p)_E^{\textnormal{alg}}$ 
of algebraic vectors, defined as
\[
\Image\left(ev\colon
\bigoplus_{\wt{k}}
\Hom_{E[\GL_2(\Zp)\times T'(p^0)]}(\pi_{\wt{k}}^*(\oo), S(U^p)_E)
\otimes_E \pi_{\wt{k}}^*(E)\rightarrow S(U^p)_E\right)
\]
where $\wt{k}$ ranges through all dominant weights,
is dense in the $E$-Banach space $S(U^p)_E$.
\end{prop}
We remark that a generalization of the control theorem
for ordinary modular forms implies that the space
$\Hom_{\oo[\GL_2(\Zp)\times T'(p^0)]}(\pi_{\wt{k}}^*(\oo), S(U^p)_E)$
is Hecke-equivariantly isomorphic to
$e'S_{\wt{k}}(U^p\GL_2(\Qp)\Iw'(p^{0,1}),E)$.
Therefore the proposition is a modified version of
the density of crystalline points from \cite{emeI} and \cite{pan}.  





\subsection{Galois representations}

Thanks to the work of \cite{ger},
we can associated a Galois representation $r_\pi$ of $\Gal_\K$
to an irreducible $G(\A_f)$-representation $\pi$
inside the space of algebraic modular form.
Moreover, if $\pi\cap S_{\wt{k}}(U^p\GL_2(\K_{w_0})\Iw(p^{0,1}),E)\neq 0$,
then $r_\pi$ is unramified at the places where 
$U_v$ is hyperspecial and ordinary at $w\in \Sigma_p\setminus\{w_0\}$.
As the Hecke operators all acts semi-simply,
the Hecke algebra $\TT(U^p)$ is reduced 
and we can glue the Galois representations from 
finite levels into a big Galois pseudo-representation
\[
    T(U^p)\colon \Gal_\K\to \TT(U^p)
\]
with the properties listed below.
\begin{enumerate}
    \item Let $S\supset S_p$ be a finite set of places $v\in\finite$
    such that $U_v$ is hyperspecial if $v\notin S$.
    Denote the maximal Galois extension
    that is unramified outside $S$ by $\K^S/\K$, then
    $T(U^p)$ factors through $Gal(\K^S/\K)$.
    \item Let $v\notin S$ be a split place
    and $\Fr_w$ be the geometric Frobenius at $w\mid v$, then
    \[
        T(U^p)(\Fr_w)=T_w^{(1)},\quad
        (\epsilon \det T(U^p))(\Fr_w)=T_w^{(2)}.
    \]
    \item Let $D_{w}$ be the decomposition group at
    $w\in S_p\setminus\{w_0\}$, then 
    $T(U^p)\vert_{D_{w}}=\Psi_{w,1}+\epsilon^{-1}\Psi_{w,2}$,
    where $\Psi_{w,1},\Psi_{w,2}\colon D_{w}\to \TT(U^p)$
    are characters and satisfy
    \begin{equation*}
	\begin{aligned}
		\Psi_{w,1}\circ \Art_{w}(\varpi_{w})&=
		U_{w}^{(1)} &
		\Psi_{w,1}\circ \Art_{w}(x)&=
		\langle 
		(\begin{smallmatrix}
			x&\\&1
		\end{smallmatrix})
		\rangle\, \text{ for }x\in \oo_{w}^{\times}\\
		\Psi_{w,2}\circ \Art_{w}(\varpi_{w})&=
		U_{w}^{(2)}/
		U_{w}^{(1)}&
		\Psi_{w,2}\circ \Art_{w}(x)&=
		\langle 
		(\begin{smallmatrix}
			1&\\&x
		\end{smallmatrix})
		\rangle\, \text{ for }x\in \oo_{w}^{\times}
	\end{aligned}
	\end{equation*}
    Moreover, $T(U^p)$ is $\Psi_{w,1}$-ordinary 
    in the sense that for all 
    $\sigma, \tau\in D_{w}$ and $\eta\in\Gal_\K$ we have
    \[
        T(U^p)(\sigma\tau\eta)
        -\Psi_{w,1}(\sigma)T(U^p)(\tau\eta)
        -\Psi_{w,2}(\tau)T(U^p)(\sigma\eta)
        +\Psi_{w,1}(\sigma)\Psi_{w',2}(\tau)T(U^p)(\eta)=0.
    \]
    \item Let $D_{w_0}$ be the decomposition group at $w_0$, then
    \[
        (\epsilon\det T(U^p))\circ \Art_{w_0}
        (\varpi_{w_0})=U_{w_0}^{(2)}\quad
        (\epsilon\det T(U^p))\circ \Art_{w_0}(x)=
        \langle 
		(\begin{smallmatrix}
			x&\\&x
		\end{smallmatrix})
		\rangle\, \text{ for }x\in \oo_{w_0}^{\times}
    \]
\end{enumerate}
We remark that a pseudo-representation
$T^{\ord}(U^p)\colon \Gal_\K\to \TT^{\ord}(U^p)$ 
with the same properties also exists and is in fact 
isomorphic to the pushforward of $T(U^p)$ via the homomorphism
$\TT(U^p)\to\TT^{\ord}(U^p)$.
Moreover, there exists characters
$\Psi_{w_0,1},\Psi_{w_0,2}\colon D_{w_0}\to \TT^{\ord}(U^p)$
such that the third property above holds for $w_0$.


\subsection{Hida family}

Let $\I$ be a local complete Noetherian ring
that is finite over $\Lambda^+$ and flat over $\oo$.
We define the space of $\I$-adic Hida families of tame level $U^p$ by
\begin{equation*}
    S^{\ord}(U^p,\I)\coloneqq 
    \{\euF\in S^{\ord}(U^p)\widehat{\otimes}_\oo\I\mid 
    (\langle u\rangle\otimes\id)\euF=
    (\id\otimes\langle u\rangle)\euF\, \text{ for }
    u\in \Lambda^+\}.
\end{equation*}
Here $(\langle u\rangle\otimes\id)$ denotes
the action through the Hecke algebra and
$(\id\otimes\langle u\rangle)$ denotes the action through $\I$.

Define $M^{\ord}(U^p,\I)=M^{\ord}(U^p)\otimes_{\Lambda^+}\I$.
From the duality between the completed homology and cohomology,
it is easy to check that we have a Hecke-equivariant isomorphism
$S^{\ord}(U^p,\I)\cong\Hom_\I(M^{\ord}(U^p,\I), \I)$.
Moreover, suppose $U^p$ is sufficiently small in the sense that
$G(\F)\cap g_i(U^pK_p)g_i^{-1}=\{1\}$ for all $i\in I$ in 
a decomposition $G(\A_f)=\bigsqcup_{i\in I} G(\F)t_i (U^pK_p)$,
then we can show that $M^{\ord}(U^p,\I)$ and $S^{\ord}(U^p,\I)$
are finite free over $\I$ 
and there exists a Hecke-equivariant isomorphism
\[
    M^{\ord}(U^p,\I)\cong \Hom_\I(S^{\ord}(U^p,\I),\I).
\]


\section{Results from p-adic local Langlands}

Fix a maximal ideal $\fm\subset \TT(U^p,\oo)$.
Enlarge $E$ if necessary, we assume that 
$\TT(U^p,\oo)/\fm$ coincides with the residue field $\fF$ of $\oo$.
Write $\TT_\fm=\TT(U^p)_\fm$ and let
$T_\fm\colon\Gal_\K\to \TT_\fm$ denote the localization of 
the pseudo-representation $T(U^p)$.
If $\omega$ denotes the Teichmuller character,
we say $T_{\fm}$ is residually reducible and locally generic at $w_0$
if there exists characters
$\bar{\delta}_1, \bar{\delta}_2\colon \Gal_{\K}\to \fF^\times$
such that 
\begin{equation}\tag{red.gen}\label{cond:red_gen}
	T_\fm\equiv \bar{\delta}_1+\bar{\delta}_2
	\mod \fm,\quad
    \text{ and }\quad
	\bar{\delta}_1\bar{\delta}_2^{-1} \vert_{D_{w_0}}
	\neq \id,\omega^{\pm}.
\end{equation}

From now on we identify $D_{w_0}$ with $\Gp$, the absolute Galois
group of $\Qp$. We shall apply techniques from $p$-adic local Langlands
to study $S(U^p,E/\oo)_\fm$ as a $\GL_2(\Qp)\times T'_p$-representation.
For a technical reason, it is necessary to make a twist as follows
so that the $\GL_2(\Qp)$-action has a central character.

Fix a continuous character $\zeta\colon \Gp\to \oo^\times$
such that $\zeta\equiv \omega\delta_1\delta_2\vert_{\Gp}\bmod \varpi$.
Then $\xi_\fm\coloneqq\zeta(\epsilon \det T_\fm)^{-1}\vert_{\Gp}$ 
is trivial modulo $\fm$ and there exists a well-defined 
square-root character $\xi_\fm^{1/2} \colon\Gp\to \TT_\fm$
since $p$ is odd.
Identify $\Qp^\times$ with the central torus of $\GL_2(\Qp)$.
We let $S(U^p,E/\oo)_\fm$ be a $Q$-representation for
\[
    Q\coloneqq \GL_2(\Qp)\times \Qp^\times \times T'_p,
\]
on which $\Qp^\times\times T_p'$ acts by the usual translation,
but $\GL_2(\Qp)$ acts with the translation
twisted by $\xi_\fm^{1/2}\circ\Art$ 
and has a central character $\epsilon\zeta$.
Take the localization $S^{\ord}(U^p,E/\oo)_\fm$ via 
$\TT(U^p)\to \TT^{\ord}(U^p)$.
We make a similar twist of the translation by $T(\Qp)\times T'_p$
on $S^{\ord}(U^p,E/\oo)_\fm$ and define an $M$-representation for 
\[
    M\coloneqq T(\Qp)\times \Qp^\times \times T'_p.
\]


\subsection{Local-global compatibility}

We very briefly review P\v ask\={u}nas' theory of blocks 
of a $p$-adic analytic group $H$ and refer the details to \cite{pask}.
Let $\Mod^{\sm}_H(\oo)$ be the category 
of $\oo[H]$-modules $V$ such that 
\begin{itemize}
    \item Each $v\in V$ is fixed by some open compact subgroup of $H$.
    \item Each $v\in V$ is annihilated by $\varpi^h$ for some $h\geq 0$.
\end{itemize}
We call $V\in \Mod^{\sm}_{H}(\oo)$ admissible if
$V^K[\varpi^h]$ is $\oo$-finite 
for any $h$ and any compact open subgroup $K$ of $H$.

Let $\Mod^{\lfin}_H(\oo)$ be the full subcategory of
$V\in \Mod^{\sm}_{H}(\oo)$  such that each $v\in V$ generates 
a $\oo[H]$-submodule of finite length.
A general formalism ensures that 
an irreducible $\pi\in \Mod^{\lfin}_H(\oo)$ admits
an injective envelope $\tilde{J}_\pi$, or dually, 
the Pontryagin dual $\pi^\vee$ 
admits a projective envelope $\tilde{P}_{\pi^\vee}$.


We say two irreducible representations $\pi,\pi'$ 
are equivalent if there exists a sequence 
$\pi=\pi_1,\cdots, \pi_n=\pi'$ of irreducible representations
such that all successive $\pi_i,\pi_{i+1}$ have nontrival extensions.
A block $\B$ is an equivalence class of irreducible representations.
Given a block $\B$, we let $\fC^\B_H(\oo)$ denote the subcategory
of all $S^\vee$, where $S\in \Mod^{\lfin}_H(\oo)$ and
all the irreducible subquotients of $S$ belong to $\B$. We have
\begin{enumerate}[label=(P\arabic*)]
\item $\Hom_H(\tilde{P}_\pi,\tilde{P}_{\pi'})=0$ if 
$\pi,\pi'$ are not equivalent and 
$\tilde{P}_\pi\in \fC^\B_H(\oo)$ if $\pi\in\B$.
\item Let $\tilde{P}_\B=\oplus_{\pi\in\B}\tilde{P}_\pi$
and $\tilde{E}_\B=\End_H(\tilde{P}_\B)$,
then the evaluation map 
$\Hom_H(\tilde{P}_\B,N)\otimes_{\tilde{E}_\B}\tilde{P}_\B\to N$
is an isomorphism if $N\in \fC^\B_H(\oo)$.
In fact $\fm(N)\coloneqq \Hom_H(\tilde{P}_\B,N)$ defines an
equivalence between $\fC^\B_H(\oo)$ and the category of pseudo-compact
$\tilde{E}_\B$-modules.
\label{P2}
\end{enumerate}


We now apply the results from \cite[\S 7]{pask}
to our settings.
Let $\chi_i\colon \Gp\to \fF^\times$ 
denote the character $\omega\bar{\delta}_i$.
If we identify $\chi_i$ with the 
$\Q^\times$-characters obtained through composition
with $\Art$,
then under the generic assumption in \eqref{cond:red_gen} the set 
$\B=\{\pi_1,\pi_2\}$ is a block of 
irreducible $\GL_2(\Qp)$-representations for
\[
\pi_1\coloneqq 
\Ind_{B}^{\GL_2}\chi_1\otimes\chi_2\omega^{-1} \text{ and }
\pi_2\coloneqq
\Ind_{B}^{\GL_2}\chi_2\otimes\chi_1\omega^{-1} .
\]
For the same technical reason, we actually use the subcategories
$\Mod^{\lfin}_{\GL_2(\Qp),\epsilon\zeta}(\oo)$ and 
$\Mod^{\lfin}_{T(\Qp),\epsilon\zeta}(\oo)$
of representations on which 
the central $\Qp^\times$ acts by $\epsilon\zeta$,
and we identify the latter category with 
$\Mod^{\lfin}_{\Qp^\times}(\oo)$ via 
$\Qp^\times\cong \smat{1&\\&*}\subset T(\Qp)$.
Let $\tilde{P}_j$ denote the projective envelope of $\pi^\vee$
and let $R^{\epsilon\zeta}$ represent
the universal pseudo-deformation of 
$\chi_1+\chi_2$ with fixed determinant $\epsilon\zeta$,
then we have the following results.
\begin{enumerate}
    \item $R^{\epsilon\zeta}$ is a regular local ring 
    of relative dimension 3 over $\oo$ and the reducibility ideal of which 
    is generated by a regular element $\xx$,
    we write $R^{\epsilon\zeta}_\red\coloneqq R^{\epsilon\zeta}/(\xx)$.
    \item $\End_{\GL_2(\Qp)}(\tilde{P}_j)$ 
    is isomorphic to $R^{\epsilon\zeta}$ as a ring and
    $\Hom_{\GL_2(\Qp)}(\tilde{P}_i,\tilde{P}_j)$ is free of rank one over
    $R^{\epsilon\zeta}$ if $i\neq j$. Moreover, let 
    $\Phi_{ij}\in\Hom_{\GL_2(\Qp)}(\tilde{P}_j, \tilde{P}_i)$
    be a generator, then $\Phi_{ji}\circ\Phi_{ij}$
    generates the reducibility ideal under the isomoprhism
    $\End_{\GL_2(\Qp)}(\tilde{P}_j)\cong R^{\epsilon\zeta}$.
    \item Let $\tilde{P}_\B=
    \tilde{P}_2\oplus \tilde{P}_2$, then $R^{\epsilon\zeta}$ is
    isomorphic to the center of 
    $\tilde{E}_\B\coloneqq
    \End_{\GL_2(\Qp)}(\tilde{P}_\B)$.
    \item Let $\Ord\colon \Mod_{\GL_2(\Qp)}(\oo)\to 
    \Mod_{T(\Qp)}(\oo)$ denote Emerton's functor
    \cite{emeI} with respect to $B$.
    Then $\Ord(\tilde{P}_{j})
    =\tilde{P}_{\chi_j^\vee}$ and 
    $\Ord\colon 
    \End_{\GL_2(\Qp)}(\tilde{P}_j)\to 
    \End_{T(\Qp)}(\tilde{P}_{\chi_j^\vee})$
    is surjective with the reducibility ideal 
    as the kernel. Therefore we can identify 
    $\End_{T(\Qp)}(\tilde{P}_{\chi_j^\vee})$
    with $R^{\epsilon\zeta}_\red$.
\end{enumerate}


On the other hand, consider the $\TT_\fm^\times$-valued character 
of $\Qp^\times\times T'_p$ defined by 
$(\epsilon\det T(U^p))$ at $w_0$
and by $\Psi_{w,1}\otimes\Psi_{w,2}$ at $w\neq w_0$
after composition with the Artin map.
Let $\upsilon\colon\Qp^\times\times T'_p\to\fF^\times$
denote the residual character.
Then $\tilde{P}=\tilde{P}_{\upsilon^\vee}$ is free of rank one over
$\tilde{E}=\End_{\Qp^\times\times T'_p}(\tilde{P})$,
which is isomorphic to the universal deformation ring of $\upsilon$.
Now $\tilde{P}_{i,\fm}\coloneqq\tilde{P}_i\otimes \tilde{P}$
is the projective envelope of $(\pi\otimes\upsilon)^\vee$.
Put $\tilde{P}_{\B,\fm}=\tilde{P}_\B\otimes \tilde{P}$ and 
$\tilde{E}_\fm=\End_{Q}(\tilde{P}_{\B,\fm})\cong 
\tilde{E}_\B\otimes \tilde{E}$,
then $R_\fm\coloneqq R^{\epsilon\zeta}\otimes \tilde{E}$
acts on $\Hom_{Q}(\tilde{P}_\fm, M(U^p)_\fm)$ in two ways:



\begin{itemize}
\item $R_\fm$ acts on $\tilde{P}_\fm$ as the center of 
$\tilde{E}_\fm=\End_{Q}(\tilde{P}_{\B,\fm})$.
\item $R_\fm$ acts on $M(U^p)_\fm$ through the homomorphism
$R_\fm\coloneqq R^{\epsilon\zeta}\otimes \tilde{E}\to \TT_\fm$, which is
induced respectively from $\xi_\fm^{1/2}\epsilon {T_\fm}\mid_{\Gp}$
and $\upsilon$ by the universal property.

\end{itemize}
We remark that since we don't know yet whether $\TT_\fm$ is Noetherian,
the homomorphism from $R^{\epsilon\zeta}$ comes from 
first applying the universal property to the Hecke algebras at finite
levels then take the inverse limit.

The following proposition is a modified version of
the local-global compatibility from \cite{pan}.

\begin{prop}\label{prop:compatibility}
    All the irreducible subquotients of $S(U^p,E/\oo)_\fm$
    are isomorphic to either $\pi_1\otimes\upsilon$ or 
    $\pi_2\otimes\upsilon$. 
    As a consequence the two $R_\fm$-actions on 
    $\Hom_{Q}(\tilde{P}_\fm, M(U^p)_\fm)$ coincide.
\end{prop}

\subsection{Reducible part of the completed homology}


We can deduce from the local-global compatibility that 
there is a projective resolution
\begin{equation*}
0\to \tilde{P}_{\B,\fm}^{\oplus r}\to 
\tilde{P}_{\B,\fm}^{\oplus r}\to 
M(U^p)_{\fm}\to 0
\end{equation*}
The first morphism above can be represented by a matrix
$A=\smat{A_{11} & A_{12}\Phi_{12}\\A_{21}\Phi_{21} & A_{22}}$
for $A_{ij}\in M_r(R_{\fm})$.
Now apply the right exact $\Ord$ to the $\GL_2(\Qp)$-part of the sequence.
We have $\Ord(\tilde{P}_{i,\fm})=\tilde{P}_{\chi_i^\vee,\fm}\coloneqq
\tilde{P}_{\chi_i^\vee}\otimes \tilde{P}$ and it can be checked that
$\Ord(M(U^p)_\fm)=M^{\ord}(U^p)_\fm$.
Thus we obtain the right exact sequence
\begin{equation}\label{eq:presentord}
	\tilde{P}_{\chi_1^\vee,\fm}^{\oplus r}\oplus 
	\tilde{P}_{\chi_2^\vee,\fm}^{\oplus r}
	\xrightarrow{\overline{A}_{11}\oplus\overline{A}_{22}}
	\tilde{P}_{\chi_1^\vee,\fm}^{\oplus r}\oplus 
	\tilde{P}_{\chi_2^\vee,\fm}^{\oplus r}
	\to M^{\ord}(U^p)_\fm\to 0
\end{equation}
where $\overline{A}_{ii}$ denote the reduction in 
$R^\red_\fm\coloneqq R^{\epsilon\zeta}_\red\otimes\tilde{E}$.
This shows that $M^{\ord}(U^p)_\fm$ is the direct sum of 
\[
    M^{\ord}(U^p)_{\fm,i}\coloneqq 
    \Hom_{M}(\tilde{P}_{\chi_i^\vee,\fm}, M^{\ord}(U^p)_\fm)
    \otimes_{R_\fm^\red}
    \tilde{P}_{\chi_i^\vee,\fm}\cong 
    \Hom_{M}(\tilde{P}_{\chi_i^\vee,\fm}, M^{\ord}(U^p)_\fm)
\]
where the isomorphism follows
since $\tilde{P}_{\chi_i^\vee,\fm}$ is free of rank one
over $R_\fm^\red\cong \End_M(\tilde{P}_{\chi_i^\vee,\fm})$.
In fact, this implies that there are exactly two maximal ideals 
$\fm_1,\fm_2$ in $\TT^{\ord}(U^p)_{\fm}$,
which are characterized by $\Psi_{2,w_0}\bmod \fm_i=\chi_i$,
and that $M^{\ord}(U^p)_{\fm,i}=M^{\ord}(U^p)_{\fm_i}$.
Another conclusion we can draw,
using that $M^{\ord}(U^p)$ is finite over $\Lambda^+$
and $\dim\Lambda^+<\dim R_\fm^\red$, is that both
$\overline{A}_{11}$ and $\overline{A}_{22}$ are injective.

This last fact has two importance consequences for us.
First, apply the snake lemma to
\begin{equation*}
    \begin{tikzcd}
    0 \arrow[r] & 
    \tilde{P}_{\B,\fm}^{\oplus r} 
	\arrow[d,"\xx"] \arrow[r,"A"] & 
	\tilde{P}_{\B,\fm}^{\oplus r} 
	\arrow[d,"\xx"] \arrow[r] & 
	M(U^p)_{\fm}
    \arrow[d,"\xx"]  \arrow[r] & 0 \\ 
    0 \arrow[r] & 
    \tilde{P}_{\B,\fm}^{\oplus r}
	\arrow[r,"A"] & 
    \tilde{P}_{\B,\fm}^{\oplus r}
	\arrow[r] &
    M(U^p)_{\fm} 
    \arrow[r] & 0 
    \end{tikzcd}
\end{equation*}
Write $M^\red\coloneqq M/\xx M$ if $M$ is an $R_\fm$-module.
We can show that $\xx$ is injective on $\tilde{P}_{\B,\fm}$.
Thus for both $j=1,2$, the space
$\Hom_{Q}(\tilde{P}_{j,\fm}, M(U^p)_{\fm}[\xx])$
is isomorphic to the kernel of the homomorphism
\begin{equation*}
	A'=\smat{\bar{A}_{ii}& \bar{A}_{ij}\Phi_{ij}\\& \bar{A}_{jj}}\colon 
	\Hom_{Q'}(\tilde{P}_{j,\fm},\tilde{P}_{\B,\fm}^{\oplus r})^{\red}\to
	\Hom_{Q'}(\tilde{P}_{j,\fm},\tilde{P}_{\B,\fm}^{\oplus r})^{\red}
\end{equation*}
which we know to be injective.
Therefore $\Hom_{Q}(\tilde{P}_{j,\fm}, M(U^p)_{\fm}[\xx])=0$
for $j=1,2$.

Second, we can observe that the same homomorphism 
above also appears in the commutative diagram 
\begin{equation*}
    \begin{tikzcd}
	    0 \arrow[r]& 
	    \Hom_{Q}(\tilde{P}_{2,\fm},\tilde{P}_{1,\fm}^{\oplus r})^{\red}
	    \arrow[r,"\bar{A}_{11}"] \arrow[d]&
	    \Hom_{Q}(\tilde{P}_{2,\fm}, \tilde{P}_{1,\fm}^{\oplus r})^{\red}
	    \arrow[d] &&\\
	    0\arrow[r] & 
	    \Hom_{Q}(\tilde{P}_{2,\fm},\tilde{P}_{\B,\fm}^{\oplus r})^{\red}
	    \arrow[r,"A'"] \arrow[d,"\Ord"] &
	    \Hom_{Q}(\tilde{P}_{2,\fm},\tilde{P}_{\B,\fm}^{\oplus r})^{\red}
	    \arrow[d,"\Ord"] \arrow[r]&
	    \Hom_{Q}(\tilde{P}_{2,\fm}, M(U^p)_{\fm})^{\red}\arrow[r]
        \arrow[d,"\Ord"] &0\\
	    0\arrow[r] & 
	    \End_{M}(\tilde{P}_{\chi_2^\vee,\fm})^{r}
	    \arrow[r,"\bar{A}_{22}"] &
	    \End_{M}(\tilde{P}_{\chi_2^\vee,\fm})^{r}
        \arrow[r]&
        \Hom_{M}(\tilde{P}_{\chi_2^\vee,\fm}, M^{\ord}(U^p)_\fm)
        \arrow[r]&0
    \end{tikzcd}
\end{equation*}
where the exactness of the columns follows from 
$\Hom_{Q}(\tilde{P}_{2,\fm}, \tilde{P}_{2,\fm})^{\red}
\cong \End_{M}(\tilde{P}_{\chi_2^\vee,\fm})$.
Since taking $\Ord$ is Hecke-equivariant,
apply the snake lemma and use the presentation \eqref{eq:presentord}
gives the result below.
\begin{prop}\label{cor:fil_by_ord}
	We have the following exact sequences of
	$\TT(U^p,\oo)_\fm\times R_{\fm}^{\red}$-modules
\begin{equation*}
\begin{aligned}
	0\to M^{\ord}(U^p)_{\fm,1}\to
	\Hom_{Q}(\tilde{P}_{2,\fm},M(U^p)_{\fm})^{\red}
    \to
	M^{\ord}(U^p)_{\fm,2}\to0\\
	0\to M^{\ord}(U^p)_{\fm,2}\to
	\Hom_{Q}(\tilde{P}_{1,\fm},M(U^p)_{\fm})^{\red}
    \to
	M^{\ord}(U^p)_{\fm,1}\to0
\end{aligned}
\end{equation*}
\end{prop}


Let $\Lambda\coloneqq\llbracket T(p^1)\rrbracket$,
which is a subalgebra of $\Lambda^+=\llbracket T(p^0)\rrbracket$.
If $U^p$ is sufficiently small, then 
$M^{\ord}(U^p)$ is finite free over $\Lambda$ and therefore
so is $\Hom_{Q}(\tilde{P}_{i,\fm},M(U^p)_{\fm})^\red$ by the proposition above.
Write $\mathbf{m}_i=\Hom_{Q}(\tilde{P}_{i,\fm},M(U^p)_{\fm})$
and suppose $s=\rank_{\Lambda}\mathbf{m}_i^\red$.
By the topological Nakayama lemma there exists a surjection
$\Lambda\llbracket \xx\rrbracket^{\oplus s}\to\mathbf{m}_i$.
Let $N$ denote the kerne and apply the snake lemma to the diagram
\[
\begin{tikzcd}
0\arrow[r]& 
N\arrow[r]\arrow[d,"\xx"]&
\Lambda\llbracket \xx\rrbracket^{\oplus s}\arrow[r]\arrow[d,"\xx"]&
\mathbf{m}_i\arrow[r]\arrow[d,"\xx"] &0\\
0\arrow[r]& 
N\arrow[r]&
\Lambda\llbracket \xx\rrbracket^{\oplus s}\arrow[r]&
\mathbf{m}_i\arrow[r] &0
\end{tikzcd}
\]
shows that $N^\red\cong\fm_i[\xx]$, which is trivial
by the first consequence above.
Therefore $N=0$ by Nakayama's lemma, as a result
$\fm_1,\fm_2$, and
$\Hom_{Q}(\tilde{P}_{\B,\fm}, M(U^p)_\fm)=\fm_1\oplus \fm_2$
are finite free over $\Lambda\llbracket \xx\rrbracket$.


From $\xx\in R_\fm\to \TT_\fm$ we can view $\TT_{\fm}$
as a $\Lambda\llbracket \xx\rrbracket$-algebra.
The local-global compatibility then implies the injectivity of
the $\Lambda\llbracket\xx\rrbracket$-algebra homomorphism
\[
    \TT_\fm\hookrightarrow
    \End_{\Lambda\llbracket\xx\rrbracket}
    (\Hom_{Q}(\tilde{P}_{\B,\fm},M(U^p)_{\fm})).
\]
We can now conclude that the right hand side is finite free
and obtain the corollary below.


\begin{cor}\label{cor:Hecke_ff}
Assume that $U^p$ is sufficiently small,
then $\TT_\fm$ is a finite free 
$\Lambda\llbracket\xx\rrbracket$-algebra.
\end{cor}



\subsection{Fundamental exact sequence}


Let $\fG$ be an abelian profinite group.
The algebra $\I\coloneqq\oo\llbracket\fG\rrbracket$
is local complete Noetherian and flat over $\oo$.
We assume $\fG=W\times \Delta$,
where $\Delta$ is a finite and $W\cong\Zp^n$.

We fix a surjective homomorphism 
$\lambda^{\ord}\colon \TT^{\ord}(U^p,\I)\to\I$
of $\Lambda^+$-algebras and let 
\begin{equation*}
    \lambda\colon 
    \TT(U^p,\oo)\to
    \TT^{\ord}(U^p,\oo)\to\I
\end{equation*}
denote the composition. 
Let $\fm_\I\subset\I$ be the maximal ideal,
$\fm=\lambda^{-1}(\fm_\I)$, and $\fm_\circ=(\lambda^\ord)^{-1}(\fm_\I)$.
We assume the maximal ideal
$\fm$ satisfies \eqref{cond:red_gen},
then we know $\fm_\circ$ is one of the only two maximal 
ideals $\fm_1,\fm_2\subset \TT^{\ord}_\fm\coloneqq\TT^{\ord}(U^p)_\fm$,
say $\fm_\circ=\fm_1$.
We now make the following assumptions on the space of Hida families
$M^{\ord}(U^p,\I)_{\fm_i}\coloneqq
M^{\ord}(U^p)_{\fm_i}\otimes_{\Lambda^+}\I$ for $i=1,2$.
\begin{enumerate}[label=(C\arabic*)]
\item There exists a nonzero $\TT_\fm$-linear map
$\Theta\colon M^{\ord}(U^p,\I)_{\fm_1}\to \I$,
where $\TT_\fm$ acts on $\I$ via $\lambda$.
\label{cond:C1}
\item There exists an ideal $\fq\subset \TT_{\fm}$
containing $\xx$ such that $\fq M^{\ord}(U^p,\I)_{\fm_2}=0$.
\label{cond:C2}
\item There exists $F\in M^{\ord}(U^p,\I)_{\fm_1}$
such that $\Theta(F)\neq0$ and $\fq F=0$.
\label{cond:C3}
\end{enumerate}

To simplify the notation, let 
$M=\Hom_{Q}(\tilde{P}_{1,\fm},M(U^p)_{\fm})\otimes_{\Lambda^+}\I$,
$M_i^{\ord}=M^{\ord}(U^p,\I)_{\fm_i}$ for $i=1,2$, and
define $S_1^{\ord}=\ker(\Theta)\subset M_1^{\ord}$.
By Proposition \ref{cor:fil_by_ord}, 
we have a right exact sequence of $\TT_\fm$-modules
\begin{equation*}
    M_2^{\ord}\to M^\red\to M_1^{\ord}\to 0
\end{equation*}
Let $S\subset M$ be the preimage of $S_1^{\ord}$ via
$M\to M^\red\to M_1^{\ord}$.
By \ref{cond:C2}
we have the commutative diagram
\[
\begin{tikzcd}
	S\arrow[r,"\xx"] \arrow[d]
	& \fq S \arrow[r] \arrow[d]
	& \fq S^{\red} \arrow[r] \arrow[d] & 0\\
	M\arrow[r,"\xx"]
	& \fq M \arrow[r]
	& \fq M^{\red} \arrow[r] & 0
\end{tikzcd}
\]
and apply the snake lemma gives a right exact sequence of $\TT_\fm$-modules
\begin{equation*}
	M/S\xrightarrow{\xx} \fq M/\fq S\to 
	\fq M_1^\red/\fq S_1^\red \to 0
\end{equation*}
Note that by definition we have $M/S\cong M_1^\ord/S_1^\ord$
and $\xx M\subset S$. We can then use \ref{cond:C2} to show that
$\fq M^{\red}/\fq S^{\red}=\fq M_1^{\ord}/\fq S_1^{\ord}$.
And with more work we can deduce the left-exactness of the sequence.


\begin{prop}\label{cor:fund}
	Under the assumptions
    \ref{cond:C1}, \ref{cond:C3}, and \ref{cond:C2},
    we have the following commutative diagram of $\TT_\fm$-modules
    where the bottom row is exact
    \begin{equation*}
    \begin{tikzcd}
    & M/S
    \arrow[r,"\xx"]\arrow[d,"\cong"] &
    (M/S)\otimes_{\TT_\fm}\fq
    \arrow[r]\arrow[d] &
    (M/S)\otimes_{\TT_\fm}\fq^\red
    \arrow[r]\arrow[d] &0\\
    0\arrow[r] &
    M^\ord_1/S^\ord_1
    \arrow[r,"\xx"] &
    \fq M/\fq S
    \arrow[r] &
    \fq M_1^{\ord}/\fq S_1^{\ord}
    \arrow[r] &0
    \end{tikzcd}
    \end{equation*}
\end{prop}

The proof of the proposition is a long commutative algebra argument
and uses Proposition \ref{cor:Hecke_ff} in an essential way.
One first uses \ref{cond:C3} and reduces the proposition 
to showing that a suitable localization of $\fq M/\fq S$,
which becomes a $\textnormal{Frac}\Lambda\llbracket\xx\rrbracket$-module,
is nontrivial.
But Proposition \ref{cor:Hecke_ff} implies that the suitable
localization of $\TT_\fm$ is a discrete valuation ring.
Therefore the localization of $\fq M/\fq S$ coincides with that 
of $M/S$, which we can show to be nontrivial by \ref{cond:C3}.


\begin{rem}
Aside from some modifications,
the formulation and the proof of the proposition
are entirely from \cite[Prop. 6.3.5]{urban},
which deals with the case of Eisenstein ideals on modular curves.
We also refer to \cite[Thm 3.4.1]{urban}
for a different proof 
which relies on the geometry of eigencurves.
\end{rem}


\section{Construction of Euler systems}

Let $\psi$ be the finite order anticyclotomic Hecke character
as in the introduction and let $W^-$ be the Galois group 
of the maximal anticyclotomic $\Zp^d$ extension 
$\tilde{\K}^a$, where $d=[\F:\Q]$.
Write $\I=\oo\llbracket W^-\rrbracket$. 
We would first like to have an eigensystem 
\[
\lambda^{\ord}\colon \TT^{\ord}(U^p)\to\I
\]
that involes the deformation of $\psi$ and
satisfies the assumptions from the previous section.

\subsection{Hida family of theta lifts}

Let $\Psi\coloneqq \psi^{-1}\langle*\rangle$,
where $\langle*\rangle\colon \Gal_\K\to \I$ is the tautological character.
In the previous work \cite{lee} of the author,
after enlarging $E$ to a finite extension
of $\bar{\Q}_p^{un}$,
we can interpolate
theta lifts of Hecke characters and obtain
\begin{itemize}
    \item a Hida family $\euF\in S^{\ord}(U^p,\I)$ which induces
    the eigensystem $\lambda^{\ord}\colon \TT^{\ord}(U^p)\to\I$
    with 
    \begin{equation}\label{eq:eigen}
        \lambda^{\ord}(T_w^{(1)})=
        \epsilon^{-1}\hat{\chi}_\circ(\varpi_w)+
        \epsilon^{-1}\hat{\chi}_\circ\Psi(\varpi_w),\quad
        \lambda^{\ord}(U_w^{(1)})=
        \epsilon^{-1}\hat{\chi}_\circ\Psi(\varpi_w).
    \end{equation}
    \item an element
    $L=\Omega_p^{-2\Sigma}\mathbf{B}(\euF,\euF)\in \I$, where 
    $\mathbf{B}\colon S^{\ord}(U^p,\I)\times S^{\ord}(U^p,\I)\to\I$ 
    is a certain Hecke-equivariant symmetric pairing, which induces
    a homomorphism $S^{\ord}(U^p,\I)\to M^{\ord}(U^p,\I)$.
\end{itemize}

Here $\chi_\circ$ is an auxiliary Hecke character of $\K$ that
defines the theta lift and $(\Omega_p,\Omega_\infty)$ 
are the CM periods associated to $\K$.
Let $\alpha$ be Hecke character such that the $p$-adic avatar
$\widehat{\alpha}$ defines a character of $W^-$.
We have shown in \cite{lee2} that if $\alpha$ 
has the infinity type $(k+1)\Sigma$ for $k\geq0$ and $\tilde{\alpha}\coloneqq\alpha/\alpha^c$,
then
\begin{equation*}
        \frac{1}{\Omega_p^{(2k+2)\Sigma}}
        \int_{W^-}\widehat{\alpha}\,L=C(\K,\chi)
        \left(\frac{2\pi}{\Omega_\infty}\right)^{(2k+2)\Sigma}
    \textnormal{Im}(\delta)^{k}(\psi^{-1}\tilde{\alpha})(z_\delta^{-1})
    \frac{\Gamma((k+2)\Sigma)}{(2\pi)^{(k+2)\Sigma}}
    L(1,\psi^{-1}\tilde{\alpha})
    \prod_{w\in \Sigma_p}E_w(\psi^{-1}\tilde{\alpha}),
\end{equation*}
where $\delta=-\delta^c\in\K$, $z_\delta\in \A_{\K,f}^\times$,
and $C(\chi,\K)\in\oo$ are constant and
$E_w(\psi^{-1}\tilde{\alpha})$ is the modified Euler factor 
\[
\frac{
(1-\psi^{-1}\tilde{\alpha}(\varpi_\bw)q_\bw^{-1})
(1-\psi\tilde{\alpha}^{-1}(\varpi_w))}
{{\varepsilon(1,(\psi^{-1}\tilde{\alpha})_w,\psi_w)}}
\]
introduced by Coates and Perrin-Rious \cite{CP89}.
From the interpolation formula we see that
$L$ coincides with the anticyclotomic $p$-adic $L$-function
of Katz and Hida-Tilouine up to a units and the constant $C(\chi,\K)$.

\begin{rem}
We have $C(\chi,\K)\neq 0$ if the central 
$L$-value of $\chi_\circ$ is nonzero.
To the limited knowledge of the author, 
the most general result towards the existence of
such an character $\chi_\circ$ with the properties
\begin{enumerate}[label={($\chi$\arabic*)}]
    \item $\chi_\circ$ restricts to $\qch_{\K/\F}|\cdot|_{\F}$,
    where $\qch_{\K/\F}$ is the quadratic character associated
    to $\K/\F$.
    \item $\chi_\circ$ is unramified at places above $p$ and has
    the infinity type $\Sigma^c$.
    \item \label{cond:chi1}
    The central $L$-value of $\chi_\circ$ is nonzero.
\end{enumerate}
is to combine the results of \cite{Rohrlich} and \cite{Hsieh2012},
which actually produces the stronger result that 
the central value is nonzero modulo $p$.
This is the main reason we impose the condition \eqref{cond:K2in}.
\end{rem}





\subsection{Main construction}

Let $\lambda\colon\TT(U^p)\to \I$ denote the composition like before,
From \eqref{eq:eigen} we have
\[
    \lambda\circ T(U^p)=
    \epsilon^{-1}\hat{\chi}_\circ+
    \epsilon^{-1}\hat{\chi}_\circ\Psi.
\]
Thus $\lambda$ is surjective by Chebotarev's density
and $\fm$ satisfies \eqref{cond:red_gen} because of 
\ref{cond:psi1in}. Moreover, 
\begin{itemize}
\item \ref{cond:C1} holds if we let 
$\Theta$ be the evaluation map 
of $M^{\ord}(U^p,\I)_{\fm_1}\cong \Hom_\I(S^{\ord}(U^p,\I)_{\fm_1},\I)$
at $\euF$.
\item the first half of \ref{cond:C3} holds if we let 
$F=\Omega_p^{-2\Sigma}\mathbf{B}(*,\euF)\in M^{\ord}(U^p,\I)_{\fm_1}$,
since the $p$-adic $L$-function $\Theta(F)=L$ is nonzero by \cite{Hida10}.
\end{itemize}
To verify the next half and \ref{cond:C2}, we first recall the general process 
of constructing cohomology classes from a generically irreducible
pseudo-representation.
Suppose $T\colon \mathcal{G}\to R$
is a two-dimensional pseudo-representation
into a Henselian local ring $R$
with maximal ideal $\fm_R$
and of odd residual characteristic.
Assume there exists distinct characters
$\bar{\delta}_i\colon \mathcal{G}\to R/\fm_R$ of $\mathcal{G}$
for $i=1,2$ such that  
$T\equiv \bar{\delta}_1+\bar{\delta}_2\bmod \fm_R$.
Fix a choice of $z\in \mathcal{G}$
with $\bar{\delta}_1(z)\neq \bar{\delta}_2(z)$
We then apply the Hensel lemma to the characteristic polynomial
\begin{equation*}
    P(z,X)\coloneqq
    X^2-T(z)X+\det(T)(z) \equiv 
    (X-\bar{\delta}_1(z))(X-\bar{\delta}_2(z))
    \mod \fm_R
\end{equation*}
and lift $\bar{\delta}_1(z)$ and $\bar{\delta}_2(z)$ 
to roots $\alpha,\beta$ of $P(z,X)$
such that $\alpha-\beta\in R^\times$.
We then define the functions
\begin{equation}\label{eq:presentation}
   A(\sigma)=
   \frac{T(\sigma z)-\beta T(\sigma)}{\alpha-\beta}\quad
   D(\sigma)=
   \frac{T(\sigma z)-\alpha T(\sigma)}{\beta-\alpha}\quad
   x(\sigma,\tau)=a(\sigma\tau)-a(\sigma)a(\tau).
\end{equation}
Then 
$A(\sigma)\equiv \bar{\delta}_1(\sigma)$,
$D(\sigma)\equiv \bar{\delta}_2(\sigma)\bmod\fm_R$,
and $x(\sigma,\tau)$
generate the reducibility ideal of $T$.

Moreover, when there exists $\sigma_0,\tau_0$ such that
$x=x(\sigma_0,\tau_0)\in R$ is not a zero divisor, the map
\begin{equation}\label{eq:present_repn}
    \sigma\mapsto 
    \begin{pmatrix}
        A(\sigma)& B(\sigma)\\
        C(\sigma) & D(\sigma)
    \end{pmatrix}\coloneqq
    \begin{pmatrix}
        A(\sigma)& x(\sigma,\tau_0)/x(\sigma_0,\tau_0)\\
        x(\sigma_0,\sigma) & D(\sigma)
    \end{pmatrix}
        \in \GL_2(R[1/x])
\end{equation}
defines a group representation 
and satisfies $x(\sigma,\tau)=B(\sigma)C(\tau)$.
Let $B\subset R[1/x]$ and $C\subset R$
be the $R$-submodules generated respectively by
$B(\sigma)$ and $C(\sigma)$ for $\sigma\in\mathcal{G}$.
The following proposition is a subcase of 
\cite[Thm 1.5.5]{BC}, which generalizes Ribet's lemma.
\begin{prop}\label{prop:BC}
Given a homomorphism $\phi\colon R\to S$ and suppose 
$\phi\circ T$ is reducible, so that 
$\delta_1\coloneqq \phi\circ A(\sigma)$ and 
$\delta_2\coloneqq \phi\circ D(\sigma)$ become characters.
Write $\delta=\delta_1\delta_2^{-1}$, then
there exists injections 
\begin{align*}
    \Hom_R(B,S)&\hookrightarrow H^1(\mathcal{G},S(\delta))\quad
    f\mapsto \delta_2(\sigma)^{-1}f(B(\sigma))\\
    \Hom_R(C,S)&\hookrightarrow H^1(\mathcal{G},S(\delta^{-1}))\quad
    f\mapsto \delta_1(\tau)^{-1}f(C(\tau))
\end{align*}
\end{prop}

Now, let $\delta_1=\epsilon^{-1}\hat{\chi}_\circ$,
$\delta_2=\epsilon^{-1}\hat{\chi}_\circ\Psi$ 
and let $\bar{\delta}_i$ be the residual characters.
As in previous section, we have two distinct characters
$\chi_i=\omega\bar{\delta}_i\mid_{\Gp}$ and let 
$R^{\epsilon\zeta}$ denote the universal deformation ring
of $\chi_1+\chi_2$.

Fix $z\in\Gp$ such that $\chi_1(z)\neq \chi_2(z)$.
We let $\alpha_0,\beta_0$ be the roots of $P(z,X)$
and define the $ R^{\epsilon\zeta}$-valued functions
$A_0(\sigma)$, $D_0(\sigma)$, and $x_0(\sigma,\tau)$
on $\Gp$ by \eqref{eq:presentation}.
Then pick $\sigma_0,\tau_0\in\Gp$ such that
$\xx=x(\sigma_0,\tau_0)$
and let $B_0(\sigma)$, $C_0(\sigma)$ be defined
by \eqref{eq:present_repn}.
Note that by definition 
$R^{\epsilon\zeta}=B_0\coloneqq(B_0(\sigma), \sigma\in\Gp)$.

Let $\alpha,\beta\in \TT_\fm$
denote the images of $\alpha_0,\beta_0$
under $R^{\zeta\epsilon}\to \TT_\fm$.
We similarly define the $\TT_\fm$-valued functions
$A(\sigma), D(\sigma)$, and $x(\sigma,\tau)$ on $\Gal_\K$
by \eqref{eq:presentation}, then
the restrictions of the functions to $\Gp$ coincide with the images of
$A_0(\sigma), D_0(\sigma)$, and $x_0(\sigma,\tau)$.
In particular $\xx=x(\sigma_0,\tau_0)$ is not a zero divisor
by Proposition \ref{cor:Hecke_ff}.
We can then define $B(\sigma)$ and $C(\sigma)$,
which also coincide with the images of $B_0(\sigma)$ and $C_0(\sigma)$
when restricted to $\Gp$.

\begin{lem}
Let $x_i(\sigma,\tau)$ denote the image of $x(\sigma,\tau)$ in
$\TT^{\ord}_{\fm_i}$ for $i=1,2$, then 
\begin{itemize}
    \item $x_1(\sigma,\tau)=0=x_2(\tau, \sigma)$ 
    for $\sigma\in D_{w_0}$ and $\tau\in\Gal_\K$.
    \item $x_1(\sigma,\tau)=0=x_2(\tau,\sigma)$ 
    for $\sigma\in \Gal_\K$ and $\tau\in D_{\bw_0}$.
\end{itemize}
\end{lem}
The lemma essentially follows from that 
the ordinary Galois representation to $\TT^{\ord}_{\fm_i}$
has $\epsilon^{-1}\hat{\chi}_\circ\Psi$ as a subrepresentation 
when $i=1$ and 
has $\epsilon^{-1}\hat{\chi}_\circ$ as a subrepresentation 
when $i=2$. We now define the ideal 
\[
    \fq=(x(\sigma,\tau_0)=B(\sigma)C(\tau_0)=B(\sigma)\xx
    \mid\sigma\in\Gal_\K)\subset \TT_\fm.
\]
Then $\xx\in\fq$ be definition;
$\fq F=0$ since the action factors through $\lambda$
and $\lambda\circ T(U^p)$ is reducible;
and $\fq M^{\ord}(U^p,\I)_{\fm_2}=0$
since $x_2(\sigma,\tau_0)=0$ by the lemma.
Thus we have verified \ref{cond:C2} and \ref{cond:C3}.

We may now apply Proposition \ref{cor:fund} and 
get the commutative diagram
\begin{equation*}
    \begin{tikzcd}
    & (M/S)
    \arrow[r]\arrow[d,equal] &
    (M/S)\otimes_{\TT_\fm}\fq
    \arrow[r,"q"]\arrow[d,"p"] &
    (M/S)\otimes_{\TT_\fm}\fq^\red
    \arrow[r]\arrow[d,"s"] &0\\
    0\arrow[r] &
    M^{\ord}_1/S^{\ord}_1
    \arrow[r] &
    \fq M/\fq S
    \arrow[r,"r"] &
    \fq M^{\ord}_1/\fq S^{\ord}_1
    \arrow[r] &0
    \end{tikzcd}
\end{equation*}

\begin{defn}\label{def:cong_map}
We define $f\in \Hom_{\TT_\fm}(B,\I)$ for
$B=(B(\sigma),\sigma\in\Gal_\K)\subset \TT_\fm[1/\xx]$
as follows.
Let $\tilde{F}\in M$ be a preimage of $F\bmod S^\ord_1$,
which is unique modulo $S$.
For $b\in B$, let $q(b)=b\xx=bC(\tau_0)\in\fq$ and define
\begin{align*}
   &y'(b)=(\tilde{F}\bmod S)\otimes q(b)\in(M/S)\otimes_{\TT_\fm}\fq\\
   &y(b)=p(y'(b))= q(b)\tilde{F}\bmod \fq S\in\fq M/\fq S
\end{align*}
On the other hand, observe that 
$(M/S)\otimes_{\TT_\fm}\fq^\red=
(M^{\ord}_1/S^{\ord}_1)\otimes_{\TT_\fm}\fq^\red$
and therefore
\begin{align*}
   q(y'(b))&=(F\bmod S^{\ord}_1)\otimes q(b)\in 
(M^{\ord}_1/S^{\ord}_1)\otimes_{\TT_\fm}\fq^\red\\
   r(y(b))&=(s\circ q)(y'(b))=
q(b)F\bmod \fq S^{\ord}_1\in\fq M^{\ord}/\fq S^{\ord}
\end{align*}
But $q(b){F}=0$ by \ref{cond:C2},
so there exists $F(b)\in M^{\ord}_1/ S^{\ord}_1$ that maps to $y(b)$.
We define $f(b)=\Theta(F(b))\in \I$.
\end{defn}

Observe that if $b\in\TT_\fm$, then 
we can take $F(b)=q F$ and $f(b)=qL$.
In particular this is the case if $b=B(\sigma)$ for $\sigma\in\Gp$.
On the other hand, if $w\in\Sigma_p\setminus\{w_0\}$ and $\sigma_1,\sigma_2\in D_{w}$ we can show that
\[
(A(\sigma_1)-D(\sigma_1))B(\sigma_2)=
(A(\sigma_2)-D(\sigma_2))B(\sigma_1).
\]
Moreover, let $\ff$ be a square-free product of split primes
such that $\ff$ is prime to $p$ and $\ff+\ff^c=\oo_\K$.
We let $\mathcal{R}$ be the set of $\fs\coloneqq\ff\ff^c$
such that $\iota_w^{-1}(\GL_2(\oo_w))\subset U^p$ for all $w\mid\ff$.
We then define $U^p_\fs\subset U^p$ by replacing $U_v$ with
$\Iw(w^{1,1})$ when $w\mid \ff$ and $w\mid v$,
and define $\I_\fs=\oo\llbracket\fG^a_\fs\rrbracket$, 
where $\fG^a_\fs$ is the Galois group of  the maximal anticyclotomic
pro-p abelain extension $\widetilde{\K}^a_\fs$ that is unramified away $p\fs$.


For $\fs\in \mathcal{R}$, 
let $\Psi_\fs\coloneqq \psi^{-1}\langle*\rangle_\fs$,
where $\langle*\rangle_\fs\colon \Gal_\K\to \I_\fs$ 
is the tautological character.
Then the construction in \cite{lee} also gives a Hida family
$\euF_\fs\in S^{\ord}(U^p_\fs,\I_\fs)$ and 
an eigensystem $\lambda_\fs\coloneqq \TT^{\ord}(U^p_\fs)\to \I_\fs$.
satisfying \eqref{eq:eigen}.
Furthermore, when $\fs, \ell\fs\in \mathcal{R}$
we have the commutative diagram
\[
\begin{tikzcd}
	& \TT(U^p_{\ell\fs})_\fm
    \arrow[d,"\varphi^{\ell\fs}_\fs"] \arrow[r,"\lambda_{\ell\fs}"]
    &\I_{\ell\fs} \arrow[d,"\phi^{\ell\fs}_\fs"]\\
	R^{\zeta\epsilon}\arrow[r]\arrow[ur]
	& \TT(U^p_\fs)_\fm \arrow[r,"\lambda_{\fs}"] &
    \I_\fs
\end{tikzcd}
\]
where $\varphi^{\ell\fs}_\fs$ is induced by the obvious inclusions
between algebraic modular forms and $\phi^{\ell\fs}_\fs$
is induced by the homomorphis $\fG^a_{\ell\fs}\to\fG^a_{\fs}$.
Then if $\ell=\fl\fl^c$ for a prime $\fl$, we can show that 
\[
\varphi^{\ell\fs}_\fs(B_{\ell\fs}(\sigma))=B_\fs(\sigma)
    \text{ if }\sigma\in\Gal_\K
    \text{ and }
    \phi^{\ell\fs}_\fs\circ f_{\ell\fs}=
    (1-\epsilon\Psi_{\fs}(\varpi_{\bar{\fl}}))
    (1-\Psi_\fs(\varpi_{\bar{\fl}}))
    \cdot (f_\fs\circ \varphi^{\ell\fs}_\fs).
\]

We now apply Proposition \ref{prop:BC} and
let $Z_\fs\in H^1(\K,\I_\fs(\Psi_\fs^{-1}))$
be the class constructed from 
$B_\fs(\sigma)$ and $f_\fs$; and
$Z_{\fs,p}\in H^1(\Gp,\I_\fs(\Psi_\fs^{-1}))$
be the class constructed from 
$B_\fs(\sigma)$ and $B_0=R^{\epsilon\zeta}\to \TT(U^p_\fs)\to\I_\fs$.
We omit the subscript when $\fs=\oo_\K$.
Then by the discussion above we obtain the following theorem.

\begin{thm}\label{thm:Bigeu}
The cohomology classes
$\{\mathcal{Z}_\fs\in H^1(\K, \I_\fs(\Psi_\fs^{-1}))\}_{\fs\in\mathcal{R}}$
has the following properties.
\begin{enumerate}
    \item There exists a finite set $S$ of primes of $\K$ such that 
    $\mathcal{Z}_{\fs}$ is unramified at $w$ if $w\notin S$ and 
    $w\nmid \fs$.
    \item If $w\in \Sigma_p\setminus\{w_0\}$, 
    $\loc_{w}(\mathcal{Z}_\fs)\in H^1(\K_{w}, \I_\fs(\Psi_\fs^{-1}))$
    is annihilated by $(\Psi_{\fs}^{-1}(\sigma)-1)$
    for any $\sigma\in D_{w}$.
    \item At $w_0$,
    $\loc_{w_0}(\mathcal{Z})\in H^1(\Qp, \I(\Psi^{-1}))$ coincides with
    $L\cdot \mathcal{Z}_{p}$.
    \item If $\ell\fs,\fs\in \mathcal{R}$ and $\ell=\fl\fl^c$
    for a prime $\fl$, then
    $\phi^{\ell\fs}_\fs(\mathcal{Z}_{\ell\fs})=
    (1-\epsilon\Psi_{\fs}(\varpi_{\bar{\fl}}))
    (1-\Psi_\fs(\varpi_{\bar{\fl}}))\cdot 
    \mathcal{Z}_{\fs}$
\end{enumerate}
\end{thm}

We call $\{\mathcal{Z}_\fs\in H^1(\K, \I_\fs(\Psi_\fs^{-1}))\}_{\fs\in\mathcal{R}}$
an anticyclotomic Euler system because
of the last property.
We also remark that the third property
should be viewed as a form of explicit reciprocity law.


\subsection{Application to the main conjecture}


Write $\Lambda_W^-=\I$ and let $\psi_\Lambda\coloneqq\psi\langle*\rangle$.
To comform to the previous notations, we replace $\Sigma_p$
with $\Sigma_p^c$ and consider instead the main conjecture for 
\[
\Sel(\psi_\Lambda,\Sigma_p^c)=\ker
    \big\{
    H^1(\K, \Lambda_W^{-*}(\psi_\Lambda))\to \prod_{w\notin \Sigma_p^c}
    H^1(I_w, \Lambda_W^{-*}(\psi_\Lambda))
    \big\},
\]
where $\Lambda_W^{-*}=\Hom^{\cts}(\Lambda_W^{-}, \Qp/\Zp)$.
From the algebraic functional equation 
and the technique from \cite{Hsieh2010}, 
we can first reduce the theorem to showing that
$(L_p(\psi_\Lambda, \Sigma_p^c))$ 
is contained in the characteristic ideal of
\begin{equation*}
    \Sel^{str}(\epsilon\psi^{-1}_\Lambda,\Sigma_p)=
    \ker\big\{
    H^1(\K,\Lambda_W^{-*}(\epsilon\psi^{-1}_\Lambda))\to 
    \prod_{w\nmid p}
    H^1(I_w,\Lambda_W^{-*}(\epsilon\psi^{-1}_\Lambda))\times
    \prod_{w\in \Sigma_p^c}
    H^1(\K_w,\Lambda_W^{-*}(\epsilon\psi^{-1}\langle*\rangle^{-1}))
    \big\}
\end{equation*}
in $E\llbracket W^-\rrbracket$.
In fact, let $L^\iota\in \Lambda_W^-$
denote the image of $L$ under the map
$\langle\gamma\rangle\mapsto \langle\gamma\rangle^{-1}$.
We can replace $L_p(\psi_\Lambda, \Sigma_p^c)$ with $(L^\iota)$
since the interpolation formula implies that the two 
generates the same ideal in $E\llbracket W^-\rrbracket$.
We then apply 
the specialization principle \cite{Och05} and further
reduce the theorem to the proposition below.
\begin{prop}
    For almost all characters $\alpha\colon W^-\to \oo$, the length of
    \begin{equation*}
\ker\big\{
H^1(\K, (E/\oo)(\epsilon(\psi\alpha)^{-1}))\to
\prod_{w\nmid p}
H^1(I_w,(E/\oo)(\epsilon(\psi\alpha)^{-1}))
\prod_{w\in\Sigma_p^c}
H^1(\K_w,(E/\oo)(\epsilon(\psi\alpha)^{-1}))
\big\}
\end{equation*}
is finite and bounded by the length of  the $\oo/(\alpha(L^\iota))$.
\end{prop}


Let $\alpha^\iota\colon \I_\fs\to \I_\fs$ be the homomorphism
induced by $\langle\gamma\rangle_{\fs}\mapsto 
\alpha^{-1}(\gamma)\langle \gamma^{-1}\rangle_{\fs}$
Then $\alpha^\iota$ defines an isomorphism between Galois modules
$\alpha^\iota\colon \I_\fs(\Psi_\fs^{-1})=
\I_\fs(\psi\langle*\rangle^{-1}_\fs)
\to \I_\fs(\psi\alpha\langle*\rangle_\fs)$
and by Shapiro lemma (cf \cite[Lem 5.8]{Schneider2016} for example), 
we then have an isomorphism between Galois modules
\begin{equation*}
    \alpha^\iota(\mathcal{Z}_\fs) \in 
    H^1(\K, \I_\fs(\psi\alpha\langle*\rangle_\fs))\cong 
    \varprojlim_{L\subset {\K}_\fs^a}
    H^1(L, \eo(\psi\alpha))
\end{equation*}
where $\Gal_\K$ acts on the left by $\langle*\rangle$
and on the right by the usual 
Galois actions on cohomology groups.
Translating the result from Theorem \ref{thm:Bigeu}, 
we have the collection of classes 
$\{ z_{L}\in H^1(L, \oo(\psi\alpha))\}
_{L\subset \widetilde{\K}^a_\fs, \,\fs\in \mathcal{R}}$ with the 
following properties.

\begin{enumerate}
    \item If $L\subset \tilde{\K}^a_\fs$, then 
    $z_{L}$ is unramified at $w$ if $w\notin S$ and $w\nmid \fs$.
    \item If $w\in \Sigma_p\setminus\{w_0\}$, 
    $\loc_{w}(z_L)\in H^1(L_w, \oo(\psi\alpha))$
    is annihilated by $((\psi\alpha)(\sigma)-1)$
    for any $\sigma\in D_{w}$.
    \item Let $z\in H^1(\K, \oo(\psi\alpha))$ and
    $z_p\in H^1(\Qp, \oo(\psi\alpha))$ be the bottom class of 
    $\alpha^\iota(\mathcal{Z})$ and $\alpha^\iota(\mathcal{Z}_p)$, then
    \[
        \loc_{w_0}(z)=\alpha(L^\iota)\cdot z_{p}\in 
        H^1(\Qp, \oo(\psi\alpha)).
    \]
    \item Let $\ell\fs,\fs\in \mathcal{R}$ with $\ell=\fl\fl^c$
    for a prime $\fl$ and define 
    $P_\fl(X)=(1-\epsilon(\psi\alpha)^{-1}(\varpi_\bl)X)
    (1-(\psi\alpha)^{-1}(\varpi_\bl)X)$.
    Suppose $L_\fs\subset \widetilde{\K}^a_\fs$ 
    and $L_{\fs\ell}\subset \widetilde{\K}^a_{\fs\ell}$ 
    and the inclusions do not hold for larger ideals, then
    \begin{equation*}
        \Cor_{L_{\fs}}^{L_{\fs\ell}}(z_{L_{\fs\ell}})=
        P_\fl(\Fr_\fl)\cdot z_{L_{\fs}}.
    \end{equation*}
\end{enumerate}

Now the proposition can be proved 
with the general machinery in \cite{Rubin}
after a slight modification.


\section*{Acknowledgement}

The author would like to thank Professor Kenichi Namikawa
for the opportunity to present 
and his hospitality during the conference.
The author would also like to thank Professor Eric Urban
and Professor Ming-Lun Hsieh for their support and encouragement
during the period of the project.








\bibliographystyle{amsalpha}
\bibliography{biblio}

\end{document}
